Reviewer: 4

Comments to the Author
This manuscript provides an econometric analysis that explains how network dependence is associated with the outcome of economic sanctions. The main argument is that sanctions are more likely to end in compliance as the target has a past history of compliance with the sender (compliance reciprocity). Sanctions are less likely to be successful as the target has a past history of sanction interactions as a target (sanction reciprocity). Thus, it is important to consider “the structure created by reciprocal interactions over time.” Overall, the manuscript has the potential to make an important contribution to the field in so far as it successfully explains the theoretical mechanism in which reciprocity affects the target’s incentive to make concessions. At this stage, I’m not fully convinced by the theory that the author provides. In case of R&R, the author should revise the manuscript to resolve these issues.

The theoretical discussion behind the proposed hypotheses needs further elaboration. On pages 6-8, author derives the key hypotheses simply from a brief review (and modification) of extant studies of sanctions and network analysis. For example, the author argues that “existing work on reciprocity and cooperation” shows that “one actor has incentive to “respond in kind” to the previous behavior of their partner.” The author applies this idea to sanctions using learning mechanism that results from iterated interactions between sender and target. This is obviously not a new theory but intuitive. Nevertheless, I see a leap of logic in the path dependence story. Why would previous compliance and resistance determine the target’s decision at a present episode given possible variations in the target and sender countries (e.g., issues at stake, leadership)? On pages 11-12, the author explains sanction reciprocity and the hypothesis for this variable: “The intuition behind this measure is to capture the concept of creating expectations of resolve: states who have been sanctioned multiple times by a sender state are likely to build up a willingness of resistance and not cooperation. This suggests that states receiving sanctions from those with whom they have been sanctioned before are likely to more slowly comply with those states.” It is not clear why multiple sanctions in the past would raise the resolve level of the target country for future episodes. Does the concept of reciprocity have any relevance to resolve? Or does reciprocity (e.g., cumulative history of compliance) simply represent the expected probability of winning in sanction episode? Where is strategic decision-making, which the author emphasizes, on the part of the target when the target repeats its behavior of concessions? Who learns what from past histories of resistance and compliance? More explanation should be provided. In doing so, it would be also helpful if the author provides some historical evidence in order to show how learning mechanism works in sanction episodes.