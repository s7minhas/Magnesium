Reviewer(s)' Comments to Author:
Reviewer: 1

Comments to the Author
I reviewed this paper for another journal not quite three months ago.  The referee reports from that submission were extraordinarily consistent and raised specific concerns about the clarity of the concepts being invoked, the theoretical argument, and the appropriateness of the particular empirical test.  All three referees indicated that they found the idea interesting, but they felt substantial and significant revisions were necessary to make the paper publishable.  At that time, I recommended a revise and resubmit decision because I felt the promise of the idea was worth pursuing even though I felt there were significant problems with the research as presented.  My recommendation this time is that the paper be rejected.  Only superficial revisions have been made and none of the serious criticisms regarding the theoretical argument or empirical test have been addressed.  For that reason, I have no confidence in the authors’ ability or willingness to make the changes necessary for this paper to be published.  I will summarize the objections.

Conceptual: The potential contribution of this work hinges on the concepts of “network” and “reciprocity” yet you do not tell us exactly what you mean by these.  This creates problems throughout the paper.  The way I normally understand reciprocity—“I will do unto you as you have done unto me”--doesn’t seem to fit.  There seems to be a strong element of “I will do unto you as I have done unto you before” as well as “I expect you to do unto me as you have done unto others” involved.  Similarly, it doesn’t seem like there is really isn’t anything that we normally consider as a ‘network’ involved.  Rather, states are interpreting each other’s actions in the context of other dyadic relationships—“have you done unto me more or less than others have done unto each other”.

Theoretical:  You haven’t developed a theoretical argument.  You haven’t laid out how you think sanctions episodes unfold or how you think networks and reciprocity influence that process.  You haven’t specified assumptions or shown how hypotheses follow from those.

Research design: You havn’t justified coding decisions or methods, you haven’t connected these to your concepts or theoretical argument (because you don’t have these) and you haven’t presented enough detail about your processes to allow replication.

Empirical test: You are using a hazard model for a situation in which cases can end in multiple ways.  That simply isn’t appropriate.

Each of the previous reviews raised these points, the author(s) can get more thorough explanations of these points and specific suggestions from that earlier set of reviews, should they choose to take them seriously.  The only changes that were made are cosmetic and, from what I can tell, only cover places where a referee made a very specific comment about a specific sentence or phrase.  As a result, I will admit that the paper reads better but the fundamental problems remain.  I still believe there is a good idea here, but a lot more is needed.  You’ve received some very good suggestions and advice—take that to heart.