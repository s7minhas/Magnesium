Reviewer: 2

Comments to the Author
International Interactions: GINI-2015-0901
When do States Say Uncle? Network Dependence and Sanctions Compliance

The research presented in the ms. is technically sophisticated and competently executed. Therefore I would be willing to accept the presented results if only I could understand what the key results are. The findings of Models 1 and 2 are all fairly straightforward. Adding compliance and santions reciprocity doesn’t affect existing ideas too much (apart from the slight ‘puzzle’ on the lack of robustness of target democracy). So far, so good. But the key insight should come from Model 3 where compliance and sanctions reciprocity are indeed statistically significant and the coefficient seem intuitive in the sense that ‘compliance’ leads to shorter and ‘sanctions’ to longer sanctions episodes. My problem is, however, that I cannot figure out what is meant by compliance reciprocity and sanctions reciprocity. Hence I don’t know whether they measure anything meaningful.

Reading the earlier sections (and in particular page 9) didn’t clarify matters for me. Three things confuse me the most:
-       How is compliance (also referred to as cooperation) measured? Particularly since the underlying data are sanctions episodes. Is compliance giving in to previous sanctions, or is it cooperating with the sender on imposing sanctions on a third party?
-       Does sanctions reciprocity mean that you impose sanction in reply to sanctions imposed on you? Or is it simply being the target of sanctions? I think that it is the latter, but then it is not 'reciprocity' strictly speaking.
-       Why is compliance/sanction reciprocity measured in relative terms.

If I got it right (and I may very well have got it wrong), compliance reciprocity measures whether a country is more likely to give in to sanctions (compared to other countries). If so, finding that it makes specific sanctions episodes last shorter is hardly surprising. Regardless, for the paper to be publishable the key concepts and insights need to be presented clearly.

Not only the core message is unclear, the whole theory section is presented in a convoluted way making it very difficult for the reader to understand what is going on and what the key contribution of the paper is. And there is a lot going on! The argument emphasizes the importance of reciprocity and the relevance of networks for learning and information-sharing. It introduces the distinction between compliance and sanctions but also refers to incentives. It distinguishes between onset and duration (or ending of) sanctions. To cap it all of, Copula methods are used to impute missing data. Maybe most problematic is that the ms. never follows up fully on these by themselves interesting observations.

For example, the emphasis on network structures suggests that the analysis would deal with higher-order dependencies in the data structure, but the eventual analysis is dyadic. Another example would be that ‘learning’ suggests that the santions network would be somehow endogenous, but that isn’t the case here. This isn’t the only ‘network’ paper to which these comments would apply, but this one promises a lot but then fails to deliver.

Finally (and a minor comment I know), the title: ‘When do States Say Uncle’ is rather obscure…fortunately, Wikipedia could help me out here.