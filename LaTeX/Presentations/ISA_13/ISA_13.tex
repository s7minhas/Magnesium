%!TEX TS-program = xelatex
% NOTE: as of 17 Sept 2012, this compiles in xelatex

\documentclass{beamer}
% \usetheme{Madrid} % My favorite!
% \usetheme{Antibes}
% \usetheme{Bergen} % This template has nagivation on the left
% \usetheme{Berkeley} % Nice details
%\usetheme{Berlin}
% \usetheme{Boadilla} % Pretty neat, soft color.
% \usetheme{Copenhagen} % sim to default, pretty sure Sunshine uses this
% \usetheme{Darmstadt} % not so good
% \usetheme{Dresden} % sim to Berlin
% \usetheme{default} 
% \usetheme{Frankfurt} % Similar to the default 
% \usetheme{Goettingen} % Navigation on right
% \usetheme{Hannover} % Navigation on left, soft color
% \usetheme{Ilmenau}
% \usetheme{Juanlespins} % don't have this .sty
% \usetheme{Madrid}
% \usetheme{Malmoe} % pretty good. No stuff on top, sim to Warsaw on bottom
% \usetheme{Marburg} % Nice Navigation on right
% \usetheme{Montpellier} % yuck
% \usetheme{Paloalto} % don't have this .sty
% \usetheme{Pittsburg} % don't have this .sty
% \usetheme{Rochester} % very plain
%\usetheme{Singapore} % similar to default
\usetheme{Warsaw}
%with an extra region at the top.
%\usecolortheme{seahorse} % Simple and clean template
% Uncomment the following line if you want %
%page numbers and using Warsaw theme%
%\setbeamertemplate{footline}[page number]
%\setbeamercovered{transparent}
\setbeamercovered{invisible}
% To remove the navigation symbols from 
% the bottom of slides%
\setbeamertemplate{navigation symbols}{} 
%\setbeamercovered{transparent}
%\usecolortheme{albatross}
%\usecolortheme{beetle}
%\usecolortheme{crane}
%\usecolortheme{dove}
%\usecolortheme{fly}
%\usecolortheme{seagull}
%\usecolortheme{wolverine}
%\usecolortheme{beaver} % I like this one
%
\usepackage{coordsys} % for number lines
\usepackage{graphicx}
\usepackage{multirow}
\usepackage{dcolumn}
\usepackage{caption}
\usepackage{subfig}
\usepackage{tikz}
%\usepackage{bm}         % For typesetting bold math (not \mathbold)
%\logo{\includegraphics[height=0.6cm]{yourlogo.eps}}
%

% font customization
% \usepackage{mathspec}
% \usepackage{xunicode}
% \usepackage{xltxtra}
% \setmainfont{Gill Sans}
% \setmathsfont(Digits,Latin,Greek){Gill Sans}

%%%%%%%%%%%%%%%%%%%%%%%%%%%%%%%%%%%%%%%%%
\title[When Do States Say Uncle? \hspace{14em} \insertframenumber/
\inserttotalframenumber]{When Do States Say Uncle? Network Dependence and Sanction Compliance}
\author{Shahryar Minhas \& Cassy Dorff}
\institute[Duke University]
{
{\emph{shahryar.minhas@duke.edu \& cassy.dorff@duke.edu}} \\
\medskip
Duke University 
}
\date{\today}

\graphicspath{{/Users/janus829/Dropbox/Research/Magnesium/Graphics/}}

\begin{document}
%%%%%%%%%%%%%%%%%%%%%%%%%%%%%%%%%%%%%%%%%
\begin{frame}
\titlepage
\end{frame}
%%%%%%%%%%%%%%%%%%%%%%%%%%%%%%%%%%%%%%%%%

%%%%%%%%%%%%%%%%%%%%%%%%%%%%%%%%%%%%%%%%%
\begin{frame}
\frametitle{Motivating Question}

\begin{itemize}
	\item When and why do states comply with economic sanctions? 
\end{itemize}

\end{frame}
%%%%%%%%%%%%%%%%%%%%%%%%%%%%%%%%%%%%%%%%%

%%%%%%%%%%%%%%%%%%%%%%%%%%%%%%%%%%%%%%%%%
\begin{frame}
\frametitle{Sanctions and domestic factors}
Previous literature has suggested sanctions ``work'' by destablizing leaders (Marinov 2005; Lektzian and Souva 2003) and focus on domestic factors that influence the effectiveness of sanctions. In addition, such work has often utilized a duration modeling approach to capture the time dependent nature of sanciton dynamics (Bolks Al-Sowayel 2000; McGillivray and Stam 2004).
\end{frame}
%%%%%%%%%%%%%%%%%%%%%%%%%%%%%%%%%%%%%%%%%

%%%%%%%%%%%%%%%%%%%%%%%%%%%%%%%%%%%%%%%%%
\begin{frame}
\frametitle{Sanctions and Network Dynamics}
While domestic conditions are important, another vein of literature suggests cross-cutting relationships and network dynamics should play a key role in understanding sanction outcomes (Martin, 1993; Drezner, 2000; Bapat and Morgan, 2009, Cranmer and Heinrich 2013). 

~\\

The importance of multilateral oordination and network dynamics are intuitive given the broader work on networks in international relationships (Hoff and Ward 2004, Cranmer and Desmarais 2012). 
\end{frame}
%%%%%%%%%%%%%%%%%%%%%%%%%%%%%%%%%%%%%%%%%

%%%%%%%%%%%%%%%%%%%%%%%%%%%%%%%%%%%%%%%%%
\begin{frame}
\frametitle{Bridging the Gaps}
NETWORK GRAPH HERE. Say we are combining duration + network effects + consideration of domestic factors. 
\end{frame}

%%%%%%%%%%%%%%%%%%%%%%%%%%%%%%%%%%%%%%%%%
\begin{frame}
\frametitle{Bridging the Gaps}
We suggest that present duration approaches fail to incorporate the network pressures instrinstic to international sanction processes. 

\begin{itemize}
\item Target states face a network of sanctioners, not just an individual sender state. We present a duration model that incorporates the interdependent nature of the international system. 
\item In addition, we draw on previous literature to explore how network pressures matter inform sanction compliance, as well as interact with domestic conditions of the target state. 
\end{itemize}
\end{frame}

%%%%%%%%%%%%%%%%%%%%%%%%%%%%%%%%%%%%%%%%%


%%%%%%%%%%%%%%%%%%%%%%%%%%%%%%%%%%%%%%%%%
\begin{frame}
\frametitle{Network Pressure Hypotheses}
\begin{itemize}
		\item \textbf{H1} Connectivity: The specific type of relationships between senders and receivers influence sanction compliance: the greater connectivity between senders and receivers in other network types (trade, alliance) the shorter the duration of the sanction. 
		\item \textbf{H2} Network Members: The greater the number of sanctioning sender states, the shorter the duration of the sanction. 
\end{itemize} 
\end{frame}
%%%%%%%%%%%%%%%%%%%%%%%%%%%%%%%%%%%%%%%%%

%%%%%%%%%%%%%%%%%%%%%%%%%%%%%%%%%%%%%%%%%
\begin{frame}
\frametitle{Institutions Hypothesis}
	Sanctions impose costs on groups within the country.  Affected groups will try to lobby the government to reach an accommodation with sanctioning states. 
	\begin{itemize}
	\item \textbf{H3} Target states with stronger democratic institutions will more quickly comply than those with weaker democratic processes. 
\end{itemize}
\end{frame}
%%%%%%%%%%%%%%%%%%%%%%%%%%%%%%%%%%%%%%%%%


%%%%%%%%%%%%%%%%%%%%%%%%%%%%%%%%%%%%%%%%%
\begin{frame}
\frametitle{Data}
\end{frame}
%%%%%%%%%%%%%%%%%%%%%%%%%%%%%%%%%%%%%%%%%

%%%%%%%%%%%%%%%%%%%%%%%%%%%%%%%%%%%%%%%%%
\begin{frame}
\frametitle{Constructing Network Measures}
\end{frame}
%%%%%%%%%%%%%%%%%%%%%%%%%%%%%%%%%%%%%%%%%

%%%%%%%%%%%%%%%%%%%%%%%%%%%%%%%%%%%%%%%%%
\begin{frame}
\frametitle{Results}

\begin{table}
\caption{Model 1: Network Effects}
\begin{tabular}{lccc}
\hline
& $\beta$ & $\sigma_{\hat{\beta}}$ & P-value \\
\hline
Number of Senders &  0.600 & 0.177 & 0.001 \\ 
Distance & -264.0 & 13.00 & 0.020 \\
Polcon & -2.660 &  1.190 & 0.025 \\
GDP per Capita (lagged) & -0.001 & 0.001 &  0.390 \\
Internal Conflict & 0.039 &  0.100 & 0.690 \\
Trade Network & 0.615 & 0.987 & 0.530 \\
Ally Network & 1.370 & 0.568 & 0.016 \\
IGO Network &  -0.023 & 0.014 & 0.110 \\
Recieved Sanctions & -0.098 & 0.107 & 0.360 \\
Religious Similarity &  -1.250 & 0.500  & 0.012 \\
\hline
\end{tabular}
\end{table}

\end{frame}

%%%%%%%%%%%%%%%%%%%%%%%%%%%%%%%%%%%%%%%%%

\begin{frame}
\frametitle{Results}

\begin{table}
\caption{Model 2: Network Effects & Institutions}
\begin{tabular}{lccc}
\hline
& $\beta$ & $\sigma_{\hat{\beta}}$ & P-value \\
\hline
Number of Senders &  1.200 & 0.412 & 0.004 \\ 
Distance & -251.0 & 112.0 & 0.024 \\
Polcon & -0.669 &  1.74 & 0.700 \\
GDP per Capita (lagged) & -0.001 & ] 0.001 &  0.350 \\
Internal Conflict & 0.0469 & 0.099 & 0.630 \\
Trade Network & 0.638 & 0.965 & 0.140 \\
Ally Network & 1.420 & 0.579 & 0.016 \\
IGO Network &  -0.023 & 0.014 & 0.110 \\
Recieved Sanctions & -0.098 & 0.107 & 0.360 \\
Religious Similarity &  -1.250 & 0.500  & 0.012 \\
\hline
\end{tabular}
\end{table}

\end{frame}


%%%%%%%%%%%%%%%%%%%%%%%%%%%%%%%%%%%%%%%%%
\begin{frame}
\frametitle{Conclusions}
\end{frame}
%%%%%%%%%%%%%%%%%%%%%%%%%%%%%%%%%%%%%%%%%
% \begin{frame}
% \frametitle{Summary}
% \scriptsize{
% 	\begin{itemize}
% 		\item In this presentation deck, we show replications of papers by Sorens \& Ruger (2012) and Lee (2013) Both these papers originally employed Amelia to impute missing values in their dataset. In the following slides, we show replications of their Amelia results and compare them to similar models run through a Bayesian framework using Sbgcop to impute missing values. 
% 		\item To implement Sbgcop we created 5,000 imputed datasets and then within the MCMCs for the models we randomly chose one of those datasets to use in estimating the parameters. 
% 	\end{itemize}
% 	}
% \end{frame}
% %%%%%%%%%%%%%%%%%%%%%%%%%%%%%%%%%%%%%%%%%
% \begin{frame}
% \frametitle{Replication of Sorens \& Ruger Model 3}
% \scriptsize{On the left hand side, we show the results of a Bayesian (black) and Frequenstist (grey) approach with listwise deletion. On the right hand side, we show the same but with having imputed values through Sbgcop and Amelia.}
% \begin{figure}
% 	\centering
% 	\begin{tabular}{ll}
% 		\subfloat{
% 		\includegraphics[width=0.48\textwidth]{/SorensRugerReplication/Graphics/coefpM3_noIMP.pdf}
% 		\label{fig:Smod1Latent}} &
% 		\subfloat{
% 		\includegraphics[width=0.48\textwidth]{/SorensRugerReplication/Graphics/coefpM3.pdf}
% 		\label{fig:Smod1Legend}}				
% 	\end{tabular}
% \end{figure}
% \end{frame}
% %%%%%%%%%%%%%%%%%%%%%%%%%%%%%%%%%%%%%%%%%
% \begin{frame}
% \frametitle{Replication of Sorens \& Ruger Model 4}
% \scriptsize{On the left hand side, we show the results of a Bayesian (black) and Frequenstist (grey) approach with listwise deletion. On the right hand side, we show the same but with having imputed values through Sbgcop and Amelia.}
% \begin{figure}
% 	\centering
% 	\begin{tabular}{ll}
% 		\subfloat{
% 		\includegraphics[width=0.48\textwidth]{/SorensRugerReplication/Graphics/coefpM4_noIMP.pdf}
% 		\label{fig:Smod1Latent}} &
% 		\subfloat{
% 		\includegraphics[width=0.48\textwidth]{/SorensRugerReplication/Graphics/coefpM4.pdf}
% 		\label{fig:Smod1Legend}}				
% 	\end{tabular}
% \end{figure}
% \end{frame}
% %%%%%%%%%%%%%%%%%%%%%%%%%%%%%%%%%%%%%%%%%
% \begin{frame}
% \frametitle{Replication of Lee}
% \scriptsize{On the left hand side, we show the results of a Bayesian (black) and Frequenstist (grey) approach with listwise deletion. On the right hand side, we show the same but with having imputed values through Sbgcop and Amelia.}
% \begin{figure}
% 	\centering
% 	\begin{tabular}{ll}
% 		\subfloat{
% 		\includegraphics[width=0.48\textwidth]{/LeeReplication/Graphics/LEE_bVm_coefpV2_noImputation.pdf}
% 		\label{fig:Smod1Latent}} &
% 		\subfloat{
% 		\includegraphics[width=0.48\textwidth]{/LeeReplication/Graphics/LEE_bVm_coefpV2.pdf}
% 		\label{fig:Smod1Legend}}				
% 	\end{tabular}
% \end{figure}
% \end{frame}
% %%%%%%%%%%%%%%%%%%%%%%%%%%%%%%%%%%%%%%%%%
% % End of slides
% \end{document} 