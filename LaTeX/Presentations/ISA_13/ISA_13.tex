%!TEX TS-program = xelatex
% NOTE: as of 17 Sept 2012, this compiles in xelatex

\documentclass{beamer}
% \usetheme{Madrid} % My favorite!
% \usetheme{Antibes}
% \usetheme{Bergen} % This template has nagivation on the left
% \usetheme{Berkeley} % Nice details
%\usetheme{Berlin}
% \usetheme{Boadilla} % Pretty neat, soft color.
% \usetheme{Copenhagen} % sim to default, pretty sure Sunshine uses this
% \usetheme{Darmstadt} % not so good
% \usetheme{Dresden} % sim to Berlin
% \usetheme{default} 
% \usetheme{Frankfurt} % Similar to the default 
% \usetheme{Goettingen} % Navigation on right
% \usetheme{Hannover} % Navigation on left, soft color
% \usetheme{Ilmenau}
% \usetheme{Juanlespins} % don't have this .sty
% \usetheme{Madrid}
% \usetheme{Malmoe} % pretty good. No stuff on top, sim to Warsaw on bottom
% \usetheme{Marburg} % Nice Navigation on right
% \usetheme{Montpellier} % yuck
% \usetheme{Paloalto} % don't have this .sty
% \usetheme{Pittsburg} % don't have this .sty
% \usetheme{Rochester} % very plain
%\usetheme{Singapore} % similar to default
\usetheme{Warsaw}
%with an extra region at the top.
%\usecolortheme{seahorse} % Simple and clean template
% Uncomment the following line if you want %
%page numbers and using Warsaw theme%
%\setbeamertemplate{footline}[page number]
%\setbeamercovered{transparent}
\setbeamercovered{invisible}
% To remove the navigation symbols from 
% the bottom of slides%
\setbeamertemplate{navigation symbols}{} 
%\setbeamercovered{transparent}
%\usecolortheme{albatross}
%\usecolortheme{beetle}
%\usecolortheme{crane}
%\usecolortheme{dove}
%\usecolortheme{fly}
%\usecolortheme{seagull}
%\usecolortheme{wolverine}
%\usecolortheme{beaver} % I like this one
%
\usepackage{coordsys} % for number lines
\usepackage{graphicx}
\usepackage{multirow}
\usepackage{dcolumn}
\usepackage{caption}
\usepackage{subfig}
\usepackage{tikz}
%\usepackage{bm}         % For typesetting bold math (not \mathbold)
%\logo{\includegraphics[height=0.6cm]{yourlogo.eps}}
%

% font customization
% \usepackage{mathspec}
% \usepackage{xunicode}
% \usepackage{xltxtra}
% \setmainfont{Gill Sans}
% \setmathsfont(Digits,Latin,Greek){Gill Sans}

%%%%%%%%%%%%%%%%%%%%%%%%%%%%%%%%%%%%%%%%%
\title[When Do States Say Uncle? \hspace{14em} \insertframenumber/
\inserttotalframenumber]{When Do States Say Uncle? Network Dependence and Sanction Compliance}
\author{Shahryar Minhas \& Cassy Dorff}
\institute[Duke University]
{
{\emph{shahryar.minhas@duke.edu \& cassy.dorff@duke.edu}} \\
\medskip
Duke University 
}
\date{\today}

\graphicspath{{/Users/janus829/Dropbox/Research/Magnesium/Graphics/}}

\begin{document}
%%%%%%%%%%%%%%%%%%%%%%%%%%%%%%%%%%%%%%%%%
\begin{frame}
\titlepage
\end{frame}
%%%%%%%%%%%%%%%%%%%%%%%%%%%%%%%%%%%%%%%%%

%%%%%%%%%%%%%%%%%%%%%%%%%%%%%%%%%%%%%%%%%
\begin{frame}
\frametitle{Motivating Question}

\begin{itemize}
	\item When and why do states comply with economic sanctions? 
	\item In this presentation we have demonstrate the necessity of incorporating network dynamics into predictive models of sanction compliance. 
	\item We show that the connectivity between target and sender states--in terms of cultural similarities, geographical proximity, and alliance patterns--plays an important and previously overlooked role on sanction outcomes. 
\end{itemize}

\end{frame}
%%%%%%%%%%%%%%%%%%%%%%%%%%%%%%%%%%%%%%%%%

%%%%%%%%%%%%%%%%%%%%%%%%%%%%%%%%%%%%%%%%%
\begin{frame}
\frametitle{Sanctions and domestic factors}
Previous literature has suggested sanctions ``work'' by destablizing leaders and other domestic factors
\begin{itemize}
\item Marinov 2005
\item Lektzian and Souva 2003
\end{itemize}

In addition, such work has often utilized a duration modeling approach to capture the time dependent nature of sanciton dynamics: 
\begin{itemize}
\item Bolks Al-Sowayel 2000
\item McGillivray and Stam 2004
\end{itemize}


\end{frame}
%%%%%%%%%%%%%%%%%%%%%%%%%%%%%%%%%%%%%%%%%

%%%%%%%%%%%%%%%%%%%%%%%%%%%%%%%%%%%%%%%%%
\begin{frame}
\frametitle{Sanctions and Network Dynamics}
While domestic conditions are important, another vein of literature suggests cross-cutting relationships and network dynamics should play a key role in understanding sanction outcomes 
\begin{itemize}
\item Martin, 1993; Drezner, 2000; Bapat and Morgan, 2009, Cranmer and Heinrich 2013. 
\end{itemize}

The importance of multilateral coordination and network dynamics are intuitive given the broader work on networks in international relationships 
\begin{itemize}
\item Hoff and Ward 2004
\item Cranmer and Desmarais 2012
\end{itemize} 
\end{frame}
%%%%%%%%%%%%%%%%%%%%%%%%%%%%%%%%%%%%%%%%%

%%%%%%%%%%%%%%%%%%%%%%%%%%%%%%%%%%%%%%%%%
\begin{frame}
\frametitle{Bridging the Gaps:}
\framesubtitle{1984 Sanction Network}
% NETWORK GRAPH HERE. Say we are combining duration + network effects + consideration of domestic factors. 
\vspace{-0.25cm}
\begin{figure}[ht]
  \centering
  \includegraphics[width=0.74\textwidth]{84net}
\end{figure}

\end{frame}
%%%%%%%%%%%%%%%%%%%%%%%%%%%%%%%%%%%%%%%%%

%%%%%%%%%%%%%%%%%%%%%%%%%%%%%%%%%%%%%%%%%
\begin{frame}
\frametitle{Bridging the Gaps}
We suggest that present duration approaches fail to incorporate the network pressures instrinstic to international sanction processes. 

\begin{itemize}
\item Target states face a network of sanctioners, not just an individual sender state. We present a duration model that incorporates the interdependent nature of the international system. 
\item In addition, we explore how network pressures inform sanction compliance, as well as interact with the domestic institutions of the target state. 
\end{itemize}
\end{frame}

%%%%%%%%%%%%%%%%%%%%%%%%%%%%%%%%%%%%%%%%%

%%%%%%%%%%%%%%%%%%%%%%%%%%%%%%%%%%%%%%%%%
\begin{frame}
\frametitle{Network Pressure Hypotheses}
\begin{itemize}
		\item \textbf{H1} Connectivity: The relationships between senders and receivers influences sanction compliance: 
		\begin{itemize}
			\item More proximate relationships translates to greater potential for pressure, which will lead to a shorter duration
		\end{itemize}
		\item \textbf{H2} Network Members: Higher number of sender states leads to greater potential for pressure, which will shorten the duration of the sanction 
\end{itemize} 
\end{frame}
%%%%%%%%%%%%%%%%%%%%%%%%%%%%%%%%%%%%%%%%%

%%%%%%%%%%%%%%%%%%%%%%%%%%%%%%%%%%%%%%%%%
\begin{frame}
\frametitle{Institutions Hypothesis}
	Sanctions impose costs on groups within the country. Affected groups will try to lobby the government to reach an accommodation with sanctioning states, and the power of these groups will be stronger in democracies.
	\begin{itemize}
	\item \textbf{H3} Target states with stronger democratic institutions that are under the pressure of sanctions will more quickly comply than those with less democratic institutions. 
\end{itemize}
\end{frame}
%%%%%%%%%%%%%%%%%%%%%%%%%%%%%%%%%%%%%%%%%

%%%%%%%%%%%%%%%%%%%%%%%%%%%%%%%%%%%%%%%%%
\begin{frame}
\frametitle{Conceptualizing Networks}

Two types of network effects that we want to capture:

\begin{itemize}
	\item Relationships between sender(s) of the sanction and a target state
	\begin{itemize}
		\item Alliances
		\item Distance
		\item Trade
		\item Religion
	\end{itemize}
	\item Pressures from overall network structure
	\begin{itemize}
		\item Number of senders
		\item How many sanctions targetstate has received
	\end{itemize}
\end{itemize}

\end{frame}
%%%%%%%%%%%%%%%%%%%%%%%%%%%%%%%%%%%%%%%%%

%%%%%%%%%%%%%%%%%%%%%%%%%%%%%%%%%%%%%%%%%
\begin{frame}
\frametitle{Constructing Network Measures:}
\framesubtitle{South Africa 1984 Network}

\begin{figure}[ht]
	\centering
	\includegraphics[width=0.75\textwidth]{sanet}
\end{figure}

\end{frame}
%%%%%%%%%%%%%%%%%%%%%%%%%%%%%%%%%%%%%%%%%

%%%%%%%%%%%%%%%%%%%%%%%%%%%%%%%%%%%%%%%%%
\begin{frame}
\frametitle{Data}

Sanction data from Threat and Imposition of Sanctions developed by Morgan (2009), covers sanction cases from 1945 to 2005 \\
\vspace{0.5cm}
The focus of this study is compliance to economic sanctions \\
\vspace{0.5cm}
We define compliance as:

\begin{itemize}
	\item Complete/Partial Acquiescence by Target to threat
	\item Negotiated Settlement
	\item Total/Partial Acquiescence by the Target State following sanctions imposition
	\item Negotiated Settlement following sanctions imposition
\end{itemize}

\end{frame}
%%%%%%%%%%%%%%%%%%%%%%%%%%%%%%%%%%%%%%%%%

%%%%%%%%%%%%%%%%%%%%%%%%%%%%%%%%%%%%%%%%%
\begin{frame}
\frametitle{Estimated Duration Model}

\begin{align*}
Compliance_{i,t} =\; & No. \; Senders_{j} + Constraints_{i,t} + Distance_{j,t} + \\
 &GDP \; Capita_{i,t-1} +  Internal \; Conflict_{i,t} + Trade_{j,t} + \\
 &Ally_{j,t} + IGOs_{j,t} + Sanc. \; Rec'd_{i,t} + Religion_{j,t} + \\
 &Constraints_{i,t}*No. \; Senders_{j} + \epsilon_{i,t}
\end{align*}

where $i$ represents the target state, $j$ represents the relationship between the set of senders and $i$, and $t$ the time period.

\end{frame}
%%%%%%%%%%%%%%%%%%%%%%%%%%%%%%%%%%%%%%%%%

%%%%%%%%%%%%%%%%%%%%%%%%%%%%%%%%%%%%%%%%%
\begin{frame}
\frametitle{Results}

%�r�0��b```b`�bf ��	H02�0pi�\�$fa $��\���f9D�
��P��qgul���N�������m���q;S5����b�v�Y`��㗼����Ͼ��糑��
�
w.2���}�t�W#���>���ԗ�/��綢��CD��;�+��}�>�i����Ru_=0���u�G:��;4-N��ݾ�qï�yw��_,.�r��yl�w/���=��ޫ�`���K�7�{��
��d�5���N�z(t���9j�e�z�w
^�+�g?M��Ԑ��E�s����ߟ�ˡ:�~ftXe���@�K�M-2��dI�OM��9�S5@\M?�Hî��F� ΢�r=dø��~��I�E
�i
ũy)�E�P	n���⒢�̼��KfqIb^r*�/���P��X�Y���������s��g^IjQ^b�Ш������B�S`��8��TB�̞��P&Pj�z�B0к�L�K`NJ��L
�TrNb1�oLPA���Ĥ�E\)�%�ziE�@@ל43��*�X���j[
�M%��D��8h4��)��9������1
���_�(%h

\begin{table}[ht]
\caption{Model 1: Network Effects}
\begin{center}
\begin{tabular}{rrrr}
  \hline
 & coef & se(coef) & Pr($>$$|$z$|$) \\ 
  \hline
Number of senders & 0.59 & 0.15 & 0.00 \\ 
  Constraints & 0.02 & 0.04 & 0.69 \\ 
  Distance & -218.15 & 97.87 & 0.03 \\ 
  GDP per Capita (lagged) & -0.00 & 0.00 & 0.43 \\ 
  Internal Conflcit & -0.03 & 0.08 & 0.74 \\ 
  Trade & 0.65 & 0.85 & 0.44 \\ 
  Ally & 1.12 & 0.49 & 0.02 \\ 
  IGO & -0.03 & 0.01 & 0.04 \\ 
  Rec'd Sanctions & -0.06 & 0.09 & 0.49 \\ 
  Religion & -0.90 & 0.40 & 0.02 \\ 
   \hline
  N= 1001 &&& \\
  Number of events= 42 &&& \\
  \hline

\end{tabular}
\end{center}
\end{table}


\end{frame}

%%%%%%%%%%%%%%%%%%%%%%%%%%%%%%%%%%%%%%%%%

\begin{frame}
\frametitle{Results:}
\framesubtitle{Duration Model Summary}

%�r�0��b```b`�bf ��	H02�0pi�\���$fa $�����s�_䦘���cT�������>͝��յkV[\���~�m��(N�?�h���q�gě�����Jwg�?���P�S���t�:]t��k���Mo������M���ܜ��j�%��u=����=2N�^lз߶LZG�M���K^��^?��Wxšv�N��
���w\�`�%q����<�ޯ��o'��ݱn�wauY��g{�����X�b��v����>!����{������?l����EǾ�8im����>���?��c�_�����6���7��Ϻ�}=��sS��p�AY��SӠl��T
W�(Ұ����D3��(�\�0n���_inRj�B~�Bqj^JjQ1T��9?���(13�&��Y\����
勻�(5;'d�$*h�$������"�W�Z����4*-'9�櫐���!,�99�P6���?���������$���Rs2ӁP�p0��ZȮE���b�����l%�I907p�$�$��]s"к<�f� c>�l)@��N7,P���F�͞�Y\��X��Xp0�aa00�a�<rH�
\begin{table}[ht]
\caption{Model 2: Network Effects \& Institutions}
\begin{center}
\begin{tabular}{rrrr}
  \hline
 & coef & se(coef) & Pr($>$$|$z$|$) \\ 
  \hline
Number of senders & 1.21 & 0.36 & 0.00 \\ 
  Constraints & -0.58 & 1.61 & 0.72 \\ 
  Distance & -253.00 & 102.73 & 0.01 \\ 
  GDP per Capita (lagged) & -0.00 & 0.00 & 0.38 \\ 
  Internal Conflcit & 0.03 & 0.09 & 0.71 \\ 
  Trade & 0.50 & 0.84 & 0.55 \\ 
  Ally & 1.44 & 0.54 & 0.01 \\ 
  IGO & -0.02 & 0.01 & 0.12 \\ 
  Rec'd Sanctions & -0.13 & 0.10 & 0.20 \\ 
  Religion & -1.33 & 0.45 & 0.00 \\ 
  Senders*Constraints & -1.66 & 0.92 & 0.07 \\ 
   \hline
     N= 1027 &&&\\
     Number of events= 44 &&&\\
     \hline
\end{tabular}
\end{center}
\end{table}

\end{frame}

%%%%%%%%%%%%%%%%%%%%%%%%%%%%%%%%%%%%%%%%%
\begin{frame}
\frametitle{Results:}
\framesubtitle{Survival Probability by Number of Senders}

\vspace{0.7cm}
\begin{figure}[ht]
	\centering
	\includegraphics[width=1\textwidth]{nosSurv}
\end{figure}

\end{frame}
%%%%%%%%%%%%%%%%%%%%%%%%%%%%%%%%%%%%%%%%%

%%%%%%%%%%%%%%%%%%%%%%%%%%%%%%%%%%%%%%%%%
\begin{frame}
\frametitle{Results:}	
\framesubtitle{Survival Probability by Other Network Variables}

\begin{figure}[ht]
	\centering
	\includegraphics[width=1\textwidth]{oNet}
\end{figure}

\end{frame}
%%%%%%%%%%%%%%%%%%%%%%%%%%%%%%%%%%%%%%%%%

%%%%%%%%%%%%%%%%%%%%%%%%%%%%%%%%%%%%%%%%%
\begin{frame}
\frametitle{Conclusions}

Next Steps
\begin{itemize}
	\item This project shows the interaction between network and domestic factors in explaining sanction compliance.
\end{itemize}
\end{frame}
%%%%%%%%%%%%%%%%%%%%%%%%%%%%%%%%%%%%%%%%%







% \begin{frame}
% \frametitle{Summary}
% \scriptsize{
% 	\begin{itemize}
% 		\item In this presentation deck, we show replications of papers by Sorens \& Ruger (2012) and Lee (2013) Both these papers originally employed Amelia to impute missing values in their dataset. In the following slides, we show replications of their Amelia results and compare them to similar models run through a Bayesian framework using Sbgcop to impute missing values. 
% 		\item To implement Sbgcop we created 5,000 imputed datasets and then within the MCMCs for the models we randomly chose one of those datasets to use in estimating the parameters. 
% 	\end{itemize}
% 	}
% \end{frame}
% %%%%%%%%%%%%%%%%%%%%%%%%%%%%%%%%%%%%%%%%%
% \begin{frame}
% \frametitle{Replication of Sorens \& Ruger Model 3}
% \scriptsize{On the left hand side, we show the results of a Bayesian (black) and Frequenstist (grey) approach with listwise deletion. On the right hand side, we show the same but with having imputed values through Sbgcop and Amelia.}
% \begin{figure}
% 	\centering
% 	\begin{tabular}{ll}
% 		\subfloat{
% 		\includegraphics[width=0.48\textwidth]{/SorensRugerReplication/Graphics/coefpM3_noIMP.pdf}
% 		\label{fig:Smod1Latent}} &
% 		\subfloat{
% 		\includegraphics[width=0.48\textwidth]{/SorensRugerReplication/Graphics/coefpM3.pdf}
% 		\label{fig:Smod1Legend}}				
% 	\end{tabular}
% \end{figure}
% \end{frame}
% %%%%%%%%%%%%%%%%%%%%%%%%%%%%%%%%%%%%%%%%%
% \begin{frame}
% \frametitle{Replication of Sorens \& Ruger Model 4}
% \scriptsize{On the left hand side, we show the results of a Bayesian (black) and Frequenstist (grey) approach with listwise deletion. On the right hand side, we show the same but with having imputed values through Sbgcop and Amelia.}
% \begin{figure}
% 	\centering
% 	\begin{tabular}{ll}
% 		\subfloat{
% 		\includegraphics[width=0.48\textwidth]{/SorensRugerReplication/Graphics/coefpM4_noIMP.pdf}
% 		\label{fig:Smod1Latent}} &
% 		\subfloat{
% 		\includegraphics[width=0.48\textwidth]{/SorensRugerReplication/Graphics/coefpM4.pdf}
% 		\label{fig:Smod1Legend}}				
% 	\end{tabular}
% \end{figure}
% \end{frame}
% %%%%%%%%%%%%%%%%%%%%%%%%%%%%%%%%%%%%%%%%%
% \begin{frame}
% \frametitle{Replication of Lee}
% \scriptsize{On the left hand side, we show the results of a Bayesian (black) and Frequenstist (grey) approach with listwise deletion. On the right hand side, we show the same but with having imputed values through Sbgcop and Amelia.}
% \begin{figure}
% 	\centering
% 	\begin{tabular}{ll}
% 		\subfloat{
% 		\includegraphics[width=0.48\textwidth]{/LeeReplication/Graphics/LEE_bVm_coefpV2_noImputation.pdf}
% 		\label{fig:Smod1Latent}} &
% 		\subfloat{
% 		\includegraphics[width=0.48\textwidth]{/LeeReplication/Graphics/LEE_bVm_coefpV2.pdf}
% 		\label{fig:Smod1Legend}}				
% 	\end{tabular}
% \end{figure}
% \end{frame}
% %%%%%%%%%%%%%%%%%%%%%%%%%%%%%%%%%%%%%%%%%
% % End of slides
% \end{document} 