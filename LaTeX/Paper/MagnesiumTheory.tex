\section{Theory}
\label{theory}

Duration approach fails to incorporate network pressures. We conceptualize network pressures in three main ways. First by capturing how other interactions between states-- such as trade-- might influence sanction compliance. Second, we consider that target states often face multiple sanctions by multiple sanctioners at one time. Third, we consider that reciprocal compliance occurs over time between states within the network. We incorporate all three of these relational effects into our duration model and assess whether these network dynamics condition the effect of domestic institutions. 

\begin{itemize}
\item ???Sanction Case Network: The relationship between sender(s) and the target matters for sanction compliance. Sanctions involving coalitions of sender(s) will be more quickly resolved than sanctions sent by just one state. Sanction cases where relationships are more proximate will be more quickly resolved.
\item Aggregate Network: Targets of sanctions often face a multitude of sanction cases at any given point in time. States under the pressure of a multitude of sanctions will more quickly resolve sanction cases than those facing only a few.
\item Target states with stronger democratic institutions that are under the pressure of sanctions will more quickly comply than those with less democratic institutions. Sanctions are designed to impose costs on key groups within countries. Affected groups will lobby the government to reach an accommodation with sanctioning states. The ability to successfully lobby is dependent upon political institutions (Manin, Przeworski and Stokes 1999; Barro 1973; Ferejohn 1986)
\end{itemize}

\section{Conceptualizing Networks}
Two types of network effects that we capture:

\begin{itemize}
	\item Sanction Case Network
	\begin{itemize}
		\item Number of senders associated with a sanction case
		\item Distance: The average distance between sender(s) and the receiver
		\item Trade: The share of total trade that the sender(s) make up for the receiver		
		\item Alliances: The proportion of sender(s) that are allied with the receiver
		\item IGOs: The average number of common IGOs that the sender(s) and receiver belong to
		\item Religion: Similarity of religious group makeups between sender(s) and the receiver
	\end{itemize}
	\item Aggregate Network
	\begin{itemize}
		\item Sanctions Received: Total number of sanctions to which the target state is currently exposed
	\end{itemize}
\end{itemize}

Add in reciprocity talk here. 