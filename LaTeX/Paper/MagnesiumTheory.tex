\section*{Accounting for Network Effects}
\label{neteffects}

In this section, we present our argument for incorporating reciprocity measures into models for predicting the time until sanction compliance, and describe our approach for capturing these features. In focusing primarily on domestic factors, as much of the extant literature has done, alternative explanations that incorporate external conditions relating to the sanctioning network have been ignored. The two explanations that we focus on are (1) reciprocity in the sanctioning network over time; (2) previous compliance reciprocity. \footnote{We call states that impose or threaten sanctions “senders” and those upon which they are imposed “receivers.”}

First, we visually present the sanction-year network. Figure \ref{fig:spaghetti} depicts the network of sanction cases ongoing and initiated by 1984. This network graph presents the entire sanction-year network.  Nodes represent states and the directed edges denote the sender and receiver of sanctions. This figure is complex, demonstrating that each yearly network contains important information about state behavior, whereby numerous states are involved in mulitple sanction cases during this individual year. Typical analysis on sanction duration does not capture sanction-year attributes and ignores how these network characteristics evolve over-time, and inform future sanction networks. In this sense, the concept of reciprocity has more than just methodological relevance, it has theoretical import as well. While studies of international relations tend to recognize the interdependent nature of state behavior in the theoretical sense, they often fail to uphold this theoretical intuition in empirical analysis. Early attempts are noteworthy (CITEs) and more recent work has continued to address this concern (CITE), yet studies on sanction compliance have not yet incorporated these insights. 

We develop two reciprocity concepts that account for any allow us to analyze the interdepenencies inherent to the evolution of sanction dynamics. First, we outline a \textit{compliance reciprocity} measure. By creating a yearly network, for each year available in our data, we are able to calculate compliance reciprocity over time. Essentially, this represents a target states' cumulative history of compliance between partners, or sender states. This allows us to capture whether those who have a history of cooperative behavior with eachother, tend to have more cooperative behavior in the future with their partner states. Or, in a duration context, states who recieve sanctions from those with whom they have a history of reciprocal compliance are likely to comply sooner to those states than they would with lower compliance reciprocity. Notably, this measure is different from a simple conceptualization of past interactions which might simply control for the number of times a state has complied. Such a simple measure completely ignores the fact that each state's actions are conditional on all other interactions between states. For example, if state $i$ complies often to state $j$, does state $i$ also comply a lot to all other partner states? Or, relative to all other interactions, does state $i$ comply more frequently and uniquely with state $j$? The latter idea is the one explicated within our  concept of compliance reciprocity. 

\textcolor{red}{this needs work}
To illustrate, consider the matrix $X_{ij}$ below, in which we have six actors in a round robin (dyadic) format.  These data are represented by the  which has the measured value of each of the thirty interactions in each entry, with the main diagonal being empty:

\singlespacing
\[
\left[
\begin{array}{cccccc}
 & X_{12}  & X_{13}  & X_{14} & X_{15} & X_{16} \\
X_{21}  &  & X_{23}  & X_{24} & X_{25} & X_{26} \\
X_{31}  & X_{32}  &    & X_{34} & X_{35} & X_{36} \\
X_{41}  & X_{42}  & X_{43}  &  & X_{45} & X_{46} \\
X_{51}  & X_{52}  & X_{53}  & X_{54} &   & X_{56} \\
X_{61}  & X_{62}  & X_{63}  & X_{64} & X_{65} &   \\
\end{array}
\right]
\]

\doublespacing
From this set up we can calculate the reciprocity effect for each specific dyad $ij$ simply subtracts the row and column effects along with the overall mean out of the value for dyad $ij$. \\
$\hat{g}_{ij} = X_{ij} - \hat{a}_i - \hat{b}_j - X_{\cdot \cdot}$

\textcolor{red}{$---------$}

We extend this intuition to our second key measure, \textit{sanction reciprocity}. Following the same idea as compliance reciprocity, this is a measure of how often a target state has received sanctions from the senders of any given sanction case--relative to all other sanction interactions of states. The intuition behind this measures is to capture the concept of resolve: states who have been sanctioned against multiple times by a sender state are likely to build up a willingness of resistance and not cooperation.  This suggests that states receiving sanctions from those with whom they have been sanctioned before are likely to more slowly comply to those states. The basic insight here is that states that continually reciprocate sanctioning behavior are signaling more conflictual interactions rather than cooperative ones. 

\begin{quote}
	\textbf{H1}: states who recieve sanctions from those with whom they have a history of reciprocal compliance are likely to comply sooner to those states than they would with lower compliance reciprocity. 
\end{quote}


\begin{quote}
	\textbf{H2}: states receiving sanctions from those with whom they have been sanctioned before are likely to more slowly comply to those states.
\end{quote}

\begin{figure}[ht]
  \centering
  \begin{tabular}{c}
	  \includegraphics[width=1\textwidth]{84net-crop} \\
	  \includegraphics[width=0.45\textwidth]{MapLegend}
  \end{tabular}
  \caption{Here we show the sanction network in 1984, nodes are colored by geographic coordinates of countries. Data for sanction cases comes from \citet{morgan2009threat}.}
  \label{fig:spaghetti}
\end{figure}
\FloatBarrier

While our concept of compliance and sanction reciprocity represent our major contribution, we also full account for key conditions at the sanction-case level. To demonstrate, we deconstruct the broader yearly network example of 1984, to zoom in and narrow our focus to observe the individual sanction case of South Africa in 1984. In this case, South Africa is the target of multiple sanctions, as shown in Figure \ref{fig:saneti}. Because of this, we construct a sanction network that represents each sanction South Africa faces during this year. As one can easily see, in most cases during 1984, South Africa faces more than one sanctioner, and these sanctioners vary across each sanction case. For example in the first network graph, in the top left of Figure \ref{fig:saneti}, we see that South Africa faces a sanction from India, Pakistan, and Jamaica. Yet in the top right network graph, we see that South Africa also faces a sanction from Canada, Sweden, the USA, Finland, and Austrialia. 

By the 1980s, the South African apartheid regime had been in power since 1948. The international community moved to sanction the apartheid regime in hopes to end the violence and delegitimize the regime. As unrest intensified international action became inevitable and multilateral economic sanctions were initiated. As \citet{kinne2013dependent} points out, the aim of delegitimizing a regime is a cooperative act between multiple actors, whereby its effectiveness is dependent on coordinated consent and action. To diplomatically exclude South Africa, successful coordination and cooperation amongst sanctioner states was key \citep{kinne2013dependent,christopher1994pattern}. Critical mass is thus acheived through a this group of state sanctioners. By also looking at the sanction-case network, we are able tocapture these state-level to test whether characteristics among sanctioners effect sanction compliance. For example, we might expect that a sanction from Pakistan, India, and Jamaica has substantially different implications than one from largerly ``western'' sanctioners such as Sweden, Canada, USA, Finland and Australia. We conceptualize these relationships as composed of ``pressures'' which likely influences the behavior of the target state. We present two hypotheses which focus on the sanction case network. First, it is intutitive that the number of senders should influence the willingness of the target state to comply because as the number of senders increases, the more constraints through multple relationships the target faces. The essential idea is that handling the demands of ten relationships is more influential than one. 


\begin{figure}[ht]
	\centering
	\includegraphics[width=1\textwidth]{saneti}
	\caption{Network for each sanction case that South Africa faced in 1984.}
	\label{fig:saneti}
\end{figure}
\FloatBarrier


We also expect that these relationships must be meaningful not just plentiful. Just as one would imagine that a person is less swayed by the demands of 10 strangers than the demands of a few close friends, we conceptualize senders as most influential when they interact with the target state on a number of dimensions.  Thus, for each sanction case we determine the number of senders but we also measure other dimensions of each sender's relationship to the target. In addition, we calculate and control for the average number of other sanctions being sent by the senders of each particular sanction case.

\begin{quote}
	\textbf{H3}: As the number of sender states increases for any given sanction case, the time to compliance will decrease. 
\end{quote}


 We now describe exactly what is meant behind our concept of ``proximity.'' In figure \ref{fig:saneti}, we show the six sanction cases faced by South Africa in 1984.\footnote{Data for sanction cases is from \citet{morgan2009threat} and is explained further in the following section.} For the most part, each sanction case involves a variety of actors with whom South Africa has differing cultural, geographic, diplomatic, and economic relationships. Within any individual sanction case we hypothesize (\textbf{H2}) that the proximity, (i.e. the ways in which the sender and target interact on a number of dimensions) of relationships between sender(s) of a sanction and a receiver influence whether a target state complies. We construct a number of covariates to test this idea that the normative closeness, or general ``proximity'' between sender(s) and receiver(s) increases sanction compliance. 

We focus on five key measures of the ``proximate'' nature of relationships First, we measure the average distance between sender(s) and reciever.\footnote{To construct this measure we use the minimum distance between countries from the Cshapes Dataset \citep{weidmann2010geography}.} We utilize the Correlates of War (COW) data to construct the remaining four variables. Our second covariate relating to proximity is trade, which we measure as the total share of the receiver's trade in that year accounted for by sender states. Third, we measure alliances as the proportion of sender(s) that are allied with the receiver. Forth, we measure the average number of common IGOs that the sender(s) and target state belong to. Last, we create a measure capturing similarity in the religous/cultural makeups across receiver and senders.\footnote{To determine religous/cultural similarity, we first calculate the correlation in religious makeups between countries for each year, data on religious makeups is taken from the COW World Religion dataset. For a single year this provides us with an $N \times N$ matrix where each $i-j$ cross-section represents how similar the populations of $i$ and $j$ are in terms of their religious denominations. We then add 1 to each of these scores so that the minimum value within a cross-section is 0 and the maximum is 2. We do this for every year providing us with $N \times N \times T$ matrices.}

\begin{quote}
	\textbf{H4}: Sanction cases where relationships between sender(s) and receiver(s) are more proximate will be more quickly resolved.
\end{quote}

Thus far we have explained how the relationships between sanctioners and target states matter for predicting compliance. This view has focused on the individual sanction case. However, the six separate sanctions that South Africa faced in 1984 can also be thought of within the context of a yearly sanction network. In figure \ref{fig:sanet}, we aggregate the six sanction networks into one where each separate sanction is denoted by a differing color. Here we hypothesize that states under the pressure of a multitude of sanctions will more quickly resolve sanction cases than those facing only a few.

\begin{quote}
	\textbf{H3}: States facing the pressure of a multitude of sanctions will more quickly resolve any one of those sanction cases.
\end{quote}

\begin{figure}[ht]
	\centering
	\includegraphics[width=0.75\textwidth]{sanet}
	\caption{All sanctions faced by South Africa in 1984 collapsed to one network}
	\label{fig:sanet}
\end{figure}
\FloatBarrier

Last, we include a number of covariates to account for domestic explanations of sanction compliance in the extant literature. First is a measure of domestic institutions from \citet{henisz2000a}. This measure provides an estimate of the degree of constraints underlying the political institutions of a country. Second, we control for whether a country is in the midst of a civil war using the PRIO Armed Conflict dataset. The expectation in the extant literature would be that countries with higher levels of internal instability would be more likely to comply to sanctions. Finally, we use a logged measure of GDP per capita and the percent change in annual GDP, from the World Bank, to account for the argument that economically succesful states are better able to weather the pressures of these agreements.

