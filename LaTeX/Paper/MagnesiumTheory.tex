% NOTE FROM SM: Lacking a theory here is really hurting us. We should consider adding a case on S. Africa or some other country so that we can actually sketch out a theory of network pressure and sanction compliance. 

%goal here is to go over how we are constructing our network measures and what network measures we use and data sources for those measures
%cassy in her lit review has already gone into explaining why networks matter and that hte lit hasnt incorp'd them so i dont need to rehash that point

\section*{Accounting for Network Effects}
\label{neteffects}

In this section, we present our argument for incorporating features of the sanction network into models for predicting the time until sanction compliance and describe our approach for capturing these features. In focusing primarily on domestic factors, as much of the extant literature has done, alternative explanations that incorporate external factors relating to the sanctioning network have been ignored. The two explanations that we focus on are, first, the types of influential relationships senders have to receiver states and, second, the network pressures created at the aggregate network level, across all sanctions in any given year. %CD: this "second" part could use something snappier, I am not sure I am summarizing it very well (?)

To illustrate the network characteristics essential to our argument, we begin with an illustration of the year 1984. Figure \ref{fig:spaghetti} depicts the network of sanction cases, and threats thereof, ongoing and initiated by 1984. Nodes represent states and the directed edges denote the sender and receiver of sanctions. This figure is complex, demonstrating that numerous states are involved in mulitple sanction cases during this one year.
%threats? we need to explain this a bit more, I don't really get what that means here.

\begin{figure}[ht]
  \centering
  \begin{tabular}{c}
	  \includegraphics[width=1\textwidth]{84net-crop} \\
	  \includegraphics[width=0.45\textwidth]{MapLegend}
  \end{tabular}
  \caption{Here we show the sanction network in 1984, nodes are colored by geographic coordinates of countries. Data for sanction cases comes from \citet{morgan2009threat}.}
  \label{fig:spaghetti}
\end{figure}
\FloatBarrier

Next, we deconstruct this network and narrow our focus to observe the indivdual country case of South Africa in 1984. In this case, South Africa is the target of multiple sanctions, as shown in Figure \ref{fig:saneti}. Because of this, we construct a sanction network that represents each sanction South Africa faces during this year. As one can easily see, in most cases during 1984, South Africa faces more than one sanctioner, and these sanctioners vary across each sanction case. For example, in the first network graph, in the top left of Figure \ref{fig:saneti}, we see that South Africa faces a sanction from India, Pakistan, and Jamaica. Yet in the top right network graph, we see that South Africa also faces a sanction from Canada, Sweden, the USA, Finland, and Austrialia. 

\begin{itemize}
	\item Average minimum distance between sender(s) and receiver from the Cshapes Dataset \citep{weidmann2010geography}
	\item Average level of trade between sender(s) and receiver and from the COW Dataset \citep{barbieri2009trading}
	\item Proportion of sender(s) that are allied with the receiver from the COW Dataset \citep{gibler2009international}
	\item Average Number of IGOs common between sender(s) and receiver from the COW Dataset \citep{pevehouse2004correlates}
	\item Similarity in the makeup of religious groups\footnote{To determine similarity we calculate the pearson correlation coefficient between the proportion of the population in various religious groups defined in the COW dataset for each country. For a single year this provides us with an $N \times N$ matrix where each $i-j$ cross-section represents how similar the populations of $i$ and $j$ are in terms of their religious denominations. We then add 1 to each of these scores so that the minimum value within a cross-section is 0 and the maximum is 2. We do this for every year providing us with $N \times N \times T$ matrices.} between sender(s) and receiver from the COW Dataset \citep{maoz2013world}
\end{itemize}

\begin{figure}[ht]
	\centering
	\includegraphics[width=1\textwidth]{saneti}
	\caption{Here we show a separate network for each sanction case that South Africa faced in 1984.}
	\label{fig:saneti}
\end{figure}
\FloatBarrier

Clearly these two networks are composed of a diverse set of unique actors, all of which have a specific relationship with the target state. We conceptualize these relationships as composed of ``pressures'' which likely influences the behavior of the target state. We present two hypotheses which focus on the sanction case network. First, it is intutitive that the number of senders likely influences the willingness of the target state to comply because as the number of senders increases, the more constraints through multple relationships the target faces. The essential idea is that handling the demands of ten relationships is more influential than one. However, we also expect that these relationships must be meaningful not juste plentiful. Just as one would imagine that a person is less swayed by the demands of 10 strangers than the demands of a few close friends, we conceptualize senders as most influential when they interact with the target state on a number of dimensions.  Thus, for each sanction case we determine the number of senders. We also calculate and control for the average number of other sanctions being sent by the senders of each particular sanction case.

\begin{quote}
	\textbf{H1}: As the number of sender states increases for any given sanction case, the time to compliance will decrease. 
\end{quote}

\begin{quote}
	\textbf{H2}: Sanction cases where relationships between sender(s) and receiver(s) are more proximate will be more quickly resolved.
\end{quote}

 Next, we describe what exactly is meant behind our concept of ``proximity.'' In figure \ref{fig:saneti}, we show the six sanction cases faced by South Africa in 1984.\footnote{Data for sanction cases is from \citet{morgan2009threat}.} For the most part, each sanction case involves a variety of actors with whom South Africa has differing cultural, geographic, diplomatic, and economic relationships. Within any individual sanction case we hypothesize (H2) that the proximity, (i.e. the ways in which the sender and target interact on a number of dimensions) of relationships between sender(s) of a sanction and a receiver influence whether a target state complies. To test this idea that the normative closeness, or general ``proximity''of relationships between sender(s) and receiver(s) in predicting sanction compliance we construct a number of covariates. 

We focus on five key measures of the ``proximate'' nature of relationships First, we measure the average distance between sender(s) and reciever. Next we utilize the Correlates of War (COW) data to construct variour measures. Our second covariate relating to proximity is trade, which we measure as the total share of trade that sender states accounts for. Third, we measure alliances as the proportion of sender(s) that are allied with the receiver. Forth, we measure the average number of common IGOs that the sender(s) and target state belong to. And Last, also using COW, we create a measure of similarity in relgiion across sender relationships.
 We construct this measure using the COW data. 

% Sanction Case Network: The relationship between sender(s) and the target matters for sanction compliance. Sanctions involving coalitions of sender(s) will be more quickly resolved than sanctions sent by just one state. Sanction cases where relationships are more proximate will be more quickly resolved.

	%\begin{itemize}
		%\item Mean Number of other sanctions being sent by senders
		% we should just write that as a control somewhere, right? 
	%	\item Distance: The average distance between sender(s) and the receiver (cshapes)
	%	\item Trade: The share of total trade that the sender(s) make up for the receiver (cow trade)
	%	\item Alliances: The proportion of sender(s) that are allied with the receiver (cow ally)
	%	\item IGOs: The average number of common IGOs that the sender(s) and receiver belong to (cow igos)
	%	\item Religion: Similarity of religious group makeups between sender(s) and the receiver (cow religion)
	%\end{itemize}
  
Thus far we have explained how the relationships between sanctioners and target states matter for predicting compliance. This view has focused on the individual sanction case. However, the six separate sanctions that South Africa faced in 1984 can also be thought of as a yearly sanction case network. In figure \ref{fig:sanet}, we aggregate the six sanction networks into one where each separate sanction is denoted by a differing color. Here we hypothesize that states under the pressure of a multitude of sanctions will more quickly resolve sanction cases than those facing only a few.\footnote{Our next step in this project is to include measures of reciprocity over time. This will allow us to test the argument presented in the earlier half of the paper where we suggest accumulated dependencies over time will influence the likeilhood of compliance (e.g. reciprocity: if country $i$ often complies with country $j$, will country $j$ be more likely to comply with country $i$?}

% Aggregate Network: Targets of sanctions often face a multitude of sanction cases at any given point in time. States under the pressure of a multitude of sanctions will more quickly resolve sanction cases than those facing only a few.

\begin{quote}
	\textbf{H3}: States facing the pressure of a multitude of sanctions will more quickly resolve any one of those sanction cases.
\end{quote}

\begin{figure}[ht]
	\centering
	\includegraphics[width=0.75\textwidth]{sanet}
	\caption{All sanctions faced by South Africa in 1984 collapsed to one network}
	\label{fig:sanet}
\end{figure}
\FloatBarrier

% We conceptualize network dynamics in terms of ``pressures.'' First how the interrelationships, e.g., cultural or geographic proximity, and interactions between states, e.g., trade, affect the time until compliance. Second, that target states, receivers, often face multiple sanctions from multiple senders at any given time. Last, we incorporate a few of the prominent target-focused explanations for sanction compliance. 
% Third, we consider that reciprocal compliance occurs over time between states within the network. We incorporate all three of these relational effects into our duration model.  NOTE FROM SM: should probably leave this out until we actually do it

% Though we hypothesize that network effects matter greatly for determining sanction compliance there are characteristics of sanctioned states that may better enable them to manage these pressures. 
% Third, we consider that reciprocal compliance occurs over time between states within the network. We incorporate all three of these relational effects into our duration model.
