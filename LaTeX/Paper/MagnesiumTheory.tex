\section*{Accounting for Network Effects}
\label{neteffects}

The focus of the extant literature on domestic explanations of sanction compliance neglects the process through which sanctions even occur. As \citet{cranmer2014reciprocity} note, the process through which sanctions are imposed is most accurately conceptualized as a strategic and multilateral phenomenon of interdependent relations. Figure \ref{fig:spaghetti} depicts the network of sanction cases and threats thereof ongoing and initiated by 1984. Nodes represent countries and the directed edges denote the sender and receiver of particular sanction cases. Clear from figure \ref{fig:spaghetti} is that the process through which sanctions have proliferated in the international system forms a complex network. Countries often face sanction pressures from multiple senders to whom they have a variety of relationships.

\begin{figure}[ht]
  \centering
  \begin{tabular}{c}
	  % \includegraphics[width=1\textwidth]{84net}
	  \includegraphics[width=1\textwidth]{84net-crop} \\
	  \includegraphics[width=0.45\textwidth]{MapLegend}
  \end{tabular}
  \caption{Here we show the sanction network in 1984, nodes are colored by geographic coordinates of countries.}
  \label{fig:spaghetti}
\end{figure}
\FloatBarrier

Many of the duration approaches in the extant literature fail to take into account the structure underlying the sanction Duration approach fails to incorporate network dynamics. We conceptualize network dynamics in terms of ``pressures.'' We specifically consider three main types of ``network pressures.'' First how other interactions between states -- such as trade -- might influence sanction compliance. Second, we consider that target states, receivers, often face multiple sanctions by multiple sanctioners, senders, at one time. Third, we consider that reciprocal compliance occurs over time between states within the network. We incorporate all three of these relational effects into our duration model. 

\begin{itemize}
	\item Sanction Case Network: The relationship between sender(s) and the target matters for sanction compliance. Sanctions involving coalitions of sender(s) will be more quickly resolved than sanctions sent by just one state. Sanction cases where relationships are more proximate will be more quickly resolved.
	\item Aggregate Network: Targets of sanctions often face a multitude of sanction cases at any given point in time. States under the pressure of a multitude of sanctions will more quickly resolve sanction cases than those facing only a few.
	\item Target states with stronger democratic institutions that are under the pressure of sanctions will more quickly comply than those with less democratic institutions. Sanctions are designed to impose costs on key groups within countries. Affected groups will lobby the government to reach an accommodation with sanctioning states. The ability to successfully lobby is dependent upon political institutions (Manin, Przeworski and Stokes 1999; Barro 1973; Ferejohn 1986)
\end{itemize}



\begin{figure}[ht]
	\centering
	\includegraphics[width=1\textwidth]{saneti}
	\caption{South Africa 1984 Sanction Case Network}
	\label{fig:saneti}
\end{figure}
\FloatBarrier


\begin{figure}[ht]
	\centering
	\includegraphics[width=0.75\textwidth]{sanet}
	\caption{South Africa 1984 Sanction Case Network}
	\label{fig:sanet}
\end{figure}
\FloatBarrier

\section*{Conceptualizing Networks}

Network pressures here can come from a variety of perspectives. 

\begin{itemize}
	\item Sanction Case Network
	\begin{itemize}
		\item Number of senders associated with a sanction case
		\item Mean Number of other sanctions being sent by senders
		\item Distance: The average distance between sender(s) and the receiver
		\item Trade: The share of total trade that the sender(s) make up for the receiver		
		\item Alliances: The proportion of sender(s) that are allied with the receiver
		\item IGOs: The average number of common IGOs that the sender(s) and receiver belong to
		\item Religion: Similarity of religious group makeups between sender(s) and the receiver
	\end{itemize}
	\item Aggregate Network
	\begin{itemize}
		\item Sanctions Received: Total number of sanctions to which the target state is currently exposed
	\end{itemize}
\end{itemize}

Add in reciprocity talk here. 