\subsection*{When do Sanctions End?}
\label{lit}

Previous work on the duration of sanctions, or when and why a target state will decide to comply with a particular sanction, has focused on both intrastate and interstate arguments, with an emphasis on the role that domestic factors play in preventing or promoting the efficacy of sanctions. \cite{marinov2005} argues that sanctions ``work'' by directly destabilizing heads of states. Accordingly, destabilization of leaders is a necessary condition for successful coercion.  This focus on internal state conditions echoes other work suggesting that sanction outcomes are dependent on domestic stability and institutions. For example, {\cite{dashti1997}} argue that if a regime is already experiencing a high level of internal conflict, such as protests or violent clashes, the onset of an economic sanction restricting trade would further weaken the regime. This heightens the cost of resistance against the sanction \citep{dashti1997}. 

Similarly, \cite{dorussen2001} suggest that domestic support determines the duration (or ``ending'') of sanctions. They argue that when target states' domestic constituencies support resistance against sanctions, leaders have greater incentives to not comply with the sanction, which effectively increases the sanction's duration. Further supporting the idea that domestic institutions condition whether and when states comply with sanctions, \cite{lektzian2007} find that differing institutional incentives between nondemocratic and democratic regimes influence the success of economic sanctions. \cite{mcgillivray2004} employ a hazard model to analyze a data set of 47 sanctions cases. They find that leadership change does strongly influence the duration of sanctions, but only in the case of non-democratic states. Similarly, \cite{bolks2000} consider the determinants of economic sanction duration, but these authors also restrict themselves to target state characteristics in determining time until compliance. These authors also look inside the target state to define domestic conditions that influence sanction outcome. They suggest that the ``decision-making'' environment can either hinder or help the leader take countermeasures against the sanction. This ``decision-making'' environment is affected by factors such as a lack of coordination between government actors and local instability. 

The literature has clearly argued that domestic conditions are an influential predictor of sanction compliance. Yet, interstate factors are also important to the decision-making environment: critical is the relationship between sanctioning and target states. Specifically, \cite{mclean2010friends} argue that research must address how dependencies between countries, such as trade relations, affect sanction outcomes. They show, for example, that if an important trading partner is part of a sanctioning coalition, then the costs for the target state are raised, and the sanctioning state is more likely to succeed. These types of external factors characterize group level dynamics that exist within each sanction case. %%MORE HERE FROM INTERNATIONAL LIT

Recent work has also successfully paired modeling techniques with theoretical propositions by utilizing hazard-based duration models. The extant literature has demonstrated that this approach more accurately captures the important time-variant dynamics relevant to understanding the sanction process \citep{bolks2000}. Clearly, if the goal of research is to both explain and predict when a target state is likely to comply to a sanction, then researchers have clear incentives to include time-variant data. Using a duration modeling approach allows for the assessment of whether a specific factor, such as political instability or regime type, increases or decreases the probability that a target country will comply with a sanction over time.

While it is intuitive to many researchers that trade dependence between target and sender states likely influences the duration of economic sanctions, and that domestic conditions influence the target state's cost-benefit analysis, these previous studies do not fully incorporate the \textit{evolution} of interaction between states over time. Interactions capture the endogenous behavior of states iteratively, and move the field forward from a more static focus testing mechanisms based singularly on unidimensional ties such as trade or conflict. Such interactions between actors over time can be captured by a network approach, which provides a deeper understanding of how the comprehensive history of sanctions and compliance between states influences future behavior vis-\`a-vis compliance. Furthermore, not accounting for possible interdependencies can cause inappropriate empirical inferences \citep{erikson2014dyadic}. \citet{cranmer2014reciprocity} also argue that the sanction literature has not yet accounted for network dynamics, and provide a network based model to explore sanction initiation. They demonstrate that the \textit{onset} of sanctions are best predicted by modeling the way in which interdependencies evolve over time and influence the future decisions made by states. \citet{hafner2008} also study sanction onset via a network approach and show that increases in bilateral trade decrease sanctioning behavior between states.

Critical concepts like these are currently given little attention in the research on sanction compliance. By avoiding network attributes, researchers miss a wealth of structural information that is critical to understanding the ebb and flow of international cooperation and conflict.\footnote{The insight that the international system is inherently a network and must be studied as such has gained increasing theoretical and empirical support in the broader international relations literature; most prominent is the work on trade networks,\cite{hoff2004modeling, ward:rainbow:2013} conflict,\cite{ward2007disputes} alliances,\cite{warren2010geometry} and intragovernmental organizations.\cite{cao2009networks,greenhill2010norm}} We provide a way to account for these network interdependencies within the context of a simple duration model.

