\subsection*{When do Sanctions End?}
\label{lit}

Previous work on the duration of sanctions, or when and why a target state will decide to comply with a particular sanction, has focused both on intrastate and interstate arguments, with an emphasis on the role that domestic factors play in preventing or promoting the efficacy of sanctions. \cite{marinov2005} argues that sanctions ``work'' by directly destabilizing heads of states. Accordingly, destabilization of leaders is a necessary condition for successful coercion.  This focus on internal state conditions echoes other work suggesting that sanction outcomes are dependent on domestic stability and domestic institutions. For example, {\cite{dashti1997}} argue that if a regime is already experiencing a high level of internal conflict, such as protests or violent clashes, the onset of an economic sanction restricting trade would further weaken the regime. This heightens the cost of resistance against the sanction \citep{dashti1997}. 

Similarly, \cite{dorussen2001} suggest that domestic support determines the duration (or ``ending'') of sanctions. They argue that when target states' domestic constituencies support resistance against sanctions, leaders have greater incentives to not comply with the sanction, which effectively increases the sanction's duration. Further supporting the idea that domestic institutions condition whether and when states comply with sanctions, \cite{lektzian2007} finds that differing institutional incentives between nondemocratic regimes than democratic ones influence the success of economic sanctions. \cite{mcgillivray2004} employ a hazard model to analyze a data set of 47 sanctions cases. They find that leadership change does strongly influence the duration of sanctions, but only in the case of non-democratic states. Similarly, \cite{bolks2000} consider the determinants of economic sanction duration, but these authors also restrict themselves to target state characteristics in determining time until compliance. These authors also look inside the target state to define domestic conditions that influence sanction outcome. They suggest that the ``decision-making'' environment can either hinder or help the leader take countermeasures against the sanction. This ``decision-making'' environment is affected by factors such as a lack of coordination between government actors and local instability. 

	%%
	%%CD: Is this review of Lektzian strong enough? Not sure if we need to say more here.
	%%

The literature has clearly argued that domestic conditions are an influential predictor of sanction compliance. Yet, other factors are also important to consider: critical is the relationship between sanctioning states and target states. Specifically, research must address how key dependencies between countries inform the strategic environment of states through an evolution of behavior in trade relations, ally ties, or geographic proximity over time; and how this complex, evolving strategic environment affects sanction outcomes \citep{mclean2010friends}. Each relationship between the sanctioner and the sanctioned takes on a slightly different influence dependent on these factors. For example, if a neighboring state is greatly dissatisfied with the target's behavior, than a conflict of interest could have more serious repercussions than for a sanctioner who is geographically removed from the target. These types of external factors characterize group level dynamics that exist within each sanction case. 

While the relationships between states might be an obvious component driving sanctions outcomes, the literature has yet to fully approach the study of sanction duration from a network perspective. In doing so, the literature continues to test theoretical ideas in a manner that ignores these rich interdependencies, and thus potentially mischaracterizes empirical evidence. Importantly, \citet{cranmer2014reciprocity} also argue that the sanction literature has not yet accounted for network dynamics. In their work they model the sanction network itself, and demonstrate that the \textit{onset} of sanctions are best predicted by modeling the way in which the network complex interdependencies, such as reciprocity, evolve over time and influence the future decisions made by states. Critical concepts like these are currently ignored in the research on sanction compliance. 

The literature has successfully paired modeling techniques with theoretical prepositions by turning to duration models. Research has recently demonstrated that a duration modeling approach more accurately captures the important time-variant dynamics relevant to understanding the sanction process. \cite{bolks2000} point out that a duration-modeling approach is able to include variables that fluctuate throughout the tenure of an individual sanction case. Clearly, if the goal of research is to both explain and predict when a target state is likely to comply to a sanction, then researchers have clear incentives to include time-variant data. Using a duration modeling approach allows for the assessment of whether a specific factor, such as political instability or regime type, increases or decreases the probability that a target country will comply with a sanction over time.

While it is intuitive to many researchers that trade dependence between target and sender states likely influences the duration of economic sanctions, and that domestic conditions influence the target's behavior, these previous studies do not fully incorporate the evolution of interaction between states over time. Interactions between actors over time can be captured by a network approach, which explores a deeper understanding of how the history of sanctions and compliance between states influences future sanction cases. By avoiding these network attributes, researchers miss a wealth of structural information that is critical to understanding the ebb and flow of international cooperation and conflict. \footnote{The insight that the international system is inherently a network and must be studied as such, is by no means original to this project, but has gained increasing theoretical and empirical support in the broader international relations literature; most prominent is the work on trade networks,\footnote{\cite{hoff2004modeling, ward:rainbow:2013}} conflict,\footnote{\cite{AnneAuthor}} alliances,\footnote{\cite{warren2010geometry}} and intragovernmental organizations.\footnote{\cite{cao2009networks,greenhill2010norm}}}

%Furthermore, current duration approaches are unable to account for the history of dependencies between countries over time, and thus ignore previous cases of compliance and sanction interdependence between target and sanctioning states.  Over time, complex interdependencies likely emerge and drive behavior between states, where if country \textit{i} complies often to country \textit{j}, country \textit{j} might also be more likely to comply to country \textit{i}. This process is typically known as reciprocity, and is one of the key network attributes we account for in our analysis below. 


