\section*{Data and Analysis}
\label{empirics}

To test the effects of network pressures on sanction compliance we use the Threat and Imposition of Sanctions (TIES) Database developed by \citet{morgan2009threat}. This database includes over 1,400 sanction case threats and initiations from 1945 to 2013. Only sanction cases initiated by 2005 are included but outcomes for cases are recorded until 2013. Our focus here is restricted to sanctions that have actually been imposed rather than simply threatened. Restricting our analysis to sanctions that have been imposed during the period of 1960 to 2005 still leaves us with over 800 unique cases. Our unit of analysis is the case-year, providing us with over 7,500 observations. For each case in the TIES database a final outcome is recorded to describe how and if the case has been resolved. We consider the target of a sanction to have complied, if the target state completely or partially acquiesces to the demands of the sanction senders or negotiates a settlement. This is the same definition of sanction success, which we refer to as compliance, that is employed by \citet{bapat2009multilateral} and \citet{bapat2013determinants}. In using this definition of compliance, approximately 36\% of cases in our dataset end with a state complying by 2013 while another 36\% remain ongoing. The remaining 27\% of cases were terminated for other reasons show below in Table \ref{tab:termCases}. Our focus here is on modeling the time until a target country complies to a sanction using the definition described above. 

\begin{table}[ht]
	\centering
	\begin{tabular}{lc}
		\hline\hline
		Outcome & Frequency \\
		\hline
		Capitulation by Sender After Imposition & 160 \\
		Stalemate after Sanctions Imposition & 71 \\
		\hline\hline
	\end{tabular}
	\caption{Outcomes of sanction cases no longer ongoing where compliance was not achieved.}
	\label{tab:termCases}	
\end{table}

\begin{center}
\textit{Modeling Approach}\\
\end{center}

Next we discuss our modeling approach. To estimate the effect of network pressures on the ability of threatened or sanctioned states to resist compliance, we use Cox proportional hazard (PH) models of the length of threat or sanction periods. Specifically, the dependent variable, sanction spell, is the number of years that a state has not complied to a threat or sanction at time $t$. We model the expected length of sanction spells as a function of a baseline hazard rate and a set of covariates that shift the baseline hazard. The Cox PH specification that we employ is:

\begin{equation}
	\log h_{i}(t | \boldsymbol{X}_{i}) \; = \; h_{0}(t) \times \exp(\boldsymbol{X}_{i} \beta),
\end{equation}

where the log-hazard rate of compliance in a sanction case, $i$, conditional on having not complied for $t$ years is a function of a common baseline hazard $h_{0}(t)$ and covariates $\boldsymbol{X}$. In employing this approach, we assume no specific functional form for the baseline hazard and instead estimate it non-parametrically from the data.\footnote{To ensure against bias in our parameter estimates we included a vector of case-level shared frailties to account for variations in unit-specific factors. We found similar results with and without the shared frailties, so we report results without the inclusion of this additional term.}  The covariates $\boldsymbol{X}$ operate multiplicatively on the hazard rate, shifting the expected risk of compliance up or down depending on the value of $\beta$ \citep{crespo2013political}.

Providing no specific functional form for the baseline hazard necessitates testing the proportional hazard assumption. \citet{keele2010proportionally} notes that not inspecting this assumption in the covariates can lead to severely biased parameter estimates. To address this issue, we first fit smoothing splines for all continuous covariates. After ascertaining that none of the continuous covariates in our model required modeling with splines, we carried out tests of non-proportionality.\footnote{For those covariates where the non-proportional effects assumption does not hold, we include interactions between the covariate and spell duration (log scale).} 

We also impute missing values to avoid excluding instances of compliance. If we employed list-wise deletion, we would lose almost 1,000 country-year observations and 100 unique sanction cases. Previous research has already highlighted how simply deleting missing observations can lead to biased results.\footnote{For example, see \citealp{rubin1976inference,honaker2010missing}.} To impute missing values, we use a copula based approach developed by \citet{hoff:2007}. Details on our imputation process and results based on the original dataset, which are nearly identical, can be found in the \nameref{appendix}.\footnote{We include summary statistics of the original and imputed datasets used for analysis in the \nameref{appendix} as well. } 

In addition to including our two key measures of reciprocity, discussed in the previous section, we control for a variety of other components identified as being important in determining whether a target states complies to a sanction. First, is a simple counter of the number of sender states involved in a sanction to account for the evidence that multilateral sanctions appear to lead to compliance more frequently than unilateral sanctions \citep{bapat2009multilateral}. However, along with many others in the literature, we expect that what determines compliance it is not simply the number of senders involved but also the relationships that a sanctioned state has with senders.

Some of the recent literature on sanction compliance has turned to examining the relationships between the senders and receiver of a sanction. Just as one would imagine that a person is less swayed by the demands of 10 strangers than the demands of a few close friends, we conceptualize senders as most influential when they interact with the target state on a number of dimensions. For example, \cite{mclean2010friends} argue that countries are more likely to comply if they are sanctioned by their major trading partners. To account for these types of explanations, we incorporate a number of variables that describe the relationship a sanctioned state has with its senders. First, we measure the average distance between sender(s) and receiver.\footnote{To construct this measure we use the minimum distance between countries from the Cshapes Dataset \citep{weidmann2010geography}.} Our second covariate relating to proximity is trade, which we measure as the total share of the receiver's trade in that year accounted for by sender states.\footnote{Data for this measure is taken from the Correlates of War (CoW) Trade dataset \citep{barbieri2009trading}.} Last, we measure alliances as the proportion of sender(s) that are allied with the receiver.\footnote{Data for this measure is obtained from the CoW Formal Alliances dataset \citep{gibler2004measuring}.}

Finally, we include a number of covariates to account for domestic explanations of sanction compliance in the extant literature. First, previous work has examined the relationship between sanctions and regime type \citep{allen2005}; to control for this we include a measure of the target states' domestic institutions from the Polity IV data.\footnote{See \cite{marshall2002polity}. Specifically, we use the ``polity2'' variable from the Polity IV data.} This measure is computed by subtracting a country's autocracy score from its democracy score, and is scaled from 0 to 20. Previous research has shown that sanctions will be more effective when the target states' domestic institutions are more democratic. Second, we control for the level of internal conflict within  a country using the weighted conflict index from the Cross National Time-series Data Archive \citep{banks2011cross}. The expectation in the extant literature is that countries with higher levels of internal instability would be more likely to comply with sanctions. Finally, we use a logged measure of GDP per capita and the percent change in annual GDP, from the World Bank, to account for the argument that economically successful states are better able to weather the pressures of these agreements.

Below we show our full model specification: 


\begin{align*} 
		Compliance_{i,t} =& \\
		&Sanction \; Reciprocity_{j,t-1} + Compliance \; Reciprocity_{j,t-1} + \\
		&No. \; Senders_{j} + Distance_{j} + Trade_{j,t-1} + Ally_{j,t-1} + \\
		&Polity_{i,t-1} + Ln(GDP \; Capita)_{i,t-1} +\\
		&GDP \; Growth_{i,t-1} + Internal \; Conflict_{i,t-1} + \epsilon_{i,t}
\end{align*}

where $i$ represents the target of the sanction, $j$ represents the relationship between the set of sender(s) for a particular sanction case and $i$, and $t$ the time period -- variables without a $t$ subscript are time-invariant.
