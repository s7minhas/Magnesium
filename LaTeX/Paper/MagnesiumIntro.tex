\section*{Introduction}
\label{intro}

Economic sanctions are a frequently used foreign policy tool in the realm of international relations. Typically, one or more states initiate sanctions against another state when they perceive the target state as non-cooperative. The trigger for economic sanctions can occur in many contexts: the target state breaks a previous agreement, the target state openly disobeys international law, or the target state engages in behavior that is simply unfavorable to the political preferences of another state. Take for example, in November of 2012 when the Obama administration imposed sanctions on the Iranian government for blocking Internet access, mobile-phone lines and satellite television channels from the public. 

Policymakers continually engage in heated debates over the use of sanctions as a means to avoid war while still taking a stand. The motivations for sanction initiation are cross-cutting, spanning across a diverse and interdependent mix of policy issues and political actors. While the concept of sanctions -- the idea that countries can put pressure on their economic ties to other countries in order to influence policy -- is relatively straightforward, the study of when and why sanctions work is complex. While earlier research on sanctions argued that sanctions have little influence on targets \citep{lam1990, dashti1997, morgan1997, drezner1998} more recent research suggests that the effectiveness of sanctions is dependent on an interaction of several factors, namely: the number of senders acting as a part of the sanctioner group and the type of issue in dispute \citep{miers2002, morgan2009threat}; the strength of domestic institutions within the target state; and the type of regime governing the target state \citep{mcgillivray2004}. 

We agree with the theoretical and empirical literatures that suggest several different, interacting conditions are at work when predicting the outcome of sanctions. We argue, however, that political scientists have thus far failed to incorporate a key factor into their analysis: network dependencies. Drawing on the work in international relations on trade and conflict, we suggest that sanction cases are best conceptualized as a network phenomenon and must be modeled as such. In each and every sanction case, not only is there a network of sanctions (i.e., how many states in the international network are sending or receiving sanctions in a given time frame) but there is also the micro-level network of the sanction case itself (i.e., there is typically a target who faces a network of sanctioners). We analyze the endogenous structures inherit to network dynamics -- such as reciprocity -- and argue that these structures must be accounted for in studies of sanction outcomes. Further, we extend on previous work suggesting duration models as the most appropriate approach for modeling sanctions outcomes by incorporating network measures into the duration framework. In doing so we are able to return to key hypotheses from the literature and assess whether factors such as domestic political institutions and internal stability influence sanction outcome once network dynamics are adequately incorporated into the model. 

We leverage the network modeling approach to produce an accurate test of when and why sanctions end. In the following section, we review previous work on compliance and introduce the network concept. We then present our central argument and hypotheses; in doing so we articulate the various ways that networks can be conceptualized in this context. Last, we present our findings and review the results.

%maybe a summary paragraph and a "we proceed as follows: or something"