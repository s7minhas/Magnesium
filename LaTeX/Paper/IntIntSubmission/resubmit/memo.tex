\documentclass[12pt,letterpaper]{article}
\usepackage{amsfonts}
\usepackage{graphicx}
\usepackage{harvard}
\renewcommand{\harvardand}{and} 
\usepackage{float}
\usepackage{endnotes}
\usepackage{amsmath}
\usepackage{setspace}
\usepackage{rotating}
\usepackage{endnotes}
\usepackage{color}
\usepackage[nofiglist, notablist]{endfloat}
\setlength\textheight{9in}
\setlength\textwidth{6.5in}
\setlength\oddsidemargin{0in}
\setlength\evensidemargin{0in}
\setlength\topmargin{0in}
\setlength\headheight{0in}
\setlength\headsep{0in}
%\doublespacing
%\raggedright
\begin{document}

\title{\large{Revision Memo ``When do States Say Uncle? Network Dependence and Sanctions Compliance GINI-2015-0901''} }
\date{\today}
\maketitle


\emph{We are grateful for the reviewer's helpful comments and the positive evaluation of this manuscript. The following responses (ours in Italics) address all the \textbf{editor's and reviewer's comments}. We are hopeful that the reviewers find the MS greatly improved and for the opportunity to potentially present our work in  \emph{International Interactions}.} \\[4ex]

\section*{Editor}
\begin{itemize}
\item In aggregate, the reviewers do not make a clear substantive recommendation at this moment. However, in my judgement the paper has the potential to be of considerable interest to many readers. As such, I am offering the opportunity to revise and resubmit it. Please pay careful attention to the reviewers comments. In this case, I believe some rather significant changes will be needed to improve the impact of the manuscript and increase the probability of a successful outcome. 

\item As you will see the reviewers are split on their judgement of the manuscript's merits. My reading was similar to several of the more positive spins on the potential of the work, but agreed with the criticisms that more clarity and translation was needed to broaden the impact of the piece. In particular, the revision should spend more time providing examples of mechanisms and being precise about how this work connects with pre-existing terminology and findings on sanctions (and again being precise about distinctions). Too much of the paper reads like a methodology exercise conducted on convenient data (which can be quite informative, but does not do the work of telling us why it is important). 

\item Moreover, some simple revisions that highlight why the specification provided brings inferences closer to theoretical expectation about the data generation process (again, from the pre-existing literature) could go a long way to broadening the impact of the manuscript. The reviewers will be looking for improved clarity in the theoretical concepts and mechanisms in the next draft.

\emph{Thank you for these comments; they served to greatly improve the manuscript. We have re-organized  and re-written sections of our paper to better emphasize how our study relates to the history of studies on reciprocity in International Relations (see the section  ``Sanctions as a Network Process $\&$ the Importance of Reciprocity''). In doing so, we streamline the paper to focus carefully on reciprocity while clarifying (and removing) extraneous language relating to learning and information sharing, which once served to muddle the paper. We have also added more information about our assumptions about reciprocity and, again, relate it to previous work. Additionally, we've added further interpretation of reciprocity through a visual example. We hope that you and the reviewers view these changes as substantial and adequate.}

\end{itemize}

\section*{Reviewer I}
\begin{itemize}
% I dont think we need to include this here since in this point the author is only complaining about us ignoring him earlier. 
% \item I reviewed this paper for another journal not quite three months ago.  The referee reports from that submission were extraordinarily consistent and raised specific concerns about the clarity of the concepts being invoked, the theoretical argument, and the appropriateness of the particular empirical test.  All three referees indicated that they found the idea interesting, but they felt substantial and significant revisions were necessary to make the paper publishable.  At that time, I recommended a revise and resubmit decision because I felt the promise of the idea was worth pursuing even though I felt there were significant problems with the research as presented.  My recommendation this time is that the paper be rejected.  Only superficial revisions have been made and none of the serious criticisms regarding the theoretical argument or empirical test have been addressed.  For that reason, I have no confidence in the authors' ability or willingness to make the changes necessary for this paper to be published.  I will summarize the objections.

\item Conceptual: The potential contribution of this work hinges on the concepts of �network� and �reciprocity� yet you do not tell us exactly what you mean by these.  This creates problems throughout the paper.  The way I normally understand reciprocity��I will do unto you as you have done unto me�--doesn't seem to fit.  There seems to be a strong element of �I will do unto you as I have done unto you before� as well as �I expect you to do unto me as you have done unto others� involved.  Similarly, it doesn't seem like there is really isn't anything that we normally consider as a �network' involved.  Rather, states are interpreting each other's actions in the context of other dyadic relationships��have you done unto me more or less than others have done unto each other�.

\item Theoretical:  You haven't developed a theoretical argument.  You haven't laid out how you think sanctions episodes unfold or how you think networks and reciprocity influence that process.  You haven't specified assumptions or shown how hypotheses follow from those.

\emph{We now spend more time justifying why, and how, sanctions are a network process. From here, we then outline our assumptions (p.8) behind reciprocity as well as additional interpretation of a graphic depiction of reciprocity. We explain the ways in which reciprocity functions in other IR-related scenarios (p.7), why reciprocity does not encourage path-dependent spirals between countries (p.8). We hope that these sections now flow in a much clearer way and serve to address your comments.}

\item Research design: You haven't justified coding decisions or methods, you haven't connected these to your concepts or theoretical argument (because you don't have these) and you haven't presented enough detail about your processes to allow replication.
\item Empirical test: You are using a hazard model for a situation in which cases can end in multiple ways.  That simply isn't appropriate.

\emph{We discuss, on multiple pages, why a duration modeling approach is both theoretically and empirically justified by our own argument as well as the large body of research employing this method elsewhere in the study of sanctions (see, for example, pages 3,4,9,and 13.)}

\emph{We consider the target of a sanction to have complied, if the target state completely or partially acquiesces to the demands of the sanction senders or negotiates a settlement. This is the same definition of sanction success, which we refer to as compliance, that is employed by \citet{bapat2009multilateral} and \citet{bapat2013determinants}. Cases in which the sender gives up are simply coded as zeros until the sender has given up and dropped afterwards. Our focus here is only on determining the relationship of reciprocity and the time until a target state complies to the senders of a sanction. } 

% Same as the above point.
% \item Each of the previous reviews raised these points, the author(s) can get more thorough explanations of these points and specific suggestions from that earlier set of reviews, should they choose to take them seriously.  The only changes that were made are cosmetic and, from what I can tell, only cover places where a referee made a very specific comment about a specific sentence or phrase.  As a result, I will admit that the paper reads better but the fundamental problems remain.  I still believe there is a good idea here, but a lot more is needed.  You've received some very good suggestions and advice�take that to heart.

\end{itemize}

\section*{Reviewer 2}
\begin{itemize}

\item The research presented in the ms. is technically sophisticated and competently executed. Therefore I would be willing to accept the presented results if only I could understand what the key results are. The findings of Models 1 and 2 are all fairly straightforward. Adding compliance and santions reciprocity doesn't affect existing ideas too much (apart from the slight �puzzle' on the lack of robustness of target democracy). So far, so good. But the key insight should come from Model 3 where compliance and sanctions reciprocity are indeed statistically significant and the coefficient seem intuitive in the sense that �compliance' leads to shorter and �sanctions' to longer sanctions episodes. My problem is, however, that I cannot figure out what is meant by compliance reciprocity and sanctions reciprocity. Hence I don't know whether they measure anything meaningful.

\item Reading the earlier sections (and in particular page 9) didn't clarify matters for me. Three things confuse me the most:
\begin{enumerate}
\item How is compliance (also referred to as cooperation) measured? Particularly since the underlying data are sanctions episodes. Is compliance giving in to previous sanctions, or is it cooperating with the sender on imposing sanctions on a third party?
\item Does sanctions reciprocity mean that you impose sanction in reply to sanctions imposed on you? Or is it simply being the target of sanctions? I think that it is the latter, but then it is not 'reciprocity' strictly speaking.
\item Why is compliance/sanction reciprocity measured in relative terms.
\end{enumerate}

% If I got it right (and I may very well have got it wrong), compliance reciprocity measures whether a country is more likely to give in to sanctions (compared to other countries). If so, finding that it makes specific sanctions episodes last shorter is hardly surprising. Regardless, for the paper to be publishable the key concepts and insights need to be presented clearly.

\item Not only the core message is unclear, the whole theory section is presented in a convoluted way making it very difficult for the reader to understand what is going on and what the key contribution of the paper is. And there is a lot going on! The argument emphasizes the importance of reciprocity and the relevance of networks for learning and information-sharing. It introduces the distinction between compliance and sanctions but also refers to incentives. It distinguishes between onset and duration (or ending of) sanctions. To cap it all of, Copula methods are used to impute missing data. Maybe most problematic is that the ms. never follows up fully on these by themselves interesting observations.

\emph{We took these comments very seriously, and ultimately (after reviewing literature on learning and information sharing) deciding to stream-line our argument entirely around reciprocity. In doing so, we've moved our explanation of sanctions-as-a-network process to the forefront of our discussion. Then we've outlined the reciprocity concept as it relates to other major IR research. Finally we now present clear assumptions around reciprocity (see p.8 for example). We do all of this prior to introducing the measurements of reciprocity. Finally, due to space constraints our copula methods are explained fully in the appendix. }

\item For example, the emphasis on network structures suggests that the analysis would deal with higher-order dependencies in the data structure, but the eventual analysis is dyadic. Another example would be that �learning' suggests that the sanctions network would be somehow endogenous, but that isn't the case here. This isn't the only �network' paper to which these comments would apply, but this one promises a lot but then fails to deliver.

\emph{These are fair concerns, and we have tried to be much more careful and precise about our language. We do think that reciprocity matters over time but have limited our claims around endogeny, which we are unable to fully address.}

\item Finally (and a minor comment I know), the title: �When do States Say Uncle' is rather obscure�fortunately, Wikipedia could help me out here.

\emph{The original title was derived from a colloquial joke in which "saying uncle" admits defeat or acquiescence. The origins of this phrase are somewhat uncertain, though there is some evidence that it largely a north american phrase. We found one article that suggests the phrase dates back to the Roman Empire during which when young children were attacked by bullies, they wouldn't be set free until they said ``Patrue, mi Patruissmo" also meaning ``Uncle, my best Uncle." Essentially, this grants the bully a title of respect, giving him or her what they wanted.}

\end{itemize} 


\section*{Reviewer 3}
\begin{itemize}
% \item In this paper, the authors explore how sanction episodes could be thought of as network phenomena and then develop a theoretical and empirical method for assessing sanctions in terms of networks and reciprocity. The approach is appealing, and the empirical work is well-executed. My primary concerns about this paper are on the theoretical side. I believe if the authors could be more to develop their theoretical expectations this paper would make a solid contribution to the literature on sanctions and the effectiveness of these measures.
\item While reading the paper, I was struck several times with the thoughts ``What exactly do they mean by this?'' or ``That's an interesting idea. I'd like to know more about that.'' In some ways, this is a good sign � I was engaged with the paper and wanted to know more. On the other hand, all of those thoughts arose during the first half of the paper leading me to the conclusions that there are several ways in which the theory is under-developed.  For example, at the top of p.5, the authors talk about the evolution (emphasis from the authors) of interaction between states. What exactly does this entail from a theoretical standpoint? The next few sentences tell us more about network analysis than they do about the way that changing relationships between states might influence the behavior of the parties in sanctions disputes.

\item I'd also like to see a bit more discussion about compliance. What exact do the authors mean with this term? Doing everything the sender asks? Altering policies to any degree? This comes up at the top of page 6, and ends with a sentence about the ``endogenous evolution between states' shared strategic environment and past reciprocal behavior.'' Again, there's some theoretical ambiguity here for me. Perhaps I'm just smart enough to get it, but if a journal believes I'm a reasonable reviewer for this paper, then this is problematic for the authors because it means that other likely readers may not get it either.

\item The authors might want to consider tapping the literature on learning. Both in the foreign policy literature and the political economy literature, there are authors writing about how learning shapes behavior. The authors suggest a learning process but do not cite any of this previous work. Another area for greater theoretical refinement comes at the top of page 8 in the discussion of sanctions and information sharing. The authors cite Morgan \& Miers, Lacy \& Niou, and Marinov, but the informational power of the network is a distinct information process and could be described in fuller detail.

\item It's also unclear at times whether the history of cooperative behavior influences compliance reciprocity via learning or information sharing. Do we know?

\item The theoretical development is over for the most part by the middle of page 8, but the authors go into two distinct concepts of reciprocity on page 9 � compliance reciprocity and sanction reciprocity. I understand how the measures are built to assess these, but what should our theoretical expectations be? I wanted to know more about sanction reciprocity conceptually.

\item Overall, I think this paper has a lot of potential. The approach is innovative and interesting, but I feel like the authors could get a lot more bang for their buck with a little more careful attention on the theoretical side. Without more theoretical development, it feels a little like the authors had a favorite method that was applied to some data on sanctions that they happened to find. I think more is going on here, but to demonstrate that more theoretical development is needed.

\end{itemize}

\section*{Reviewer 4}
\begin{itemize}

\item This manuscript provides an econometric analysis that explains how network dependence is associated with the outcome of economic sanctions. The main argument is that sanctions are more likely to end in compliance as the target has a past history of compliance with the sender (compliance reciprocity). Sanctions are less likely to be successful as the target has a past history of sanction interactions as a target (sanction reciprocity). Thus, it is important to consider �the structure created by reciprocal interactions over time.� Overall, the manuscript has the potential to make an important contribution to the field in so far as it successfully explains the theoretical mechanism in which reciprocity affects the target's incentive to make concessions. At this stage, I'm not fully convinced by the theory that the author provides. In case of R\&R, the author should revise the manuscript to resolve these issues.

\item The theoretical discussion behind the proposed hypotheses needs further elaboration. On pages 6-8, author derives the key hypotheses simply from a brief review (and modification) of extant studies of sanctions and network analysis. For example, the author argues that �existing work on reciprocity and cooperation� shows that �one actor has incentive to �respond in kind� to the previous behavior of their partner.� The author applies this idea to sanctions using learning mechanism that results from iterated interactions between sender and target. This is obviously not a new theory but intuitive. Nevertheless, I see a leap of logic in the path dependence story. Why would previous compliance and resistance determine the target's decision at a present episode given possible variations in the target and sender countries (e.g., issues at stake, leadership)? 

\item On pages 11-12, the author explains sanction reciprocity and the hypothesis for this variable: �The intuition behind this measure is to capture the concept of creating expectations of resolve: states who have been sanctioned multiple times by a sender state are likely to build up a willingness of resistance and not cooperation. This suggests that states receiving sanctions from those with whom they have been sanctioned before are likely to more slowly comply with those states.� It is not clear why multiple sanctions in the past would raise the resolve level of the target country for future episodes. Does the concept of reciprocity have any relevance to resolve? Or does reciprocity (e.g., cumulative history of compliance) simply represent the expected probability of winning in sanction episode? 

\item Where is strategic decision-making, which the author emphasizes, on the part of the target when the target repeats its behavior of concessions? Who learns what from past histories of resistance and compliance? More explanation should be provided. In doing so, it would be also helpful if the author provides some historical evidence in order to show how learning mechanism works in sanction episodes.

\end{itemize}

\bibliographystyle{../../APSR}
\bibliography{../../magRefs.bib}

\end{document}
