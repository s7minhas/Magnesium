\documentclass[12pt,letterpaper]{article}
\usepackage{amsfonts}
\usepackage{graphicx}
\usepackage{harvard}
\renewcommand{\harvardand}{and} 
\usepackage{float}
\usepackage{endnotes}
\usepackage{amsmath}
\usepackage{setspace}
\usepackage{rotating}
\usepackage{endnotes}
\usepackage{color}
\usepackage[nofiglist, notablist]{endfloat}

\setlength\textheight{9in}
\setlength\textwidth{6.5in}
\setlength\oddsidemargin{0in}
\setlength\evensidemargin{0in}
\setlength\topmargin{0in}
\setlength\headheight{0in}
\setlength\headsep{0in}
%\doublespacing
%\raggedright
\begin{document}

\title{\large{Revision Memo ``When do States Say Uncle? Network Dependence and Sanctions Compliance GINI-2015-0901''} }
\date{\today}
\maketitle


\emph{We are grateful for the reviewer's helpful comments and the positive evaluation of this manuscript. The following responses (ours in Italics) address all the \textbf{editor's and reviewer's}  substantive comments. We are hopeful that the reviewers find the MS greatly improved and for the opportunity to potentially present our work in  \emph{International Interactions}.} \\[4ex]

\section*{Editor}
\begin{itemize}
\item In aggregate, the reviewers do not make a clear substantive recommendation at this moment. However, in my judgement the paper has the potential to be of considerable interest to many readers. As such, I am offering the opportunity to revise and resubmit it. Please pay careful attention to the reviewers comments. In this case, I believe some rather significant changes will be needed to improve the impact of the manuscript and increase the probability of a successful outcome. 

As you will see the reviewers are split on their judgement of the manuscript's merits. My reading was similar to several of the more positive spins on the potential of the work, but agreed with the criticisms that more clarity and translation was needed to broaden the impact of the piece. In particular, the revision should spend more time providing examples of mechanisms and being precise about how this work connects with pre-existing terminology and findings on sanctions (and again being precise about distinctions). Too much of the paper reads like a methodology exercise conducted on convenient data (which can be quite informative, but does not do the work of telling us why it is important). 

Moreover, some simple revisions that highlight why the specification provided brings inferences closer to theoretical expectation about the data generation process (again, from the pre-existing literature) could go a long way to broadening the impact of the manuscript. The reviewers will be looking for improved clarity in the theoretical concepts and mechanisms in the next draft.

\emph{Thank you for these comments; they served to greatly improve the manuscript. We have re-organized  and re-written sections of our paper to better emphasize how our study relates to the history of studies on reciprocity in International Relations (see the section  ``Sanctions as a Network Process $\&$ the Importance of Reciprocity''). In doing so, we streamline the paper to focus carefully on reciprocity while clarifying (and removing) extraneous language relating to learning and information sharing, which once served to muddle the paper. We have also added more information about our assumptions about reciprocity and, again, relate it to previous work. Additionally, we've added further interpretation of our visual depiction of reciprocity (Figure 2). We hope that you and the reviewers view these changes as substantial and adequate.}

\end{itemize}

\section*{Reviewer I}
\begin{itemize}

\item Conceptual: The potential contribution of this work hinges on the concepts of ``network'' and ``reciprocity'' yet you do not tell us exactly what you mean by these.  This creates problems throughout the paper.  The way I normally understand reciprocity ``I will do unto you as you have done unto me''--doesn't seem to fit.  There seems to be a strong element of ``I will do unto you as I have done unto you before'' as well as ``I expect you to do unto me as you have done unto others'' involved.  Similarly, it doesn't seem like there is really isn't anything that we normally consider as a ``network'' involved.  Rather, states are interpreting each other's actions in the context of other dyadic relationships ``have you done unto me more or less than others have done unto each other''.

\emph{The concept of reciprocity that we operationalize here comes with the interpretation that a certain type of behavior by a state $i$ towards another $j$ is likely to induce similar behavior from $j$ to $i$ in the next period. This is a common definition of reciprocity that has been used in many previous studies (e.g., Hoff and Ward 2004). In terms of how nothing characteristic of a network is being presented we were not quite sure what the reviewer meant. The crux of the approach that we utilize to study sanctions and compliances to sanctions is the social relations model developed by Kenny (1994). This is the same approach that lies at the core of many popular network based approaches such as latent space models. The idea here is that from the network we can decompose nodal specific and second order dependencies such as reciprocity, which is our focus here. R1 might be referring to the lack of discussion around third order dependencies (e.g., transitivity), which is a common argument presented in many network analysis application (e.g., Ward, Siverson and Cao 2007). We refrain from an exploration of third order dependencies in the case of sanction compliance because we have little theoretical reason to believe that the type of logic captures by empirical measures of transitivity would have a direct role in shaping our understanding of sanction compliance. For example, a transitivity type story would imply that if $i$ has complied to $j$ and $j$ has complied to $k$ then $i$ will be more likely to comply to $k$ in the next period. We have no argument to present that would call for that type of logic. }

\emph{Within the context of the manuscript, we now spend more time justifying why, and how, sanctions are a network process. From here, we then outline our assumptions (p.8) behind reciprocity as well as additional interpretation of a graphic depiction of reciprocity. We explain the ways in which reciprocity functions in other IR-related scenarios (p.7), why reciprocity does not encourage path-dependent spirals between countries (p.8). We hope that these sections now flow in a much clearer way and serve to address your comments.}

\item Research design: You haven't justified coding decisions or methods, you haven't connected these to your concepts or theoretical argument (because you don't have these) and you haven't presented enough detail about your processes to allow replication. Empirical test: You are using a hazard model for a situation in which cases can end in multiple ways.

\emph{We discuss, on multiple pages, why a duration modeling approach is both theoretically and empirically justified by our own argument as well as the large body of research employing this method elsewhere in the study of sanctions (see, for example, pages 3,4,9,and 13.)}

\emph{We consider the target of a sanction to have complied, if the target state completely or partially acquiesces to the demands of the sanction senders or negotiates a settlement. This is the same definition of sanction success, which we refer to as compliance, that is employed by Bapat and Morgan (2009) and Bapat, Heinrich, Kobayashi and Morgan (2013). Cases in which the sender gives up are simply coded as zeros until the sender has given up and dropped afterwards. Our focus here is only on determining the relationship of reciprocity and the time until a target state complies to the senders of a sanction.} 

\end{itemize}

\section*{Reviewer 2}
\begin{itemize}

\item The research presented in the ms. is technically sophisticated and competently executed. Therefore I would be willing to accept the presented results if only I could understand what the key results are. The findings of Models 1 and 2 are all fairly straightforward. Adding compliance and sanctions reciprocity doesn't affect existing ideas too much (apart from the slight ``puzzle'' on the lack of robustness of target democracy). So far, so good. But the key insight should come from Model 3 where compliance and sanctions reciprocity are indeed statistically significant and the coefficient seem intuitive in the sense that ``compliance'' leads to shorter and ``sanctions'' to longer sanctions episodes. My problem is, however, that I cannot figure out what is meant by compliance reciprocity and sanctions reciprocity. Hence I don't know whether they measure anything meaningful.\\

\emph{[Reviewer 2 provides an outline of his/her main points, which we address in reply to the more detailed follow up comments:]}

\begin{itemize}
\item Reading the earlier sections (and in particular page 9) didn't clarify matters for me. Three things confuse me the most:
  \begin{enumerate}
  \item How is compliance (also referred to as cooperation) measured? Particularly since the underlying data are sanctions episodes. Is compliance giving in to previous sanctions, or is it cooperating with the sender on imposing sanctions on a third party?
  \item Does sanctions reciprocity mean that you impose sanction in reply to sanctions imposed on you? Or is it simply being the target of sanctions? I think that it is the latter, but then it is not 'reciprocity' strictly speaking.
  \item Why is compliance/sanction reciprocity measured in relative terms.
  \end{enumerate}
\end{itemize}

\emph{We now spend more time justifying why, and how, sanctions are a network process. From here, we then outline our assumptions (p.8) behind reciprocity as well as included additional interpretation of a graphic depiction of reciprocity. We now also explain the ways in which reciprocity functions in other IR-related scenarios (p.7), why reciprocity does not encourage path-dependent spirals between countries (p.8), and how reciprocity logic ``works" in other IR examples.  After walking through this, we now introduce compliance and sanction reciprocity and discuss them prior to explaining their measurement (p.9, ``Compliance reciprocity represents a target states' cumulative history of compliance with a particular sender relative to all others in the network"). Within that section we have also added a brief discussion on why we measure reciprocity in relative terms. We hope that these sections now flow in a much clearer way and serve to address your comments.}

\item Not only the core message is unclear, the whole theory section is presented in a convoluted way making it very difficult for the reader to understand what is going on and what the key contribution of the paper is. And there is a lot going on! The argument emphasizes the importance of reciprocity and the relevance of networks for learning and information-sharing. It introduces the distinction between compliance and sanctions but also refers to incentives. It distinguishes between onset and duration (or ending of) sanctions. To cap it all of, Copula methods are used to impute missing data. Maybe most problematic is that the ms. never follows up fully on these by themselves interesting observations.

\emph{We took these comments very seriously, and ultimately (after reviewing literature on learning and information sharing) decided to stream-line our argument entirely around reciprocity. In doing so, we've moved our explanation of sanctions-as-a-network process to the forefront of our discussion. Then we've outlined the reciprocity concept as it relates to other major IR research and present clear assumptions around reciprocity (see p.8 for). We now do all of this prior to introducing the measurements of reciprocity. We also discuss the duration framework both as it relates to pre-existing literature (p.3, 4, 9) as well as explain our own use (p.12), in which we assess sanctions in terms of sanction-year. Finally, due to space constraints our copula methods are explained fully in the appendix. One of the authors is currently working on a separate manuscript in which Copula methods are discussed in more detail and compared to extant approaches for multiple imputation in political science. }

\item For example, the emphasis on network structures suggests that the analysis would deal with higher-order dependencies in the data structure, but the eventual analysis is dyadic. Another example would be that ``learning'' suggests that the sanctions network would be somehow endogenous, but that isn't the case here. This isn't the only ``network'' paper to which these comments would apply, but this one promises a lot but then fails to deliver.

\emph{These are fair concerns, and we have tried to be much more careful and precise about our language. We see benefit in leaving the DV as ``dyadic'' while incorporating network attributes on the right-hand side of the model. Though we do note that the measurement and incorporation of our sanction and compliance reciprocity measures explicitly accounts for at least second order dependencies. We do not make any attempt to capture third order dependencies as there is little theoretical reason to expect that they will have much of an effect in helping us to understand when a target country complies to a sanction from a set of sender. While we think that reciprocity matters over time we have now limited our claims around endogeneity which we cannot systematically capture.}

\item Finally (and a minor comment I know), the title: ``When do States Say Uncle'' is rather obscure�fortunately, Wikipedia could help me out here.

\emph{The original title was derived from a colloquial phrase in which ``saying uncle" admits defeat or acquiescence. The origins of this phrase are somewhat uncertain, though there is some evidence that it is largely a north american phrase. We found one article (via Mental Floss) that suggests the phrase dates back to the Roman Empire during which when young children were attacked by bullies, they wouldn't be set free until they said ``Patrue, mi Patruissmo" also meaning ``Uncle, my best Uncle." Essentially, this grants the bully a title of respect, giving him or her what they wanted. After investigating this a bit more, we are pretty sure this is a well known enough phrase to keep it in the title, but would happily change it at the Editor's request! }

\end{itemize} 


\section*{Reviewer 3}
\begin{itemize}

\item While reading the paper, I was struck several times with the thoughts ``What exactly do they mean by this?'' or ``That's an interesting idea. I'd like to know more about that.'' In some ways, this is a good sign I was engaged with the paper and wanted to know more. On the other hand, all of those thoughts arose during the first half of the paper leading me to the conclusions that there are several ways in which the theory is under-developed.  For example, at the top of p.5, the authors talk about the evolution (emphasis from the authors) of interaction between states. What exactly does this entail from a theoretical standpoint? The next few sentences tell us more about network analysis than they do about the way that changing relationships between states might influence the behavior of the parties in sanctions disputes.

 I'd also like to see a bit more discussion about compliance. What exact do the authors mean with this term? Doing everything the sender asks? Altering policies to any degree? This comes up at the top of page 6, and ends with a sentence about the ``endogenous evolution between states' shared strategic environment and past reciprocal behavior.'' Again, there's some theoretical ambiguity here for me. Perhaps I'm just smart enough to get it, but if a journal believes I'm a reasonable reviewer for this paper, then this is problematic for the authors because it means that other likely readers may not get it either.

\emph{We have now lessened claims about endogeneity; as you pointed out, we are unable to really get at this. Instead we emphasize that reciprocity forms over time and can be calculated temporally. In the re-organization of our argument, we now present the intuition behind sanction reciprocity and compliance reciprocity prior to any discussion of measurement. We hope that this clarifies previous confusion over what this concept captures. Further, we also descriptively demonstrate compliance reciprocity via our discussion of Figure 2 (p.11). Compliance reciprocity includes both partial and full acquiescence, i.e. ``We consider the target of a sanction to have complied, if the target state completely or partially acquiesces to the demands of the sanction senders or negotiates a settlement." Our definition and operationalization of compliance are from Bapat and Morgan (2009,  pg.12).}

\item The authors might want to consider tapping the literature on learning. Both in the foreign policy literature and the political economy literature, there are authors writing about how learning shapes behavior. The authors suggest a learning process but do not cite any of this previous work. Another area for greater theoretical refinement comes at the top of page 8 in the discussion of sanctions and information sharing. The authors cite Morgan \& Miers, Lacy \& Niou, and Marinov, but the informational power of the network is a distinct information process and could be described in fuller detail.

\emph{Thank you for this feedback. We did look into the (specially economic) literature on learning, but ultimately we decided to instead draw on previous work on reciprocity in IR. Reciprocity inherently involves the share of information and the formation of shared expectations between actors (pages 7-8). We cleaned the manuscript of extraneous language referencing learning and focus explicitly on reciprocity. }

% \item It's also unclear at times whether the history of cooperative behavior influences compliance reciprocity via learning or information sharing. Do we know? 
% not we dont

\item The theoretical development is over for the most part by the middle of page 8, but the authors go into two distinct concepts of reciprocity on page 9 compliance reciprocity and sanction reciprocity. I understand how the measures are built to assess these, but what should our theoretical expectations be? I wanted to know more about sanction reciprocity conceptually.

\emph{We have added in a longer discussion in terms of our specific theory and on our theoretical expectations for the effect of sanction and compliance reciprocity in p.9 and the beginning of p.10. }

\end{itemize}

\section*{Reviewer 4}
\begin{itemize}

\item The theoretical discussion behind the proposed hypotheses needs further elaboration. On pages 6-8, author derives the key hypotheses simply from a brief review (and modification) of extant studies of sanctions and network analysis. For example, the author argues that ``existing work on reciprocity and cooperation'' shows that one actor has incentive to ``respond in kind'' to the previous behavior of their partner. The author applies this idea to sanctions using learning mechanism that results from iterated interactions between sender and target. This is obviously not a new theory but intuitive. Nevertheless, I see a leap of logic in the path dependence story. Why would previous compliance and resistance determine the target's decision at a present episode given possible variations in the target and sender countries? 

% I think we should just leave this in without the quotes
\emph{Thank you for these comments. We thought that your understanding of the main intuition in the paper was correct, which enabled us to elaborate on your suggestions. We've clarified the assumptions of reciprocity by incorporating previous work in IR as well as listing out our assumptions (p.8): ``Specifically, to establish the existence of reciprocity we assume that (1) reciprocity is best conceptualized as a long-term mechanism that develops a common expectation of behavior between states; (2) departures from established expectations of behavior are possible (Moore 1995); and (3) reciprocity is a necessary but not sufficient condition for influencing strategic behavior between states (Goldstein 2001). To the first point, early considerations of reciprocity characterized it as a reactive, short-term process (e.g. tit for tat ``reactionary'' responses). Based on the evidence from (Rajmaira and Ward 1990) and others, we assume that reciprocity is a long-term process, born out by multiple interactions between actors over time. In this way, reciprocity becomes a norm of behavior, or a set of expectations about how actors engage with one another, over time. The temporal effects of reciprocity are empirically demonstrated in literature focused both on super power behavior (Rajmaira and Ward 1990) and behavior in regional conflicts (Goldstein 1997). Our second and third assumptions relate to one another in that we acknowledge other influences might effect state behavior and that at different periods states could have incentives to break away from established norms of behavior; this is important because it suggests that state behavior is not easily locked into unidirectional spirals (Moore 1995).''}

\item On pages 11-12, the author explains sanction reciprocity and the hypothesis for this variable: The intuition behind this measure is to capture the concept of creating expectations of resolve: states who have been sanctioned multiple times by a sender state are likely to build up a willingness of resistance and not cooperation. This suggests that states receiving sanctions from those with whom they have been sanctioned before are likely to more slowly comply with those states. It is not clear why multiple sanctions in the past would raise the resolve level of the target country for future episodes. Does the concept of reciprocity have any relevance to resolve? Or does reciprocity (e.g., cumulative history of compliance) simply represent the expected probability of winning in sanction episode? 

\emph{Thank you for these excellent comments. Good key question, one think we have clarified in the MS.  The sanction reciprocity measure captures more than just  ``having multiple sanctions in the past'' instead, it captures the dyadic history of sanctions between a pair of countries relative to the sanction history that those pairs have with all other countries. More specifically, this measure captures the likelihood of $i$ sending a sanction to $j$ after removing the tendency of $i$ to send a sanction to any other country and for $j$ to have received a sanction from any other country. Thus higher scores on this measure can be associated with two countries that are simply more likely to use sanctions against one another than they they are for any other pair in the sanctioning network. We argue that a country which has received a sanction from a country with which it has a high sanction reciprocity score is likely to comply more slowly than with a country that it does not have this type of relationship with. The basic insight is that states whom continually respond to sanctions by sending sanctions of their own are signaling more conflictual rather than cooperative behavior over time, and that this is likely characteristic of a pair of countries that are more resistant to cooperating with one another.  }

% we got no historical episodes to present here, so in case you know what to say i think we should jsut cut this from the memo
% \item Where is strategic decision-making, which the author emphasizes, on the part of the target when the target repeats its behavior of concessions? Who learns what from past histories of resistance and compliance? More explanation should be provided. In doing so, it would be also helpful if the author provides some historical evidence in order to show how learning mechanism works in sanction episodes.

\end{itemize}

\end{document}
