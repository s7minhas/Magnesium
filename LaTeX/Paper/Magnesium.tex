\documentclass[12pt,onesided]{amsart} 

%%%%%%%%%%%%%%%%%%%%%%%%%%%%%%%%%%%%%%%%%%%%%%%%%%
%%%%%%%%%%%%%%%%%%%% PREAMBLE %%%%%%%%%%%%%%%%%%%%
%%%%%%%%%%%%%%%%%%%%%%%%%%%%%%%%%%%%%%%%%%%%%%%%%%


% -------------------- defaults -------------------- %
% load lots o' packages

% layout control
\usepackage{geometry}
\geometry{verbose,tmargin=1.25in,bmargin=1.25in,lmargin=1.1in,rmargin=1.1in}
\usepackage{rotating}
\usepackage{fancyhdr}
\usepackage{parallel}
\usepackage{parcolumns}

% math typesetting
\usepackage{array}
\usepackage{amsmath}
\usepackage{amssymb}
\usepackage{amsfonts}

% tables
\usepackage{tabularx}
\usepackage{booktabs}
\usepackage{multicol}
\usepackage{multirow}
\usepackage{longtable}

\usepackage[%
decimalsymbol=.,
digitsep=fullstop
]{siunitx}

% to adapt caption style
\usepackage[font={small},labelfont=bf]{caption}
 
% references
\usepackage[longnamesfirst]{natbib}
\bibpunct{(}{)}{;}{a}{}{,}
\usepackage{nameref}

% footnotes at bottom
\usepackage[bottom]{footmisc}

% to change enumeration symbols begin{enumerate}[(a)]
\usepackage{enumerate}

% to make enumerations and itemizations within paragraphs or
% lines. f.i. begin{inparaenum} for (a) is (b) and (c)
\usepackage{paralist}

% to colorize links in document. See color specification below
\usepackage[x11names]{xcolor}

% for multiple references and insertion of the word "figure" or "table"
% \usepackage{cleveref}

% load the hyper-references package and set document info
\usepackage[pdftex]{hyperref}

% graphics stuff
\usepackage{subfig}
\usepackage{graphicx}
\usepackage[space]{grffile} % allows us to specify directories that have spaces
\usepackage[section]{placeins} % prevents floats from moving past a \FloatBarrier or section
\usepackage{tikz}
% \usepackage{pgfplots}

% Spacing
\usepackage[doublespacing]{setspace}

% define clickable links and their colors
\hypersetup{
	unicode=false,          % non-Latin characters in Acrobat's bookmarks
	pdftoolbar=true,        % show Acrobat's toolbar?
	pdfmenubar=true,        % show Acrobat's menu?
	pdffitwindow=false,     % window fit to page when opened
	pdfstartview={FitH},    % fits the width of the page to the window
	pdfnewwindow=true,%
	pdfauthor={Cassy Dorff and Shahryar Minhas},%
	pdftitle={Title},%
	colorlinks,%
	citecolor=black,%
	filecolor=black,%
	linkcolor=black,%
	urlcolor=RoyalBlue4%
	}

% Including External Code
\usepackage{verbatim}
\usepackage{listings}
\lstset{
	language=R,
	basicstyle=\scriptsize\ttfamily,
	commentstyle=\ttfamily\color{gray},
	numbers=left,
	numberstyle=\ttfamily\color{gray}\footnotesize,
	stepnumber=1,
	numbersep=5pt,
	backgroundcolor=\color{white},
	showspaces=false,
	showstringspaces=false,
	showtabs=false,
	frame=single,
	tabsize=2,
	captionpos=b,
	breaklines=true,
	breakatwhitespace=false,
	title=\lstname,
	escapeinside={},
	keywordstyle={},
	morekeywords={}
	}

% -------------------------------------------------- %


% -------------------- title -------------------- %

\title[Network Dependence and Sanction Compliance]{When Do States Say Uncle? Network Dependence and Sanction Compliance}
% \date{Draft Copy \today}

\author[Dorff]{Cassy Dorff}
\address{Cassy Dorff: Department of Political Science}
\curraddr{Duke University, Durham, NC, 27708, USA}
\email{cassy.dorff@duke.edu}

\author[Minhas]{Shahryar Minhas}
\address{Shahryar Minhas: Department of Political Science}
\curraddr{Duke University, Durham, NC, 27708, USA}
\email{shahryar.minhas@duke.edu}

\thanks{This paper was prepared for 72$^{nd}$ annual Midwest Political Science Association Conference in Chicago, April 2-6 2014. We are grateful for comments on earlier versions of this paper received at the ISSS-ISAC Conference in Washington, D.C., October 4-6 2013. }

\setlength{\headheight}{15pt}
\setlength{\headsep}{20pt}
\pagestyle{fancyplain}
 
\fancyhf{}
 
\lhead{\fancyplain{}{}}
\chead{\fancyplain{}{}}
\rhead{\fancyplain{}{}}
\rfoot{\fancyplain{}{}}

% ----------------------------------------------- %


% -------------------- customizations -------------------- %

% define the includegraphics search path
% \graphicspath{{Graphics/}}

% easy commands for number propers
\newcommand{\first}{$1^{\text{st}}$}
\newcommand{\second}{$2^{\text{nd}}$}
\newcommand{\third}{$3^{\text{rd}}$}
\newcommand{\nth}[1]{${#1}^{\text{th}}$}

% easy command for boldface math symbols
\newcommand{\mbs}[1]{\boldsymbol{#1}}

% define bibliography style
%\bibliographystyle{/Users/janus829/Documents/APSR}

\graphicspath{{/Users/cassydorff/Dropbox/Research/Magnesium/Graphics/}}
%\graphicspath{{/Users/janus829/Dropbox/Research/Magnesium/Graphics/}}

% -------------------------------------------------------- %


%%%%%%%%%%%%%%%%%%%%%%%%%%%%%%%%%%%%%%%%%%%%%%%%%%
%%%%%%%%%%%%%%%%%%%% DOCUMENT %%%%%%%%%%%%%%%%%%%%
%%%%%%%%%%%%%%%%%%%%%%%%%%%%%%%%%%%%%%%%%%%%%%%%%%

\doublespacing 

\begin{document}

\maketitle\thispagestyle{empty}

\begin{abstract}

\singlespacing{This article explores when and why states comply with sanctions. Previous literature has suggested a duration modeling approach is needed to adequately capture the time it takes for a sanction to "work." This approach, however, has failed to carefully account for important dynamics relevant to the modeling of sanction outcomes. Namely, present duration approaches fail to incorporate the network effects intrinsic to international sanction processes. At any given time, target states typically face both a network of sanctioners for an individual sanction case, as well as a general network of sanctioners including senders from multiple cases. We present a model that incorporates this interdependent nature of the international system by including network effects within the duration model. In addition, we are able to test whether traditional conditions that the literature claims as critical for predicting sanction compliance, such as domestic institutions, are still influential once network dynamics are adequately modeled. In doing so we are able to test two key hypothesis: (1) does dependence between the target state and its sanctioning network increase the probability of target compliance; and (2) do domestic institutions condition network effects?}

\end{abstract}

\newpage\setcounter{page}{1} 

%%%%%%%% INTRO %%%%%%%%
\section*{Introduction}
\label{intro}

Economic sanctions are a frequently used foreign policy tool in the realm of international relations. Typically, one or more states initiate sanctions against another state when they perceive the target state as non-cooperative. The trigger for economic sanctions can occur in many contexts: the target state breaks a previous agreement, the target state openly disobeys international law, or the target state engages in behavior that is simply unfavorable to the political preferences of another state. Take for example, in November of 2012 when the Obama administration imposed sanctions on the Iranian government for blocking Internet access, mobile-phone lines and satellite television channels from the public. 

Policymakers continually engage in heated debates over the use of sanctions as a means to avoid war while still taking a stand. The motivations for sanction initiation are cross-cutting, spanning a diverse and interdependent mix of policy issues and political actors. While the concept of sanctions -- the idea that countries can put pressure on their economic ties to other countries in order to influence policy -- is relatively straightforward, the study of when and why sanctions work is complex. While earlier research on sanctions argued that sanctions have little influence on targets \citep{lam1990, dashti1997, morgan1997, drezner1998} more recent research suggests that the effectiveness of sanctions is dependent on an interaction of several factors, namely: the number of senders acting as a part of the sanctioner group and the type of issue in dispute \citep{miers2002, morgan2009threat}; the strength of domestic institutions within the target state; and the type of regime governing the target state \citep{mcgillivray2004}. Notably, the literature has not emphasized the role of past actions and the way in with these interactions influence future behavior.

We agree with the theoretical and empirical literatures that suggest several different, interacting conditions are at work when predicting the outcome of sanctions. We argue, however, that political scientists have thus far failed to incorporate a key factor into their analysis: reciprocity within the sanction network. Drawing on the work in international relations on trade and conflict, we suggest that sanction cases are best conceptualized as a network phenomenon and must be modeled as such. Reciprocity is a not a new concept to the field of international studies, but has its roots in previous theories of cooperation \citep{keohane1989reciprocity}. Yet the study of sanctions has not yet addressed how the intutition of reciprocity's effect on state behavior might condition the effects of 
other variables on sanction compliance. 

We analyze this key endogenous structure inherit to network dynamics, reciprocity, and argue that the structure created by reciprocial interactions over time must be accounted for in studies of sanction outcomes. Further, we extend on previous work suggesting duration models as the most appropriate approach for modeling sanctions outcomes by incorporating network measures into the duration framework. In doing so we are able to return to key hypotheses from the literature and assess whether factors such as domestic political institutions and internal stability influence sanction outcome once network dynamics are adequately incorporated into the model. 

We leverage the network modeling approach to produce an accurate test of when and why sanctions end. In the following section, we review previous work on compliance and introduce the network concept. We then present our central argument and hypotheses; in doing so we articulate the various ways that networks can be conceptualized in this context. Last, we present our findings and review the results.

%maybe a summary paragraph and a "we proceed as follows: or something"


%%%%%%%%%%%%%%%%%%%%%%%

%%%%% Lit Review %%%%%
\subsection*{When do Sanctions End?}
\label{lit}

Previous work on the duration of sanctions, or when and why a target state will decide to comply with a particular sanction, has more recently focused on the role of domestic factors. \cite{marinov2005} argues that sanctions ``work'' by destabilizing the leaders of the governments that sanctions punish. This focus on internal state conditions echoes other work which suggests that sanction outcomes are dependent on domestic stability and domestic institutions. For example, if a regime is already experiencing a high level of internal conflict, such as protest or violent clashes, the onset of an economic sanction against trade would weaken the regime even more. This heightens the cost of resistance against the sanction \citep{dashti1997}. 

Similarly, \citep{dorussen2001} suggest that domestic support determines the duration (or ``ending'') of sanctions whereby when the target state's domestic constituency supports resistance against the sanction, the leader has greater incentive to not comply with the sanction, which effectively increases the sanction's duration. Further supporting the idea that domestic institutions condition whether and when states comply with sanctions, \cite{lektzian2007} argues that because of differing institutional incentives, economic sanctions are more likely to succeed against nondemocratic regimes than democratic ones. While all of these studies present empirical evidence for the general claim that domestic factors condition sanction outcome, none of them account for third party effects, or network level dependencies. 

Research on compliance has historically utilized a logit or probit-estimation approach. However, some have demonstrated that a duration modeling approach more accurately captures the important time-variant dynamics relevant to understanding the sanction process. \cite{bolks2000} point out that a duration-modeling approach is able to include variables that fluctuate throughout the tenure of an individual sanction case. Clearly, if the goal of research is to understand and predict when a target state is likely to comply to a sanction, then researchers have clear incentives to include time-variant data. Using a duration modeling approach allows for the assessment of whether over time a specific factor, such as political instability or regime type, increases or decreases the probability that a target country will comply with a sanction.

\cite{mcgillivray2004} employ a hazard model to analyze a data set of 47 sanctions cases. They find that leadership change does strongly influence the duration of sanctions, but only in the case of non-democratic states. Similarly, \cite{bolks2000} consider the determinants of economic sanction duration using a duration model approach. These authors also look inside the target state to define domestic conditions that influence sanction outcome. They suggest that the ``decision-making'' environment can either hinder or help the leader take countermeasures against the sanction. This ``decision-making'' environment is affected by factors such as a lack of coordination between government actors and local instability. 

Clearly, domestic conditions seem to matter for predicting sanction compliance. While researchers have successfully applied duration approaches, the literature can be improved on in two main ways. First, it remains unclear whether external factors also influence the duration until compliance. It is important to consider whether network of sanctioners for each sanction case are critical trade partners, allies, or neighbors with the target state. Each relationship between the sanctioner and the sanctioned takes on a slightly different form dependent on these factors. If a neighboring state is greatly dissatisfied with the target's behavior, this conflict of interest could have more serious repercussions that a sanctioner who is geographically removed from the target. These types of external factors are housed within the network of sanctioners for each and every sanction case. Such factors have been incorporated into previous analysis as largely dyadic or monadic variables, but this approach fails to capture and account for the complex interdependence structure that international politics inherently exhibits. Take, again, for example international trade dynamics. 

While it is intuitive to many researchers that trade dependence between target and sender states likely influences the duration of economic sanctions, in order to adequately measure trade interactions, one has to analyze the trade \textit{network} relevant to each sanction case, which accounts for the fact that trade between dyads is not an independent process. By avoiding these network attributes, researchers miss a wealth of structural information that is critical to understanding the ebb and flow of international cooperation and conflict. The insight that the international system is inherently a network and must be studied as such, is by no means original to this project, but has gained increasing support in the literature; most prominent is the work on trade networks \citep{hoff2004modeling} , conflict \citep{dorff2013}, alliances \citep{warren2010geometry} and intragovernmental organizations \citep{cao2009networks,greenhill2010norm}. 

Second, current duration approaches are unable to account for the history of dependencies between countries over time, and thus ignore previous cases of compliance and sanction interdependence between target and sanctioning states.  Over time, complex interdependencies likely emerge and drive behavior between states, where if country \textit{i} complies often to country \textit{j}, country \textit{j} might also be more likely to comply to country \textit{i}. This process is typically known as reciprocity, and is one of the network attributes we account for in our analysis below. Importantly, \citet{cranmer2014reciprocity} also argue that the sanction literature has not yet accounted for network dynamics. In their work they model the sanction network itself, and demonstrate that onset of sanction cases are best predicted by modeling the way in which the network complex interdependencies, such as reciprocity, evolve over time and influence the future decisions made by states. Critical concepts like these are currently ignored in the research on sanction compliance. This paper aims to fill this gap. 


%%%%%%%%%%%%%%%
% NOTES ON THINGS TO ADD
%%%%%%%%%%%%%%%

%do we only look at compliance (not just a sender ``quitting''?)
%Things we don't account for : threats and sanction "issues" driving compliance (see And and Peksen 2007)
%We also need to address why we want to study compliance/duration (mcgillivray and stam, as well as Bolks, have more on this)
%%%%%%%%%%%%%%%%%%%%%%

%%%%% Theory %%%%%
% NOTE FROM SM: Lacking a theory here is really hurting us. We should consider adding a case on S. Africa or some other country so that we can actually sketch out a theory of network pressure and sanction compliance. 

\section*{Accounting for Network Effects}
\label{neteffects}

The focus of the extant literature on domestic explanations of sanction compliance neglects the process through which sanctions even occur. As \citet{cranmer2014reciprocity} note, the process through which sanctions are imposed is most accurately conceptualized as a strategic and multilateral phenomenon of interdependent relations. Figure \ref{fig:spaghetti} depicts the network of sanction cases and threats thereof ongoing and initiated by 1984. Nodes represent countries and the directed edges denote the sender and receiver of particular sanction cases. Clear from figure \ref{fig:spaghetti} is that the process through which sanctions have proliferated in the international system forms a complex network. 

\begin{figure}[ht]
  \centering
  \begin{tabular}{c}
	  % \includegraphics[width=1\textwidth]{84net}
	  \includegraphics[width=1\textwidth]{84net-crop} \\
	  \includegraphics[width=0.45\textwidth]{MapLegend}
  \end{tabular}
  \caption{Here we show the sanction network in 1984, nodes are colored by geographic coordinates of countries.}
  \label{fig:spaghetti}
\end{figure}
\FloatBarrier

Many of the duration approaches in the extant literature focus solely on domestic factors. In doing so, they fail to take into account the types of network pressures that are brought to bear through sanctions. One such network pressure involves the interrelationships and interactions between senders and receiver in any particular sanction case. In figure \ref{fig:saneti}, we show the six sanction cases faced by South Africa in 1984.\footnote{Data for sanction cases is from \citet{morgan2009threat}.} For the most part each sanction case involves a variety of actors with whom South Africa has differing cultural, geographic, diplomatic, and economic relationships. Within any individual sanction case we hypothesize that the proximity of relationships between sender(s) of a sanction and a receiver matter for determining the time and if the sanctioned state complies. 

% Sanction Case Network: The relationship between sender(s) and the target matters for sanction compliance. Sanctions involving coalitions of sender(s) will be more quickly resolved than sanctions sent by just one state. Sanction cases where relationships are more proximate will be more quickly resolved.

\begin{quote}
	\textbf{H1}: Sanction cases where relationships between sender(s) and receiver(s) are more proximate will be more quickly resolved.
\end{quote}

\begin{figure}[ht]
	\centering
	\includegraphics[width=1\textwidth]{saneti}
	\caption{Here we show a separate network for each sanction case that South Africa faced in 1984.}
	\label{fig:saneti}
\end{figure}
\FloatBarrier

To test the effect of relationships between sender(s) and receiver(s) in predicting sanction compliance we construct a number of covariates. First for each sanction case we determine the number of senders. We also calculate the average number of other sanctions being sent by the senders of each particular sanction case.  

	\begin{itemize}
		\item Number of senders associated with a sanction case
		\item Mean Number of other sanctions being sent by senders
		\item Distance: The average distance between sender(s) and the receiver
		\item Trade: The share of total trade that the sender(s) make up for the receiver		
		\item Alliances: The proportion of sender(s) that are allied with the receiver
		\item IGOs: The average number of common IGOs that the sender(s) and receiver belong to
		\item Religion: Similarity of religious group makeups between sender(s) and the receiver
	\end{itemize}

Additionally, the six separate sanctions that South Africa faced in 1984 can also be thought of as a yearly sanction case network. In figure \ref{fig:sanet}, we aggregate the six sanction networks into one where each separate sanction is denoted by a differing color. Here we hypothesize that states under the pressure of a multitude of sanctions will more quickly resolve sanction cases than those facing only a few.

% Aggregate Network: Targets of sanctions often face a multitude of sanction cases at any given point in time. States under the pressure of a multitude of sanctions will more quickly resolve sanction cases than those facing only a few.

\begin{quote}
	\textbf{H2}: States facing the pressure of a multitude of sanctions will more quickly resolve any one of those sanction cases.
\end{quote}

\begin{figure}[ht]
	\centering
	\includegraphics[width=0.75\textwidth]{sanet}
	\caption{South Africa 1984 Sanction Case Network}
	\label{fig:sanet}
\end{figure}
\FloatBarrier

	\begin{itemize}
		\item Sanctions Received: Total number of sanctions to which the target state is currently exposed
	\end{itemize}

% We conceptualize network dynamics in terms of ``pressures.'' First how the interrelationships, e.g., cultural or geographic proximity, and interactions between states, e.g., trade, affect the time until compliance. Second, that target states, receivers, often face multiple sanctions from multiple senders at any given time. Last, we incorporate a few of the prominent target-focused explanations for sanction compliance. 
% Third, we consider that reciprocal compliance occurs over time between states within the network. We incorporate all three of these relational effects into our duration model.  NOTE FROM SM: should probably leave this out until we actually do it



% Though we hypothesize that network effects matter greatly for determining sanction compliance there are characteristics of sanctioned states that may better enable them to manage these pressures. 

% Target states with stronger democratic institutions that are under the pressure of sanctions will more quickly comply than those with less democratic institutions. Sanctions are designed to impose costs on key groups within countries. Affected groups will lobby the government to reach an accommodation with sanctioning states. The ability to successfully lobby is dependent upon political institutions (Manin, Przeworski and Stokes 1999; Barro 1973; Ferejohn 1986)
%%%%%%%%%%%%%%%%%%%%%%

%%%%% Empirics %%%%%
\section*{Data and Analysis}
\label{empirics}

To test the effects of network pressures on sanction compliance we use the Threat and Imposition of Sanctions (TIES) Database developed by \citet{morgan2009threat}. This database includes over 1,400 sanction case threats and initiations from 1945 to 2013.\footnote{Only sanction cases threatened and initiated up until 2005 are included but outcomes for cases are recorded up until 2013.} Our focus here is restricted to threats and sanctions that are prompted as the result of economic issues such as expropriation, trade practices, and implementation of economic reforms. 

% Need to provide reason for why we are restricting to economic sanctions
% Economic sanctions are a complex and interdependent phenomenon in which a state or numerous states cut or threaten to cut regular economic relations with another state in order to obtain a policy change.

Restricting our analysis to threats or sanctions stemming from these issues during the period of 1960 to 2005 leaves us with over 800 cases. Our unit of analysis is the case-year, providing us with a total of 5,303 observations. For each case in the TIES database a final outcome is recorded to describe how and if the case has been resolved. The purpose of our analysis is to assess the time until a state complies to a threat or sanction. We consider a case to have been resolved by compliance if the target state completely or partially acquiesces to the demands of the sanction senders or negotiates a settlement.
	
In using this definition of compliance, approximately 37\% of cases in our dataset end with a state complying by 2013 while 40\% remain ongoing. The remaining 23\% of cases were terminated for other reasons show below in table \ref{tab:termCases}.

\begin{table}[ht]
	\centering
	\begin{tabular}{lc}
		\hline\hline
		Outcome & Frequency \\
		\hline
		Capitulation by Sender in Threat Stage & 77 \\
		Stalemate in the Threat Stage & 11 \\		
		Capitulation by Sender After Imposition & 58 \\
		Stalemate after Sanctions Imposition & 38 \\
		\hline\hline
	\end{tabular}
	\caption{Outcomes of threat and sanction cases no longer ongoing where compliance was not achieved.}
	\label{tab:termCases}	
\end{table}
\FloatBarrier

% Add in something about imputaiton with sbgcop

\subsection*{Modeling Approach} 

Next we discuss our modeling approach. To estimate the effect of network pressures on the ability of a threatened or sanctioned states to resist compliance, we use Cox proportional hazard (PH) models of the length of threat or sanction periods. Specifically, the dependent variable, sanction spell, is the number of years that a state has not complied to a threat or sanction at time $t$. We model the expected length of sanction spells as a function of a baseline hazard rate and a set of covariates that shift the baseline hazard. The Cox PH specification that we employ is:

\begin{center}
	$\log h_{i}(t | \boldsymbol{X}_{i}) \; = \; h_{0}(t) \times \exp(\boldsymbol{X}_{i} \beta)$,
\end{center}

where the log-hazard rate of compliance in a sanction case, $i$, conditional on having not complied for $t$ years is a function of a common baseline hazard $h_{0}(t)$ and covariates $\boldsymbol{X}$. In employing this approach, we assume no specific functional form for the baseline hazard and instead estimate it non-parametrically from the data. The covariates $\boldsymbol{X}$ operate multiplicatevely on the hazard rate, shifting the expected risk of compliance up or down depending on the value of $\beta$ \citep{crespo2013political}.\footnote{To ensure against bias in our parameter estimates we included a vector of case-level shared frailties to account for variations in unit-specific factors. We found similar results with and without the shared frailities, so we report results without the inclusion of this additional term.} 

Providing no specific functional form for the baseline hazard necessitates testing the proportional hazard assumption. \citet{keele2010proportionally} notes that not inspecting this assumption in the covariates can lead to severely biased parameter estimates. To address this issue, we first fit smoothing splines for all continuous covariates. After ascertaining that none of the continuous covariates in our model required modeling with splines, we carried out tests of non-proportionality. For those covariates where the non-proportional effects assumption does not hold, we include interactions between the covariate and spell duration (log scale). The only covariate showing evidence of non-proportionality is the average similarity of religious profiles.

We also imputed missing values to avoid excluding instances of compliance. If we employed list-wise deletion, we would lose over 438 country-year observations, 32 of which contained instances in which a state complied to a sanction. Previous research (e.g., see \citealp{rubin1976inference,honaker2010missing}) has already highlighted how simply deleting missing observations can lead to biased results. To impute missing values, we use a copula based approach developed by \citet{hoff:2007}. Details on our imputation process and results based on the original unimputed dataset, which are nearly identical, can be found in the \nameref{appendix}. 

Below we show our full model specification: 

\begin{align*}
		Compliance_{i,t} =& \\
		&Sanction \; Reciprocity_{j,t-1} + Compliance \; Reciprocity_{j,t-1} + \\
		&No. \; Senders_{j,t} + Distance_{j,t} + Trade_{j,t} + Ally_{j,t} + \\
		&Constraints_{i,t-1} + Ln(GDP \; Capita)_{i,t-1} +\\
		&GDP \; Growth_{i,t-1} + Internal \; Conflict_{i,t} + \epsilon_{i,t}
\end{align*}

where $i$ represents the target of the sanction, $j$ represents the relationship between the set of sender(s) for a particular sanction case and $i$, and $t$ the time period.
%%%%%%%%%%%%%%%%%%%%%%

%%%%% Empirics %%%%%
\section*{Results}
\label{Results} 


Table~\ref{tab:regResults} displays the results from our model. To contrast with findings from the extant literature, we run the model in three ways. The first column tests explanations of sanction compliance centered on target state characteristics. In the second column, we add covariates to account for the relationship that a target state has with the senders of the sanction, and the last model incorporates our reciprocity measures. 

As we go from left to right across the Table, we see that explanations of sanction compliance centered on the target state become less prominent. We find no support for the argument that high levels of internal stability may prompt a country to comply. In Models 1 and 2, we find that countries with higher levels of GDP per capita take longer to comply to a sanction case, but after accounting for the network level characteristics the coefficient estimate for that variable more than halves. Our findings for the effect of a country's regime type on sanction compliance also diverge from what the extant literature would predict. Instead of finding that democracies are more likely to comply to a sanction case, we see that countries rated as more democratic take longer to comply than countries that are more autocratic. However, this effect is marginal at best, and once we control for the relationship between sender and target states this effect also becomes much weaker. 

\newpage
% latex table generated in R 3.1.2 by xtable 1.7-4 package
% Sun May 24 13:11:58 2015
\begin{table}[ht]
\centering
{\normalsize
\begin{tabular}{lccc}
 Variable & Model 1 & Model 2 & Model 3 \\ 
  \hline
\hline
Compliance Reciprocity$_{j,t-1}$ &  &  & $0.348^{\ast\ast}$ \\ 
   &  &  & (0.115) \\ 
  Sanction Reciprocity$_{j,t-1}$ &  &  & $-0.15^{\ast\ast}$ \\ 
   &  &  & (0.057) \\ 
   \hline
Number of Senders$_{j,t}$ &  & $0.369^{\ast\ast}$ & $0.356^{\ast\ast}$ \\ 
   &  & (0.086) & (0.086) \\ 
  Distance$_{j,t}$ &  & $1.416^{\ast\ast}$ & $1.426^{\ast\ast}$ \\ 
   &  & (0.28) & (0.284) \\ 
  Trade$_{j,t}$ &  & -1.153 & 3.618 \\ 
   &  & (25.874) & (25.931) \\ 
  Ally$_{j,t}$ &  & 0.05 & 0.09 \\ 
   &  & (0.271) & (0.277) \\ 
   \hline
Polity$_{i,t-1}$ & 0.032 & $0.069^{\ast\ast}$ & $0.067^{\ast\ast}$ \\ 
   & (0.026) & (0.028) & (0.028) \\ 
  Ln(GDP per capita)$_{i,t-1}$ & $-0.331^{\ast\ast}$ & $-0.439^{\ast\ast}$ & $-0.388^{\ast\ast}$ \\ 
   & (0.1) & (0.114) & (0.121) \\ 
  GDP Growth$_{i,t-1}$ & 0.037 & 0.051 & 0.037 \\ 
   & (0.033) & (0.035) & (0.035) \\ 
  Population$_{i,t-1}$ & $-0.197^{\ast\ast}$ & $-0.193^{\ast\ast}$ & -0.142 \\ 
   & (0.083) & (0.096) & (0.1) \\ 
  Internal Conflict$_{i,t-1}$ & 0.001 & 0.003 & 0.002 \\ 
   & (0.025) & (0.025) & (0.025) \\ 
   \hline
n & 3777 & 3764 & 3764 \\ 
  Events & 72 & 72 & 72 \\ 
  Likelihood ratio test & 20.18 (0) & 55.2 (0) & 64.09 (0) \\ 
   \hline
\hline
\end{tabular}
}
\caption{Duration model with time varying covariates estimated using Cox Proportional Hazards. Standard errors in parentheses. $^{**}$ and $^{*}$ indicate significance at $p< 0.05 $ and $p< 0.10 $, respectively.} 
\label{tab:regResults}
\end{table}

\newpage

We find strong support for our third hypothesis which states that target states are more likely to comply to sanctions involving multiple actors, and the effect of this variable remains consistent even after controlling for our network level covariates in Model 3. Because it is difficult to interpret the substantive meaning of point estimates from the hazard function in Table \ref{tab:regResults}, we utilize Kaplan-Meier estimates of survival probabilities in Figure \ref{fig:surv1}. The y-axis represents the probability of survival, or in this case the probability that a country will not comply to a sanction, and the x-axis represents time since sanction initiation (measured in years). The grey line represents the scenario where the number of senders variable is set to a low value, in this case one, and the black line represents the scenario where this variable is set to its high value, in this case five. All the other covariates were set to their median. 

Using Figure \ref{fig:surv1} we can quickly see that there is a stark difference in the likelihood of non-compliance between a sanction case involving single and multiple senders. After just five years the probability of non-compliance drops to approximately 50\%, whereas a sanction from a single sender by that time still has almost an 80\% chance of non-compliance. This finding is in sharp contrast to previous research arguing that multilateral sanctions can actually be counterproductive.\footnote{e.g., see \citealp{drezner2000bargaining}}

\newpage
\begin{figure}[ht]
	\centering
	\caption{Survival probabilities reflecting variation in the number of senders in a sanction case. Grey designates the scenario where the number of senders is set to its minimum value and black the scenario where it is set to its high value (90th percentile).}
	% \includegraphics[width=1\textwidth]{nosSurv}
	\resizebox{0.7\textwidth}{!}{% Created by tikzDevice version 0.7.0 on 2015-05-17 09:25:41
% !TEX encoding = UTF-8 Unicode
\begin{tikzpicture}[x=1pt,y=1pt]
\definecolor[named]{fillColor}{rgb}{1.00,1.00,1.00}
\path[use as bounding box,fill=fillColor,fill opacity=0.00] (0,0) rectangle (433.62,289.08);
\begin{scope}
\path[clip] (  0.00,  0.00) rectangle (433.62,289.08);
\definecolor[named]{drawColor}{rgb}{0.00,0.00,0.00}

\path[draw=drawColor,line width= 0.4pt,line join=round,line cap=round] ( 49.20, 61.20) -- (408.42, 61.20);

\path[draw=drawColor,line width= 0.4pt,line join=round,line cap=round] ( 49.20, 61.20) -- ( 49.20, 55.20);

\path[draw=drawColor,line width= 0.4pt,line join=round,line cap=round] (109.07, 61.20) -- (109.07, 55.20);

\path[draw=drawColor,line width= 0.4pt,line join=round,line cap=round] (168.94, 61.20) -- (168.94, 55.20);

\path[draw=drawColor,line width= 0.4pt,line join=round,line cap=round] (228.81, 61.20) -- (228.81, 55.20);

\path[draw=drawColor,line width= 0.4pt,line join=round,line cap=round] (288.68, 61.20) -- (288.68, 55.20);

\path[draw=drawColor,line width= 0.4pt,line join=round,line cap=round] (348.55, 61.20) -- (348.55, 55.20);

\path[draw=drawColor,line width= 0.4pt,line join=round,line cap=round] (408.42, 61.20) -- (408.42, 55.20);

\node[text=drawColor,anchor=base,inner sep=0pt, outer sep=0pt, scale=  1.00] at ( 49.20, 39.60) {0};

\node[text=drawColor,anchor=base,inner sep=0pt, outer sep=0pt, scale=  1.00] at (109.07, 39.60) {5};

\node[text=drawColor,anchor=base,inner sep=0pt, outer sep=0pt, scale=  1.00] at (168.94, 39.60) {10};

\node[text=drawColor,anchor=base,inner sep=0pt, outer sep=0pt, scale=  1.00] at (228.81, 39.60) {15};

\node[text=drawColor,anchor=base,inner sep=0pt, outer sep=0pt, scale=  1.00] at (288.68, 39.60) {20};

\node[text=drawColor,anchor=base,inner sep=0pt, outer sep=0pt, scale=  1.00] at (348.55, 39.60) {25};

\node[text=drawColor,anchor=base,inner sep=0pt, outer sep=0pt, scale=  1.00] at (408.42, 39.60) {30};

\path[draw=drawColor,line width= 0.4pt,line join=round,line cap=round] ( 49.20, 67.82) -- ( 49.20,233.26);

\path[draw=drawColor,line width= 0.4pt,line join=round,line cap=round] ( 49.20, 67.82) -- ( 43.20, 67.82);

\path[draw=drawColor,line width= 0.4pt,line join=round,line cap=round] ( 49.20,100.91) -- ( 43.20,100.91);

\path[draw=drawColor,line width= 0.4pt,line join=round,line cap=round] ( 49.20,134.00) -- ( 43.20,134.00);

\path[draw=drawColor,line width= 0.4pt,line join=round,line cap=round] ( 49.20,167.08) -- ( 43.20,167.08);

\path[draw=drawColor,line width= 0.4pt,line join=round,line cap=round] ( 49.20,200.17) -- ( 43.20,200.17);

\path[draw=drawColor,line width= 0.4pt,line join=round,line cap=round] ( 49.20,233.26) -- ( 43.20,233.26);

\node[text=drawColor,anchor=base east,inner sep=0pt, outer sep=0pt, scale=  1.00] at ( 37.20, 64.37) {0.0};

\node[text=drawColor,anchor=base east,inner sep=0pt, outer sep=0pt, scale=  1.00] at ( 37.20, 97.46) {0.2};

\node[text=drawColor,anchor=base east,inner sep=0pt, outer sep=0pt, scale=  1.00] at ( 37.20,130.55) {0.4};

\node[text=drawColor,anchor=base east,inner sep=0pt, outer sep=0pt, scale=  1.00] at ( 37.20,163.64) {0.6};

\node[text=drawColor,anchor=base east,inner sep=0pt, outer sep=0pt, scale=  1.00] at ( 37.20,196.73) {0.8};

\node[text=drawColor,anchor=base east,inner sep=0pt, outer sep=0pt, scale=  1.00] at ( 37.20,229.82) {1.0};
\end{scope}
\begin{scope}
\path[clip] (  0.00,  0.00) rectangle (433.62,289.08);
\definecolor[named]{drawColor}{rgb}{0.00,0.00,0.00}

\node[text=drawColor,anchor=base,inner sep=0pt, outer sep=0pt, scale=  1.00] at (228.81, 15.60) {Time (Years)};

\node[text=drawColor,rotate= 90.00,anchor=base,inner sep=0pt, outer sep=0pt, scale=  1.00] at ( 10.80,150.54) {Survival Probability};
\end{scope}
\begin{scope}
\path[clip] ( 49.20, 61.20) rectangle (408.42,239.88);
\definecolor[named]{drawColor}{rgb}{0.66,0.66,0.66}

\path[draw=drawColor,line width= 0.4pt,line join=round,line cap=round] ( 49.20,233.26) --
	( 61.17,233.26) --
	( 61.17,229.38) --
	( 73.15,229.38) --
	( 73.15,224.99) --
	( 85.12,224.99) --
	( 85.12,220.53) --
	( 97.10,220.53) --
	( 97.10,217.96) --
	(109.07,217.96) --
	(109.07,216.94) --
	(121.04,216.94) --
	(121.04,216.58) --
	(133.02,216.58) --
	(133.02,216.21) --
	(144.99,216.21) --
	(144.99,215.79) --
	(156.97,215.79) --
	(156.97,215.30) --
	(168.94,215.30) --
	(168.94,214.79) --
	(204.86,214.79) --
	(204.86,213.97) --
	(216.84,213.97) --
	(216.84,212.78) --
	(264.73,212.78) --
	(264.73,209.90) --
	(288.68,209.90) --
	(288.68,206.28) --
	(300.65,206.28) --
	(300.65,202.66) --
	(336.58,202.66) --
	(336.58,199.14) --
	(433.62,199.14);

\path[draw=drawColor,line width= 0.4pt,line join=round,line cap=round] ( 57.99,229.38) -- ( 64.36,229.38);

\path[draw=drawColor,line width= 0.4pt,line join=round,line cap=round] ( 61.17,226.20) -- ( 61.17,232.56);

\path[draw=drawColor,line width= 0.4pt,line join=round,line cap=round] ( 69.97,224.99) -- ( 76.33,224.99);

\path[draw=drawColor,line width= 0.4pt,line join=round,line cap=round] ( 73.15,221.81) -- ( 73.15,228.17);

\path[draw=drawColor,line width= 0.4pt,line join=round,line cap=round] ( 81.94,220.53) -- ( 88.30,220.53);

\path[draw=drawColor,line width= 0.4pt,line join=round,line cap=round] ( 85.12,217.35) -- ( 85.12,223.72);

\path[draw=drawColor,line width= 0.4pt,line join=round,line cap=round] ( 93.91,217.96) -- (100.28,217.96);

\path[draw=drawColor,line width= 0.4pt,line join=round,line cap=round] ( 97.10,214.77) -- ( 97.10,221.14);

\path[draw=drawColor,line width= 0.4pt,line join=round,line cap=round] (105.89,216.94) -- (112.25,216.94);

\path[draw=drawColor,line width= 0.4pt,line join=round,line cap=round] (109.07,213.76) -- (109.07,220.12);

\path[draw=drawColor,line width= 0.4pt,line join=round,line cap=round] (117.86,216.58) -- (124.23,216.58);

\path[draw=drawColor,line width= 0.4pt,line join=round,line cap=round] (121.04,213.39) -- (121.04,219.76);

\path[draw=drawColor,line width= 0.4pt,line join=round,line cap=round] (129.84,216.21) -- (136.20,216.21);

\path[draw=drawColor,line width= 0.4pt,line join=round,line cap=round] (133.02,213.03) -- (133.02,219.39);

\path[draw=drawColor,line width= 0.4pt,line join=round,line cap=round] (141.81,215.79) -- (148.17,215.79);

\path[draw=drawColor,line width= 0.4pt,line join=round,line cap=round] (144.99,212.61) -- (144.99,218.97);

\path[draw=drawColor,line width= 0.4pt,line join=round,line cap=round] (153.78,215.30) -- (160.15,215.30);

\path[draw=drawColor,line width= 0.4pt,line join=round,line cap=round] (156.97,212.12) -- (156.97,218.48);

\path[draw=drawColor,line width= 0.4pt,line join=round,line cap=round] (165.76,214.79) -- (172.12,214.79);

\path[draw=drawColor,line width= 0.4pt,line join=round,line cap=round] (168.94,211.61) -- (168.94,217.97);

\path[draw=drawColor,line width= 0.4pt,line join=round,line cap=round] (177.73,214.79) -- (184.10,214.79);

\path[draw=drawColor,line width= 0.4pt,line join=round,line cap=round] (180.91,211.61) -- (180.91,217.97);

\path[draw=drawColor,line width= 0.4pt,line join=round,line cap=round] (189.71,214.79) -- (196.07,214.79);

\path[draw=drawColor,line width= 0.4pt,line join=round,line cap=round] (192.89,211.61) -- (192.89,217.97);

\path[draw=drawColor,line width= 0.4pt,line join=round,line cap=round] (201.68,213.97) -- (208.04,213.97);

\path[draw=drawColor,line width= 0.4pt,line join=round,line cap=round] (204.86,210.79) -- (204.86,217.15);

\path[draw=drawColor,line width= 0.4pt,line join=round,line cap=round] (213.65,212.78) -- (220.02,212.78);

\path[draw=drawColor,line width= 0.4pt,line join=round,line cap=round] (216.84,209.60) -- (216.84,215.96);

\path[draw=drawColor,line width= 0.4pt,line join=round,line cap=round] (225.63,212.78) -- (231.99,212.78);

\path[draw=drawColor,line width= 0.4pt,line join=round,line cap=round] (228.81,209.60) -- (228.81,215.96);

\path[draw=drawColor,line width= 0.4pt,line join=round,line cap=round] (237.60,212.78) -- (243.97,212.78);

\path[draw=drawColor,line width= 0.4pt,line join=round,line cap=round] (240.78,209.60) -- (240.78,215.96);

\path[draw=drawColor,line width= 0.4pt,line join=round,line cap=round] (249.58,212.78) -- (255.94,212.78);

\path[draw=drawColor,line width= 0.4pt,line join=round,line cap=round] (252.76,209.60) -- (252.76,215.96);

\path[draw=drawColor,line width= 0.4pt,line join=round,line cap=round] (261.55,209.90) -- (267.91,209.90);

\path[draw=drawColor,line width= 0.4pt,line join=round,line cap=round] (264.73,206.72) -- (264.73,213.08);

\path[draw=drawColor,line width= 0.4pt,line join=round,line cap=round] (273.52,209.90) -- (279.89,209.90);

\path[draw=drawColor,line width= 0.4pt,line join=round,line cap=round] (276.71,206.72) -- (276.71,213.08);

\path[draw=drawColor,line width= 0.4pt,line join=round,line cap=round] (285.50,206.28) -- (291.86,206.28);

\path[draw=drawColor,line width= 0.4pt,line join=round,line cap=round] (288.68,203.10) -- (288.68,209.46);

\path[draw=drawColor,line width= 0.4pt,line join=round,line cap=round] (297.47,202.66) -- (303.84,202.66);

\path[draw=drawColor,line width= 0.4pt,line join=round,line cap=round] (300.65,199.48) -- (300.65,205.85);

\path[draw=drawColor,line width= 0.4pt,line join=round,line cap=round] (309.45,202.66) -- (315.81,202.66);

\path[draw=drawColor,line width= 0.4pt,line join=round,line cap=round] (312.63,199.48) -- (312.63,205.85);

\path[draw=drawColor,line width= 0.4pt,line join=round,line cap=round] (321.42,202.66) -- (327.78,202.66);

\path[draw=drawColor,line width= 0.4pt,line join=round,line cap=round] (324.60,199.48) -- (324.60,205.85);

\path[draw=drawColor,line width= 0.4pt,line join=round,line cap=round] (333.39,199.14) -- (339.76,199.14);

\path[draw=drawColor,line width= 0.4pt,line join=round,line cap=round] (336.58,195.96) -- (336.58,202.33);

\path[draw=drawColor,line width= 0.4pt,line join=round,line cap=round] (345.37,199.14) -- (351.73,199.14);

\path[draw=drawColor,line width= 0.4pt,line join=round,line cap=round] (348.55,195.96) -- (348.55,202.33);

\path[draw=drawColor,line width= 0.4pt,line join=round,line cap=round] (357.34,199.14) -- (363.71,199.14);

\path[draw=drawColor,line width= 0.4pt,line join=round,line cap=round] (360.52,195.96) -- (360.52,202.33);

\path[draw=drawColor,line width= 0.4pt,line join=round,line cap=round] (369.32,199.14) -- (375.68,199.14);

\path[draw=drawColor,line width= 0.4pt,line join=round,line cap=round] (372.50,195.96) -- (372.50,202.33);

\path[draw=drawColor,line width= 0.4pt,line join=round,line cap=round] (381.29,199.14) -- (387.65,199.14);

\path[draw=drawColor,line width= 0.4pt,line join=round,line cap=round] (384.47,195.96) -- (384.47,202.33);

\path[draw=drawColor,line width= 0.4pt,line join=round,line cap=round] (393.26,199.14) -- (399.63,199.14);

\path[draw=drawColor,line width= 0.4pt,line join=round,line cap=round] (396.45,195.96) -- (396.45,202.33);

\path[draw=drawColor,line width= 0.4pt,line join=round,line cap=round] (405.24,199.14) -- (411.60,199.14);

\path[draw=drawColor,line width= 0.4pt,line join=round,line cap=round] (408.42,195.96) -- (408.42,202.33);

\path[draw=drawColor,line width= 0.4pt,line join=round,line cap=round] (417.21,199.14) -- (423.58,199.14);

\path[draw=drawColor,line width= 0.4pt,line join=round,line cap=round] (420.39,195.96) -- (420.39,202.33);

\path[draw=drawColor,line width= 0.4pt,line join=round,line cap=round] (429.19,199.14) -- (433.62,199.14);

\path[draw=drawColor,line width= 0.4pt,line join=round,line cap=round] (432.37,195.96) -- (432.37,202.33);
\definecolor[named]{drawColor}{rgb}{0.00,0.00,0.00}

\path[draw=drawColor,line width= 0.4pt,line join=round,line cap=round] ( 49.20,233.26) --
	( 61.17,233.26) --
	( 61.17,217.54) --
	( 73.15,217.54) --
	( 73.15,201.19) --
	( 85.12,201.19) --
	( 85.12,186.00) --
	( 97.10,186.00) --
	( 97.10,177.84) --
	(109.07,177.84) --
	(109.07,174.75) --
	(121.04,174.75) --
	(121.04,173.65) --
	(133.02,173.65) --
	(133.02,172.57) --
	(144.99,172.57) --
	(144.99,171.32) --
	(156.97,171.32) --
	(156.97,169.88) --
	(168.94,169.88) --
	(168.94,168.41) --
	(204.86,168.41) --
	(204.86,166.08) --
	(216.84,166.08) --
	(216.84,162.77) --
	(264.73,162.77) --
	(264.73,155.08) --
	(288.68,155.08) --
	(288.68,146.12) --
	(300.65,146.12) --
	(300.65,137.87) --
	(336.58,137.87) --
	(336.58,130.50) --
	(433.62,130.50);

\path[draw=drawColor,line width= 0.4pt,line join=round,line cap=round] ( 57.99,217.54) -- ( 64.36,217.54);

\path[draw=drawColor,line width= 0.4pt,line join=round,line cap=round] ( 61.17,214.36) -- ( 61.17,220.73);

\path[draw=drawColor,line width= 0.4pt,line join=round,line cap=round] ( 69.97,201.19) -- ( 76.33,201.19);

\path[draw=drawColor,line width= 0.4pt,line join=round,line cap=round] ( 73.15,198.01) -- ( 73.15,204.37);

\path[draw=drawColor,line width= 0.4pt,line join=round,line cap=round] ( 81.94,186.00) -- ( 88.30,186.00);

\path[draw=drawColor,line width= 0.4pt,line join=round,line cap=round] ( 85.12,182.82) -- ( 85.12,189.18);

\path[draw=drawColor,line width= 0.4pt,line join=round,line cap=round] ( 93.91,177.84) -- (100.28,177.84);

\path[draw=drawColor,line width= 0.4pt,line join=round,line cap=round] ( 97.10,174.66) -- ( 97.10,181.02);

\path[draw=drawColor,line width= 0.4pt,line join=round,line cap=round] (105.89,174.75) -- (112.25,174.75);

\path[draw=drawColor,line width= 0.4pt,line join=round,line cap=round] (109.07,171.56) -- (109.07,177.93);

\path[draw=drawColor,line width= 0.4pt,line join=round,line cap=round] (117.86,173.65) -- (124.23,173.65);

\path[draw=drawColor,line width= 0.4pt,line join=round,line cap=round] (121.04,170.47) -- (121.04,176.84);

\path[draw=drawColor,line width= 0.4pt,line join=round,line cap=round] (129.84,172.57) -- (136.20,172.57);

\path[draw=drawColor,line width= 0.4pt,line join=round,line cap=round] (133.02,169.38) -- (133.02,175.75);

\path[draw=drawColor,line width= 0.4pt,line join=round,line cap=round] (141.81,171.32) -- (148.17,171.32);

\path[draw=drawColor,line width= 0.4pt,line join=round,line cap=round] (144.99,168.13) -- (144.99,174.50);

\path[draw=drawColor,line width= 0.4pt,line join=round,line cap=round] (153.78,169.88) -- (160.15,169.88);

\path[draw=drawColor,line width= 0.4pt,line join=round,line cap=round] (156.97,166.70) -- (156.97,173.07);

\path[draw=drawColor,line width= 0.4pt,line join=round,line cap=round] (165.76,168.41) -- (172.12,168.41);

\path[draw=drawColor,line width= 0.4pt,line join=round,line cap=round] (168.94,165.23) -- (168.94,171.59);

\path[draw=drawColor,line width= 0.4pt,line join=round,line cap=round] (177.73,168.41) -- (184.10,168.41);

\path[draw=drawColor,line width= 0.4pt,line join=round,line cap=round] (180.91,165.23) -- (180.91,171.59);

\path[draw=drawColor,line width= 0.4pt,line join=round,line cap=round] (189.71,168.41) -- (196.07,168.41);

\path[draw=drawColor,line width= 0.4pt,line join=round,line cap=round] (192.89,165.23) -- (192.89,171.59);

\path[draw=drawColor,line width= 0.4pt,line join=round,line cap=round] (201.68,166.08) -- (208.04,166.08);

\path[draw=drawColor,line width= 0.4pt,line join=round,line cap=round] (204.86,162.90) -- (204.86,169.26);

\path[draw=drawColor,line width= 0.4pt,line join=round,line cap=round] (213.65,162.77) -- (220.02,162.77);

\path[draw=drawColor,line width= 0.4pt,line join=round,line cap=round] (216.84,159.58) -- (216.84,165.95);

\path[draw=drawColor,line width= 0.4pt,line join=round,line cap=round] (225.63,162.77) -- (231.99,162.77);

\path[draw=drawColor,line width= 0.4pt,line join=round,line cap=round] (228.81,159.58) -- (228.81,165.95);

\path[draw=drawColor,line width= 0.4pt,line join=round,line cap=round] (237.60,162.77) -- (243.97,162.77);

\path[draw=drawColor,line width= 0.4pt,line join=round,line cap=round] (240.78,159.58) -- (240.78,165.95);

\path[draw=drawColor,line width= 0.4pt,line join=round,line cap=round] (249.58,162.77) -- (255.94,162.77);

\path[draw=drawColor,line width= 0.4pt,line join=round,line cap=round] (252.76,159.58) -- (252.76,165.95);

\path[draw=drawColor,line width= 0.4pt,line join=round,line cap=round] (261.55,155.08) -- (267.91,155.08);

\path[draw=drawColor,line width= 0.4pt,line join=round,line cap=round] (264.73,151.90) -- (264.73,158.27);

\path[draw=drawColor,line width= 0.4pt,line join=round,line cap=round] (273.52,155.08) -- (279.89,155.08);

\path[draw=drawColor,line width= 0.4pt,line join=round,line cap=round] (276.71,151.90) -- (276.71,158.27);

\path[draw=drawColor,line width= 0.4pt,line join=round,line cap=round] (285.50,146.12) -- (291.86,146.12);

\path[draw=drawColor,line width= 0.4pt,line join=round,line cap=round] (288.68,142.93) -- (288.68,149.30);

\path[draw=drawColor,line width= 0.4pt,line join=round,line cap=round] (297.47,137.87) -- (303.84,137.87);

\path[draw=drawColor,line width= 0.4pt,line join=round,line cap=round] (300.65,134.69) -- (300.65,141.06);

\path[draw=drawColor,line width= 0.4pt,line join=round,line cap=round] (309.45,137.87) -- (315.81,137.87);

\path[draw=drawColor,line width= 0.4pt,line join=round,line cap=round] (312.63,134.69) -- (312.63,141.06);

\path[draw=drawColor,line width= 0.4pt,line join=round,line cap=round] (321.42,137.87) -- (327.78,137.87);

\path[draw=drawColor,line width= 0.4pt,line join=round,line cap=round] (324.60,134.69) -- (324.60,141.06);

\path[draw=drawColor,line width= 0.4pt,line join=round,line cap=round] (333.39,130.50) -- (339.76,130.50);

\path[draw=drawColor,line width= 0.4pt,line join=round,line cap=round] (336.58,127.32) -- (336.58,133.68);

\path[draw=drawColor,line width= 0.4pt,line join=round,line cap=round] (345.37,130.50) -- (351.73,130.50);

\path[draw=drawColor,line width= 0.4pt,line join=round,line cap=round] (348.55,127.32) -- (348.55,133.68);

\path[draw=drawColor,line width= 0.4pt,line join=round,line cap=round] (357.34,130.50) -- (363.71,130.50);

\path[draw=drawColor,line width= 0.4pt,line join=round,line cap=round] (360.52,127.32) -- (360.52,133.68);

\path[draw=drawColor,line width= 0.4pt,line join=round,line cap=round] (369.32,130.50) -- (375.68,130.50);

\path[draw=drawColor,line width= 0.4pt,line join=round,line cap=round] (372.50,127.32) -- (372.50,133.68);

\path[draw=drawColor,line width= 0.4pt,line join=round,line cap=round] (381.29,130.50) -- (387.65,130.50);

\path[draw=drawColor,line width= 0.4pt,line join=round,line cap=round] (384.47,127.32) -- (384.47,133.68);

\path[draw=drawColor,line width= 0.4pt,line join=round,line cap=round] (393.26,130.50) -- (399.63,130.50);

\path[draw=drawColor,line width= 0.4pt,line join=round,line cap=round] (396.45,127.32) -- (396.45,133.68);

\path[draw=drawColor,line width= 0.4pt,line join=round,line cap=round] (405.24,130.50) -- (411.60,130.50);

\path[draw=drawColor,line width= 0.4pt,line join=round,line cap=round] (408.42,127.32) -- (408.42,133.68);

\path[draw=drawColor,line width= 0.4pt,line join=round,line cap=round] (417.21,130.50) -- (423.58,130.50);

\path[draw=drawColor,line width= 0.4pt,line join=round,line cap=round] (420.39,127.32) -- (420.39,133.68);

\path[draw=drawColor,line width= 0.4pt,line join=round,line cap=round] (429.19,130.50) -- (433.62,130.50);

\path[draw=drawColor,line width= 0.4pt,line join=round,line cap=round] (432.37,127.32) -- (432.37,133.68);
\end{scope}
\end{tikzpicture}
}
	\label{fig:surv1}
\end{figure}
\newpage

Like the extant literature, we do find strong evidence for the ability of trading partners to obtain quick and successful resolutions to sanction cases.\footnote{e.g., \citealp{mclean2010friends}} This is in line with our fourth hypothesis, that more proximate relationships between sender and recievers result in quicker compliance by the target state. But support for this general proximity hypothesis is limited to trade. States are not likely to comply more quickly to sanctions sent by allies, and are actually less likely to comply to neighbors. The graphs in Figure \ref{fig:surv2} demonstrate the effects of the ``proximate'' relationships that were significant in Table \ref{tab:regResults}. The panel on the left shows the effect for distance. Here we can see that even though the coefficient estimate for this variable was significant, its substantive meaning is almost nil. For trade, on the other hand, there is a strong effect. We can clearly see that when a target state faces a sanction from countries with whom they frequently trade, they are likely to comply sooner.

\newpage
\begin{figure}[ht]
	\centering
	\caption{Survival probabilities by ``proximity'' covariates. Grey designates scenarios in which the covariate is set to its minimum value and black where it is set to its high value (90th percentile).}
	% \includegraphics[width=1\textwidth]{oNet}
	\resizebox{1\textwidth}{!}{% Created by tikzDevice version 0.7.0 on 2014-06-13 12:42:25
% !TEX encoding = UTF-8 Unicode
\begin{tikzpicture}[x=1pt,y=1pt]
\definecolor[named]{fillColor}{rgb}{1.00,1.00,1.00}
\path[use as bounding box,fill=fillColor,fill opacity=0.00] (0,0) rectangle (578.16,216.81);
\begin{scope}
\path[clip] (  0.00,  0.00) rectangle (578.16,216.81);
\definecolor[named]{drawColor}{rgb}{0.00,0.00,0.00}

\path[draw=drawColor,line width= 0.4pt,line join=round,line cap=round] ( 49.20, 61.20) -- (263.88, 61.20);

\path[draw=drawColor,line width= 0.4pt,line join=round,line cap=round] ( 49.20, 61.20) -- ( 49.20, 55.20);

\path[draw=drawColor,line width= 0.4pt,line join=round,line cap=round] ( 84.98, 61.20) -- ( 84.98, 55.20);

\path[draw=drawColor,line width= 0.4pt,line join=round,line cap=round] (120.76, 61.20) -- (120.76, 55.20);

\path[draw=drawColor,line width= 0.4pt,line join=round,line cap=round] (156.54, 61.20) -- (156.54, 55.20);

\path[draw=drawColor,line width= 0.4pt,line join=round,line cap=round] (192.32, 61.20) -- (192.32, 55.20);

\path[draw=drawColor,line width= 0.4pt,line join=round,line cap=round] (228.10, 61.20) -- (228.10, 55.20);

\path[draw=drawColor,line width= 0.4pt,line join=round,line cap=round] (263.88, 61.20) -- (263.88, 55.20);

\node[text=drawColor,anchor=base,inner sep=0pt, outer sep=0pt, scale=  1.00] at ( 49.20, 39.60) {0};

\node[text=drawColor,anchor=base,inner sep=0pt, outer sep=0pt, scale=  1.00] at ( 84.98, 39.60) {5};

\node[text=drawColor,anchor=base,inner sep=0pt, outer sep=0pt, scale=  1.00] at (120.76, 39.60) {10};

\node[text=drawColor,anchor=base,inner sep=0pt, outer sep=0pt, scale=  1.00] at (156.54, 39.60) {15};

\node[text=drawColor,anchor=base,inner sep=0pt, outer sep=0pt, scale=  1.00] at (192.32, 39.60) {20};

\node[text=drawColor,anchor=base,inner sep=0pt, outer sep=0pt, scale=  1.00] at (228.10, 39.60) {25};

\node[text=drawColor,anchor=base,inner sep=0pt, outer sep=0pt, scale=  1.00] at (263.88, 39.60) {30};

\path[draw=drawColor,line width= 0.4pt,line join=round,line cap=round] ( 49.20, 65.14) -- ( 49.20,163.67);

\path[draw=drawColor,line width= 0.4pt,line join=round,line cap=round] ( 49.20, 65.14) -- ( 43.20, 65.14);

\path[draw=drawColor,line width= 0.4pt,line join=round,line cap=round] ( 49.20, 84.85) -- ( 43.20, 84.85);

\path[draw=drawColor,line width= 0.4pt,line join=round,line cap=round] ( 49.20,104.55) -- ( 43.20,104.55);

\path[draw=drawColor,line width= 0.4pt,line join=round,line cap=round] ( 49.20,124.26) -- ( 43.20,124.26);

\path[draw=drawColor,line width= 0.4pt,line join=round,line cap=round] ( 49.20,143.96) -- ( 43.20,143.96);

\path[draw=drawColor,line width= 0.4pt,line join=round,line cap=round] ( 49.20,163.67) -- ( 43.20,163.67);

\node[text=drawColor,anchor=base east,inner sep=0pt, outer sep=0pt, scale=  1.00] at ( 37.20, 61.70) {0.0};

\node[text=drawColor,anchor=base east,inner sep=0pt, outer sep=0pt, scale=  1.00] at ( 37.20, 81.40) {0.2};

\node[text=drawColor,anchor=base east,inner sep=0pt, outer sep=0pt, scale=  1.00] at ( 37.20,101.11) {0.4};

\node[text=drawColor,anchor=base east,inner sep=0pt, outer sep=0pt, scale=  1.00] at ( 37.20,120.81) {0.6};

\node[text=drawColor,anchor=base east,inner sep=0pt, outer sep=0pt, scale=  1.00] at ( 37.20,140.52) {0.8};

\node[text=drawColor,anchor=base east,inner sep=0pt, outer sep=0pt, scale=  1.00] at ( 37.20,160.23) {1.0};

\path[draw=drawColor,line width= 0.4pt,line join=round,line cap=round] ( 49.20, 61.20) --
	(263.88, 61.20) --
	(263.88,167.61) --
	( 49.20,167.61) --
	( 49.20, 61.20);
\end{scope}
\begin{scope}
\path[clip] (  0.00,  0.00) rectangle (289.08,216.81);
\definecolor[named]{drawColor}{rgb}{0.00,0.00,0.00}

\node[text=drawColor,anchor=base,inner sep=0pt, outer sep=0pt, scale=  1.20] at (156.54,188.07) {\bfseries Distance};
\end{scope}
\begin{scope}
\path[clip] ( 49.20, 61.20) rectangle (263.88,167.61);
\definecolor[named]{drawColor}{rgb}{0.65,0.06,0.08}

\path[draw=drawColor,line width= 0.4pt,line join=round,line cap=round] ( 49.20,163.67) --
	( 56.36,163.67) --
	( 56.36,157.83) --
	( 63.51,157.83) --
	( 63.51,153.54) --
	( 70.67,153.54) --
	( 70.67,151.47) --
	( 77.82,151.47) --
	( 77.82,149.75) --
	( 84.98,149.75) --
	( 84.98,149.39) --
	( 92.14,149.39) --
	( 92.14,149.16) --
	( 99.29,149.16) --
	( 99.29,148.94) --
	(106.45,148.94) --
	(106.45,148.68) --
	(113.60,148.68) --
	(113.60,147.64) --
	(127.92,147.64) --
	(127.92,147.32) --
	(142.23,147.32) --
	(142.23,146.93) --
	(149.38,146.93) --
	(149.38,146.29) --
	(156.54,146.29) --
	(156.54,145.62) --
	(178.01,145.62) --
	(178.01,144.68) --
	(192.32,144.68) --
	(192.32,143.55) --
	(199.48,143.55) --
	(199.48,142.52) --
	(464.25,142.52) --
	(464.25,142.52);

\path[draw=drawColor,line width= 0.4pt,line join=round,line cap=round] ( 53.17,157.83) -- ( 59.54,157.83);

\path[draw=drawColor,line width= 0.4pt,line join=round,line cap=round] ( 56.36,154.65) -- ( 56.36,161.01);

\path[draw=drawColor,line width= 0.4pt,line join=round,line cap=round] ( 60.33,153.54) -- ( 66.69,153.54);

\path[draw=drawColor,line width= 0.4pt,line join=round,line cap=round] ( 63.51,150.36) -- ( 63.51,156.72);

\path[draw=drawColor,line width= 0.4pt,line join=round,line cap=round] ( 67.49,151.47) -- ( 73.85,151.47);

\path[draw=drawColor,line width= 0.4pt,line join=round,line cap=round] ( 70.67,148.29) -- ( 70.67,154.66);

\path[draw=drawColor,line width= 0.4pt,line join=round,line cap=round] ( 74.64,149.75) -- ( 81.01,149.75);

\path[draw=drawColor,line width= 0.4pt,line join=round,line cap=round] ( 77.82,146.57) -- ( 77.82,152.93);

\path[draw=drawColor,line width= 0.4pt,line join=round,line cap=round] ( 81.80,149.39) -- ( 88.16,149.39);

\path[draw=drawColor,line width= 0.4pt,line join=round,line cap=round] ( 84.98,146.20) -- ( 84.98,152.57);

\path[draw=drawColor,line width= 0.4pt,line join=round,line cap=round] ( 88.95,149.16) -- ( 95.32,149.16);

\path[draw=drawColor,line width= 0.4pt,line join=round,line cap=round] ( 92.14,145.98) -- ( 92.14,152.34);

\path[draw=drawColor,line width= 0.4pt,line join=round,line cap=round] ( 96.11,148.94) -- (102.47,148.94);

\path[draw=drawColor,line width= 0.4pt,line join=round,line cap=round] ( 99.29,145.76) -- ( 99.29,152.12);

\path[draw=drawColor,line width= 0.4pt,line join=round,line cap=round] (103.27,148.68) -- (109.63,148.68);

\path[draw=drawColor,line width= 0.4pt,line join=round,line cap=round] (106.45,145.50) -- (106.45,151.86);

\path[draw=drawColor,line width= 0.4pt,line join=round,line cap=round] (110.42,147.64) -- (116.79,147.64);

\path[draw=drawColor,line width= 0.4pt,line join=round,line cap=round] (113.60,144.46) -- (113.60,150.82);

\path[draw=drawColor,line width= 0.4pt,line join=round,line cap=round] (117.58,147.64) -- (123.94,147.64);

\path[draw=drawColor,line width= 0.4pt,line join=round,line cap=round] (120.76,144.46) -- (120.76,150.82);

\path[draw=drawColor,line width= 0.4pt,line join=round,line cap=round] (124.73,147.32) -- (131.10,147.32);

\path[draw=drawColor,line width= 0.4pt,line join=round,line cap=round] (127.92,144.14) -- (127.92,150.51);

\path[draw=drawColor,line width= 0.4pt,line join=round,line cap=round] (131.89,147.32) -- (138.25,147.32);

\path[draw=drawColor,line width= 0.4pt,line join=round,line cap=round] (135.07,144.14) -- (135.07,150.51);

\path[draw=drawColor,line width= 0.4pt,line join=round,line cap=round] (139.05,146.93) -- (145.41,146.93);

\path[draw=drawColor,line width= 0.4pt,line join=round,line cap=round] (142.23,143.75) -- (142.23,150.11);

\path[draw=drawColor,line width= 0.4pt,line join=round,line cap=round] (146.20,146.29) -- (152.57,146.29);

\path[draw=drawColor,line width= 0.4pt,line join=round,line cap=round] (149.38,143.11) -- (149.38,149.47);

\path[draw=drawColor,line width= 0.4pt,line join=round,line cap=round] (153.36,145.62) -- (159.72,145.62);

\path[draw=drawColor,line width= 0.4pt,line join=round,line cap=round] (156.54,142.44) -- (156.54,148.80);

\path[draw=drawColor,line width= 0.4pt,line join=round,line cap=round] (160.51,145.62) -- (166.88,145.62);

\path[draw=drawColor,line width= 0.4pt,line join=round,line cap=round] (163.70,142.44) -- (163.70,148.80);

\path[draw=drawColor,line width= 0.4pt,line join=round,line cap=round] (167.67,145.62) -- (174.03,145.62);

\path[draw=drawColor,line width= 0.4pt,line join=round,line cap=round] (170.85,142.44) -- (170.85,148.80);

\path[draw=drawColor,line width= 0.4pt,line join=round,line cap=round] (174.83,144.68) -- (181.19,144.68);

\path[draw=drawColor,line width= 0.4pt,line join=round,line cap=round] (178.01,141.49) -- (178.01,147.86);

\path[draw=drawColor,line width= 0.4pt,line join=round,line cap=round] (181.98,144.68) -- (188.35,144.68);

\path[draw=drawColor,line width= 0.4pt,line join=round,line cap=round] (185.16,141.49) -- (185.16,147.86);

\path[draw=drawColor,line width= 0.4pt,line join=round,line cap=round] (189.14,143.55) -- (195.50,143.55);

\path[draw=drawColor,line width= 0.4pt,line join=round,line cap=round] (192.32,140.36) -- (192.32,146.73);

\path[draw=drawColor,line width= 0.4pt,line join=round,line cap=round] (196.29,142.52) -- (202.66,142.52);

\path[draw=drawColor,line width= 0.4pt,line join=round,line cap=round] (199.48,139.34) -- (199.48,145.70);

\path[draw=drawColor,line width= 0.4pt,line join=round,line cap=round] (203.45,142.52) -- (209.81,142.52);

\path[draw=drawColor,line width= 0.4pt,line join=round,line cap=round] (206.63,139.34) -- (206.63,145.70);

\path[draw=drawColor,line width= 0.4pt,line join=round,line cap=round] (210.61,142.52) -- (216.97,142.52);

\path[draw=drawColor,line width= 0.4pt,line join=round,line cap=round] (213.79,139.34) -- (213.79,145.70);

\path[draw=drawColor,line width= 0.4pt,line join=round,line cap=round] (217.76,142.52) -- (224.13,142.52);

\path[draw=drawColor,line width= 0.4pt,line join=round,line cap=round] (220.94,139.34) -- (220.94,145.70);

\path[draw=drawColor,line width= 0.4pt,line join=round,line cap=round] (224.92,142.52) -- (231.28,142.52);

\path[draw=drawColor,line width= 0.4pt,line join=round,line cap=round] (228.10,139.34) -- (228.10,145.70);

\path[draw=drawColor,line width= 0.4pt,line join=round,line cap=round] (232.07,142.52) -- (238.44,142.52);

\path[draw=drawColor,line width= 0.4pt,line join=round,line cap=round] (235.26,139.34) -- (235.26,145.70);

\path[draw=drawColor,line width= 0.4pt,line join=round,line cap=round] (239.23,142.52) -- (245.59,142.52);

\path[draw=drawColor,line width= 0.4pt,line join=round,line cap=round] (242.41,139.34) -- (242.41,145.70);

\path[draw=drawColor,line width= 0.4pt,line join=round,line cap=round] (246.39,142.52) -- (252.75,142.52);

\path[draw=drawColor,line width= 0.4pt,line join=round,line cap=round] (249.57,139.34) -- (249.57,145.70);

\path[draw=drawColor,line width= 0.4pt,line join=round,line cap=round] (253.54,142.52) -- (259.91,142.52);

\path[draw=drawColor,line width= 0.4pt,line join=round,line cap=round] (256.72,139.34) -- (256.72,145.70);

\path[draw=drawColor,line width= 0.4pt,line join=round,line cap=round] (260.70,142.52) -- (267.06,142.52);

\path[draw=drawColor,line width= 0.4pt,line join=round,line cap=round] (263.88,139.34) -- (263.88,145.70);

\path[draw=drawColor,line width= 0.4pt,line join=round,line cap=round] (267.85,142.52) -- (274.22,142.52);

\path[draw=drawColor,line width= 0.4pt,line join=round,line cap=round] (271.04,139.34) -- (271.04,145.70);

\path[draw=drawColor,line width= 0.4pt,line join=round,line cap=round] (275.01,142.52) -- (281.37,142.52);

\path[draw=drawColor,line width= 0.4pt,line join=round,line cap=round] (278.19,139.34) -- (278.19,145.70);

\path[draw=drawColor,line width= 0.4pt,line join=round,line cap=round] (282.17,142.52) -- (288.53,142.52);

\path[draw=drawColor,line width= 0.4pt,line join=round,line cap=round] (285.35,139.34) -- (285.35,145.70);

\path[draw=drawColor,line width= 0.4pt,line join=round,line cap=round] (289.32,142.52) -- (295.69,142.52);

\path[draw=drawColor,line width= 0.4pt,line join=round,line cap=round] (292.50,139.34) -- (292.50,145.70);

\path[draw=drawColor,line width= 0.4pt,line join=round,line cap=round] (296.48,142.52) -- (302.84,142.52);

\path[draw=drawColor,line width= 0.4pt,line join=round,line cap=round] (299.66,139.34) -- (299.66,145.70);

\path[draw=drawColor,line width= 0.4pt,line join=round,line cap=round] (303.63,142.52) -- (310.00,142.52);

\path[draw=drawColor,line width= 0.4pt,line join=round,line cap=round] (306.82,139.34) -- (306.82,145.70);

\path[draw=drawColor,line width= 0.4pt,line join=round,line cap=round] (310.79,142.52) -- (317.15,142.52);

\path[draw=drawColor,line width= 0.4pt,line join=round,line cap=round] (313.97,139.34) -- (313.97,145.70);

\path[draw=drawColor,line width= 0.4pt,line join=round,line cap=round] (317.95,142.52) -- (324.31,142.52);

\path[draw=drawColor,line width= 0.4pt,line join=round,line cap=round] (321.13,139.34) -- (321.13,145.70);

\path[draw=drawColor,line width= 0.4pt,line join=round,line cap=round] (325.10,142.52) -- (331.47,142.52);

\path[draw=drawColor,line width= 0.4pt,line join=round,line cap=round] (328.28,139.34) -- (328.28,145.70);

\path[draw=drawColor,line width= 0.4pt,line join=round,line cap=round] (332.26,142.52) -- (338.62,142.52);

\path[draw=drawColor,line width= 0.4pt,line join=round,line cap=round] (335.44,139.34) -- (335.44,145.70);

\path[draw=drawColor,line width= 0.4pt,line join=round,line cap=round] (339.41,142.52) -- (345.78,142.52);

\path[draw=drawColor,line width= 0.4pt,line join=round,line cap=round] (342.60,139.34) -- (342.60,145.70);

\path[draw=drawColor,line width= 0.4pt,line join=round,line cap=round] (346.57,142.52) -- (352.93,142.52);

\path[draw=drawColor,line width= 0.4pt,line join=round,line cap=round] (349.75,139.34) -- (349.75,145.70);

\path[draw=drawColor,line width= 0.4pt,line join=round,line cap=round] (353.73,142.52) -- (360.09,142.52);

\path[draw=drawColor,line width= 0.4pt,line join=round,line cap=round] (356.91,139.34) -- (356.91,145.70);

\path[draw=drawColor,line width= 0.4pt,line join=round,line cap=round] (360.88,142.52) -- (367.25,142.52);

\path[draw=drawColor,line width= 0.4pt,line join=round,line cap=round] (364.06,139.34) -- (364.06,145.70);

\path[draw=drawColor,line width= 0.4pt,line join=round,line cap=round] (368.04,142.52) -- (374.40,142.52);

\path[draw=drawColor,line width= 0.4pt,line join=round,line cap=round] (371.22,139.34) -- (371.22,145.70);

\path[draw=drawColor,line width= 0.4pt,line join=round,line cap=round] (375.19,142.52) -- (381.56,142.52);

\path[draw=drawColor,line width= 0.4pt,line join=round,line cap=round] (378.38,139.34) -- (378.38,145.70);

\path[draw=drawColor,line width= 0.4pt,line join=round,line cap=round] (382.35,142.52) -- (388.71,142.52);

\path[draw=drawColor,line width= 0.4pt,line join=round,line cap=round] (385.53,139.34) -- (385.53,145.70);

\path[draw=drawColor,line width= 0.4pt,line join=round,line cap=round] (389.51,142.52) -- (395.87,142.52);

\path[draw=drawColor,line width= 0.4pt,line join=round,line cap=round] (392.69,139.34) -- (392.69,145.70);

\path[draw=drawColor,line width= 0.4pt,line join=round,line cap=round] (396.66,142.52) -- (403.03,142.52);

\path[draw=drawColor,line width= 0.4pt,line join=round,line cap=round] (399.84,139.34) -- (399.84,145.70);

\path[draw=drawColor,line width= 0.4pt,line join=round,line cap=round] (403.82,142.52) -- (410.18,142.52);

\path[draw=drawColor,line width= 0.4pt,line join=round,line cap=round] (407.00,139.34) -- (407.00,145.70);

\path[draw=drawColor,line width= 0.4pt,line join=round,line cap=round] (410.97,142.52) -- (417.34,142.52);

\path[draw=drawColor,line width= 0.4pt,line join=round,line cap=round] (414.16,139.34) -- (414.16,145.70);

\path[draw=drawColor,line width= 0.4pt,line join=round,line cap=round] (418.13,142.52) -- (424.49,142.52);

\path[draw=drawColor,line width= 0.4pt,line join=round,line cap=round] (421.31,139.34) -- (421.31,145.70);

\path[draw=drawColor,line width= 0.4pt,line join=round,line cap=round] (425.29,142.52) -- (431.65,142.52);

\path[draw=drawColor,line width= 0.4pt,line join=round,line cap=round] (428.47,139.34) -- (428.47,145.70);

\path[draw=drawColor,line width= 0.4pt,line join=round,line cap=round] (432.44,142.52) -- (438.81,142.52);

\path[draw=drawColor,line width= 0.4pt,line join=round,line cap=round] (435.62,139.34) -- (435.62,145.70);

\path[draw=drawColor,line width= 0.4pt,line join=round,line cap=round] (439.60,142.52) -- (445.96,142.52);

\path[draw=drawColor,line width= 0.4pt,line join=round,line cap=round] (442.78,139.34) -- (442.78,145.70);

\path[draw=drawColor,line width= 0.4pt,line join=round,line cap=round] (446.75,142.52) -- (453.12,142.52);

\path[draw=drawColor,line width= 0.4pt,line join=round,line cap=round] (449.94,139.34) -- (449.94,145.70);

\path[draw=drawColor,line width= 0.4pt,line join=round,line cap=round] (453.91,142.52) -- (460.27,142.52);

\path[draw=drawColor,line width= 0.4pt,line join=round,line cap=round] (457.09,139.34) -- (457.09,145.70);

\path[draw=drawColor,line width= 0.4pt,line join=round,line cap=round] (461.07,142.52) -- (467.43,142.52);

\path[draw=drawColor,line width= 0.4pt,line join=round,line cap=round] (464.25,139.34) -- (464.25,145.70);
\definecolor[named]{drawColor}{rgb}{0.03,0.32,0.61}

\path[draw=drawColor,line width= 0.4pt,line join=round,line cap=round] ( 49.20,163.67) --
	( 56.36,163.67) --
	( 56.36,153.31) --
	( 63.51,153.31) --
	( 63.51,146.02) --
	( 70.67,146.02) --
	( 70.67,142.61) --
	( 77.82,142.61) --
	( 77.82,139.82) --
	( 84.98,139.82) --
	( 84.98,139.24) --
	( 92.14,139.24) --
	( 92.14,138.87) --
	( 99.29,138.87) --
	( 99.29,138.52) --
	(106.45,138.52) --
	(106.45,138.12) --
	(113.60,138.12) --
	(113.60,136.46) --
	(127.92,136.46) --
	(127.92,135.97) --
	(142.23,135.97) --
	(142.23,135.35) --
	(149.38,135.35) --
	(149.38,134.36) --
	(156.54,134.36) --
	(156.54,133.33) --
	(178.01,133.33) --
	(178.01,131.87) --
	(192.32,131.87) --
	(192.32,130.16) --
	(199.48,130.16) --
	(199.48,128.62) --
	(464.25,128.62) --
	(464.25,128.62);

\path[draw=drawColor,line width= 0.4pt,line join=round,line cap=round] ( 53.17,153.31) -- ( 59.54,153.31);

\path[draw=drawColor,line width= 0.4pt,line join=round,line cap=round] ( 56.36,150.13) -- ( 56.36,156.49);

\path[draw=drawColor,line width= 0.4pt,line join=round,line cap=round] ( 60.33,146.02) -- ( 66.69,146.02);

\path[draw=drawColor,line width= 0.4pt,line join=round,line cap=round] ( 63.51,142.84) -- ( 63.51,149.20);

\path[draw=drawColor,line width= 0.4pt,line join=round,line cap=round] ( 67.49,142.61) -- ( 73.85,142.61);

\path[draw=drawColor,line width= 0.4pt,line join=round,line cap=round] ( 70.67,139.43) -- ( 70.67,145.80);

\path[draw=drawColor,line width= 0.4pt,line join=round,line cap=round] ( 74.64,139.82) -- ( 81.01,139.82);

\path[draw=drawColor,line width= 0.4pt,line join=round,line cap=round] ( 77.82,136.64) -- ( 77.82,143.00);

\path[draw=drawColor,line width= 0.4pt,line join=round,line cap=round] ( 81.80,139.24) -- ( 88.16,139.24);

\path[draw=drawColor,line width= 0.4pt,line join=round,line cap=round] ( 84.98,136.06) -- ( 84.98,142.42);

\path[draw=drawColor,line width= 0.4pt,line join=round,line cap=round] ( 88.95,138.87) -- ( 95.32,138.87);

\path[draw=drawColor,line width= 0.4pt,line join=round,line cap=round] ( 92.14,135.69) -- ( 92.14,142.05);

\path[draw=drawColor,line width= 0.4pt,line join=round,line cap=round] ( 96.11,138.52) -- (102.47,138.52);

\path[draw=drawColor,line width= 0.4pt,line join=round,line cap=round] ( 99.29,135.34) -- ( 99.29,141.70);

\path[draw=drawColor,line width= 0.4pt,line join=round,line cap=round] (103.27,138.12) -- (109.63,138.12);

\path[draw=drawColor,line width= 0.4pt,line join=round,line cap=round] (106.45,134.93) -- (106.45,141.30);

\path[draw=drawColor,line width= 0.4pt,line join=round,line cap=round] (110.42,136.46) -- (116.79,136.46);

\path[draw=drawColor,line width= 0.4pt,line join=round,line cap=round] (113.60,133.28) -- (113.60,139.65);

\path[draw=drawColor,line width= 0.4pt,line join=round,line cap=round] (117.58,136.46) -- (123.94,136.46);

\path[draw=drawColor,line width= 0.4pt,line join=round,line cap=round] (120.76,133.28) -- (120.76,139.65);

\path[draw=drawColor,line width= 0.4pt,line join=round,line cap=round] (124.73,135.97) -- (131.10,135.97);

\path[draw=drawColor,line width= 0.4pt,line join=round,line cap=round] (127.92,132.79) -- (127.92,139.15);

\path[draw=drawColor,line width= 0.4pt,line join=round,line cap=round] (131.89,135.97) -- (138.25,135.97);

\path[draw=drawColor,line width= 0.4pt,line join=round,line cap=round] (135.07,132.79) -- (135.07,139.15);

\path[draw=drawColor,line width= 0.4pt,line join=round,line cap=round] (139.05,135.35) -- (145.41,135.35);

\path[draw=drawColor,line width= 0.4pt,line join=round,line cap=round] (142.23,132.17) -- (142.23,138.53);

\path[draw=drawColor,line width= 0.4pt,line join=round,line cap=round] (146.20,134.36) -- (152.57,134.36);

\path[draw=drawColor,line width= 0.4pt,line join=round,line cap=round] (149.38,131.18) -- (149.38,137.54);

\path[draw=drawColor,line width= 0.4pt,line join=round,line cap=round] (153.36,133.33) -- (159.72,133.33);

\path[draw=drawColor,line width= 0.4pt,line join=round,line cap=round] (156.54,130.14) -- (156.54,136.51);

\path[draw=drawColor,line width= 0.4pt,line join=round,line cap=round] (160.51,133.33) -- (166.88,133.33);

\path[draw=drawColor,line width= 0.4pt,line join=round,line cap=round] (163.70,130.14) -- (163.70,136.51);

\path[draw=drawColor,line width= 0.4pt,line join=round,line cap=round] (167.67,133.33) -- (174.03,133.33);

\path[draw=drawColor,line width= 0.4pt,line join=round,line cap=round] (170.85,130.14) -- (170.85,136.51);

\path[draw=drawColor,line width= 0.4pt,line join=round,line cap=round] (174.83,131.87) -- (181.19,131.87);

\path[draw=drawColor,line width= 0.4pt,line join=round,line cap=round] (178.01,128.69) -- (178.01,135.05);

\path[draw=drawColor,line width= 0.4pt,line join=round,line cap=round] (181.98,131.87) -- (188.35,131.87);

\path[draw=drawColor,line width= 0.4pt,line join=round,line cap=round] (185.16,128.69) -- (185.16,135.05);

\path[draw=drawColor,line width= 0.4pt,line join=round,line cap=round] (189.14,130.16) -- (195.50,130.16);

\path[draw=drawColor,line width= 0.4pt,line join=round,line cap=round] (192.32,126.97) -- (192.32,133.34);

\path[draw=drawColor,line width= 0.4pt,line join=round,line cap=round] (196.29,128.62) -- (202.66,128.62);

\path[draw=drawColor,line width= 0.4pt,line join=round,line cap=round] (199.48,125.44) -- (199.48,131.80);

\path[draw=drawColor,line width= 0.4pt,line join=round,line cap=round] (203.45,128.62) -- (209.81,128.62);

\path[draw=drawColor,line width= 0.4pt,line join=round,line cap=round] (206.63,125.44) -- (206.63,131.80);

\path[draw=drawColor,line width= 0.4pt,line join=round,line cap=round] (210.61,128.62) -- (216.97,128.62);

\path[draw=drawColor,line width= 0.4pt,line join=round,line cap=round] (213.79,125.44) -- (213.79,131.80);

\path[draw=drawColor,line width= 0.4pt,line join=round,line cap=round] (217.76,128.62) -- (224.13,128.62);

\path[draw=drawColor,line width= 0.4pt,line join=round,line cap=round] (220.94,125.44) -- (220.94,131.80);

\path[draw=drawColor,line width= 0.4pt,line join=round,line cap=round] (224.92,128.62) -- (231.28,128.62);

\path[draw=drawColor,line width= 0.4pt,line join=round,line cap=round] (228.10,125.44) -- (228.10,131.80);

\path[draw=drawColor,line width= 0.4pt,line join=round,line cap=round] (232.07,128.62) -- (238.44,128.62);

\path[draw=drawColor,line width= 0.4pt,line join=round,line cap=round] (235.26,125.44) -- (235.26,131.80);

\path[draw=drawColor,line width= 0.4pt,line join=round,line cap=round] (239.23,128.62) -- (245.59,128.62);

\path[draw=drawColor,line width= 0.4pt,line join=round,line cap=round] (242.41,125.44) -- (242.41,131.80);

\path[draw=drawColor,line width= 0.4pt,line join=round,line cap=round] (246.39,128.62) -- (252.75,128.62);

\path[draw=drawColor,line width= 0.4pt,line join=round,line cap=round] (249.57,125.44) -- (249.57,131.80);

\path[draw=drawColor,line width= 0.4pt,line join=round,line cap=round] (253.54,128.62) -- (259.91,128.62);

\path[draw=drawColor,line width= 0.4pt,line join=round,line cap=round] (256.72,125.44) -- (256.72,131.80);

\path[draw=drawColor,line width= 0.4pt,line join=round,line cap=round] (260.70,128.62) -- (267.06,128.62);

\path[draw=drawColor,line width= 0.4pt,line join=round,line cap=round] (263.88,125.44) -- (263.88,131.80);

\path[draw=drawColor,line width= 0.4pt,line join=round,line cap=round] (267.85,128.62) -- (274.22,128.62);

\path[draw=drawColor,line width= 0.4pt,line join=round,line cap=round] (271.04,125.44) -- (271.04,131.80);

\path[draw=drawColor,line width= 0.4pt,line join=round,line cap=round] (275.01,128.62) -- (281.37,128.62);

\path[draw=drawColor,line width= 0.4pt,line join=round,line cap=round] (278.19,125.44) -- (278.19,131.80);

\path[draw=drawColor,line width= 0.4pt,line join=round,line cap=round] (282.17,128.62) -- (288.53,128.62);

\path[draw=drawColor,line width= 0.4pt,line join=round,line cap=round] (285.35,125.44) -- (285.35,131.80);

\path[draw=drawColor,line width= 0.4pt,line join=round,line cap=round] (289.32,128.62) -- (295.69,128.62);

\path[draw=drawColor,line width= 0.4pt,line join=round,line cap=round] (292.50,125.44) -- (292.50,131.80);

\path[draw=drawColor,line width= 0.4pt,line join=round,line cap=round] (296.48,128.62) -- (302.84,128.62);

\path[draw=drawColor,line width= 0.4pt,line join=round,line cap=round] (299.66,125.44) -- (299.66,131.80);

\path[draw=drawColor,line width= 0.4pt,line join=round,line cap=round] (303.63,128.62) -- (310.00,128.62);

\path[draw=drawColor,line width= 0.4pt,line join=round,line cap=round] (306.82,125.44) -- (306.82,131.80);

\path[draw=drawColor,line width= 0.4pt,line join=round,line cap=round] (310.79,128.62) -- (317.15,128.62);

\path[draw=drawColor,line width= 0.4pt,line join=round,line cap=round] (313.97,125.44) -- (313.97,131.80);

\path[draw=drawColor,line width= 0.4pt,line join=round,line cap=round] (317.95,128.62) -- (324.31,128.62);

\path[draw=drawColor,line width= 0.4pt,line join=round,line cap=round] (321.13,125.44) -- (321.13,131.80);

\path[draw=drawColor,line width= 0.4pt,line join=round,line cap=round] (325.10,128.62) -- (331.47,128.62);

\path[draw=drawColor,line width= 0.4pt,line join=round,line cap=round] (328.28,125.44) -- (328.28,131.80);

\path[draw=drawColor,line width= 0.4pt,line join=round,line cap=round] (332.26,128.62) -- (338.62,128.62);

\path[draw=drawColor,line width= 0.4pt,line join=round,line cap=round] (335.44,125.44) -- (335.44,131.80);

\path[draw=drawColor,line width= 0.4pt,line join=round,line cap=round] (339.41,128.62) -- (345.78,128.62);

\path[draw=drawColor,line width= 0.4pt,line join=round,line cap=round] (342.60,125.44) -- (342.60,131.80);

\path[draw=drawColor,line width= 0.4pt,line join=round,line cap=round] (346.57,128.62) -- (352.93,128.62);

\path[draw=drawColor,line width= 0.4pt,line join=round,line cap=round] (349.75,125.44) -- (349.75,131.80);

\path[draw=drawColor,line width= 0.4pt,line join=round,line cap=round] (353.73,128.62) -- (360.09,128.62);

\path[draw=drawColor,line width= 0.4pt,line join=round,line cap=round] (356.91,125.44) -- (356.91,131.80);

\path[draw=drawColor,line width= 0.4pt,line join=round,line cap=round] (360.88,128.62) -- (367.25,128.62);

\path[draw=drawColor,line width= 0.4pt,line join=round,line cap=round] (364.06,125.44) -- (364.06,131.80);

\path[draw=drawColor,line width= 0.4pt,line join=round,line cap=round] (368.04,128.62) -- (374.40,128.62);

\path[draw=drawColor,line width= 0.4pt,line join=round,line cap=round] (371.22,125.44) -- (371.22,131.80);

\path[draw=drawColor,line width= 0.4pt,line join=round,line cap=round] (375.19,128.62) -- (381.56,128.62);

\path[draw=drawColor,line width= 0.4pt,line join=round,line cap=round] (378.38,125.44) -- (378.38,131.80);

\path[draw=drawColor,line width= 0.4pt,line join=round,line cap=round] (382.35,128.62) -- (388.71,128.62);

\path[draw=drawColor,line width= 0.4pt,line join=round,line cap=round] (385.53,125.44) -- (385.53,131.80);

\path[draw=drawColor,line width= 0.4pt,line join=round,line cap=round] (389.51,128.62) -- (395.87,128.62);

\path[draw=drawColor,line width= 0.4pt,line join=round,line cap=round] (392.69,125.44) -- (392.69,131.80);

\path[draw=drawColor,line width= 0.4pt,line join=round,line cap=round] (396.66,128.62) -- (403.03,128.62);

\path[draw=drawColor,line width= 0.4pt,line join=round,line cap=round] (399.84,125.44) -- (399.84,131.80);

\path[draw=drawColor,line width= 0.4pt,line join=round,line cap=round] (403.82,128.62) -- (410.18,128.62);

\path[draw=drawColor,line width= 0.4pt,line join=round,line cap=round] (407.00,125.44) -- (407.00,131.80);

\path[draw=drawColor,line width= 0.4pt,line join=round,line cap=round] (410.97,128.62) -- (417.34,128.62);

\path[draw=drawColor,line width= 0.4pt,line join=round,line cap=round] (414.16,125.44) -- (414.16,131.80);

\path[draw=drawColor,line width= 0.4pt,line join=round,line cap=round] (418.13,128.62) -- (424.49,128.62);

\path[draw=drawColor,line width= 0.4pt,line join=round,line cap=round] (421.31,125.44) -- (421.31,131.80);

\path[draw=drawColor,line width= 0.4pt,line join=round,line cap=round] (425.29,128.62) -- (431.65,128.62);

\path[draw=drawColor,line width= 0.4pt,line join=round,line cap=round] (428.47,125.44) -- (428.47,131.80);

\path[draw=drawColor,line width= 0.4pt,line join=round,line cap=round] (432.44,128.62) -- (438.81,128.62);

\path[draw=drawColor,line width= 0.4pt,line join=round,line cap=round] (435.62,125.44) -- (435.62,131.80);

\path[draw=drawColor,line width= 0.4pt,line join=round,line cap=round] (439.60,128.62) -- (445.96,128.62);

\path[draw=drawColor,line width= 0.4pt,line join=round,line cap=round] (442.78,125.44) -- (442.78,131.80);

\path[draw=drawColor,line width= 0.4pt,line join=round,line cap=round] (446.75,128.62) -- (453.12,128.62);

\path[draw=drawColor,line width= 0.4pt,line join=round,line cap=round] (449.94,125.44) -- (449.94,131.80);

\path[draw=drawColor,line width= 0.4pt,line join=round,line cap=round] (453.91,128.62) -- (460.27,128.62);

\path[draw=drawColor,line width= 0.4pt,line join=round,line cap=round] (457.09,125.44) -- (457.09,131.80);

\path[draw=drawColor,line width= 0.4pt,line join=round,line cap=round] (461.07,128.62) -- (467.43,128.62);

\path[draw=drawColor,line width= 0.4pt,line join=round,line cap=round] (464.25,125.44) -- (464.25,131.80);
\end{scope}
\begin{scope}
\path[clip] (  0.00,  0.00) rectangle (289.08,216.81);
\definecolor[named]{drawColor}{rgb}{0.00,0.00,0.00}

\node[text=drawColor,rotate= 90.00,anchor=base,inner sep=0pt, outer sep=0pt, scale=  1.00] at ( 10.80,114.41) {Survival Prob.};

\node[text=drawColor,anchor=base,inner sep=0pt, outer sep=0pt, scale=  1.00] at (156.54, 15.60) {Time (Years)};
\end{scope}
\begin{scope}
\path[clip] (  0.00,  0.00) rectangle (578.16,216.81);
\definecolor[named]{drawColor}{rgb}{0.00,0.00,0.00}

\path[draw=drawColor,line width= 0.4pt,line join=round,line cap=round] (338.28, 61.20) -- (552.96, 61.20);

\path[draw=drawColor,line width= 0.4pt,line join=round,line cap=round] (338.28, 61.20) -- (338.28, 55.20);

\path[draw=drawColor,line width= 0.4pt,line join=round,line cap=round] (374.06, 61.20) -- (374.06, 55.20);

\path[draw=drawColor,line width= 0.4pt,line join=round,line cap=round] (409.84, 61.20) -- (409.84, 55.20);

\path[draw=drawColor,line width= 0.4pt,line join=round,line cap=round] (445.62, 61.20) -- (445.62, 55.20);

\path[draw=drawColor,line width= 0.4pt,line join=round,line cap=round] (481.40, 61.20) -- (481.40, 55.20);

\path[draw=drawColor,line width= 0.4pt,line join=round,line cap=round] (517.18, 61.20) -- (517.18, 55.20);

\path[draw=drawColor,line width= 0.4pt,line join=round,line cap=round] (552.96, 61.20) -- (552.96, 55.20);

\node[text=drawColor,anchor=base,inner sep=0pt, outer sep=0pt, scale=  1.00] at (338.28, 39.60) {0};

\node[text=drawColor,anchor=base,inner sep=0pt, outer sep=0pt, scale=  1.00] at (374.06, 39.60) {5};

\node[text=drawColor,anchor=base,inner sep=0pt, outer sep=0pt, scale=  1.00] at (409.84, 39.60) {10};

\node[text=drawColor,anchor=base,inner sep=0pt, outer sep=0pt, scale=  1.00] at (445.62, 39.60) {15};

\node[text=drawColor,anchor=base,inner sep=0pt, outer sep=0pt, scale=  1.00] at (481.40, 39.60) {20};

\node[text=drawColor,anchor=base,inner sep=0pt, outer sep=0pt, scale=  1.00] at (517.18, 39.60) {25};

\node[text=drawColor,anchor=base,inner sep=0pt, outer sep=0pt, scale=  1.00] at (552.96, 39.60) {30};

\path[draw=drawColor,line width= 0.4pt,line join=round,line cap=round] (338.28, 65.14) -- (338.28,163.67);

\path[draw=drawColor,line width= 0.4pt,line join=round,line cap=round] (338.28, 65.14) -- (332.28, 65.14);

\path[draw=drawColor,line width= 0.4pt,line join=round,line cap=round] (338.28, 84.85) -- (332.28, 84.85);

\path[draw=drawColor,line width= 0.4pt,line join=round,line cap=round] (338.28,104.55) -- (332.28,104.55);

\path[draw=drawColor,line width= 0.4pt,line join=round,line cap=round] (338.28,124.26) -- (332.28,124.26);

\path[draw=drawColor,line width= 0.4pt,line join=round,line cap=round] (338.28,143.96) -- (332.28,143.96);

\path[draw=drawColor,line width= 0.4pt,line join=round,line cap=round] (338.28,163.67) -- (332.28,163.67);

\node[text=drawColor,anchor=base east,inner sep=0pt, outer sep=0pt, scale=  1.00] at (326.28, 61.70) {0.0};

\node[text=drawColor,anchor=base east,inner sep=0pt, outer sep=0pt, scale=  1.00] at (326.28, 81.40) {0.2};

\node[text=drawColor,anchor=base east,inner sep=0pt, outer sep=0pt, scale=  1.00] at (326.28,101.11) {0.4};

\node[text=drawColor,anchor=base east,inner sep=0pt, outer sep=0pt, scale=  1.00] at (326.28,120.81) {0.6};

\node[text=drawColor,anchor=base east,inner sep=0pt, outer sep=0pt, scale=  1.00] at (326.28,140.52) {0.8};

\node[text=drawColor,anchor=base east,inner sep=0pt, outer sep=0pt, scale=  1.00] at (326.28,160.23) {1.0};

\path[draw=drawColor,line width= 0.4pt,line join=round,line cap=round] (338.28, 61.20) --
	(552.96, 61.20) --
	(552.96,167.61) --
	(338.28,167.61) --
	(338.28, 61.20);
\end{scope}
\begin{scope}
\path[clip] (289.08,  0.00) rectangle (578.16,216.81);
\definecolor[named]{drawColor}{rgb}{0.00,0.00,0.00}

\node[text=drawColor,anchor=base,inner sep=0pt, outer sep=0pt, scale=  1.20] at (445.62,188.07) {\bfseries Trade};
\end{scope}
\begin{scope}
\path[clip] (338.28, 61.20) rectangle (552.96,167.61);
\definecolor[named]{drawColor}{rgb}{0.65,0.06,0.08}

\path[draw=drawColor,line width= 0.4pt,line join=round,line cap=round] (338.28,163.67) --
	(345.44,163.67) --
	(345.44,158.32) --
	(352.59,158.32) --
	(352.59,154.37) --
	(359.75,154.37) --
	(359.75,152.47) --
	(366.90,152.47) --
	(366.90,150.87) --
	(374.06,150.87) --
	(374.06,150.54) --
	(381.22,150.54) --
	(381.22,150.32) --
	(388.37,150.32) --
	(388.37,150.12) --
	(395.53,150.12) --
	(395.53,149.88) --
	(402.68,149.88) --
	(402.68,148.92) --
	(417.00,148.92) --
	(417.00,148.62) --
	(431.31,148.62) --
	(431.31,148.26) --
	(438.46,148.26) --
	(438.46,147.67) --
	(445.62,147.67) --
	(445.62,147.04) --
	(467.09,147.04) --
	(467.09,146.16) --
	(481.40,146.16) --
	(481.40,145.11) --
	(488.56,145.11) --
	(488.56,144.15) --
	(578.16,144.15);

\path[draw=drawColor,line width= 0.4pt,line join=round,line cap=round] (342.25,158.32) -- (348.62,158.32);

\path[draw=drawColor,line width= 0.4pt,line join=round,line cap=round] (345.44,155.14) -- (345.44,161.51);

\path[draw=drawColor,line width= 0.4pt,line join=round,line cap=round] (349.41,154.37) -- (355.77,154.37);

\path[draw=drawColor,line width= 0.4pt,line join=round,line cap=round] (352.59,151.19) -- (352.59,157.56);

\path[draw=drawColor,line width= 0.4pt,line join=round,line cap=round] (356.57,152.47) -- (362.93,152.47);

\path[draw=drawColor,line width= 0.4pt,line join=round,line cap=round] (359.75,149.28) -- (359.75,155.65);

\path[draw=drawColor,line width= 0.4pt,line join=round,line cap=round] (363.72,150.87) -- (370.09,150.87);

\path[draw=drawColor,line width= 0.4pt,line join=round,line cap=round] (366.90,147.69) -- (366.90,154.05);

\path[draw=drawColor,line width= 0.4pt,line join=round,line cap=round] (370.88,150.54) -- (377.24,150.54);

\path[draw=drawColor,line width= 0.4pt,line join=round,line cap=round] (374.06,147.35) -- (374.06,153.72);

\path[draw=drawColor,line width= 0.4pt,line join=round,line cap=round] (378.03,150.32) -- (384.40,150.32);

\path[draw=drawColor,line width= 0.4pt,line join=round,line cap=round] (381.22,147.14) -- (381.22,153.50);

\path[draw=drawColor,line width= 0.4pt,line join=round,line cap=round] (385.19,150.12) -- (391.55,150.12);

\path[draw=drawColor,line width= 0.4pt,line join=round,line cap=round] (388.37,146.94) -- (388.37,153.30);

\path[draw=drawColor,line width= 0.4pt,line join=round,line cap=round] (392.35,149.88) -- (398.71,149.88);

\path[draw=drawColor,line width= 0.4pt,line join=round,line cap=round] (395.53,146.70) -- (395.53,153.07);

\path[draw=drawColor,line width= 0.4pt,line join=round,line cap=round] (399.50,148.92) -- (405.87,148.92);

\path[draw=drawColor,line width= 0.4pt,line join=round,line cap=round] (402.68,145.73) -- (402.68,152.10);

\path[draw=drawColor,line width= 0.4pt,line join=round,line cap=round] (406.66,148.92) -- (413.02,148.92);

\path[draw=drawColor,line width= 0.4pt,line join=round,line cap=round] (409.84,145.73) -- (409.84,152.10);

\path[draw=drawColor,line width= 0.4pt,line join=round,line cap=round] (413.81,148.62) -- (420.18,148.62);

\path[draw=drawColor,line width= 0.4pt,line join=round,line cap=round] (417.00,145.44) -- (417.00,151.80);

\path[draw=drawColor,line width= 0.4pt,line join=round,line cap=round] (420.97,148.62) -- (427.33,148.62);

\path[draw=drawColor,line width= 0.4pt,line join=round,line cap=round] (424.15,145.44) -- (424.15,151.80);

\path[draw=drawColor,line width= 0.4pt,line join=round,line cap=round] (428.13,148.26) -- (434.49,148.26);

\path[draw=drawColor,line width= 0.4pt,line join=round,line cap=round] (431.31,145.07) -- (431.31,151.44);

\path[draw=drawColor,line width= 0.4pt,line join=round,line cap=round] (435.28,147.67) -- (441.65,147.67);

\path[draw=drawColor,line width= 0.4pt,line join=round,line cap=round] (438.46,144.48) -- (438.46,150.85);

\path[draw=drawColor,line width= 0.4pt,line join=round,line cap=round] (442.44,147.04) -- (448.80,147.04);

\path[draw=drawColor,line width= 0.4pt,line join=round,line cap=round] (445.62,143.86) -- (445.62,150.23);

\path[draw=drawColor,line width= 0.4pt,line join=round,line cap=round] (449.59,147.04) -- (455.96,147.04);

\path[draw=drawColor,line width= 0.4pt,line join=round,line cap=round] (452.78,143.86) -- (452.78,150.23);

\path[draw=drawColor,line width= 0.4pt,line join=round,line cap=round] (456.75,147.04) -- (463.11,147.04);

\path[draw=drawColor,line width= 0.4pt,line join=round,line cap=round] (459.93,143.86) -- (459.93,150.23);

\path[draw=drawColor,line width= 0.4pt,line join=round,line cap=round] (463.91,146.16) -- (470.27,146.16);

\path[draw=drawColor,line width= 0.4pt,line join=round,line cap=round] (467.09,142.98) -- (467.09,149.34);

\path[draw=drawColor,line width= 0.4pt,line join=round,line cap=round] (471.06,146.16) -- (477.43,146.16);

\path[draw=drawColor,line width= 0.4pt,line join=round,line cap=round] (474.24,142.98) -- (474.24,149.34);

\path[draw=drawColor,line width= 0.4pt,line join=round,line cap=round] (478.22,145.11) -- (484.58,145.11);

\path[draw=drawColor,line width= 0.4pt,line join=round,line cap=round] (481.40,141.93) -- (481.40,148.29);

\path[draw=drawColor,line width= 0.4pt,line join=round,line cap=round] (485.37,144.15) -- (491.74,144.15);

\path[draw=drawColor,line width= 0.4pt,line join=round,line cap=round] (488.56,140.97) -- (488.56,147.34);

\path[draw=drawColor,line width= 0.4pt,line join=round,line cap=round] (492.53,144.15) -- (498.89,144.15);

\path[draw=drawColor,line width= 0.4pt,line join=round,line cap=round] (495.71,140.97) -- (495.71,147.34);

\path[draw=drawColor,line width= 0.4pt,line join=round,line cap=round] (499.69,144.15) -- (506.05,144.15);

\path[draw=drawColor,line width= 0.4pt,line join=round,line cap=round] (502.87,140.97) -- (502.87,147.34);

\path[draw=drawColor,line width= 0.4pt,line join=round,line cap=round] (506.84,144.15) -- (513.21,144.15);

\path[draw=drawColor,line width= 0.4pt,line join=round,line cap=round] (510.02,140.97) -- (510.02,147.34);

\path[draw=drawColor,line width= 0.4pt,line join=round,line cap=round] (514.00,144.15) -- (520.36,144.15);

\path[draw=drawColor,line width= 0.4pt,line join=round,line cap=round] (517.18,140.97) -- (517.18,147.34);

\path[draw=drawColor,line width= 0.4pt,line join=round,line cap=round] (521.15,144.15) -- (527.52,144.15);

\path[draw=drawColor,line width= 0.4pt,line join=round,line cap=round] (524.34,140.97) -- (524.34,147.34);

\path[draw=drawColor,line width= 0.4pt,line join=round,line cap=round] (528.31,144.15) -- (534.67,144.15);

\path[draw=drawColor,line width= 0.4pt,line join=round,line cap=round] (531.49,140.97) -- (531.49,147.34);

\path[draw=drawColor,line width= 0.4pt,line join=round,line cap=round] (535.47,144.15) -- (541.83,144.15);

\path[draw=drawColor,line width= 0.4pt,line join=round,line cap=round] (538.65,140.97) -- (538.65,147.34);

\path[draw=drawColor,line width= 0.4pt,line join=round,line cap=round] (542.62,144.15) -- (548.99,144.15);

\path[draw=drawColor,line width= 0.4pt,line join=round,line cap=round] (545.80,140.97) -- (545.80,147.34);

\path[draw=drawColor,line width= 0.4pt,line join=round,line cap=round] (549.78,144.15) -- (556.14,144.15);

\path[draw=drawColor,line width= 0.4pt,line join=round,line cap=round] (552.96,140.97) -- (552.96,147.34);

\path[draw=drawColor,line width= 0.4pt,line join=round,line cap=round] (556.93,144.15) -- (563.30,144.15);

\path[draw=drawColor,line width= 0.4pt,line join=round,line cap=round] (560.12,140.97) -- (560.12,147.34);

\path[draw=drawColor,line width= 0.4pt,line join=round,line cap=round] (564.09,144.15) -- (570.45,144.15);

\path[draw=drawColor,line width= 0.4pt,line join=round,line cap=round] (567.27,140.97) -- (567.27,147.34);

\path[draw=drawColor,line width= 0.4pt,line join=round,line cap=round] (571.25,144.15) -- (577.61,144.15);

\path[draw=drawColor,line width= 0.4pt,line join=round,line cap=round] (574.43,140.97) -- (574.43,147.34);
\definecolor[named]{drawColor}{rgb}{0.03,0.32,0.61}

\path[draw=drawColor,line width= 0.4pt,line join=round,line cap=round] (338.28,163.67) --
	(345.44,163.67) --
	(345.44,144.40) --
	(352.59,144.40) --
	(352.59,132.08) --
	(359.75,132.08) --
	(359.75,126.67) --
	(366.90,126.67) --
	(366.90,122.40) --
	(374.06,122.40) --
	(374.06,121.53) --
	(381.22,121.53) --
	(381.22,120.98) --
	(388.37,120.98) --
	(388.37,120.47) --
	(395.53,120.47) --
	(395.53,119.87) --
	(402.68,119.87) --
	(402.68,117.47) --
	(417.00,117.47) --
	(417.00,116.76) --
	(431.31,116.76) --
	(431.31,115.88) --
	(438.46,115.88) --
	(438.46,114.49) --
	(445.62,114.49) --
	(445.62,113.05) --
	(467.09,113.05) --
	(467.09,111.07) --
	(481.40,111.07) --
	(481.40,108.79) --
	(488.56,108.79) --
	(488.56,106.79) --
	(578.16,106.79);

\path[draw=drawColor,line width= 0.4pt,line join=round,line cap=round] (342.25,144.40) -- (348.62,144.40);

\path[draw=drawColor,line width= 0.4pt,line join=round,line cap=round] (345.44,141.22) -- (345.44,147.58);

\path[draw=drawColor,line width= 0.4pt,line join=round,line cap=round] (349.41,132.08) -- (355.77,132.08);

\path[draw=drawColor,line width= 0.4pt,line join=round,line cap=round] (352.59,128.90) -- (352.59,135.26);

\path[draw=drawColor,line width= 0.4pt,line join=round,line cap=round] (356.57,126.67) -- (362.93,126.67);

\path[draw=drawColor,line width= 0.4pt,line join=round,line cap=round] (359.75,123.49) -- (359.75,129.85);

\path[draw=drawColor,line width= 0.4pt,line join=round,line cap=round] (363.72,122.40) -- (370.09,122.40);

\path[draw=drawColor,line width= 0.4pt,line join=round,line cap=round] (366.90,119.22) -- (366.90,125.58);

\path[draw=drawColor,line width= 0.4pt,line join=round,line cap=round] (370.88,121.53) -- (377.24,121.53);

\path[draw=drawColor,line width= 0.4pt,line join=round,line cap=round] (374.06,118.35) -- (374.06,124.71);

\path[draw=drawColor,line width= 0.4pt,line join=round,line cap=round] (378.03,120.98) -- (384.40,120.98);

\path[draw=drawColor,line width= 0.4pt,line join=round,line cap=round] (381.22,117.80) -- (381.22,124.16);

\path[draw=drawColor,line width= 0.4pt,line join=round,line cap=round] (385.19,120.47) -- (391.55,120.47);

\path[draw=drawColor,line width= 0.4pt,line join=round,line cap=round] (388.37,117.28) -- (388.37,123.65);

\path[draw=drawColor,line width= 0.4pt,line join=round,line cap=round] (392.35,119.87) -- (398.71,119.87);

\path[draw=drawColor,line width= 0.4pt,line join=round,line cap=round] (395.53,116.68) -- (395.53,123.05);

\path[draw=drawColor,line width= 0.4pt,line join=round,line cap=round] (399.50,117.47) -- (405.87,117.47);

\path[draw=drawColor,line width= 0.4pt,line join=round,line cap=round] (402.68,114.29) -- (402.68,120.65);

\path[draw=drawColor,line width= 0.4pt,line join=round,line cap=round] (406.66,117.47) -- (413.02,117.47);

\path[draw=drawColor,line width= 0.4pt,line join=round,line cap=round] (409.84,114.29) -- (409.84,120.65);

\path[draw=drawColor,line width= 0.4pt,line join=round,line cap=round] (413.81,116.76) -- (420.18,116.76);

\path[draw=drawColor,line width= 0.4pt,line join=round,line cap=round] (417.00,113.58) -- (417.00,119.94);

\path[draw=drawColor,line width= 0.4pt,line join=round,line cap=round] (420.97,116.76) -- (427.33,116.76);

\path[draw=drawColor,line width= 0.4pt,line join=round,line cap=round] (424.15,113.58) -- (424.15,119.94);

\path[draw=drawColor,line width= 0.4pt,line join=round,line cap=round] (428.13,115.88) -- (434.49,115.88);

\path[draw=drawColor,line width= 0.4pt,line join=round,line cap=round] (431.31,112.70) -- (431.31,119.06);

\path[draw=drawColor,line width= 0.4pt,line join=round,line cap=round] (435.28,114.49) -- (441.65,114.49);

\path[draw=drawColor,line width= 0.4pt,line join=round,line cap=round] (438.46,111.30) -- (438.46,117.67);

\path[draw=drawColor,line width= 0.4pt,line join=round,line cap=round] (442.44,113.05) -- (448.80,113.05);

\path[draw=drawColor,line width= 0.4pt,line join=round,line cap=round] (445.62,109.87) -- (445.62,116.23);

\path[draw=drawColor,line width= 0.4pt,line join=round,line cap=round] (449.59,113.05) -- (455.96,113.05);

\path[draw=drawColor,line width= 0.4pt,line join=round,line cap=round] (452.78,109.87) -- (452.78,116.23);

\path[draw=drawColor,line width= 0.4pt,line join=round,line cap=round] (456.75,113.05) -- (463.11,113.05);

\path[draw=drawColor,line width= 0.4pt,line join=round,line cap=round] (459.93,109.87) -- (459.93,116.23);

\path[draw=drawColor,line width= 0.4pt,line join=round,line cap=round] (463.91,111.07) -- (470.27,111.07);

\path[draw=drawColor,line width= 0.4pt,line join=round,line cap=round] (467.09,107.89) -- (467.09,114.25);

\path[draw=drawColor,line width= 0.4pt,line join=round,line cap=round] (471.06,111.07) -- (477.43,111.07);

\path[draw=drawColor,line width= 0.4pt,line join=round,line cap=round] (474.24,107.89) -- (474.24,114.25);

\path[draw=drawColor,line width= 0.4pt,line join=round,line cap=round] (478.22,108.79) -- (484.58,108.79);

\path[draw=drawColor,line width= 0.4pt,line join=round,line cap=round] (481.40,105.61) -- (481.40,111.97);

\path[draw=drawColor,line width= 0.4pt,line join=round,line cap=round] (485.37,106.79) -- (491.74,106.79);

\path[draw=drawColor,line width= 0.4pt,line join=round,line cap=round] (488.56,103.61) -- (488.56,109.97);

\path[draw=drawColor,line width= 0.4pt,line join=round,line cap=round] (492.53,106.79) -- (498.89,106.79);

\path[draw=drawColor,line width= 0.4pt,line join=round,line cap=round] (495.71,103.61) -- (495.71,109.97);

\path[draw=drawColor,line width= 0.4pt,line join=round,line cap=round] (499.69,106.79) -- (506.05,106.79);

\path[draw=drawColor,line width= 0.4pt,line join=round,line cap=round] (502.87,103.61) -- (502.87,109.97);

\path[draw=drawColor,line width= 0.4pt,line join=round,line cap=round] (506.84,106.79) -- (513.21,106.79);

\path[draw=drawColor,line width= 0.4pt,line join=round,line cap=round] (510.02,103.61) -- (510.02,109.97);

\path[draw=drawColor,line width= 0.4pt,line join=round,line cap=round] (514.00,106.79) -- (520.36,106.79);

\path[draw=drawColor,line width= 0.4pt,line join=round,line cap=round] (517.18,103.61) -- (517.18,109.97);

\path[draw=drawColor,line width= 0.4pt,line join=round,line cap=round] (521.15,106.79) -- (527.52,106.79);

\path[draw=drawColor,line width= 0.4pt,line join=round,line cap=round] (524.34,103.61) -- (524.34,109.97);

\path[draw=drawColor,line width= 0.4pt,line join=round,line cap=round] (528.31,106.79) -- (534.67,106.79);

\path[draw=drawColor,line width= 0.4pt,line join=round,line cap=round] (531.49,103.61) -- (531.49,109.97);

\path[draw=drawColor,line width= 0.4pt,line join=round,line cap=round] (535.47,106.79) -- (541.83,106.79);

\path[draw=drawColor,line width= 0.4pt,line join=round,line cap=round] (538.65,103.61) -- (538.65,109.97);

\path[draw=drawColor,line width= 0.4pt,line join=round,line cap=round] (542.62,106.79) -- (548.99,106.79);

\path[draw=drawColor,line width= 0.4pt,line join=round,line cap=round] (545.80,103.61) -- (545.80,109.97);

\path[draw=drawColor,line width= 0.4pt,line join=round,line cap=round] (549.78,106.79) -- (556.14,106.79);

\path[draw=drawColor,line width= 0.4pt,line join=round,line cap=round] (552.96,103.61) -- (552.96,109.97);

\path[draw=drawColor,line width= 0.4pt,line join=round,line cap=round] (556.93,106.79) -- (563.30,106.79);

\path[draw=drawColor,line width= 0.4pt,line join=round,line cap=round] (560.12,103.61) -- (560.12,109.97);

\path[draw=drawColor,line width= 0.4pt,line join=round,line cap=round] (564.09,106.79) -- (570.45,106.79);

\path[draw=drawColor,line width= 0.4pt,line join=round,line cap=round] (567.27,103.61) -- (567.27,109.97);

\path[draw=drawColor,line width= 0.4pt,line join=round,line cap=round] (571.25,106.79) -- (577.61,106.79);

\path[draw=drawColor,line width= 0.4pt,line join=round,line cap=round] (574.43,103.61) -- (574.43,109.97);
\end{scope}
\begin{scope}
\path[clip] (289.08,  0.00) rectangle (578.16,216.81);
\definecolor[named]{drawColor}{rgb}{0.00,0.00,0.00}

\node[text=drawColor,anchor=base,inner sep=0pt, outer sep=0pt, scale=  1.00] at (445.62, 15.60) {Time (Years)};
\end{scope}
\end{tikzpicture}
}	
	\label{fig:surv2}
\end{figure}
\newpage

Our key hypotheses relate to the effect of compliance and sanction reciprocity. We incorporate these variables in the last column of Table \ref{tab:regResults}, and here we find that target states comply much more quickly to sanctions sent from countries with whom they have a strong history of reciprocal compliance relative to others in the network. On the left side of Figure \ref{fig:surv3}, we can see that after just five years, the probability of non-compliance in sanction cases where target and sender states have a history of reciprocal compliance is approximately 40\%, compared to about 95\% when this history does not exist between senders and receivers. Thus the compliance reciprocity variable tells a story of how friendly reciprocal relations in the past can explain future friendly interactions. 

\newpage
\begin{figure}[ht]
	\centering
	\caption{Survival probabilities predicted by network level covariates. Grey designates the scenario where the number of senders is set to its minimum value and black the scenario where it is set to its high value high value (90th percentile).}
	% \includegraphics[width=1\textwidth]{oNet}
	\resizebox{1\textwidth}{!}{% Created by tikzDevice version 0.6.2 on 2014-03-22 08:39:18
% !TEX encoding = UTF-8 Unicode
\begin{tikzpicture}[x=1pt,y=1pt]
\definecolor[named]{drawColor}{rgb}{0.00,0.00,0.00}
\definecolor[named]{fillColor}{rgb}{1.00,1.00,1.00}
\fill[color=fillColor,fill opacity=0.00,] (0,0) rectangle (578.16,216.81);
\begin{scope}
\path[clip] (  0.00,  0.00) rectangle (578.16,216.81);
\definecolor[named]{drawColor}{rgb}{0.00,0.00,0.00}

\draw[color=drawColor,line cap=round,line join=round,fill opacity=0.00,] ( 49.20, 61.20) -- (263.88, 61.20);

\draw[color=drawColor,line cap=round,line join=round,fill opacity=0.00,] ( 49.20, 61.20) -- ( 49.20, 55.20);

\draw[color=drawColor,line cap=round,line join=round,fill opacity=0.00,] ( 84.98, 61.20) -- ( 84.98, 55.20);

\draw[color=drawColor,line cap=round,line join=round,fill opacity=0.00,] (120.76, 61.20) -- (120.76, 55.20);

\draw[color=drawColor,line cap=round,line join=round,fill opacity=0.00,] (156.54, 61.20) -- (156.54, 55.20);

\draw[color=drawColor,line cap=round,line join=round,fill opacity=0.00,] (192.32, 61.20) -- (192.32, 55.20);

\draw[color=drawColor,line cap=round,line join=round,fill opacity=0.00,] (228.10, 61.20) -- (228.10, 55.20);

\draw[color=drawColor,line cap=round,line join=round,fill opacity=0.00,] (263.88, 61.20) -- (263.88, 55.20);

\node[color=drawColor,anchor=base,inner sep=0pt, outer sep=0pt, scale=  1.00] at ( 49.20, 37.20) {0};

\node[color=drawColor,anchor=base,inner sep=0pt, outer sep=0pt, scale=  1.00] at ( 84.98, 37.20) {5};

\node[color=drawColor,anchor=base,inner sep=0pt, outer sep=0pt, scale=  1.00] at (120.76, 37.20) {10};

\node[color=drawColor,anchor=base,inner sep=0pt, outer sep=0pt, scale=  1.00] at (156.54, 37.20) {15};

\node[color=drawColor,anchor=base,inner sep=0pt, outer sep=0pt, scale=  1.00] at (192.32, 37.20) {20};

\node[color=drawColor,anchor=base,inner sep=0pt, outer sep=0pt, scale=  1.00] at (228.10, 37.20) {25};

\node[color=drawColor,anchor=base,inner sep=0pt, outer sep=0pt, scale=  1.00] at (263.88, 37.20) {30};

\draw[color=drawColor,line cap=round,line join=round,fill opacity=0.00,] ( 49.20, 65.14) -- ( 49.20,163.67);

\draw[color=drawColor,line cap=round,line join=round,fill opacity=0.00,] ( 49.20, 65.14) -- ( 43.20, 65.14);

\draw[color=drawColor,line cap=round,line join=round,fill opacity=0.00,] ( 49.20, 81.56) -- ( 43.20, 81.56);

\draw[color=drawColor,line cap=round,line join=round,fill opacity=0.00,] ( 49.20, 97.98) -- ( 43.20, 97.98);

\draw[color=drawColor,line cap=round,line join=round,fill opacity=0.00,] ( 49.20,114.40) -- ( 43.20,114.40);

\draw[color=drawColor,line cap=round,line join=round,fill opacity=0.00,] ( 49.20,130.83) -- ( 43.20,130.83);

\draw[color=drawColor,line cap=round,line join=round,fill opacity=0.00,] ( 49.20,147.25) -- ( 43.20,147.25);

\draw[color=drawColor,line cap=round,line join=round,fill opacity=0.00,] ( 49.20,163.67) -- ( 43.20,163.67);

\node[color=drawColor,anchor=base east,inner sep=0pt, outer sep=0pt, scale=  1.00] at ( 37.20, 61.70) {0.4};

\node[color=drawColor,anchor=base east,inner sep=0pt, outer sep=0pt, scale=  1.00] at ( 37.20, 78.12) {0.5};

\node[color=drawColor,anchor=base east,inner sep=0pt, outer sep=0pt, scale=  1.00] at ( 37.20, 94.54) {0.6};

\node[color=drawColor,anchor=base east,inner sep=0pt, outer sep=0pt, scale=  1.00] at ( 37.20,110.96) {0.7};

\node[color=drawColor,anchor=base east,inner sep=0pt, outer sep=0pt, scale=  1.00] at ( 37.20,127.38) {0.8};

\node[color=drawColor,anchor=base east,inner sep=0pt, outer sep=0pt, scale=  1.00] at ( 37.20,143.80) {0.9};

\node[color=drawColor,anchor=base east,inner sep=0pt, outer sep=0pt, scale=  1.00] at ( 37.20,160.23) {1.0};

\draw[color=drawColor,line cap=round,line join=round,fill opacity=0.00,] ( 49.20, 61.20) --
	(263.88, 61.20) --
	(263.88,167.61) --
	( 49.20,167.61) --
	( 49.20, 61.20);
\end{scope}
\begin{scope}
\path[clip] (  0.00,  0.00) rectangle (289.08,216.81);
\definecolor[named]{drawColor}{rgb}{0.00,0.00,0.00}

\node[color=drawColor,anchor=base,inner sep=0pt, outer sep=0pt, scale=  1.20] at (156.54,188.07) {\bfseries Sanc. Sent$_{j,t-1}$};
\end{scope}
\begin{scope}
\path[clip] ( 49.20, 61.20) rectangle (263.88,167.61);
\definecolor[named]{drawColor}{rgb}{0.65,0.06,0.08}

\draw[color=drawColor,line cap=round,line join=round,fill opacity=0.00,] ( 49.20,163.67) --
	( 56.36,163.67) --
	( 56.36,155.80) --
	( 63.51,155.80) --
	( 63.51,149.03) --
	( 70.67,149.03) --
	( 70.67,144.29) --
	( 77.82,144.29) --
	( 77.82,143.12) --
	( 84.98,143.12) --
	( 84.98,142.02) --
	( 92.14,142.02) --
	( 92.14,141.56) --
	( 99.29,141.56) --
	( 99.29,141.00) --
	(113.60,141.00) --
	(113.60,140.32) --
	(149.38,140.32) --
	(149.38,139.95) --
	(156.54,139.95) --
	(156.54,139.66) --
	(464.25,139.66) --
	(464.25,139.66);

\draw[color=drawColor,line cap=round,line join=round,fill opacity=0.00,] ( 53.17,155.80) -- ( 59.54,155.80);

\draw[color=drawColor,line cap=round,line join=round,fill opacity=0.00,] ( 56.36,152.61) -- ( 56.36,158.98);

\draw[color=drawColor,line cap=round,line join=round,fill opacity=0.00,] ( 60.33,149.03) -- ( 66.69,149.03);

\draw[color=drawColor,line cap=round,line join=round,fill opacity=0.00,] ( 63.51,145.84) -- ( 63.51,152.21);

\draw[color=drawColor,line cap=round,line join=round,fill opacity=0.00,] ( 67.49,144.29) -- ( 73.85,144.29);

\draw[color=drawColor,line cap=round,line join=round,fill opacity=0.00,] ( 70.67,141.11) -- ( 70.67,147.47);

\draw[color=drawColor,line cap=round,line join=round,fill opacity=0.00,] ( 74.64,143.12) -- ( 81.01,143.12);

\draw[color=drawColor,line cap=round,line join=round,fill opacity=0.00,] ( 77.82,139.94) -- ( 77.82,146.30);

\draw[color=drawColor,line cap=round,line join=round,fill opacity=0.00,] ( 81.80,142.02) -- ( 88.16,142.02);

\draw[color=drawColor,line cap=round,line join=round,fill opacity=0.00,] ( 84.98,138.84) -- ( 84.98,145.21);

\draw[color=drawColor,line cap=round,line join=round,fill opacity=0.00,] ( 88.95,141.56) -- ( 95.32,141.56);

\draw[color=drawColor,line cap=round,line join=round,fill opacity=0.00,] ( 92.14,138.38) -- ( 92.14,144.74);

\draw[color=drawColor,line cap=round,line join=round,fill opacity=0.00,] ( 96.11,141.00) -- (102.47,141.00);

\draw[color=drawColor,line cap=round,line join=round,fill opacity=0.00,] ( 99.29,137.82) -- ( 99.29,144.18);

\draw[color=drawColor,line cap=round,line join=round,fill opacity=0.00,] (103.27,141.00) -- (109.63,141.00);

\draw[color=drawColor,line cap=round,line join=round,fill opacity=0.00,] (106.45,137.82) -- (106.45,144.18);

\draw[color=drawColor,line cap=round,line join=round,fill opacity=0.00,] (110.42,140.32) -- (116.79,140.32);

\draw[color=drawColor,line cap=round,line join=round,fill opacity=0.00,] (113.60,137.14) -- (113.60,143.51);

\draw[color=drawColor,line cap=round,line join=round,fill opacity=0.00,] (117.58,140.32) -- (123.94,140.32);

\draw[color=drawColor,line cap=round,line join=round,fill opacity=0.00,] (120.76,137.14) -- (120.76,143.51);

\draw[color=drawColor,line cap=round,line join=round,fill opacity=0.00,] (124.73,140.32) -- (131.10,140.32);

\draw[color=drawColor,line cap=round,line join=round,fill opacity=0.00,] (127.92,137.14) -- (127.92,143.51);

\draw[color=drawColor,line cap=round,line join=round,fill opacity=0.00,] (131.89,140.32) -- (138.25,140.32);

\draw[color=drawColor,line cap=round,line join=round,fill opacity=0.00,] (135.07,137.14) -- (135.07,143.51);

\draw[color=drawColor,line cap=round,line join=round,fill opacity=0.00,] (139.05,140.32) -- (145.41,140.32);

\draw[color=drawColor,line cap=round,line join=round,fill opacity=0.00,] (142.23,137.14) -- (142.23,143.51);

\draw[color=drawColor,line cap=round,line join=round,fill opacity=0.00,] (146.20,139.95) -- (152.57,139.95);

\draw[color=drawColor,line cap=round,line join=round,fill opacity=0.00,] (149.38,136.76) -- (149.38,143.13);

\draw[color=drawColor,line cap=round,line join=round,fill opacity=0.00,] (153.36,139.66) -- (159.72,139.66);

\draw[color=drawColor,line cap=round,line join=round,fill opacity=0.00,] (156.54,136.48) -- (156.54,142.84);

\draw[color=drawColor,line cap=round,line join=round,fill opacity=0.00,] (160.51,139.66) -- (166.88,139.66);

\draw[color=drawColor,line cap=round,line join=round,fill opacity=0.00,] (163.70,136.48) -- (163.70,142.84);

\draw[color=drawColor,line cap=round,line join=round,fill opacity=0.00,] (167.67,139.66) -- (174.03,139.66);

\draw[color=drawColor,line cap=round,line join=round,fill opacity=0.00,] (170.85,136.48) -- (170.85,142.84);

\draw[color=drawColor,line cap=round,line join=round,fill opacity=0.00,] (174.83,139.66) -- (181.19,139.66);

\draw[color=drawColor,line cap=round,line join=round,fill opacity=0.00,] (178.01,136.48) -- (178.01,142.84);

\draw[color=drawColor,line cap=round,line join=round,fill opacity=0.00,] (181.98,139.66) -- (188.35,139.66);

\draw[color=drawColor,line cap=round,line join=round,fill opacity=0.00,] (185.16,136.48) -- (185.16,142.84);

\draw[color=drawColor,line cap=round,line join=round,fill opacity=0.00,] (189.14,139.66) -- (195.50,139.66);

\draw[color=drawColor,line cap=round,line join=round,fill opacity=0.00,] (192.32,136.48) -- (192.32,142.84);

\draw[color=drawColor,line cap=round,line join=round,fill opacity=0.00,] (196.29,139.66) -- (202.66,139.66);

\draw[color=drawColor,line cap=round,line join=round,fill opacity=0.00,] (199.48,136.48) -- (199.48,142.84);

\draw[color=drawColor,line cap=round,line join=round,fill opacity=0.00,] (203.45,139.66) -- (209.81,139.66);

\draw[color=drawColor,line cap=round,line join=round,fill opacity=0.00,] (206.63,136.48) -- (206.63,142.84);

\draw[color=drawColor,line cap=round,line join=round,fill opacity=0.00,] (210.61,139.66) -- (216.97,139.66);

\draw[color=drawColor,line cap=round,line join=round,fill opacity=0.00,] (213.79,136.48) -- (213.79,142.84);

\draw[color=drawColor,line cap=round,line join=round,fill opacity=0.00,] (217.76,139.66) -- (224.13,139.66);

\draw[color=drawColor,line cap=round,line join=round,fill opacity=0.00,] (220.94,136.48) -- (220.94,142.84);

\draw[color=drawColor,line cap=round,line join=round,fill opacity=0.00,] (224.92,139.66) -- (231.28,139.66);

\draw[color=drawColor,line cap=round,line join=round,fill opacity=0.00,] (228.10,136.48) -- (228.10,142.84);

\draw[color=drawColor,line cap=round,line join=round,fill opacity=0.00,] (232.07,139.66) -- (238.44,139.66);

\draw[color=drawColor,line cap=round,line join=round,fill opacity=0.00,] (235.26,136.48) -- (235.26,142.84);

\draw[color=drawColor,line cap=round,line join=round,fill opacity=0.00,] (239.23,139.66) -- (245.59,139.66);

\draw[color=drawColor,line cap=round,line join=round,fill opacity=0.00,] (242.41,136.48) -- (242.41,142.84);

\draw[color=drawColor,line cap=round,line join=round,fill opacity=0.00,] (246.39,139.66) -- (252.75,139.66);

\draw[color=drawColor,line cap=round,line join=round,fill opacity=0.00,] (249.57,136.48) -- (249.57,142.84);

\draw[color=drawColor,line cap=round,line join=round,fill opacity=0.00,] (253.54,139.66) -- (259.91,139.66);

\draw[color=drawColor,line cap=round,line join=round,fill opacity=0.00,] (256.72,136.48) -- (256.72,142.84);

\draw[color=drawColor,line cap=round,line join=round,fill opacity=0.00,] (260.70,139.66) -- (267.06,139.66);

\draw[color=drawColor,line cap=round,line join=round,fill opacity=0.00,] (263.88,136.48) -- (263.88,142.84);

\draw[color=drawColor,line cap=round,line join=round,fill opacity=0.00,] (267.85,139.66) -- (274.22,139.66);

\draw[color=drawColor,line cap=round,line join=round,fill opacity=0.00,] (271.04,136.48) -- (271.04,142.84);

\draw[color=drawColor,line cap=round,line join=round,fill opacity=0.00,] (275.01,139.66) -- (281.37,139.66);

\draw[color=drawColor,line cap=round,line join=round,fill opacity=0.00,] (278.19,136.48) -- (278.19,142.84);

\draw[color=drawColor,line cap=round,line join=round,fill opacity=0.00,] (282.17,139.66) -- (288.53,139.66);

\draw[color=drawColor,line cap=round,line join=round,fill opacity=0.00,] (285.35,136.48) -- (285.35,142.84);

\draw[color=drawColor,line cap=round,line join=round,fill opacity=0.00,] (289.32,139.66) -- (295.69,139.66);

\draw[color=drawColor,line cap=round,line join=round,fill opacity=0.00,] (292.50,136.48) -- (292.50,142.84);

\draw[color=drawColor,line cap=round,line join=round,fill opacity=0.00,] (296.48,139.66) -- (302.84,139.66);

\draw[color=drawColor,line cap=round,line join=round,fill opacity=0.00,] (299.66,136.48) -- (299.66,142.84);

\draw[color=drawColor,line cap=round,line join=round,fill opacity=0.00,] (303.63,139.66) -- (310.00,139.66);

\draw[color=drawColor,line cap=round,line join=round,fill opacity=0.00,] (306.82,136.48) -- (306.82,142.84);

\draw[color=drawColor,line cap=round,line join=round,fill opacity=0.00,] (310.79,139.66) -- (317.15,139.66);

\draw[color=drawColor,line cap=round,line join=round,fill opacity=0.00,] (313.97,136.48) -- (313.97,142.84);

\draw[color=drawColor,line cap=round,line join=round,fill opacity=0.00,] (317.95,139.66) -- (324.31,139.66);

\draw[color=drawColor,line cap=round,line join=round,fill opacity=0.00,] (321.13,136.48) -- (321.13,142.84);

\draw[color=drawColor,line cap=round,line join=round,fill opacity=0.00,] (325.10,139.66) -- (331.47,139.66);

\draw[color=drawColor,line cap=round,line join=round,fill opacity=0.00,] (328.28,136.48) -- (328.28,142.84);

\draw[color=drawColor,line cap=round,line join=round,fill opacity=0.00,] (332.26,139.66) -- (338.62,139.66);

\draw[color=drawColor,line cap=round,line join=round,fill opacity=0.00,] (335.44,136.48) -- (335.44,142.84);

\draw[color=drawColor,line cap=round,line join=round,fill opacity=0.00,] (339.41,139.66) -- (345.78,139.66);

\draw[color=drawColor,line cap=round,line join=round,fill opacity=0.00,] (342.60,136.48) -- (342.60,142.84);

\draw[color=drawColor,line cap=round,line join=round,fill opacity=0.00,] (346.57,139.66) -- (352.93,139.66);

\draw[color=drawColor,line cap=round,line join=round,fill opacity=0.00,] (349.75,136.48) -- (349.75,142.84);

\draw[color=drawColor,line cap=round,line join=round,fill opacity=0.00,] (353.73,139.66) -- (360.09,139.66);

\draw[color=drawColor,line cap=round,line join=round,fill opacity=0.00,] (356.91,136.48) -- (356.91,142.84);

\draw[color=drawColor,line cap=round,line join=round,fill opacity=0.00,] (360.88,139.66) -- (367.25,139.66);

\draw[color=drawColor,line cap=round,line join=round,fill opacity=0.00,] (364.06,136.48) -- (364.06,142.84);

\draw[color=drawColor,line cap=round,line join=round,fill opacity=0.00,] (368.04,139.66) -- (374.40,139.66);

\draw[color=drawColor,line cap=round,line join=round,fill opacity=0.00,] (371.22,136.48) -- (371.22,142.84);

\draw[color=drawColor,line cap=round,line join=round,fill opacity=0.00,] (375.19,139.66) -- (381.56,139.66);

\draw[color=drawColor,line cap=round,line join=round,fill opacity=0.00,] (378.38,136.48) -- (378.38,142.84);

\draw[color=drawColor,line cap=round,line join=round,fill opacity=0.00,] (382.35,139.66) -- (388.71,139.66);

\draw[color=drawColor,line cap=round,line join=round,fill opacity=0.00,] (385.53,136.48) -- (385.53,142.84);

\draw[color=drawColor,line cap=round,line join=round,fill opacity=0.00,] (389.51,139.66) -- (395.87,139.66);

\draw[color=drawColor,line cap=round,line join=round,fill opacity=0.00,] (392.69,136.48) -- (392.69,142.84);

\draw[color=drawColor,line cap=round,line join=round,fill opacity=0.00,] (396.66,139.66) -- (403.03,139.66);

\draw[color=drawColor,line cap=round,line join=round,fill opacity=0.00,] (399.84,136.48) -- (399.84,142.84);

\draw[color=drawColor,line cap=round,line join=round,fill opacity=0.00,] (403.82,139.66) -- (410.18,139.66);

\draw[color=drawColor,line cap=round,line join=round,fill opacity=0.00,] (407.00,136.48) -- (407.00,142.84);

\draw[color=drawColor,line cap=round,line join=round,fill opacity=0.00,] (410.97,139.66) -- (417.34,139.66);

\draw[color=drawColor,line cap=round,line join=round,fill opacity=0.00,] (414.16,136.48) -- (414.16,142.84);

\draw[color=drawColor,line cap=round,line join=round,fill opacity=0.00,] (418.13,139.66) -- (424.49,139.66);

\draw[color=drawColor,line cap=round,line join=round,fill opacity=0.00,] (421.31,136.48) -- (421.31,142.84);

\draw[color=drawColor,line cap=round,line join=round,fill opacity=0.00,] (425.29,139.66) -- (431.65,139.66);

\draw[color=drawColor,line cap=round,line join=round,fill opacity=0.00,] (428.47,136.48) -- (428.47,142.84);

\draw[color=drawColor,line cap=round,line join=round,fill opacity=0.00,] (432.44,139.66) -- (438.81,139.66);

\draw[color=drawColor,line cap=round,line join=round,fill opacity=0.00,] (435.62,136.48) -- (435.62,142.84);

\draw[color=drawColor,line cap=round,line join=round,fill opacity=0.00,] (439.60,139.66) -- (445.96,139.66);

\draw[color=drawColor,line cap=round,line join=round,fill opacity=0.00,] (442.78,136.48) -- (442.78,142.84);

\draw[color=drawColor,line cap=round,line join=round,fill opacity=0.00,] (446.75,139.66) -- (453.12,139.66);

\draw[color=drawColor,line cap=round,line join=round,fill opacity=0.00,] (449.94,136.48) -- (449.94,142.84);

\draw[color=drawColor,line cap=round,line join=round,fill opacity=0.00,] (453.91,139.66) -- (460.27,139.66);

\draw[color=drawColor,line cap=round,line join=round,fill opacity=0.00,] (457.09,136.48) -- (457.09,142.84);

\draw[color=drawColor,line cap=round,line join=round,fill opacity=0.00,] (461.07,139.66) -- (467.43,139.66);

\draw[color=drawColor,line cap=round,line join=round,fill opacity=0.00,] (464.25,136.48) -- (464.25,142.84);
\definecolor[named]{drawColor}{rgb}{0.03,0.32,0.61}

\draw[color=drawColor,line cap=round,line join=round,fill opacity=0.00,] ( 49.20,163.67) --
	( 56.36,163.67) --
	( 56.36,140.57) --
	( 63.51,140.57) --
	( 63.51,122.56) --
	( 70.67,122.56) --
	( 70.67,110.93) --
	( 77.82,110.93) --
	( 77.82,108.18) --
	( 84.98,108.18) --
	( 84.98,105.64) --
	( 92.14,105.64) --
	( 92.14,104.58) --
	( 99.29,104.58) --
	( 99.29,103.30) --
	(113.60,103.30) --
	(113.60,101.78) --
	(149.38,101.78) --
	(149.38,100.94) --
	(156.54,100.94) --
	(156.54,100.30) --
	(464.25,100.30) --
	(464.25,100.30);

\draw[color=drawColor,line cap=round,line join=round,fill opacity=0.00,] ( 53.17,140.57) -- ( 59.54,140.57);

\draw[color=drawColor,line cap=round,line join=round,fill opacity=0.00,] ( 56.36,137.39) -- ( 56.36,143.76);

\draw[color=drawColor,line cap=round,line join=round,fill opacity=0.00,] ( 60.33,122.56) -- ( 66.69,122.56);

\draw[color=drawColor,line cap=round,line join=round,fill opacity=0.00,] ( 63.51,119.38) -- ( 63.51,125.74);

\draw[color=drawColor,line cap=round,line join=round,fill opacity=0.00,] ( 67.49,110.93) -- ( 73.85,110.93);

\draw[color=drawColor,line cap=round,line join=round,fill opacity=0.00,] ( 70.67,107.75) -- ( 70.67,114.11);

\draw[color=drawColor,line cap=round,line join=round,fill opacity=0.00,] ( 74.64,108.18) -- ( 81.01,108.18);

\draw[color=drawColor,line cap=round,line join=round,fill opacity=0.00,] ( 77.82,104.99) -- ( 77.82,111.36);

\draw[color=drawColor,line cap=round,line join=round,fill opacity=0.00,] ( 81.80,105.64) -- ( 88.16,105.64);

\draw[color=drawColor,line cap=round,line join=round,fill opacity=0.00,] ( 84.98,102.46) -- ( 84.98,108.82);

\draw[color=drawColor,line cap=round,line join=round,fill opacity=0.00,] ( 88.95,104.58) -- ( 95.32,104.58);

\draw[color=drawColor,line cap=round,line join=round,fill opacity=0.00,] ( 92.14,101.40) -- ( 92.14,107.76);

\draw[color=drawColor,line cap=round,line join=round,fill opacity=0.00,] ( 96.11,103.30) -- (102.47,103.30);

\draw[color=drawColor,line cap=round,line join=round,fill opacity=0.00,] ( 99.29,100.12) -- ( 99.29,106.48);

\draw[color=drawColor,line cap=round,line join=round,fill opacity=0.00,] (103.27,103.30) -- (109.63,103.30);

\draw[color=drawColor,line cap=round,line join=round,fill opacity=0.00,] (106.45,100.12) -- (106.45,106.48);

\draw[color=drawColor,line cap=round,line join=round,fill opacity=0.00,] (110.42,101.78) -- (116.79,101.78);

\draw[color=drawColor,line cap=round,line join=round,fill opacity=0.00,] (113.60, 98.59) -- (113.60,104.96);

\draw[color=drawColor,line cap=round,line join=round,fill opacity=0.00,] (117.58,101.78) -- (123.94,101.78);

\draw[color=drawColor,line cap=round,line join=round,fill opacity=0.00,] (120.76, 98.59) -- (120.76,104.96);

\draw[color=drawColor,line cap=round,line join=round,fill opacity=0.00,] (124.73,101.78) -- (131.10,101.78);

\draw[color=drawColor,line cap=round,line join=round,fill opacity=0.00,] (127.92, 98.59) -- (127.92,104.96);

\draw[color=drawColor,line cap=round,line join=round,fill opacity=0.00,] (131.89,101.78) -- (138.25,101.78);

\draw[color=drawColor,line cap=round,line join=round,fill opacity=0.00,] (135.07, 98.59) -- (135.07,104.96);

\draw[color=drawColor,line cap=round,line join=round,fill opacity=0.00,] (139.05,101.78) -- (145.41,101.78);

\draw[color=drawColor,line cap=round,line join=round,fill opacity=0.00,] (142.23, 98.59) -- (142.23,104.96);

\draw[color=drawColor,line cap=round,line join=round,fill opacity=0.00,] (146.20,100.94) -- (152.57,100.94);

\draw[color=drawColor,line cap=round,line join=round,fill opacity=0.00,] (149.38, 97.75) -- (149.38,104.12);

\draw[color=drawColor,line cap=round,line join=round,fill opacity=0.00,] (153.36,100.30) -- (159.72,100.30);

\draw[color=drawColor,line cap=round,line join=round,fill opacity=0.00,] (156.54, 97.12) -- (156.54,103.48);

\draw[color=drawColor,line cap=round,line join=round,fill opacity=0.00,] (160.51,100.30) -- (166.88,100.30);

\draw[color=drawColor,line cap=round,line join=round,fill opacity=0.00,] (163.70, 97.12) -- (163.70,103.48);

\draw[color=drawColor,line cap=round,line join=round,fill opacity=0.00,] (167.67,100.30) -- (174.03,100.30);

\draw[color=drawColor,line cap=round,line join=round,fill opacity=0.00,] (170.85, 97.12) -- (170.85,103.48);

\draw[color=drawColor,line cap=round,line join=round,fill opacity=0.00,] (174.83,100.30) -- (181.19,100.30);

\draw[color=drawColor,line cap=round,line join=round,fill opacity=0.00,] (178.01, 97.12) -- (178.01,103.48);

\draw[color=drawColor,line cap=round,line join=round,fill opacity=0.00,] (181.98,100.30) -- (188.35,100.30);

\draw[color=drawColor,line cap=round,line join=round,fill opacity=0.00,] (185.16, 97.12) -- (185.16,103.48);

\draw[color=drawColor,line cap=round,line join=round,fill opacity=0.00,] (189.14,100.30) -- (195.50,100.30);

\draw[color=drawColor,line cap=round,line join=round,fill opacity=0.00,] (192.32, 97.12) -- (192.32,103.48);

\draw[color=drawColor,line cap=round,line join=round,fill opacity=0.00,] (196.29,100.30) -- (202.66,100.30);

\draw[color=drawColor,line cap=round,line join=round,fill opacity=0.00,] (199.48, 97.12) -- (199.48,103.48);

\draw[color=drawColor,line cap=round,line join=round,fill opacity=0.00,] (203.45,100.30) -- (209.81,100.30);

\draw[color=drawColor,line cap=round,line join=round,fill opacity=0.00,] (206.63, 97.12) -- (206.63,103.48);

\draw[color=drawColor,line cap=round,line join=round,fill opacity=0.00,] (210.61,100.30) -- (216.97,100.30);

\draw[color=drawColor,line cap=round,line join=round,fill opacity=0.00,] (213.79, 97.12) -- (213.79,103.48);

\draw[color=drawColor,line cap=round,line join=round,fill opacity=0.00,] (217.76,100.30) -- (224.13,100.30);

\draw[color=drawColor,line cap=round,line join=round,fill opacity=0.00,] (220.94, 97.12) -- (220.94,103.48);

\draw[color=drawColor,line cap=round,line join=round,fill opacity=0.00,] (224.92,100.30) -- (231.28,100.30);

\draw[color=drawColor,line cap=round,line join=round,fill opacity=0.00,] (228.10, 97.12) -- (228.10,103.48);

\draw[color=drawColor,line cap=round,line join=round,fill opacity=0.00,] (232.07,100.30) -- (238.44,100.30);

\draw[color=drawColor,line cap=round,line join=round,fill opacity=0.00,] (235.26, 97.12) -- (235.26,103.48);

\draw[color=drawColor,line cap=round,line join=round,fill opacity=0.00,] (239.23,100.30) -- (245.59,100.30);

\draw[color=drawColor,line cap=round,line join=round,fill opacity=0.00,] (242.41, 97.12) -- (242.41,103.48);

\draw[color=drawColor,line cap=round,line join=round,fill opacity=0.00,] (246.39,100.30) -- (252.75,100.30);

\draw[color=drawColor,line cap=round,line join=round,fill opacity=0.00,] (249.57, 97.12) -- (249.57,103.48);

\draw[color=drawColor,line cap=round,line join=round,fill opacity=0.00,] (253.54,100.30) -- (259.91,100.30);

\draw[color=drawColor,line cap=round,line join=round,fill opacity=0.00,] (256.72, 97.12) -- (256.72,103.48);

\draw[color=drawColor,line cap=round,line join=round,fill opacity=0.00,] (260.70,100.30) -- (267.06,100.30);

\draw[color=drawColor,line cap=round,line join=round,fill opacity=0.00,] (263.88, 97.12) -- (263.88,103.48);

\draw[color=drawColor,line cap=round,line join=round,fill opacity=0.00,] (267.85,100.30) -- (274.22,100.30);

\draw[color=drawColor,line cap=round,line join=round,fill opacity=0.00,] (271.04, 97.12) -- (271.04,103.48);

\draw[color=drawColor,line cap=round,line join=round,fill opacity=0.00,] (275.01,100.30) -- (281.37,100.30);

\draw[color=drawColor,line cap=round,line join=round,fill opacity=0.00,] (278.19, 97.12) -- (278.19,103.48);

\draw[color=drawColor,line cap=round,line join=round,fill opacity=0.00,] (282.17,100.30) -- (288.53,100.30);

\draw[color=drawColor,line cap=round,line join=round,fill opacity=0.00,] (285.35, 97.12) -- (285.35,103.48);

\draw[color=drawColor,line cap=round,line join=round,fill opacity=0.00,] (289.32,100.30) -- (295.69,100.30);

\draw[color=drawColor,line cap=round,line join=round,fill opacity=0.00,] (292.50, 97.12) -- (292.50,103.48);

\draw[color=drawColor,line cap=round,line join=round,fill opacity=0.00,] (296.48,100.30) -- (302.84,100.30);

\draw[color=drawColor,line cap=round,line join=round,fill opacity=0.00,] (299.66, 97.12) -- (299.66,103.48);

\draw[color=drawColor,line cap=round,line join=round,fill opacity=0.00,] (303.63,100.30) -- (310.00,100.30);

\draw[color=drawColor,line cap=round,line join=round,fill opacity=0.00,] (306.82, 97.12) -- (306.82,103.48);

\draw[color=drawColor,line cap=round,line join=round,fill opacity=0.00,] (310.79,100.30) -- (317.15,100.30);

\draw[color=drawColor,line cap=round,line join=round,fill opacity=0.00,] (313.97, 97.12) -- (313.97,103.48);

\draw[color=drawColor,line cap=round,line join=round,fill opacity=0.00,] (317.95,100.30) -- (324.31,100.30);

\draw[color=drawColor,line cap=round,line join=round,fill opacity=0.00,] (321.13, 97.12) -- (321.13,103.48);

\draw[color=drawColor,line cap=round,line join=round,fill opacity=0.00,] (325.10,100.30) -- (331.47,100.30);

\draw[color=drawColor,line cap=round,line join=round,fill opacity=0.00,] (328.28, 97.12) -- (328.28,103.48);

\draw[color=drawColor,line cap=round,line join=round,fill opacity=0.00,] (332.26,100.30) -- (338.62,100.30);

\draw[color=drawColor,line cap=round,line join=round,fill opacity=0.00,] (335.44, 97.12) -- (335.44,103.48);

\draw[color=drawColor,line cap=round,line join=round,fill opacity=0.00,] (339.41,100.30) -- (345.78,100.30);

\draw[color=drawColor,line cap=round,line join=round,fill opacity=0.00,] (342.60, 97.12) -- (342.60,103.48);

\draw[color=drawColor,line cap=round,line join=round,fill opacity=0.00,] (346.57,100.30) -- (352.93,100.30);

\draw[color=drawColor,line cap=round,line join=round,fill opacity=0.00,] (349.75, 97.12) -- (349.75,103.48);

\draw[color=drawColor,line cap=round,line join=round,fill opacity=0.00,] (353.73,100.30) -- (360.09,100.30);

\draw[color=drawColor,line cap=round,line join=round,fill opacity=0.00,] (356.91, 97.12) -- (356.91,103.48);

\draw[color=drawColor,line cap=round,line join=round,fill opacity=0.00,] (360.88,100.30) -- (367.25,100.30);

\draw[color=drawColor,line cap=round,line join=round,fill opacity=0.00,] (364.06, 97.12) -- (364.06,103.48);

\draw[color=drawColor,line cap=round,line join=round,fill opacity=0.00,] (368.04,100.30) -- (374.40,100.30);

\draw[color=drawColor,line cap=round,line join=round,fill opacity=0.00,] (371.22, 97.12) -- (371.22,103.48);

\draw[color=drawColor,line cap=round,line join=round,fill opacity=0.00,] (375.19,100.30) -- (381.56,100.30);

\draw[color=drawColor,line cap=round,line join=round,fill opacity=0.00,] (378.38, 97.12) -- (378.38,103.48);

\draw[color=drawColor,line cap=round,line join=round,fill opacity=0.00,] (382.35,100.30) -- (388.71,100.30);

\draw[color=drawColor,line cap=round,line join=round,fill opacity=0.00,] (385.53, 97.12) -- (385.53,103.48);

\draw[color=drawColor,line cap=round,line join=round,fill opacity=0.00,] (389.51,100.30) -- (395.87,100.30);

\draw[color=drawColor,line cap=round,line join=round,fill opacity=0.00,] (392.69, 97.12) -- (392.69,103.48);

\draw[color=drawColor,line cap=round,line join=round,fill opacity=0.00,] (396.66,100.30) -- (403.03,100.30);

\draw[color=drawColor,line cap=round,line join=round,fill opacity=0.00,] (399.84, 97.12) -- (399.84,103.48);

\draw[color=drawColor,line cap=round,line join=round,fill opacity=0.00,] (403.82,100.30) -- (410.18,100.30);

\draw[color=drawColor,line cap=round,line join=round,fill opacity=0.00,] (407.00, 97.12) -- (407.00,103.48);

\draw[color=drawColor,line cap=round,line join=round,fill opacity=0.00,] (410.97,100.30) -- (417.34,100.30);

\draw[color=drawColor,line cap=round,line join=round,fill opacity=0.00,] (414.16, 97.12) -- (414.16,103.48);

\draw[color=drawColor,line cap=round,line join=round,fill opacity=0.00,] (418.13,100.30) -- (424.49,100.30);

\draw[color=drawColor,line cap=round,line join=round,fill opacity=0.00,] (421.31, 97.12) -- (421.31,103.48);

\draw[color=drawColor,line cap=round,line join=round,fill opacity=0.00,] (425.29,100.30) -- (431.65,100.30);

\draw[color=drawColor,line cap=round,line join=round,fill opacity=0.00,] (428.47, 97.12) -- (428.47,103.48);

\draw[color=drawColor,line cap=round,line join=round,fill opacity=0.00,] (432.44,100.30) -- (438.81,100.30);

\draw[color=drawColor,line cap=round,line join=round,fill opacity=0.00,] (435.62, 97.12) -- (435.62,103.48);

\draw[color=drawColor,line cap=round,line join=round,fill opacity=0.00,] (439.60,100.30) -- (445.96,100.30);

\draw[color=drawColor,line cap=round,line join=round,fill opacity=0.00,] (442.78, 97.12) -- (442.78,103.48);

\draw[color=drawColor,line cap=round,line join=round,fill opacity=0.00,] (446.75,100.30) -- (453.12,100.30);

\draw[color=drawColor,line cap=round,line join=round,fill opacity=0.00,] (449.94, 97.12) -- (449.94,103.48);

\draw[color=drawColor,line cap=round,line join=round,fill opacity=0.00,] (453.91,100.30) -- (460.27,100.30);

\draw[color=drawColor,line cap=round,line join=round,fill opacity=0.00,] (457.09, 97.12) -- (457.09,103.48);

\draw[color=drawColor,line cap=round,line join=round,fill opacity=0.00,] (461.07,100.30) -- (467.43,100.30);

\draw[color=drawColor,line cap=round,line join=round,fill opacity=0.00,] (464.25, 97.12) -- (464.25,103.48);
\end{scope}
\begin{scope}
\path[clip] (  0.00,  0.00) rectangle (289.08,216.81);
\definecolor[named]{drawColor}{rgb}{0.00,0.00,0.00}

\node[rotate= 90.00,color=drawColor,anchor=base,inner sep=0pt, outer sep=0pt, scale=  1.00] at ( 13.20,114.41) {Survival Prob.};

\node[color=drawColor,anchor=base,inner sep=0pt, outer sep=0pt, scale=  1.00] at (156.54, 13.20) {Time (Years)};
\end{scope}
\begin{scope}
\path[clip] (338.28, 61.20) rectangle (552.96,167.61);
\end{scope}
\begin{scope}
\path[clip] (  0.00,  0.00) rectangle (578.16,216.81);
\definecolor[named]{drawColor}{rgb}{0.00,0.00,0.00}

\draw[color=drawColor,line cap=round,line join=round,fill opacity=0.00,] (338.28, 61.20) -- (552.96, 61.20);

\draw[color=drawColor,line cap=round,line join=round,fill opacity=0.00,] (338.28, 61.20) -- (338.28, 55.20);

\draw[color=drawColor,line cap=round,line join=round,fill opacity=0.00,] (374.06, 61.20) -- (374.06, 55.20);

\draw[color=drawColor,line cap=round,line join=round,fill opacity=0.00,] (409.84, 61.20) -- (409.84, 55.20);

\draw[color=drawColor,line cap=round,line join=round,fill opacity=0.00,] (445.62, 61.20) -- (445.62, 55.20);

\draw[color=drawColor,line cap=round,line join=round,fill opacity=0.00,] (481.40, 61.20) -- (481.40, 55.20);

\draw[color=drawColor,line cap=round,line join=round,fill opacity=0.00,] (517.18, 61.20) -- (517.18, 55.20);

\draw[color=drawColor,line cap=round,line join=round,fill opacity=0.00,] (552.96, 61.20) -- (552.96, 55.20);

\node[color=drawColor,anchor=base,inner sep=0pt, outer sep=0pt, scale=  1.00] at (338.28, 37.20) {0};

\node[color=drawColor,anchor=base,inner sep=0pt, outer sep=0pt, scale=  1.00] at (374.06, 37.20) {5};

\node[color=drawColor,anchor=base,inner sep=0pt, outer sep=0pt, scale=  1.00] at (409.84, 37.20) {10};

\node[color=drawColor,anchor=base,inner sep=0pt, outer sep=0pt, scale=  1.00] at (445.62, 37.20) {15};

\node[color=drawColor,anchor=base,inner sep=0pt, outer sep=0pt, scale=  1.00] at (481.40, 37.20) {20};

\node[color=drawColor,anchor=base,inner sep=0pt, outer sep=0pt, scale=  1.00] at (517.18, 37.20) {25};

\node[color=drawColor,anchor=base,inner sep=0pt, outer sep=0pt, scale=  1.00] at (552.96, 37.20) {30};

\draw[color=drawColor,line cap=round,line join=round,fill opacity=0.00,] (338.28, 65.14) -- (338.28,163.67);

\draw[color=drawColor,line cap=round,line join=round,fill opacity=0.00,] (338.28, 65.14) -- (332.28, 65.14);

\draw[color=drawColor,line cap=round,line join=round,fill opacity=0.00,] (338.28, 81.56) -- (332.28, 81.56);

\draw[color=drawColor,line cap=round,line join=round,fill opacity=0.00,] (338.28, 97.98) -- (332.28, 97.98);

\draw[color=drawColor,line cap=round,line join=round,fill opacity=0.00,] (338.28,114.40) -- (332.28,114.40);

\draw[color=drawColor,line cap=round,line join=round,fill opacity=0.00,] (338.28,130.83) -- (332.28,130.83);

\draw[color=drawColor,line cap=round,line join=round,fill opacity=0.00,] (338.28,147.25) -- (332.28,147.25);

\draw[color=drawColor,line cap=round,line join=round,fill opacity=0.00,] (338.28,163.67) -- (332.28,163.67);

\node[color=drawColor,anchor=base east,inner sep=0pt, outer sep=0pt, scale=  1.00] at (326.28, 61.70) {0.4};

\node[color=drawColor,anchor=base east,inner sep=0pt, outer sep=0pt, scale=  1.00] at (326.28, 78.12) {0.5};

\node[color=drawColor,anchor=base east,inner sep=0pt, outer sep=0pt, scale=  1.00] at (326.28, 94.54) {0.6};

\node[color=drawColor,anchor=base east,inner sep=0pt, outer sep=0pt, scale=  1.00] at (326.28,110.96) {0.7};

\node[color=drawColor,anchor=base east,inner sep=0pt, outer sep=0pt, scale=  1.00] at (326.28,127.38) {0.8};

\node[color=drawColor,anchor=base east,inner sep=0pt, outer sep=0pt, scale=  1.00] at (326.28,143.80) {0.9};

\node[color=drawColor,anchor=base east,inner sep=0pt, outer sep=0pt, scale=  1.00] at (326.28,160.23) {1.0};

\draw[color=drawColor,line cap=round,line join=round,fill opacity=0.00,] (338.28, 61.20) --
	(552.96, 61.20) --
	(552.96,167.61) --
	(338.28,167.61) --
	(338.28, 61.20);
\end{scope}
\begin{scope}
\path[clip] (289.08,  0.00) rectangle (578.16,216.81);
\definecolor[named]{drawColor}{rgb}{0.00,0.00,0.00}

\node[color=drawColor,anchor=base,inner sep=0pt, outer sep=0pt, scale=  1.20] at (445.62,188.07) {\bfseries Sanc. Rec'd$_{i,t-1}$};
\end{scope}
\begin{scope}
\path[clip] (338.28, 61.20) rectangle (552.96,167.61);
\definecolor[named]{drawColor}{rgb}{0.65,0.06,0.08}

\draw[color=drawColor,line cap=round,line join=round,fill opacity=0.00,] (338.28,163.67) --
	(345.44,163.67) --
	(345.44,153.12) --
	(352.59,153.12) --
	(352.59,144.19) --
	(359.75,144.19) --
	(359.75,138.03) --
	(366.90,138.03) --
	(366.90,136.52) --
	(374.06,136.52) --
	(374.06,135.11) --
	(381.22,135.11) --
	(381.22,134.51) --
	(388.37,134.51) --
	(388.37,133.79) --
	(402.68,133.79) --
	(402.68,132.93) --
	(438.46,132.93) --
	(438.46,132.44) --
	(445.62,132.44) --
	(445.62,132.08) --
	(578.16,132.08);

\draw[color=drawColor,line cap=round,line join=round,fill opacity=0.00,] (342.25,153.12) -- (348.62,153.12);

\draw[color=drawColor,line cap=round,line join=round,fill opacity=0.00,] (345.44,149.94) -- (345.44,156.30);

\draw[color=drawColor,line cap=round,line join=round,fill opacity=0.00,] (349.41,144.19) -- (355.77,144.19);

\draw[color=drawColor,line cap=round,line join=round,fill opacity=0.00,] (352.59,141.01) -- (352.59,147.37);

\draw[color=drawColor,line cap=round,line join=round,fill opacity=0.00,] (356.57,138.03) -- (362.93,138.03);

\draw[color=drawColor,line cap=round,line join=round,fill opacity=0.00,] (359.75,134.85) -- (359.75,141.21);

\draw[color=drawColor,line cap=round,line join=round,fill opacity=0.00,] (363.72,136.52) -- (370.09,136.52);

\draw[color=drawColor,line cap=round,line join=round,fill opacity=0.00,] (366.90,133.34) -- (366.90,139.70);

\draw[color=drawColor,line cap=round,line join=round,fill opacity=0.00,] (370.88,135.11) -- (377.24,135.11);

\draw[color=drawColor,line cap=round,line join=round,fill opacity=0.00,] (374.06,131.93) -- (374.06,138.29);

\draw[color=drawColor,line cap=round,line join=round,fill opacity=0.00,] (378.03,134.51) -- (384.40,134.51);

\draw[color=drawColor,line cap=round,line join=round,fill opacity=0.00,] (381.22,131.33) -- (381.22,137.70);

\draw[color=drawColor,line cap=round,line join=round,fill opacity=0.00,] (385.19,133.79) -- (391.55,133.79);

\draw[color=drawColor,line cap=round,line join=round,fill opacity=0.00,] (388.37,130.61) -- (388.37,136.97);

\draw[color=drawColor,line cap=round,line join=round,fill opacity=0.00,] (392.35,133.79) -- (398.71,133.79);

\draw[color=drawColor,line cap=round,line join=round,fill opacity=0.00,] (395.53,130.61) -- (395.53,136.97);

\draw[color=drawColor,line cap=round,line join=round,fill opacity=0.00,] (399.50,132.93) -- (405.87,132.93);

\draw[color=drawColor,line cap=round,line join=round,fill opacity=0.00,] (402.68,129.74) -- (402.68,136.11);

\draw[color=drawColor,line cap=round,line join=round,fill opacity=0.00,] (406.66,132.93) -- (413.02,132.93);

\draw[color=drawColor,line cap=round,line join=round,fill opacity=0.00,] (409.84,129.74) -- (409.84,136.11);

\draw[color=drawColor,line cap=round,line join=round,fill opacity=0.00,] (413.81,132.93) -- (420.18,132.93);

\draw[color=drawColor,line cap=round,line join=round,fill opacity=0.00,] (417.00,129.74) -- (417.00,136.11);

\draw[color=drawColor,line cap=round,line join=round,fill opacity=0.00,] (420.97,132.93) -- (427.33,132.93);

\draw[color=drawColor,line cap=round,line join=round,fill opacity=0.00,] (424.15,129.74) -- (424.15,136.11);

\draw[color=drawColor,line cap=round,line join=round,fill opacity=0.00,] (428.13,132.93) -- (434.49,132.93);

\draw[color=drawColor,line cap=round,line join=round,fill opacity=0.00,] (431.31,129.74) -- (431.31,136.11);

\draw[color=drawColor,line cap=round,line join=round,fill opacity=0.00,] (435.28,132.44) -- (441.65,132.44);

\draw[color=drawColor,line cap=round,line join=round,fill opacity=0.00,] (438.46,129.26) -- (438.46,135.63);

\draw[color=drawColor,line cap=round,line join=round,fill opacity=0.00,] (442.44,132.08) -- (448.80,132.08);

\draw[color=drawColor,line cap=round,line join=round,fill opacity=0.00,] (445.62,128.89) -- (445.62,135.26);

\draw[color=drawColor,line cap=round,line join=round,fill opacity=0.00,] (449.59,132.08) -- (455.96,132.08);

\draw[color=drawColor,line cap=round,line join=round,fill opacity=0.00,] (452.78,128.89) -- (452.78,135.26);

\draw[color=drawColor,line cap=round,line join=round,fill opacity=0.00,] (456.75,132.08) -- (463.11,132.08);

\draw[color=drawColor,line cap=round,line join=round,fill opacity=0.00,] (459.93,128.89) -- (459.93,135.26);

\draw[color=drawColor,line cap=round,line join=round,fill opacity=0.00,] (463.91,132.08) -- (470.27,132.08);

\draw[color=drawColor,line cap=round,line join=round,fill opacity=0.00,] (467.09,128.89) -- (467.09,135.26);

\draw[color=drawColor,line cap=round,line join=round,fill opacity=0.00,] (471.06,132.08) -- (477.43,132.08);

\draw[color=drawColor,line cap=round,line join=round,fill opacity=0.00,] (474.24,128.89) -- (474.24,135.26);

\draw[color=drawColor,line cap=round,line join=round,fill opacity=0.00,] (478.22,132.08) -- (484.58,132.08);

\draw[color=drawColor,line cap=round,line join=round,fill opacity=0.00,] (481.40,128.89) -- (481.40,135.26);

\draw[color=drawColor,line cap=round,line join=round,fill opacity=0.00,] (485.37,132.08) -- (491.74,132.08);

\draw[color=drawColor,line cap=round,line join=round,fill opacity=0.00,] (488.56,128.89) -- (488.56,135.26);

\draw[color=drawColor,line cap=round,line join=round,fill opacity=0.00,] (492.53,132.08) -- (498.89,132.08);

\draw[color=drawColor,line cap=round,line join=round,fill opacity=0.00,] (495.71,128.89) -- (495.71,135.26);

\draw[color=drawColor,line cap=round,line join=round,fill opacity=0.00,] (499.69,132.08) -- (506.05,132.08);

\draw[color=drawColor,line cap=round,line join=round,fill opacity=0.00,] (502.87,128.89) -- (502.87,135.26);

\draw[color=drawColor,line cap=round,line join=round,fill opacity=0.00,] (506.84,132.08) -- (513.21,132.08);

\draw[color=drawColor,line cap=round,line join=round,fill opacity=0.00,] (510.02,128.89) -- (510.02,135.26);

\draw[color=drawColor,line cap=round,line join=round,fill opacity=0.00,] (514.00,132.08) -- (520.36,132.08);

\draw[color=drawColor,line cap=round,line join=round,fill opacity=0.00,] (517.18,128.89) -- (517.18,135.26);

\draw[color=drawColor,line cap=round,line join=round,fill opacity=0.00,] (521.15,132.08) -- (527.52,132.08);

\draw[color=drawColor,line cap=round,line join=round,fill opacity=0.00,] (524.34,128.89) -- (524.34,135.26);

\draw[color=drawColor,line cap=round,line join=round,fill opacity=0.00,] (528.31,132.08) -- (534.67,132.08);

\draw[color=drawColor,line cap=round,line join=round,fill opacity=0.00,] (531.49,128.89) -- (531.49,135.26);

\draw[color=drawColor,line cap=round,line join=round,fill opacity=0.00,] (535.47,132.08) -- (541.83,132.08);

\draw[color=drawColor,line cap=round,line join=round,fill opacity=0.00,] (538.65,128.89) -- (538.65,135.26);

\draw[color=drawColor,line cap=round,line join=round,fill opacity=0.00,] (542.62,132.08) -- (548.99,132.08);

\draw[color=drawColor,line cap=round,line join=round,fill opacity=0.00,] (545.80,128.89) -- (545.80,135.26);

\draw[color=drawColor,line cap=round,line join=round,fill opacity=0.00,] (549.78,132.08) -- (556.14,132.08);

\draw[color=drawColor,line cap=round,line join=round,fill opacity=0.00,] (552.96,128.89) -- (552.96,135.26);

\draw[color=drawColor,line cap=round,line join=round,fill opacity=0.00,] (556.93,132.08) -- (563.30,132.08);

\draw[color=drawColor,line cap=round,line join=round,fill opacity=0.00,] (560.12,128.89) -- (560.12,135.26);

\draw[color=drawColor,line cap=round,line join=round,fill opacity=0.00,] (564.09,132.08) -- (570.45,132.08);

\draw[color=drawColor,line cap=round,line join=round,fill opacity=0.00,] (567.27,128.89) -- (567.27,135.26);

\draw[color=drawColor,line cap=round,line join=round,fill opacity=0.00,] (571.25,132.08) -- (577.61,132.08);

\draw[color=drawColor,line cap=round,line join=round,fill opacity=0.00,] (574.43,128.89) -- (574.43,135.26);
\definecolor[named]{drawColor}{rgb}{0.03,0.32,0.61}

\draw[color=drawColor,line cap=round,line join=round,fill opacity=0.00,] (338.28,163.67) --
	(345.44,163.67) --
	(345.44,132.73) --
	(352.59,132.73) --
	(352.59,109.88) --
	(359.75,109.88) --
	(359.75, 95.77) --
	(366.90, 95.77) --
	(366.90, 92.51) --
	(374.06, 92.51) --
	(374.06, 89.53) --
	(381.22, 89.53) --
	(381.22, 88.30) --
	(388.37, 88.30) --
	(388.37, 86.81) --
	(402.68, 86.81) --
	(402.68, 85.05) --
	(438.46, 85.05) --
	(438.46, 84.09) --
	(445.62, 84.09) --
	(445.62, 83.35) --
	(578.16, 83.35);

\draw[color=drawColor,line cap=round,line join=round,fill opacity=0.00,] (342.25,132.73) -- (348.62,132.73);

\draw[color=drawColor,line cap=round,line join=round,fill opacity=0.00,] (345.44,129.55) -- (345.44,135.91);

\draw[color=drawColor,line cap=round,line join=round,fill opacity=0.00,] (349.41,109.88) -- (355.77,109.88);

\draw[color=drawColor,line cap=round,line join=round,fill opacity=0.00,] (352.59,106.70) -- (352.59,113.06);

\draw[color=drawColor,line cap=round,line join=round,fill opacity=0.00,] (356.57, 95.77) -- (362.93, 95.77);

\draw[color=drawColor,line cap=round,line join=round,fill opacity=0.00,] (359.75, 92.59) -- (359.75, 98.95);

\draw[color=drawColor,line cap=round,line join=round,fill opacity=0.00,] (363.72, 92.51) -- (370.09, 92.51);

\draw[color=drawColor,line cap=round,line join=round,fill opacity=0.00,] (366.90, 89.33) -- (366.90, 95.69);

\draw[color=drawColor,line cap=round,line join=round,fill opacity=0.00,] (370.88, 89.53) -- (377.24, 89.53);

\draw[color=drawColor,line cap=round,line join=round,fill opacity=0.00,] (374.06, 86.35) -- (374.06, 92.72);

\draw[color=drawColor,line cap=round,line join=round,fill opacity=0.00,] (378.03, 88.30) -- (384.40, 88.30);

\draw[color=drawColor,line cap=round,line join=round,fill opacity=0.00,] (381.22, 85.12) -- (381.22, 91.48);

\draw[color=drawColor,line cap=round,line join=round,fill opacity=0.00,] (385.19, 86.81) -- (391.55, 86.81);

\draw[color=drawColor,line cap=round,line join=round,fill opacity=0.00,] (388.37, 83.63) -- (388.37, 90.00);

\draw[color=drawColor,line cap=round,line join=round,fill opacity=0.00,] (392.35, 86.81) -- (398.71, 86.81);

\draw[color=drawColor,line cap=round,line join=round,fill opacity=0.00,] (395.53, 83.63) -- (395.53, 90.00);

\draw[color=drawColor,line cap=round,line join=round,fill opacity=0.00,] (399.50, 85.05) -- (405.87, 85.05);

\draw[color=drawColor,line cap=round,line join=round,fill opacity=0.00,] (402.68, 81.87) -- (402.68, 88.24);

\draw[color=drawColor,line cap=round,line join=round,fill opacity=0.00,] (406.66, 85.05) -- (413.02, 85.05);

\draw[color=drawColor,line cap=round,line join=round,fill opacity=0.00,] (409.84, 81.87) -- (409.84, 88.24);

\draw[color=drawColor,line cap=round,line join=round,fill opacity=0.00,] (413.81, 85.05) -- (420.18, 85.05);

\draw[color=drawColor,line cap=round,line join=round,fill opacity=0.00,] (417.00, 81.87) -- (417.00, 88.24);

\draw[color=drawColor,line cap=round,line join=round,fill opacity=0.00,] (420.97, 85.05) -- (427.33, 85.05);

\draw[color=drawColor,line cap=round,line join=round,fill opacity=0.00,] (424.15, 81.87) -- (424.15, 88.24);

\draw[color=drawColor,line cap=round,line join=round,fill opacity=0.00,] (428.13, 85.05) -- (434.49, 85.05);

\draw[color=drawColor,line cap=round,line join=round,fill opacity=0.00,] (431.31, 81.87) -- (431.31, 88.24);

\draw[color=drawColor,line cap=round,line join=round,fill opacity=0.00,] (435.28, 84.09) -- (441.65, 84.09);

\draw[color=drawColor,line cap=round,line join=round,fill opacity=0.00,] (438.46, 80.90) -- (438.46, 87.27);

\draw[color=drawColor,line cap=round,line join=round,fill opacity=0.00,] (442.44, 83.35) -- (448.80, 83.35);

\draw[color=drawColor,line cap=round,line join=round,fill opacity=0.00,] (445.62, 80.17) -- (445.62, 86.54);

\draw[color=drawColor,line cap=round,line join=round,fill opacity=0.00,] (449.59, 83.35) -- (455.96, 83.35);

\draw[color=drawColor,line cap=round,line join=round,fill opacity=0.00,] (452.78, 80.17) -- (452.78, 86.54);

\draw[color=drawColor,line cap=round,line join=round,fill opacity=0.00,] (456.75, 83.35) -- (463.11, 83.35);

\draw[color=drawColor,line cap=round,line join=round,fill opacity=0.00,] (459.93, 80.17) -- (459.93, 86.54);

\draw[color=drawColor,line cap=round,line join=round,fill opacity=0.00,] (463.91, 83.35) -- (470.27, 83.35);

\draw[color=drawColor,line cap=round,line join=round,fill opacity=0.00,] (467.09, 80.17) -- (467.09, 86.54);

\draw[color=drawColor,line cap=round,line join=round,fill opacity=0.00,] (471.06, 83.35) -- (477.43, 83.35);

\draw[color=drawColor,line cap=round,line join=round,fill opacity=0.00,] (474.24, 80.17) -- (474.24, 86.54);

\draw[color=drawColor,line cap=round,line join=round,fill opacity=0.00,] (478.22, 83.35) -- (484.58, 83.35);

\draw[color=drawColor,line cap=round,line join=round,fill opacity=0.00,] (481.40, 80.17) -- (481.40, 86.54);

\draw[color=drawColor,line cap=round,line join=round,fill opacity=0.00,] (485.37, 83.35) -- (491.74, 83.35);

\draw[color=drawColor,line cap=round,line join=round,fill opacity=0.00,] (488.56, 80.17) -- (488.56, 86.54);

\draw[color=drawColor,line cap=round,line join=round,fill opacity=0.00,] (492.53, 83.35) -- (498.89, 83.35);

\draw[color=drawColor,line cap=round,line join=round,fill opacity=0.00,] (495.71, 80.17) -- (495.71, 86.54);

\draw[color=drawColor,line cap=round,line join=round,fill opacity=0.00,] (499.69, 83.35) -- (506.05, 83.35);

\draw[color=drawColor,line cap=round,line join=round,fill opacity=0.00,] (502.87, 80.17) -- (502.87, 86.54);

\draw[color=drawColor,line cap=round,line join=round,fill opacity=0.00,] (506.84, 83.35) -- (513.21, 83.35);

\draw[color=drawColor,line cap=round,line join=round,fill opacity=0.00,] (510.02, 80.17) -- (510.02, 86.54);

\draw[color=drawColor,line cap=round,line join=round,fill opacity=0.00,] (514.00, 83.35) -- (520.36, 83.35);

\draw[color=drawColor,line cap=round,line join=round,fill opacity=0.00,] (517.18, 80.17) -- (517.18, 86.54);

\draw[color=drawColor,line cap=round,line join=round,fill opacity=0.00,] (521.15, 83.35) -- (527.52, 83.35);

\draw[color=drawColor,line cap=round,line join=round,fill opacity=0.00,] (524.34, 80.17) -- (524.34, 86.54);

\draw[color=drawColor,line cap=round,line join=round,fill opacity=0.00,] (528.31, 83.35) -- (534.67, 83.35);

\draw[color=drawColor,line cap=round,line join=round,fill opacity=0.00,] (531.49, 80.17) -- (531.49, 86.54);

\draw[color=drawColor,line cap=round,line join=round,fill opacity=0.00,] (535.47, 83.35) -- (541.83, 83.35);

\draw[color=drawColor,line cap=round,line join=round,fill opacity=0.00,] (538.65, 80.17) -- (538.65, 86.54);

\draw[color=drawColor,line cap=round,line join=round,fill opacity=0.00,] (542.62, 83.35) -- (548.99, 83.35);

\draw[color=drawColor,line cap=round,line join=round,fill opacity=0.00,] (545.80, 80.17) -- (545.80, 86.54);

\draw[color=drawColor,line cap=round,line join=round,fill opacity=0.00,] (549.78, 83.35) -- (556.14, 83.35);

\draw[color=drawColor,line cap=round,line join=round,fill opacity=0.00,] (552.96, 80.17) -- (552.96, 86.54);

\draw[color=drawColor,line cap=round,line join=round,fill opacity=0.00,] (556.93, 83.35) -- (563.30, 83.35);

\draw[color=drawColor,line cap=round,line join=round,fill opacity=0.00,] (560.12, 80.17) -- (560.12, 86.54);

\draw[color=drawColor,line cap=round,line join=round,fill opacity=0.00,] (564.09, 83.35) -- (570.45, 83.35);

\draw[color=drawColor,line cap=round,line join=round,fill opacity=0.00,] (567.27, 80.17) -- (567.27, 86.54);

\draw[color=drawColor,line cap=round,line join=round,fill opacity=0.00,] (571.25, 83.35) -- (577.61, 83.35);

\draw[color=drawColor,line cap=round,line join=round,fill opacity=0.00,] (574.43, 80.17) -- (574.43, 86.54);
\end{scope}
\begin{scope}
\path[clip] (289.08,  0.00) rectangle (578.16,216.81);
\definecolor[named]{drawColor}{rgb}{0.00,0.00,0.00}

\node[color=drawColor,anchor=base,inner sep=0pt, outer sep=0pt, scale=  1.00] at (445.62, 13.20) {Time (Years)};
\end{scope}
\end{tikzpicture}
}	
	\label{fig:surv3}	
\end{figure}
\newpage

Our sanction reciprocity measure tells a similar story, but focuses on the consequences of past reciprocal adverse relations. Here we can see that countries whom have sanctioned each other in the past without complying to one another are not likely to comply to one another in the present. On the right side of Figure \ref{fig:surv3}, we can again see that within just five years the probability of non-compliance in a case where target and sender states have not had adverse past relations is half compared to a case where past adverse relations are present. This points to important consequences for sender states, namely that continuous sanctioning of a particular state without that target every compyling  may build up a resistance to compliance to future sanctions. 

\subsection*{Performance}

To assess the accuracy and performance of these estimates we employ a six-fold cross validation procedure.\footnote{Results of analysis were similar when employing a 10-fold cross validation as well, however, we limit to showing six here for the sake of space.} We use this procedure both to determine the robustness of our coefficient estimates when estimated on different subsamples of our dataset, and to assess how well the results of our model would generalize to an independent dataset. To begin the cross-validation, we split the 653 sanction cases in our dataset into six approximately equal subsets. We then run each model shown in Table \ref{tab:regResults} six times, where in each iteration we left out one subsample to use as a test set. This allows us to compare the prediction accuracy of each model, thereby helping us to determine the gains from incorporating the reciprocity covariates that are key to our argument.

First, however, we show the results for our reciprocity covariates when we rerun our survival analysis on each of the six folds from the cross-validation. This analysis helps us to understand whether some of the subsets in our dataset follow a different pattern than what is in the broader set.\footnote{\cite{beck2008time}} Figure \ref{fig:crossval} shows that this is not the case for the analysis we present here, the coefficient estimates for compliance and sanction reciprocity remain consistent across each of these subsamples.\footnote{The parameter estimates for the other covariates also remain consistent across each of the six folds but we leave them out here due to space constraints.}

\newpage
\begin{figure}[ht]
	\centering
	\caption{Reciprocity coefficient estimates from each of the six-folds of the cross validation procedure.}
	\resizebox{1\textwidth}{!}{% Created by tikzDevice version 0.8.1 on 2015-05-25 07:37:44
% !TEX encoding = UTF-8 Unicode
\begin{tikzpicture}[x=1pt,y=1pt]
\definecolor{fillColor}{RGB}{255,255,255}
\path[use as bounding box,fill=fillColor,fill opacity=0.00] (0,0) rectangle (505.89,216.81);
\begin{scope}
\path[clip] (  0.00,  0.00) rectangle (505.89,216.81);
\definecolor{drawColor}{RGB}{255,255,255}
\definecolor{fillColor}{RGB}{255,255,255}

\path[draw=drawColor,line width= 0.6pt,line join=round,line cap=round,fill=fillColor] (  0.00,  0.00) rectangle (505.89,216.81);
\end{scope}
\begin{scope}
\path[clip] ( 39.69, 50.67) rectangle (251.57,192.13);
\definecolor{fillColor}{RGB}{255,255,255}

\path[fill=fillColor] ( 39.69, 50.67) rectangle (251.57,192.13);
\definecolor{drawColor}{RGB}{0,0,0}

\path[draw=drawColor,draw opacity=0.30,line width= 0.3pt,line join=round] ( 60.19, 67.01) -- ( 60.19,157.97);

\path[draw=drawColor,draw opacity=0.30,line width= 0.3pt,line join=round] ( 94.37, 84.84) -- ( 94.37,171.68);

\path[draw=drawColor,draw opacity=0.30,line width= 0.3pt,line join=round] (128.54, 81.00) -- (128.54,173.77);

\path[draw=drawColor,draw opacity=0.30,line width= 0.3pt,line join=round] (162.72, 76.75) -- (162.72,165.85);

\path[draw=drawColor,draw opacity=0.30,line width= 0.3pt,line join=round] (196.89, 90.43) -- (196.89,183.18);

\path[draw=drawColor,draw opacity=0.30,line width= 0.3pt,line join=round] (231.07, 95.99) -- (231.07,185.70);
\definecolor{drawColor}{RGB}{0,0,0}

\path[draw=drawColor,line width= 1.1pt,line join=round] ( 60.19, 74.32) -- ( 60.19,150.66);

\path[draw=drawColor,line width= 1.1pt,line join=round] ( 94.37, 91.82) -- ( 94.37,164.70);

\path[draw=drawColor,line width= 1.1pt,line join=round] (128.54, 88.46) -- (128.54,166.31);

\path[draw=drawColor,line width= 1.1pt,line join=round] (162.72, 83.91) -- (162.72,158.69);

\path[draw=drawColor,line width= 1.1pt,line join=round] (196.89, 97.88) -- (196.89,175.73);

\path[draw=drawColor,line width= 1.1pt,line join=round] (231.07,103.20) -- (231.07,178.49);

\path[draw=drawColor,line width= 0.6pt,dash pattern=on 4pt off 4pt ,line join=round] ( 39.69, 57.10) -- (251.57, 57.10);
\definecolor{fillColor}{RGB}{0,0,0}

\path[draw=drawColor,line width= 0.4pt,line join=round,line cap=round,fill=fillColor] ( 60.19,112.49) circle (  2.85);

\path[draw=drawColor,line width= 0.4pt,line join=round,line cap=round,fill=fillColor] ( 94.37,128.26) circle (  2.85);

\path[draw=drawColor,line width= 0.4pt,line join=round,line cap=round,fill=fillColor] (128.54,127.39) circle (  2.85);

\path[draw=drawColor,line width= 0.4pt,line join=round,line cap=round,fill=fillColor] (162.72,121.30) circle (  2.85);

\path[draw=drawColor,line width= 0.4pt,line join=round,line cap=round,fill=fillColor] (196.89,136.80) circle (  2.85);

\path[draw=drawColor,line width= 0.4pt,line join=round,line cap=round,fill=fillColor] (231.07,140.84) circle (  2.85);

\path[draw=drawColor,line width= 0.6pt,line join=round] ( 58.48,157.97) --
	( 61.90,157.97);

\path[draw=drawColor,line width= 0.6pt,line join=round] ( 60.19,157.97) --
	( 60.19, 67.01);

\path[draw=drawColor,line width= 0.6pt,line join=round] ( 58.48, 67.01) --
	( 61.90, 67.01);

\path[draw=drawColor,line width= 0.6pt,line join=round] ( 92.66,171.68) --
	( 96.08,171.68);

\path[draw=drawColor,line width= 0.6pt,line join=round] ( 94.37,171.68) --
	( 94.37, 84.84);

\path[draw=drawColor,line width= 0.6pt,line join=round] ( 92.66, 84.84) --
	( 96.08, 84.84);

\path[draw=drawColor,line width= 0.6pt,line join=round] (126.83,173.77) --
	(130.25,173.77);

\path[draw=drawColor,line width= 0.6pt,line join=round] (128.54,173.77) --
	(128.54, 81.00);

\path[draw=drawColor,line width= 0.6pt,line join=round] (126.83, 81.00) --
	(130.25, 81.00);

\path[draw=drawColor,line width= 0.6pt,line join=round] (161.01,165.85) --
	(164.43,165.85);

\path[draw=drawColor,line width= 0.6pt,line join=round] (162.72,165.85) --
	(162.72, 76.75);

\path[draw=drawColor,line width= 0.6pt,line join=round] (161.01, 76.75) --
	(164.43, 76.75);

\path[draw=drawColor,line width= 0.6pt,line join=round] (195.18,183.18) --
	(198.60,183.18);

\path[draw=drawColor,line width= 0.6pt,line join=round] (196.89,183.18) --
	(196.89, 90.43);

\path[draw=drawColor,line width= 0.6pt,line join=round] (195.18, 90.43) --
	(198.60, 90.43);

\path[draw=drawColor,line width= 0.6pt,line join=round] (229.36,185.70) --
	(232.78,185.70);

\path[draw=drawColor,line width= 0.6pt,line join=round] (231.07,185.70) --
	(231.07, 95.99);

\path[draw=drawColor,line width= 0.6pt,line join=round] (229.36, 95.99) --
	(232.78, 95.99);
\definecolor{drawColor}{gray}{0.50}

\path[draw=drawColor,line width= 0.6pt,line join=round,line cap=round] ( 39.69, 50.67) rectangle (251.57,192.13);
\end{scope}
\begin{scope}
\path[clip] (281.96, 50.67) rectangle (493.85,192.13);
\definecolor{fillColor}{RGB}{255,255,255}

\path[fill=fillColor] (281.96, 50.67) rectangle (493.85,192.13);
\definecolor{drawColor}{RGB}{0,0,0}

\path[draw=drawColor,draw opacity=0.30,line width= 0.3pt,line join=round] (302.46, 94.34) -- (302.46,184.13);

\path[draw=drawColor,draw opacity=0.30,line width= 0.3pt,line join=round] (336.64, 79.61) -- (336.64,166.21);

\path[draw=drawColor,draw opacity=0.30,line width= 0.3pt,line join=round] (370.81, 66.95) -- (370.81,161.82);

\path[draw=drawColor,draw opacity=0.30,line width= 0.3pt,line join=round] (404.99, 78.48) -- (404.99,167.69);

\path[draw=drawColor,draw opacity=0.30,line width= 0.3pt,line join=round] (439.16, 57.10) -- (439.16,151.50);

\path[draw=drawColor,draw opacity=0.30,line width= 0.3pt,line join=round] (473.34, 60.25) -- (473.34,150.65);
\definecolor{drawColor}{RGB}{0,0,0}

\path[draw=drawColor,line width= 1.1pt,line join=round] (302.46,101.55) -- (302.46,176.91);

\path[draw=drawColor,line width= 1.1pt,line join=round] (336.64, 86.57) -- (336.64,159.25);

\path[draw=drawColor,line width= 1.1pt,line join=round] (370.81, 74.58) -- (370.81,154.20);

\path[draw=drawColor,line width= 1.1pt,line join=round] (404.99, 85.65) -- (404.99,160.52);

\path[draw=drawColor,line width= 1.1pt,line join=round] (439.16, 64.69) -- (439.16,143.91);

\path[draw=drawColor,line width= 1.1pt,line join=round] (473.34, 67.52) -- (473.34,143.38);

\path[draw=drawColor,line width= 0.6pt,dash pattern=on 4pt off 4pt ,line join=round] (281.96,185.70) -- (493.85,185.70);
\definecolor{fillColor}{RGB}{0,0,0}

\path[draw=drawColor,line width= 0.4pt,line join=round,line cap=round,fill=fillColor] (302.46,139.23) circle (  2.85);

\path[draw=drawColor,line width= 0.4pt,line join=round,line cap=round,fill=fillColor] (336.64,122.91) circle (  2.85);

\path[draw=drawColor,line width= 0.4pt,line join=round,line cap=round,fill=fillColor] (370.81,114.39) circle (  2.85);

\path[draw=drawColor,line width= 0.4pt,line join=round,line cap=round,fill=fillColor] (404.99,123.08) circle (  2.85);

\path[draw=drawColor,line width= 0.4pt,line join=round,line cap=round,fill=fillColor] (439.16,104.30) circle (  2.85);

\path[draw=drawColor,line width= 0.4pt,line join=round,line cap=round,fill=fillColor] (473.34,105.45) circle (  2.85);

\path[draw=drawColor,line width= 0.6pt,line join=round] (300.76,184.13) --
	(304.17,184.13);

\path[draw=drawColor,line width= 0.6pt,line join=round] (302.46,184.13) --
	(302.46, 94.34);

\path[draw=drawColor,line width= 0.6pt,line join=round] (300.76, 94.34) --
	(304.17, 94.34);

\path[draw=drawColor,line width= 0.6pt,line join=round] (334.93,166.21) --
	(338.35,166.21);

\path[draw=drawColor,line width= 0.6pt,line join=round] (336.64,166.21) --
	(336.64, 79.61);

\path[draw=drawColor,line width= 0.6pt,line join=round] (334.93, 79.61) --
	(338.35, 79.61);

\path[draw=drawColor,line width= 0.6pt,line join=round] (369.11,161.82) --
	(372.52,161.82);

\path[draw=drawColor,line width= 0.6pt,line join=round] (370.81,161.82) --
	(370.81, 66.95);

\path[draw=drawColor,line width= 0.6pt,line join=round] (369.11, 66.95) --
	(372.52, 66.95);

\path[draw=drawColor,line width= 0.6pt,line join=round] (403.28,167.69) --
	(406.70,167.69);

\path[draw=drawColor,line width= 0.6pt,line join=round] (404.99,167.69) --
	(404.99, 78.48);

\path[draw=drawColor,line width= 0.6pt,line join=round] (403.28, 78.48) --
	(406.70, 78.48);

\path[draw=drawColor,line width= 0.6pt,line join=round] (437.46,151.50) --
	(440.87,151.50);

\path[draw=drawColor,line width= 0.6pt,line join=round] (439.16,151.50) --
	(439.16, 57.10);

\path[draw=drawColor,line width= 0.6pt,line join=round] (437.46, 57.10) --
	(440.87, 57.10);

\path[draw=drawColor,line width= 0.6pt,line join=round] (471.63,150.65) --
	(475.05,150.65);

\path[draw=drawColor,line width= 0.6pt,line join=round] (473.34,150.65) --
	(473.34, 60.25);

\path[draw=drawColor,line width= 0.6pt,line join=round] (471.63, 60.25) --
	(475.05, 60.25);
\definecolor{drawColor}{gray}{0.50}

\path[draw=drawColor,line width= 0.6pt,line join=round,line cap=round] (281.96, 50.67) rectangle (493.85,192.13);
\end{scope}
\begin{scope}
\path[clip] (  0.00,  0.00) rectangle (505.89,216.81);
\definecolor{drawColor}{gray}{0.50}
\definecolor{fillColor}{gray}{0.80}

\path[draw=drawColor,line width= 0.2pt,line join=round,line cap=round,fill=fillColor] ( 39.69,192.13) rectangle (251.57,204.77);
\definecolor{drawColor}{RGB}{0,0,0}

\node[text=drawColor,anchor=base,inner sep=0pt, outer sep=0pt, scale=  0.96] at (145.63,195.14) {Compliance Reciprocity$_{j,t-1}$};
\end{scope}
\begin{scope}
\path[clip] (  0.00,  0.00) rectangle (505.89,216.81);
\definecolor{drawColor}{gray}{0.50}
\definecolor{fillColor}{gray}{0.80}

\path[draw=drawColor,line width= 0.2pt,line join=round,line cap=round,fill=fillColor] (281.96,192.13) rectangle (493.85,204.77);
\definecolor{drawColor}{RGB}{0,0,0}

\node[text=drawColor,anchor=base,inner sep=0pt, outer sep=0pt, scale=  0.96] at (387.90,195.14) {Sanction Reciprocity$_{j,t-1}$};
\end{scope}
\begin{scope}
\path[clip] (  0.00,  0.00) rectangle (505.89,216.81);
\definecolor{drawColor}{RGB}{0,0,0}

\node[text=drawColor,anchor=base east,inner sep=0pt, outer sep=0pt, scale=  0.96] at ( 32.57, 53.79) {0.0};

\node[text=drawColor,anchor=base east,inner sep=0pt, outer sep=0pt, scale=  0.96] at ( 32.57, 86.88) {0.1};

\node[text=drawColor,anchor=base east,inner sep=0pt, outer sep=0pt, scale=  0.96] at ( 32.57,119.97) {0.2};

\node[text=drawColor,anchor=base east,inner sep=0pt, outer sep=0pt, scale=  0.96] at ( 32.57,153.05) {0.3};

\node[text=drawColor,anchor=base east,inner sep=0pt, outer sep=0pt, scale=  0.96] at ( 32.57,186.14) {0.4};
\end{scope}
\begin{scope}
\path[clip] (  0.00,  0.00) rectangle (505.89,216.81);
\definecolor{drawColor}{RGB}{0,0,0}

\node[text=drawColor,anchor=base east,inner sep=0pt, outer sep=0pt, scale=  0.96] at (274.85, 53.29) {-0.20};

\node[text=drawColor,anchor=base east,inner sep=0pt, outer sep=0pt, scale=  0.96] at (274.85, 85.57) {-0.15};

\node[text=drawColor,anchor=base east,inner sep=0pt, outer sep=0pt, scale=  0.96] at (274.85,117.84) {-0.10};

\node[text=drawColor,anchor=base east,inner sep=0pt, outer sep=0pt, scale=  0.96] at (274.85,150.12) {-0.05};

\node[text=drawColor,anchor=base east,inner sep=0pt, outer sep=0pt, scale=  0.96] at (274.85,182.39) {0.00};
\end{scope}
\begin{scope}
\path[clip] (  0.00,  0.00) rectangle (505.89,216.81);
\definecolor{drawColor}{RGB}{0,0,0}

\node[text=drawColor,rotate= 45.00,anchor=base east,inner sep=0pt, outer sep=0pt, scale=  0.96] at ( 64.87, 38.88) {Fold 1};

\node[text=drawColor,rotate= 45.00,anchor=base east,inner sep=0pt, outer sep=0pt, scale=  0.96] at ( 99.04, 38.88) {Fold 2};

\node[text=drawColor,rotate= 45.00,anchor=base east,inner sep=0pt, outer sep=0pt, scale=  0.96] at (133.22, 38.88) {Fold 3};

\node[text=drawColor,rotate= 45.00,anchor=base east,inner sep=0pt, outer sep=0pt, scale=  0.96] at (167.39, 38.88) {Fold 4};

\node[text=drawColor,rotate= 45.00,anchor=base east,inner sep=0pt, outer sep=0pt, scale=  0.96] at (201.57, 38.88) {Fold 5};

\node[text=drawColor,rotate= 45.00,anchor=base east,inner sep=0pt, outer sep=0pt, scale=  0.96] at (235.74, 38.88) {Fold 6};
\end{scope}
\begin{scope}
\path[clip] (  0.00,  0.00) rectangle (505.89,216.81);
\definecolor{drawColor}{RGB}{0,0,0}

\node[text=drawColor,rotate= 45.00,anchor=base east,inner sep=0pt, outer sep=0pt, scale=  0.96] at (307.14, 38.88) {Fold 1};

\node[text=drawColor,rotate= 45.00,anchor=base east,inner sep=0pt, outer sep=0pt, scale=  0.96] at (341.31, 38.88) {Fold 2};

\node[text=drawColor,rotate= 45.00,anchor=base east,inner sep=0pt, outer sep=0pt, scale=  0.96] at (375.49, 38.88) {Fold 3};

\node[text=drawColor,rotate= 45.00,anchor=base east,inner sep=0pt, outer sep=0pt, scale=  0.96] at (409.66, 38.88) {Fold 4};

\node[text=drawColor,rotate= 45.00,anchor=base east,inner sep=0pt, outer sep=0pt, scale=  0.96] at (443.84, 38.88) {Fold 5};

\node[text=drawColor,rotate= 45.00,anchor=base east,inner sep=0pt, outer sep=0pt, scale=  0.96] at (478.02, 38.88) {Fold 6};
\end{scope}
\end{tikzpicture}
}
	\label{fig:crossval}
\end{figure}
\newpage

A key question that remains, however, is whether we are able to better explain sanction compliance through the incorporation of these network level covariates. Figure \ref{fig:auc} shows out-of-sample time-dependent AUC results from the six-fold cross validation procedure. When calculating the time-dependent AUC we vary the time parameter to range from 0 to 15 years.\footnote{Time-dependent AUCs were computed using the formula provided by \citet{chambless2006estimation}.} We set the max for 15 years because only 4\% of sanction cases in our dataset that extend past 15 years end with compliance by a target state. This leads to the AUC statistics for each model. After that time point, the accuracy of any of the models begin to coalesce. Before the 15 year mark, however, we see noticeable variation in the time-dependent AUC statistics for each model. Most importantly, we can see that Model 3, where we incorporate our network level covariates, provides a noticeably higher AUC than the alternatives that we examined. Even simply accounting for proximate relationships, as we did in Model 2, does not provide a noticeably higher level of performance than target-state focused explanations.

\newpage
\begin{figure}[ht]
	\centering
	\caption{Out-of-sample time-dependent AUC statistics from six-fold cross validation procedure for each model shown in Table \ref{tab:regResults}. The solid line represents Model 1, the dotted line represents Model2, and the dashed line represents Model 3.}
	\resizebox{0.8\textwidth}{!}{% Created by tikzDevice version 0.7.0 on 2014-08-02 16:07:00
% !TEX encoding = UTF-8 Unicode
\begin{tikzpicture}[x=1pt,y=1pt]
\definecolor[named]{fillColor}{rgb}{1.00,1.00,1.00}
\path[use as bounding box,fill=fillColor,fill opacity=0.00] (0,0) rectangle (578.16,361.35);
\begin{scope}
\path[clip] (  0.00,  0.00) rectangle (578.16,361.35);
\definecolor[named]{drawColor}{rgb}{1.00,1.00,1.00}
\definecolor[named]{fillColor}{rgb}{1.00,1.00,1.00}

\path[draw=drawColor,line width= 0.6pt,line join=round,line cap=round,fill=fillColor] ( -0.00,  0.00) rectangle (578.16,361.35);
\end{scope}
\begin{scope}
\path[clip] ( 39.69,193.18) rectangle (200.24,336.67);
\definecolor[named]{fillColor}{rgb}{1.00,1.00,1.00}

\path[fill=fillColor] ( 39.69,193.18) rectangle (200.24,336.67);
\definecolor[named]{drawColor}{rgb}{0.00,0.00,0.00}

\path[draw=drawColor,line width= 1.1pt,line join=round] ( 56.11,201.72) --
	( 65.23,203.71) --
	( 74.35,206.42) --
	( 83.47,210.08) --
	( 92.60,215.14) --
	(101.72,221.77) --
	(110.84,230.18) --
	(119.96,240.18) --
	(129.08,251.47) --
	(138.21,263.38) --
	(147.33,274.54) --
	(156.45,284.60) --
	(165.57,292.95) --
	(174.70,299.65) --
	(183.82,305.08) --
	(192.94,309.07);

\path[draw=drawColor,line width= 1.1pt,dash pattern=on 2pt off 2pt ,line join=round] ( 56.11,208.30) --
	( 65.23,210.23) --
	( 74.35,212.84) --
	( 83.47,216.38) --
	( 92.60,221.20) --
	(101.72,227.43) --
	(110.84,235.16) --
	(119.96,244.24) --
	(129.08,254.53) --
	(138.21,265.60) --
	(147.33,276.33) --
	(156.45,286.47) --
	(165.57,295.20) --
	(174.70,302.39) --
	(183.82,308.22) --
	(192.94,312.43);

\path[draw=drawColor,line width= 1.1pt,dash pattern=on 4pt off 2pt ,line join=round] ( 56.11,215.75) --
	( 65.23,218.58) --
	( 74.35,222.28) --
	( 83.47,227.01) --
	( 92.60,233.11) --
	(101.72,240.50) --
	(110.84,249.18) --
	(119.96,258.80) --
	(129.08,269.11) --
	(138.21,279.54) --
	(147.33,289.03) --
	(156.45,297.41) --
	(165.57,304.19) --
	(174.70,309.47) --
	(183.82,313.54) --
	(192.94,316.36);
\definecolor[named]{drawColor}{rgb}{0.50,0.50,0.50}

\path[draw=drawColor,line width= 0.6pt,line join=round,line cap=round] ( 39.69,193.18) rectangle (200.24,336.67);
\end{scope}
\begin{scope}
\path[clip] (222.63,193.18) rectangle (383.18,336.67);
\definecolor[named]{fillColor}{rgb}{1.00,1.00,1.00}

\path[fill=fillColor] (222.63,193.18) rectangle (383.18,336.67);
\definecolor[named]{drawColor}{rgb}{0.00,0.00,0.00}

\path[draw=drawColor,line width= 1.1pt,line join=round] (239.05,201.46) --
	(248.17,203.47) --
	(257.29,206.15) --
	(266.41,209.60) --
	(275.53,214.17) --
	(284.66,219.99) --
	(293.78,227.19) --
	(302.90,235.65) --
	(312.02,245.34) --
	(321.15,255.77) --
	(330.27,266.03) --
	(339.39,275.84) --
	(348.51,284.42) --
	(357.63,291.66) --
	(366.76,297.66) --
	(375.88,302.18);

\path[draw=drawColor,line width= 1.1pt,dash pattern=on 2pt off 2pt ,line join=round] (239.05,206.03) --
	(248.17,207.75) --
	(257.29,210.00) --
	(266.41,212.90) --
	(275.53,216.73) --
	(284.66,221.64) --
	(293.78,227.74) --
	(302.90,234.97) --
	(312.02,243.37) --
	(321.15,252.63) --
	(330.27,262.12) --
	(339.39,271.73) --
	(348.51,280.73) --
	(357.63,288.82) --
	(366.76,295.80) --
	(375.88,301.15);

\path[draw=drawColor,line width= 1.1pt,dash pattern=on 4pt off 2pt ,line join=round] (239.05,215.56) --
	(248.17,218.44) --
	(257.29,222.14) --
	(266.41,226.74) --
	(275.53,232.56) --
	(284.66,239.64) --
	(293.78,247.96) --
	(302.90,257.21) --
	(312.02,267.11) --
	(321.15,277.00) --
	(330.27,285.99) --
	(339.39,293.96) --
	(348.51,300.52) --
	(357.63,305.85) --
	(366.76,310.23) --
	(375.88,313.60);
\definecolor[named]{drawColor}{rgb}{0.50,0.50,0.50}

\path[draw=drawColor,line width= 0.6pt,line join=round,line cap=round] (222.63,193.18) rectangle (383.18,336.67);
\end{scope}
\begin{scope}
\path[clip] (405.56,193.18) rectangle (566.12,336.67);
\definecolor[named]{fillColor}{rgb}{1.00,1.00,1.00}

\path[fill=fillColor] (405.56,193.18) rectangle (566.12,336.67);
\definecolor[named]{drawColor}{rgb}{0.00,0.00,0.00}

\path[draw=drawColor,line width= 1.1pt,line join=round] (421.98,206.56) --
	(431.11,208.43) --
	(440.23,210.92) --
	(449.35,214.11) --
	(458.47,218.34) --
	(467.60,223.73) --
	(476.72,230.47) --
	(485.84,238.42) --
	(494.96,247.45) --
	(504.08,257.22) --
	(513.21,266.82) --
	(522.33,276.06) --
	(531.45,284.44) --
	(540.57,291.95) --
	(549.70,298.64) --
	(558.82,303.77);

\path[draw=drawColor,line width= 1.1pt,dash pattern=on 2pt off 2pt ,line join=round] (421.98,219.03) --
	(431.11,220.26) --
	(440.23,221.93) --
	(449.35,224.22) --
	(458.47,227.45) --
	(467.60,231.81) --
	(476.72,237.41) --
	(485.84,244.13) --
	(494.96,251.84) --
	(504.08,260.31) --
	(513.21,268.84) --
	(522.33,277.32) --
	(531.45,285.26) --
	(540.57,292.58) --
	(549.70,299.21) --
	(558.82,304.36);

\path[draw=drawColor,line width= 1.1pt,dash pattern=on 4pt off 2pt ,line join=round] (421.98,227.27) --
	(431.11,230.16) --
	(440.23,233.82) --
	(449.35,238.31) --
	(458.47,244.03) --
	(467.60,251.02) --
	(476.72,259.28) --
	(485.84,268.24) --
	(494.96,277.30) --
	(504.08,285.77) --
	(513.21,292.84) --
	(522.33,298.66) --
	(531.45,303.29) --
	(540.57,307.06) --
	(549.70,310.23) --
	(558.82,312.58);
\definecolor[named]{drawColor}{rgb}{0.50,0.50,0.50}

\path[draw=drawColor,line width= 0.6pt,line join=round,line cap=round] (405.56,193.18) rectangle (566.12,336.67);
\end{scope}
\begin{scope}
\path[clip] ( 39.69, 34.03) rectangle (200.24,177.53);
\definecolor[named]{fillColor}{rgb}{1.00,1.00,1.00}

\path[fill=fillColor] ( 39.69, 34.03) rectangle (200.24,177.53);
\definecolor[named]{drawColor}{rgb}{0.00,0.00,0.00}

\path[draw=drawColor,line width= 1.1pt,line join=round] ( 56.11, 53.67) --
	( 65.23, 56.70) --
	( 74.35, 60.36) --
	( 83.47, 64.61) --
	( 92.60, 69.62) --
	(101.72, 75.33) --
	(110.84, 81.78) --
	(119.96, 88.98) --
	(129.08, 96.91) --
	(138.21,105.49) --
	(147.33,113.91) --
	(156.45,122.10) --
	(165.57,129.52) --
	(174.70,136.05) --
	(183.82,141.55) --
	(192.94,145.51);

\path[draw=drawColor,line width= 1.1pt,dash pattern=on 2pt off 2pt ,line join=round] ( 56.11, 71.78) --
	( 65.23, 72.70) --
	( 74.35, 73.49) --
	( 83.47, 74.79) --
	( 92.60, 77.19) --
	(101.72, 80.94) --
	(110.84, 85.99) --
	(119.96, 92.13) --
	(129.08, 99.16) --
	(138.21,106.95) --
	(147.33,114.81) --
	(156.45,122.74) --
	(165.57,130.24) --
	(174.70,137.22) --
	(183.82,143.46) --
	(192.94,148.26);

\path[draw=drawColor,line width= 1.1pt,dash pattern=on 4pt off 2pt ,line join=round] ( 56.11, 77.20) --
	( 65.23, 80.10) --
	( 74.35, 82.94) --
	( 83.47, 86.31) --
	( 92.60, 90.91) --
	(101.72, 96.89) --
	(110.84,104.06) --
	(119.96,111.96) --
	(129.08,120.12) --
	(138.21,128.19) --
	(147.33,135.32) --
	(156.45,141.62) --
	(165.57,146.92) --
	(174.70,151.40) --
	(183.82,155.19) --
	(192.94,158.01);
\definecolor[named]{drawColor}{rgb}{0.50,0.50,0.50}

\path[draw=drawColor,line width= 0.6pt,line join=round,line cap=round] ( 39.69, 34.03) rectangle (200.24,177.53);
\end{scope}
\begin{scope}
\path[clip] (222.63, 34.03) rectangle (383.18,177.53);
\definecolor[named]{fillColor}{rgb}{1.00,1.00,1.00}

\path[fill=fillColor] (222.63, 34.03) rectangle (383.18,177.53);
\definecolor[named]{drawColor}{rgb}{0.00,0.00,0.00}

\path[draw=drawColor,line width= 1.1pt,line join=round] (239.05, 45.83) --
	(248.17, 47.53) --
	(257.29, 49.75) --
	(266.41, 52.66) --
	(275.53, 56.61) --
	(284.66, 61.76) --
	(293.78, 68.25) --
	(302.90, 76.08) --
	(312.02, 85.12) --
	(321.15, 94.98) --
	(330.27,104.73) --
	(339.39,114.15) --
	(348.51,122.72) --
	(357.63,130.40) --
	(366.76,137.17) --
	(375.88,142.43);

\path[draw=drawColor,line width= 1.1pt,dash pattern=on 2pt off 2pt ,line join=round] (239.05, 51.93) --
	(248.17, 53.42) --
	(257.29, 55.42) --
	(266.41, 58.09) --
	(275.53, 61.74) --
	(284.66, 66.48) --
	(293.78, 72.28) --
	(302.90, 79.05) --
	(312.02, 86.65) --
	(321.15, 94.94) --
	(330.27,103.36) --
	(339.39,111.97) --
	(348.51,120.35) --
	(357.63,128.42) --
	(366.76,136.07) --
	(375.88,142.41);

\path[draw=drawColor,line width= 1.1pt,dash pattern=on 4pt off 2pt ,line join=round] (239.05, 65.25) --
	(248.17, 68.43) --
	(257.29, 72.38) --
	(266.41, 77.30) --
	(275.53, 83.58) --
	(284.66, 91.21) --
	(293.78, 99.90) --
	(302.90,109.10) --
	(312.02,118.15) --
	(321.15,126.47) --
	(330.27,133.42) --
	(339.39,139.23) --
	(348.51,143.97) --
	(357.63,147.94) --
	(366.76,151.30) --
	(375.88,153.85);
\definecolor[named]{drawColor}{rgb}{0.50,0.50,0.50}

\path[draw=drawColor,line width= 0.6pt,line join=round,line cap=round] (222.63, 34.03) rectangle (383.18,177.53);
\end{scope}
\begin{scope}
\path[clip] (405.56, 34.03) rectangle (566.12,177.53);
\definecolor[named]{fillColor}{rgb}{1.00,1.00,1.00}

\path[fill=fillColor] (405.56, 34.03) rectangle (566.12,177.53);
\definecolor[named]{drawColor}{rgb}{0.00,0.00,0.00}

\path[draw=drawColor,line width= 1.1pt,line join=round] (421.98, 50.04) --
	(431.11, 51.76) --
	(440.23, 53.99) --
	(449.35, 56.78) --
	(458.47, 60.52) --
	(467.60, 65.38) --
	(476.72, 71.60) --
	(485.84, 79.18) --
	(494.96, 88.16) --
	(504.08, 98.23) --
	(513.21,108.27) --
	(522.33,118.01) --
	(531.45,126.86) --
	(540.57,134.63) --
	(549.70,141.02) --
	(558.82,145.51);

\path[draw=drawColor,line width= 1.1pt,dash pattern=on 2pt off 2pt ,line join=round] (421.98, 58.85) --
	(431.11, 59.32) --
	(440.23, 60.41) --
	(449.35, 62.36) --
	(458.47, 65.43) --
	(467.60, 69.62) --
	(476.72, 74.92) --
	(485.84, 81.21) --
	(494.96, 88.57) --
	(504.08, 96.86) --
	(513.21,105.41) --
	(522.33,114.26) --
	(531.45,122.97) --
	(540.57,131.42) --
	(549.70,139.20) --
	(558.82,145.29);

\path[draw=drawColor,line width= 1.1pt,dash pattern=on 4pt off 2pt ,line join=round] (421.98, 67.62) --
	(431.11, 69.87) --
	(440.23, 73.06) --
	(449.35, 77.32) --
	(458.47, 82.91) --
	(467.60, 89.64) --
	(476.72, 97.41) --
	(485.84,105.92) --
	(494.96,115.07) --
	(504.08,124.37) --
	(513.21,132.76) --
	(522.33,140.07) --
	(531.45,145.99) --
	(540.57,150.60) --
	(549.70,154.02) --
	(558.82,156.31);
\definecolor[named]{drawColor}{rgb}{0.50,0.50,0.50}

\path[draw=drawColor,line width= 0.6pt,line join=round,line cap=round] (405.56, 34.03) rectangle (566.12,177.53);
\end{scope}
\begin{scope}
\path[clip] (  0.00,  0.00) rectangle (578.16,361.35);
\definecolor[named]{drawColor}{rgb}{0.50,0.50,0.50}
\definecolor[named]{fillColor}{rgb}{0.80,0.80,0.80}

\path[draw=drawColor,line width= 0.2pt,line join=round,line cap=round,fill=fillColor] ( 39.69,336.67) rectangle (200.24,349.31);
\definecolor[named]{drawColor}{rgb}{0.00,0.00,0.00}

\node[text=drawColor,anchor=base,inner sep=0pt, outer sep=0pt, scale=  0.96] at (119.96,339.68) {Fold 1};
\end{scope}
\begin{scope}
\path[clip] (  0.00,  0.00) rectangle (578.16,361.35);
\definecolor[named]{drawColor}{rgb}{0.50,0.50,0.50}
\definecolor[named]{fillColor}{rgb}{0.80,0.80,0.80}

\path[draw=drawColor,line width= 0.2pt,line join=round,line cap=round,fill=fillColor] (222.63,336.67) rectangle (383.18,349.31);
\definecolor[named]{drawColor}{rgb}{0.00,0.00,0.00}

\node[text=drawColor,anchor=base,inner sep=0pt, outer sep=0pt, scale=  0.96] at (302.90,339.68) {Fold 2};
\end{scope}
\begin{scope}
\path[clip] (  0.00,  0.00) rectangle (578.16,361.35);
\definecolor[named]{drawColor}{rgb}{0.50,0.50,0.50}
\definecolor[named]{fillColor}{rgb}{0.80,0.80,0.80}

\path[draw=drawColor,line width= 0.2pt,line join=round,line cap=round,fill=fillColor] (405.56,336.67) rectangle (566.12,349.31);
\definecolor[named]{drawColor}{rgb}{0.00,0.00,0.00}

\node[text=drawColor,anchor=base,inner sep=0pt, outer sep=0pt, scale=  0.96] at (485.84,339.68) {Fold 3};
\end{scope}
\begin{scope}
\path[clip] (  0.00,  0.00) rectangle (578.16,361.35);
\definecolor[named]{drawColor}{rgb}{0.50,0.50,0.50}
\definecolor[named]{fillColor}{rgb}{0.80,0.80,0.80}

\path[draw=drawColor,line width= 0.2pt,line join=round,line cap=round,fill=fillColor] ( 39.69,177.53) rectangle (200.24,190.16);
\definecolor[named]{drawColor}{rgb}{0.00,0.00,0.00}

\node[text=drawColor,anchor=base,inner sep=0pt, outer sep=0pt, scale=  0.96] at (119.96,180.54) {Fold 4};
\end{scope}
\begin{scope}
\path[clip] (  0.00,  0.00) rectangle (578.16,361.35);
\definecolor[named]{drawColor}{rgb}{0.50,0.50,0.50}
\definecolor[named]{fillColor}{rgb}{0.80,0.80,0.80}

\path[draw=drawColor,line width= 0.2pt,line join=round,line cap=round,fill=fillColor] (222.63,177.53) rectangle (383.18,190.16);
\definecolor[named]{drawColor}{rgb}{0.00,0.00,0.00}

\node[text=drawColor,anchor=base,inner sep=0pt, outer sep=0pt, scale=  0.96] at (302.90,180.54) {Fold 5};
\end{scope}
\begin{scope}
\path[clip] (  0.00,  0.00) rectangle (578.16,361.35);
\definecolor[named]{drawColor}{rgb}{0.50,0.50,0.50}
\definecolor[named]{fillColor}{rgb}{0.80,0.80,0.80}

\path[draw=drawColor,line width= 0.2pt,line join=round,line cap=round,fill=fillColor] (405.56,177.53) rectangle (566.12,190.16);
\definecolor[named]{drawColor}{rgb}{0.00,0.00,0.00}

\node[text=drawColor,anchor=base,inner sep=0pt, outer sep=0pt, scale=  0.96] at (485.84,180.54) {Fold 6};
\end{scope}
\begin{scope}
\path[clip] (  0.00,  0.00) rectangle (578.16,361.35);
\definecolor[named]{drawColor}{rgb}{0.00,0.00,0.00}

\node[text=drawColor,anchor=base east,inner sep=0pt, outer sep=0pt, scale=  0.96] at ( 32.57,199.50) {0.6};

\node[text=drawColor,anchor=base east,inner sep=0pt, outer sep=0pt, scale=  0.96] at ( 32.57,230.56) {0.7};

\node[text=drawColor,anchor=base east,inner sep=0pt, outer sep=0pt, scale=  0.96] at ( 32.57,261.62) {0.8};

\node[text=drawColor,anchor=base east,inner sep=0pt, outer sep=0pt, scale=  0.96] at ( 32.57,292.68) {0.9};

\node[text=drawColor,anchor=base east,inner sep=0pt, outer sep=0pt, scale=  0.96] at ( 32.57,323.74) {1.0};
\end{scope}
\begin{scope}
\path[clip] (  0.00,  0.00) rectangle (578.16,361.35);
\definecolor[named]{drawColor}{rgb}{0.00,0.00,0.00}

\node[text=drawColor,anchor=base east,inner sep=0pt, outer sep=0pt, scale=  0.96] at (215.51,199.50) {0.6};

\node[text=drawColor,anchor=base east,inner sep=0pt, outer sep=0pt, scale=  0.96] at (215.51,230.56) {0.7};

\node[text=drawColor,anchor=base east,inner sep=0pt, outer sep=0pt, scale=  0.96] at (215.51,261.62) {0.8};

\node[text=drawColor,anchor=base east,inner sep=0pt, outer sep=0pt, scale=  0.96] at (215.51,292.68) {0.9};

\node[text=drawColor,anchor=base east,inner sep=0pt, outer sep=0pt, scale=  0.96] at (215.51,323.74) {1.0};
\end{scope}
\begin{scope}
\path[clip] (  0.00,  0.00) rectangle (578.16,361.35);
\definecolor[named]{drawColor}{rgb}{0.00,0.00,0.00}

\node[text=drawColor,anchor=base east,inner sep=0pt, outer sep=0pt, scale=  0.96] at (398.45,199.50) {0.6};

\node[text=drawColor,anchor=base east,inner sep=0pt, outer sep=0pt, scale=  0.96] at (398.45,230.56) {0.7};

\node[text=drawColor,anchor=base east,inner sep=0pt, outer sep=0pt, scale=  0.96] at (398.45,261.62) {0.8};

\node[text=drawColor,anchor=base east,inner sep=0pt, outer sep=0pt, scale=  0.96] at (398.45,292.68) {0.9};

\node[text=drawColor,anchor=base east,inner sep=0pt, outer sep=0pt, scale=  0.96] at (398.45,323.74) {1.0};
\end{scope}
\begin{scope}
\path[clip] (  0.00,  0.00) rectangle (578.16,361.35);
\definecolor[named]{drawColor}{rgb}{0.00,0.00,0.00}

\node[text=drawColor,anchor=base east,inner sep=0pt, outer sep=0pt, scale=  0.96] at ( 32.57, 40.36) {0.6};

\node[text=drawColor,anchor=base east,inner sep=0pt, outer sep=0pt, scale=  0.96] at ( 32.57, 71.42) {0.7};

\node[text=drawColor,anchor=base east,inner sep=0pt, outer sep=0pt, scale=  0.96] at ( 32.57,102.48) {0.8};

\node[text=drawColor,anchor=base east,inner sep=0pt, outer sep=0pt, scale=  0.96] at ( 32.57,133.54) {0.9};

\node[text=drawColor,anchor=base east,inner sep=0pt, outer sep=0pt, scale=  0.96] at ( 32.57,164.60) {1.0};
\end{scope}
\begin{scope}
\path[clip] (  0.00,  0.00) rectangle (578.16,361.35);
\definecolor[named]{drawColor}{rgb}{0.00,0.00,0.00}

\node[text=drawColor,anchor=base east,inner sep=0pt, outer sep=0pt, scale=  0.96] at (215.51, 40.36) {0.6};

\node[text=drawColor,anchor=base east,inner sep=0pt, outer sep=0pt, scale=  0.96] at (215.51, 71.42) {0.7};

\node[text=drawColor,anchor=base east,inner sep=0pt, outer sep=0pt, scale=  0.96] at (215.51,102.48) {0.8};

\node[text=drawColor,anchor=base east,inner sep=0pt, outer sep=0pt, scale=  0.96] at (215.51,133.54) {0.9};

\node[text=drawColor,anchor=base east,inner sep=0pt, outer sep=0pt, scale=  0.96] at (215.51,164.60) {1.0};
\end{scope}
\begin{scope}
\path[clip] (  0.00,  0.00) rectangle (578.16,361.35);
\definecolor[named]{drawColor}{rgb}{0.00,0.00,0.00}

\node[text=drawColor,anchor=base east,inner sep=0pt, outer sep=0pt, scale=  0.96] at (398.45, 40.36) {0.6};

\node[text=drawColor,anchor=base east,inner sep=0pt, outer sep=0pt, scale=  0.96] at (398.45, 71.42) {0.7};

\node[text=drawColor,anchor=base east,inner sep=0pt, outer sep=0pt, scale=  0.96] at (398.45,102.48) {0.8};

\node[text=drawColor,anchor=base east,inner sep=0pt, outer sep=0pt, scale=  0.96] at (398.45,133.54) {0.9};

\node[text=drawColor,anchor=base east,inner sep=0pt, outer sep=0pt, scale=  0.96] at (398.45,164.60) {1.0};
\end{scope}
\begin{scope}
\path[clip] (  0.00,  0.00) rectangle (578.16,361.35);
\definecolor[named]{drawColor}{rgb}{0.00,0.00,0.00}

\node[text=drawColor,anchor=base,inner sep=0pt, outer sep=0pt, scale=  0.96] at ( 46.98, 20.31) {0};

\node[text=drawColor,anchor=base,inner sep=0pt, outer sep=0pt, scale=  0.96] at ( 74.35, 20.31) {3};

\node[text=drawColor,anchor=base,inner sep=0pt, outer sep=0pt, scale=  0.96] at (101.72, 20.31) {6};

\node[text=drawColor,anchor=base,inner sep=0pt, outer sep=0pt, scale=  0.96] at (129.08, 20.31) {9};

\node[text=drawColor,anchor=base,inner sep=0pt, outer sep=0pt, scale=  0.96] at (156.45, 20.31) {12};

\node[text=drawColor,anchor=base,inner sep=0pt, outer sep=0pt, scale=  0.96] at (183.82, 20.31) {15};
\end{scope}
\begin{scope}
\path[clip] (  0.00,  0.00) rectangle (578.16,361.35);
\definecolor[named]{drawColor}{rgb}{0.00,0.00,0.00}

\node[text=drawColor,anchor=base,inner sep=0pt, outer sep=0pt, scale=  0.96] at (229.92, 20.31) {0};

\node[text=drawColor,anchor=base,inner sep=0pt, outer sep=0pt, scale=  0.96] at (257.29, 20.31) {3};

\node[text=drawColor,anchor=base,inner sep=0pt, outer sep=0pt, scale=  0.96] at (284.66, 20.31) {6};

\node[text=drawColor,anchor=base,inner sep=0pt, outer sep=0pt, scale=  0.96] at (312.02, 20.31) {9};

\node[text=drawColor,anchor=base,inner sep=0pt, outer sep=0pt, scale=  0.96] at (339.39, 20.31) {12};

\node[text=drawColor,anchor=base,inner sep=0pt, outer sep=0pt, scale=  0.96] at (366.76, 20.31) {15};
\end{scope}
\begin{scope}
\path[clip] (  0.00,  0.00) rectangle (578.16,361.35);
\definecolor[named]{drawColor}{rgb}{0.00,0.00,0.00}

\node[text=drawColor,anchor=base,inner sep=0pt, outer sep=0pt, scale=  0.96] at (412.86, 20.31) {0};

\node[text=drawColor,anchor=base,inner sep=0pt, outer sep=0pt, scale=  0.96] at (440.23, 20.31) {3};

\node[text=drawColor,anchor=base,inner sep=0pt, outer sep=0pt, scale=  0.96] at (467.60, 20.31) {6};

\node[text=drawColor,anchor=base,inner sep=0pt, outer sep=0pt, scale=  0.96] at (494.96, 20.31) {9};

\node[text=drawColor,anchor=base,inner sep=0pt, outer sep=0pt, scale=  0.96] at (522.33, 20.31) {12};

\node[text=drawColor,anchor=base,inner sep=0pt, outer sep=0pt, scale=  0.96] at (549.70, 20.31) {15};
\end{scope}
\begin{scope}
\path[clip] (  0.00,  0.00) rectangle (578.16,361.35);
\definecolor[named]{drawColor}{rgb}{0.00,0.00,0.00}

\node[text=drawColor,anchor=base,inner sep=0pt, outer sep=0pt, scale=  1.20] at (302.90,  3.01) {Time (years)};
\end{scope}
\begin{scope}
\path[clip] (  0.00,  0.00) rectangle (578.16,361.35);
\definecolor[named]{drawColor}{rgb}{0.00,0.00,0.00}

\node[text=drawColor,rotate= 90.00,anchor=base,inner sep=0pt, outer sep=0pt, scale=  1.20] at ( 14.29,185.35) {Time-dependent AUC};
\end{scope}
\end{tikzpicture}
}
	\label{fig:auc}
\end{figure}
\newpage
%%%%%%%%%%%%%%%%%%%%%%

%%%%% Conclusion %%%%%
\section*{Conclusion}
\label{conclusion}

Interational economic sanctions are not disappearing from global politics anytime soon. Indeed, in response to the current crisis in the Ukraine, the European Union and the United States are both launching sanction intiatives against Russia, and Russia is now poised to respond with retaliative sanctions. Such an exchange highlights one key theme of this paper: sanctions are driven by the interdependent nature of the international system. 

 We have outlined both theoretical and empirical reasons for why sanctioning behavior between states constitutes a network, and thus requires scholars to incorporate network attributes into the study of sanction compliance. We have then demonstrated the key role of reciprocity in determining the duration of economic sanctions in the international network, while also assessing longstanding hypothesis from the literature. In doing so, we are able to construct a more accurate representation of sanction dynamics than has yet been presented in the literature. 

We find strong support for the influence of reciprocal compliance as well as the number of sender states within the sanction network, suggesting that the most effective sanctions are likely to be those composed of a higher number of senders with this shared compliance history. Similarly, we highlight a previously underexamined aspect of sanction behavior and show that countries whom have sanctioned each other in the past without complying to one another are not liekly to comply to another in the present. We also find support that bolsters previous claims from the literature: even when accounting for network effects, trade partners are more likely to successfully utilize sanctions against target states.  




\newpage
\section*{Appendix}
\label{appendix}

\subsection*{Imputation Procedure}

The copula based approach developed by \citet{hoff:2007} is estimated through a Markov chain Monte Carlo (MCMC) algorithm. We run the MCMC for 6,000 imputations, saving every sixth imputation, using the \texttt{sbgcop.mcmc} function in the \texttt{sbgcop} package in $\mathcal{R}$. To account for time trends and obtain better performance from this imputaiton procedure, we create five lags of each variable, except for polity, prior to imputation. Every imputation of the MCMC leads to the creation of one dataset with all missing values imputed. Running this algorithm on our dataset then produces a total of 1,000 imputed datasets. Results across these 1,000 imputed datasets are then averaged, thereby accounting for a portion of the uncertainty in the imputed values. We then used the average of the results from these 1,000 imputed datasets to generate the regression estimates in table \ref{tab:regResults}. 

Regression results on the unimputed dataset are shown in table \ref{tab:regResultsNoImp}. The results, particularly for our reciprocity variables, are nearly identical. 

% latex table generated in R 3.1.2 by xtable 1.7-4 package
% Sun May 17 06:32:03 2015
\begin{table}[ht]
\centering
{\normalsize
\begin{tabular}{lccc}
 Variable & Model 1 & Model 2 & Model 3 \\ 
  \hline
\hline
Compliance Reciprocity$_{j,t-1}$ &  &  & $0.36^{\ast\ast}$ \\ 
   &  &  & (0.114) \\ 
  Sanction Reciprocity$_{j,t-1}$ &  &  & $-0.155^{\ast\ast}$ \\ 
   &  &  & (0.058) \\ 
   \hline
Number of Senders$_{j,t}$ &  & $0.439^{\ast\ast}$ & $0.429^{\ast\ast}$ \\ 
   &  & (0.092) & (0.093) \\ 
  Distance$_{j,t}$ &  & $1.314^{\ast\ast}$ & $1.345^{\ast\ast}$ \\ 
   &  & (0.295) & (0.3) \\ 
  Trade$_{j,t}$ &  & $80^{\ast}$ & $93.478^{\ast\ast}$ \\ 
   &  & (46.616) & (45.695) \\ 
  Ally$_{j,t}$ &  & -0.065 & -0.064 \\ 
   &  & (0.289) & (0.297) \\ 
   \hline
Polity$_{i,t-1}$ & 0.05 & $0.087^{\ast\ast}$ & $0.083^{\ast\ast}$ \\ 
   & (0.03) & (0.032) & (0.033) \\ 
  Ln(GDP per capita)$_{i,t-1}$ & $-0.393^{\ast\ast}$ & $-0.372^{\ast\ast}$ & $-0.337^{\ast\ast}$ \\ 
   & (0.103) & (0.126) & (0.131) \\ 
  GDP Growth$_{i,t-1}$ & 0.041 & 0.06 & 0.044 \\ 
   & (0.036) & (0.038) & (0.037) \\ 
  Population$_{i,t-1}$ & $-0.232^{\ast\ast}$ & -0.126 & -0.036 \\ 
   & (0.091) & (0.111) & (0.12) \\ 
  Internal Conflict$_{i,t-1}$ & -0.003 & 0.002 & 0 \\ 
   & (0.03) & (0.029) & (0.029) \\ 
   \hline
n & 3592 & 3579 & 3579 \\ 
  Events & 64 & 64 & 64 \\ 
  Likelihood ratio test & 21.25 (0) & 56.76 (0) & 66.98 (0) \\ 
   \hline
\hline
\end{tabular}
}
\caption{Duration model on unimputed data with time varying covariates estimated using Cox Proportional Hazards. Standard errors in parentheses. $^{**}$ and $^{*}$ indicate significance at $p< 0.05 $ and $p< 0.10 $, respectively.} 
\label{tab:regResultsNoImp}
\end{table}

\FloatBarrier

\newpage
%%%%%%%%%%%%%%%%%%%%%%

\newpage
\bibliographystyle{apsr}
\bibliography{magRefs.bib}

\end{document} 