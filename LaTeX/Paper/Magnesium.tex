\documentclass[12pt,onesided]{amsart} 

%%%%%%%%%%%%%%%%%%%%%%%%%%%%%%%%%%%%%%%%%%%%%%%%%%
%%%%%%%%%%%%%%%%%%%% PREAMBLE %%%%%%%%%%%%%%%%%%%%
%%%%%%%%%%%%%%%%%%%%%%%%%%%%%%%%%%%%%%%%%%%%%%%%%%


% -------------------- defaults -------------------- %
% load lots o' packages

% layout control
\usepackage{geometry}
\geometry{verbose,tmargin=1.25in,bmargin=1.25in,lmargin=1.1in,rmargin=1.1in}
\usepackage{rotating}
\usepackage{fancyhdr}
\usepackage{parallel}
\usepackage{parcolumns}

% math typesetting
\usepackage{array}
\usepackage{amsmath}
\usepackage{amssymb}
\usepackage{amsfonts}

% tables
\usepackage{tabularx}
\usepackage{booktabs}
\usepackage{multicol}
\usepackage{multirow}
\usepackage{longtable}

\usepackage[%
decimalsymbol=.,
digitsep=fullstop
]{siunitx}

% to adapt caption style
\usepackage[font={small},labelfont=bf]{caption}
 
% references
\usepackage[longnamesfirst]{natbib}
\bibpunct{(}{)}{;}{a}{}{,}
\usepackage{nameref}

% footnotes at bottom
\usepackage[bottom]{footmisc}

% to change enumeration symbols begin{enumerate}[(a)]
\usepackage{enumerate}

% to make enumerations and itemizations within paragraphs or
% lines. f.i. begin{inparaenum} for (a) is (b) and (c)
\usepackage{paralist}

% to colorize links in document. See color specification below
\usepackage[x11names]{xcolor}

% for multiple references and insertion of the word "figure" or "table"
% \usepackage{cleveref}

% load the hyper-references package and set document info
\usepackage[pdftex]{hyperref}

% graphics stuff
\usepackage{subfig}
\usepackage{graphicx}
\usepackage[space]{grffile} % allows us to specify directories that have spaces
\usepackage[section]{placeins} % prevents floats from moving past a \FloatBarrier or section
\usepackage{tikz}
% \usepackage{pgfplots}

% Spacing
\usepackage[doublespacing]{setspace}

% define clickable links and their colors
\hypersetup{
	unicode=false,          % non-Latin characters in Acrobat's bookmarks
	pdftoolbar=true,        % show Acrobat's toolbar?
	pdfmenubar=true,        % show Acrobat's menu?
	pdffitwindow=false,     % window fit to page when opened
	pdfstartview={FitH},    % fits the width of the page to the window
	pdfnewwindow=true,%
	pdfauthor={Cassy Dorff and Shahryar Minhas},%
	pdftitle={Title},%
	colorlinks,%
	citecolor=black,%
	filecolor=black,%
	linkcolor=black,%
	urlcolor=RoyalBlue4%
	}

% Including External Code
\usepackage{verbatim}
\usepackage{listings}
\lstset{
	language=R,
	basicstyle=\scriptsize\ttfamily,
	commentstyle=\ttfamily\color{gray},
	numbers=left,
	numberstyle=\ttfamily\color{gray}\footnotesize,
	stepnumber=1,
	numbersep=5pt,
	backgroundcolor=\color{white},
	showspaces=false,
	showstringspaces=false,
	showtabs=false,
	frame=single,
	tabsize=2,
	captionpos=b,
	breaklines=true,
	breakatwhitespace=false,
	title=\lstname,
	escapeinside={},
	keywordstyle={},
	morekeywords={}
	}

% -------------------------------------------------- %


% -------------------- title -------------------- %

\title{When Do States Say Uncle? Network Dependence and Sanction Compliance}
\date{Draft Copy \today}

\author[Dorff]{Cassy Dorff}
\address{Cassy Dorff: Department of Political Science}
\curraddr{Duke University, Durham, NC, 27708, USA}
\email{cassy.dorff@duke.edu}

\author[Minhas]{Shahryar Minhas}
\address{Shahryar Minhas: Department of Political Science}
\curraddr{Duke University, Durham, NC, 27708, USA}
\email{shahryar.minhas@duke.edu}

\thanks{This paper was prepared for 72$^{nd}$ annual Midwest Political Science Association Conference in Chicago, April 2-6 2014. We are grateful for comments on earlier versions of this paper received at the ISSS-ISAC Conference in Washington, D.C., October 4-6 2013. }

% \setlength{\headheight}{15pt}
% \setlength{\headsep}{20pt}
% \pagestyle{fancyplain}
 
% \fancyhf{}
 
% \lhead{\fancyplain{}{}}
% \chead{\fancyplain{}{}}
% \rhead{\fancyplain{}{}}
% \rfoot{\fancyplain{}{}}

% ----------------------------------------------- %


% -------------------- customizations -------------------- %

% define the includegraphics search path
% \graphicspath{{Graphics/}}

% easy commands for number propers
\newcommand{\first}{$1^{\text{st}}$}
\newcommand{\second}{$2^{\text{nd}}$}
\newcommand{\third}{$3^{\text{rd}}$}
\newcommand{\nth}[1]{${#1}^{\text{th}}$}

% easy command for boldface math symbols
\newcommand{\mbs}[1]{\boldsymbol{#1}}

% define bibliography style
%\bibliographystyle{/Users/janus829/Documents/APSR}

   % \graphicspath{{/Users/cassydorff/Dropbox/Research/Magnesium/Graphics/}}
\graphicspath{{/Users/janus829/Dropbox/Research/Magnesium/Graphics/}}

% -------------------------------------------------------- %


%%%%%%%%%%%%%%%%%%%%%%%%%%%%%%%%%%%%%%%%%%%%%%%%%%
%%%%%%%%%%%%%%%%%%%% DOCUMENT %%%%%%%%%%%%%%%%%%%%
%%%%%%%%%%%%%%%%%%%%%%%%%%%%%%%%%%%%%%%%%%%%%%%%%%

\doublespacing 

\begin{document}

\maketitle\thispagestyle{empty}

\begin{abstract}

\singlespacing{In this paper we investigate when and why states comply with economic sanctions.  The literature has suggested two ways of modeling this phenomenon. First, previous literature has suggested a duration modeling approach to capture the time it takes for a sanction to ``work". This approach, however, has failed to carefully account for important dynamics relevant to the modeling of sanction outcomes. Namely, present duration approaches fail to incorporate the network pressures instrinstic to international sanction processes. Target states face a network of sanctioners, not just an individual sender state. We present a duration model that incorporates the interdependent nature of the international system by assessing how the number of senders as well as the connectivity between senders and receivers, influences sanciton outcomes. In doing so we are able to test two prominent hypotheses from the literature: (1) does dependence between the target state and its sanctioning network decrease the time until target compliance; and (2) are democratic states more susceptible to these network pressures than others? We demonstrate the necessity of incorporating network dynamics into understanding the time until sanction compliance and show that the connectivity between target and sender states--in terms of cultural similarities, geographical proximity, and alliance patterns--plays an important and previously overlooked role on sanction outcomes.}

\end{abstract}

\newpage\setcounter{page}{1}

%%%%%%%% INTRO %%%%%%%%
\section{Introduction}
\label{intro}

I completely agree that understanding why states try to string together as broad a base as possible when constructing sanctions is key. But we should take that a step further to understanding if there are a set of states which are crucial to determining the power of sanctions in bringing about change. For example, the US and EU put sanctions on N. Korea but what do they care, they have no strong economic ties with them nor do they share any alliances with them, etc. But if China were to join then N. Korea may begin to pay attention because it relies on China for economic support and in many other ways. This is what I was trying to get at with the economic and reputational argument. The idea that sanctions are not just about finding ways to harm a state directly but also about isolating it from the rest of the international system. 

From the economic side, the US will try to get as many of Magnesia's trading partners to join the sanctions because it knows that's the best way to make Magnesia really feel the pain of the sanctions. For example, with Iran the US spent a good deal of effort roping in countries like China and Russia because they were Iran's major export market. Having them participate in sanctions deprives Iran of important sources of income for their economy. 

From the reputational side, I was being completely unclear, lol. But I was using it as a catch-all for all the other non-economic ties that bind states - so forget about phrasing it as reputation. Also I wasn't trying to imply that there would be costs to sanction joiners from the sanctioned state in the future, there very well might be but I think that's a future question. All I am saying is that there are non-economic ties that bind states such as things like shared cultural heritage and maybe even geographic proximity. When states with these close ties join in sanctions against poor, old Magnesia then that would do more damage to regime stability in Magnesia than if the sanctioning states were just a bunch of gun-toting hillbillies a world away. 

I agree that this latter argument is underdeveloped but I think it would be interesting for us to be able to talk about the proximity of states on some space other than simply geographic, or economic. For example, don't we think that sanctions on Syria from the US mean less to it than sanctions from Saudi Arabia and that sanctions from Saudi Arabia mean less to it than sanctions from Iran? I don't think that we can capture that type of variation with purely an economic variable. 

In terms of types of sanctions yeah I think we should be as broad as possible but the key point here is that sanctions, at least from our network perspective, are only effective if they can isolate the target state from those that it relies on or considers as its peers. 

We haven't at all talked about the all important monadic level here yet either...countries like Syria, N Korea, and Iran feel free to plough forward with their plans despite a bevy of sanctions from multiple countries. Additionally, Iran saw no regime change despite years of sanctions from everybody. Last there is a major sample bias problem as well. Maybe target states always back down before its close economic and other peer countries sign onto sanctions, so just the threat of them joining in the sanctions is enough to make the target state back down. 
%%%%%%%%%%%%%%%%%%%%%%%

%%%%% Lit Review %%%%%
\subsection{Literature Review}
\label{lit}

\begin{itemize}
	\item Domestic factors: marinov 2005, Lektzian and Souva 2003
	\item Duration modeling: Bolks Al Sowayel 2000, McGillivray and Stam 2004
	\item Network: Bapat and Morgan 2009, Cranmer and Heinrich 2013
	\item Networks in IR: Ward and Cranmer etc. 
\end{itemize}
%%%%%%%%%%%%%%%%%%%%%%

%%%%% Theory %%%%%
% NOTE FROM SM: Lacking a theory here is really hurting us. We should consider adding a case on S. Africa or some other country so that we can actually sketch out a theory of network pressure and sanction compliance. 

%goal here is to go over how we are constructing our network measures and what network measures we use and data sources for those measures
%cassy in her lit review has already gone into explaining why networks matter and that hte lit hasnt incorp'd them so i dont need to rehash that point

\section*{Accounting for Network Effects}
\label{neteffects}

In this section, we present our argument for incorporating features of the sanction network into models for predicting the time until sanction compliance and describe our approach for capturing these features. In focusing primarily on domestic factors, as much of the extant literature has done, alternative explanations that incorporate external factors relating to the sanctioning network have been ignored. The two explanations that we focus on are, first, the types of influential relationships senders have to receiver states and, second, the network pressures created at the aggregate network level, across all sanctions in any given year. %CD: this "second" part could use something snappier, I am not sure I am summarizing it very well (?)

To illustrate the network characteristics essential to our argument, we begin with an illustration of the year 1984. Figure \ref{fig:spaghetti} depicts the network of sanction cases, and threats thereof, ongoing and initiated by 1984. Nodes represent states and the directed edges denote the sender and receiver of sanctions. This figure is complex, demonstrating that numerous states are involved in mulitple sanction cases during this one year.
%threats? we need to explain this a bit more, I don't really get what that means here.

\begin{figure}[ht]
  \centering
  \begin{tabular}{c}
	  \includegraphics[width=1\textwidth]{84net-crop} \\
	  \includegraphics[width=0.45\textwidth]{MapLegend}
  \end{tabular}
  \caption{Here we show the sanction network in 1984, nodes are colored by geographic coordinates of countries. Data for sanction cases comes from \citet{morgan2009threat}.}
  \label{fig:spaghetti}
\end{figure}
\FloatBarrier

Next, we deconstruct this network and narrow our focus to observe the indivdual country case of South Africa in 1984. In this case, South Africa is the target of multiple sanctions, as shown in Figure \ref{fig:saneti}. Because of this, we construct a sanction network that represents each sanction South Africa faces during this year. As one can easily see, in most cases during 1984, South Africa faces more than one sanctioner, and these sanctioners vary across each sanction case. For example, in the first network graph, in the top left of Figure \ref{fig:saneti}, we see that South Africa faces a sanction from India, Pakistan, and Jamaica. Yet in the top right network graph, we see that South Africa also faces a sanction from Canada, Sweden, the USA, Finland, and Austrialia. 

\begin{itemize}
	\item Average minimum distance between sender(s) and receiver from the Cshapes Dataset \citep{weidmann2010geography}
	\item Average level of trade between sender(s) and receiver and from the COW Dataset \citep{barbieri2009trading}
	\item Proportion of sender(s) that are allied with the receiver from the COW Dataset \citep{gibler2009international}
	\item Average Number of IGOs common between sender(s) and receiver from the COW Dataset \citep{pevehouse2004correlates}
	\item Similarity in the makeup of religious groups\footnote{To determine similarity we calculate the pearson correlation coefficient between the proportion of the population in various religious groups defined in the COW dataset for each country. For a single year this provides us with an $N \times N$ matrix where each $i-j$ cross-section represents how similar the populations of $i$ and $j$ are in terms of their religious denominations. We then add 1 to each of these scores so that the minimum value within a cross-section is 0 and the maximum is 2. We do this for every year providing us with $N \times N \times T$ matrices.} between sender(s) and receiver from the COW Dataset \citep{maoz2013world}
\end{itemize}

\begin{figure}[ht]
	\centering
	\includegraphics[width=1\textwidth]{saneti}
	\caption{Here we show a separate network for each sanction case that South Africa faced in 1984.}
	\label{fig:saneti}
\end{figure}
\FloatBarrier

Clearly these two networks are composed of a diverse set of unique actors, all of which have a specific relationship with the target state. We conceptualize these relationships as composed of ``pressures'' which likely influences the behavior of the target state. We present two hypotheses which focus on the sanction case network. First, it is intutitive that the number of senders likely influences the willingness of the target state to comply because as the number of senders increases, the more constraints through multple relationships the target faces. The essential idea is that handling the demands of ten relationships is more influential than one. However, we also expect that these relationships must be meaningful not juste plentiful. Just as one would imagine that a person is less swayed by the demands of 10 strangers than the demands of a few close friends, we conceptualize senders as most influential when they interact with the target state on a number of dimensions.  Thus, for each sanction case we determine the number of senders. We also calculate and control for the average number of other sanctions being sent by the senders of each particular sanction case.

\begin{quote}
	\textbf{H1}: As the number of sender states increases for any given sanction case, the time to compliance will decrease. 
\end{quote}

\begin{quote}
	\textbf{H2}: Sanction cases where relationships between sender(s) and receiver(s) are more proximate will be more quickly resolved.
\end{quote}

 Next, we describe what exactly is meant behind our concept of ``proximity.'' In figure \ref{fig:saneti}, we show the six sanction cases faced by South Africa in 1984.\footnote{Data for sanction cases is from \citet{morgan2009threat}.} For the most part, each sanction case involves a variety of actors with whom South Africa has differing cultural, geographic, diplomatic, and economic relationships. Within any individual sanction case we hypothesize (H2) that the proximity, (i.e. the ways in which the sender and target interact on a number of dimensions) of relationships between sender(s) of a sanction and a receiver influence whether a target state complies. To test this idea that the normative closeness, or general ``proximity''of relationships between sender(s) and receiver(s) in predicting sanction compliance we construct a number of covariates. 

We focus on five key measures of the ``proximate'' nature of relationships First, we measure the average distance between sender(s) and reciever. Next we utilize the Correlates of War (COW) data to construct variour measures. Our second covariate relating to proximity is trade, which we measure as the total share of trade that sender states accounts for. Third, we measure alliances as the proportion of sender(s) that are allied with the receiver. Forth, we measure the average number of common IGOs that the sender(s) and target state belong to. And Last, also using COW, we create a measure of similarity in relgiion across sender relationships.
 We construct this measure using the COW data. 

% Sanction Case Network: The relationship between sender(s) and the target matters for sanction compliance. Sanctions involving coalitions of sender(s) will be more quickly resolved than sanctions sent by just one state. Sanction cases where relationships are more proximate will be more quickly resolved.

	%\begin{itemize}
		%\item Mean Number of other sanctions being sent by senders
		% we should just write that as a control somewhere, right? 
	%	\item Distance: The average distance between sender(s) and the receiver (cshapes)
	%	\item Trade: The share of total trade that the sender(s) make up for the receiver (cow trade)
	%	\item Alliances: The proportion of sender(s) that are allied with the receiver (cow ally)
	%	\item IGOs: The average number of common IGOs that the sender(s) and receiver belong to (cow igos)
	%	\item Religion: Similarity of religious group makeups between sender(s) and the receiver (cow religion)
	%\end{itemize}
  
Thus far we have explained how the relationships between sanctioners and target states matter for predicting compliance. This view has focused on the individual sanction case. However, the six separate sanctions that South Africa faced in 1984 can also be thought of as a yearly sanction case network. In figure \ref{fig:sanet}, we aggregate the six sanction networks into one where each separate sanction is denoted by a differing color. Here we hypothesize that states under the pressure of a multitude of sanctions will more quickly resolve sanction cases than those facing only a few.\footnote{Our next step in this project is to include measures of reciprocity over time. This will allow us to test the argument presented in the earlier half of the paper where we suggest accumulated dependencies over time will influence the likeilhood of compliance (e.g. reciprocity: if country $i$ often complies with country $j$, will country $j$ be more likely to comply with country $i$?}

% Aggregate Network: Targets of sanctions often face a multitude of sanction cases at any given point in time. States under the pressure of a multitude of sanctions will more quickly resolve sanction cases than those facing only a few.

\begin{quote}
	\textbf{H3}: States facing the pressure of a multitude of sanctions will more quickly resolve any one of those sanction cases.
\end{quote}

\begin{figure}[ht]
	\centering
	\includegraphics[width=0.75\textwidth]{sanet}
	\caption{All sanctions faced by South Africa in 1984 collapsed to one network}
	\label{fig:sanet}
\end{figure}
\FloatBarrier

% We conceptualize network dynamics in terms of ``pressures.'' First how the interrelationships, e.g., cultural or geographic proximity, and interactions between states, e.g., trade, affect the time until compliance. Second, that target states, receivers, often face multiple sanctions from multiple senders at any given time. Last, we incorporate a few of the prominent target-focused explanations for sanction compliance. 
% Third, we consider that reciprocal compliance occurs over time between states within the network. We incorporate all three of these relational effects into our duration model.  NOTE FROM SM: should probably leave this out until we actually do it

% Though we hypothesize that network effects matter greatly for determining sanction compliance there are characteristics of sanctioned states that may better enable them to manage these pressures. 
% Third, we consider that reciprocal compliance occurs over time between states within the network. We incorporate all three of these relational effects into our duration model.

%%%%%%%%%%%%%%%%%%%%%%

%%%%% Empirics %%%%%
\section*{Empirics}
\label{empirics}

To test the effects of network pressures on sanction compliance we use the Threat and Imposition of Sanctions (TIES) Database developed by \citet{morgan2009threat}. This database includes over 1,400 sanction case initiations and outcomes from 1945 to 2005. Our focus here is restricted to sanctions that are prompted as the result of an economic issue. The TIES database categorizes the issue(s) involved in the threat or impositions of sanctions, we focus on four:

\begin{itemize}
	\item Release citizens, property, or material
	\item Improve environmental policies
	\item Trade practices
	\item Implement economic reform
\end{itemize}

Restricting our analysis to sanctions stemming from these issues during the period of 1984 to 2005 leaves us with 184 sanction cases. Our unit of analysis is the sanction case-year, providing us with a total of 1,920 observations. Our dependent variable measures whether states are complying. 


We define compliance as:
	\begin{itemize}
		\item Complete/Partial Acquiescence by Target to threat
		\item Negotiated Settlement
		\item Total/Partial Acquiescence by the Target State following sanctions imposition
		\item Negotiated Settlement following sanctions imposition
	\end{itemize}
	
\subsection{Modeling Approach} 
\begin{align*}
		Compliance_{i,t} =\; & No. \; Senders_{j} + Distance_{j,t} + Trade_{j,t}  + \\
		 &Ally_{j,t} + IGOs_{j,t} + Religion_{j,t} +\\
 		 &Sanc. \; Rec'd_{i,t} + \\
		 &Constraints_{i,t} + GDP \; Capita_{i,t-1} +\\
		 & Internal \; Conflict_{i,t} +\\
		 &Constraints_{i,t}*No. \; Senders_{j} + \epsilon_{i,t}
	\end{align*}
	
\begin{itemize}
	\item $i$ represents the target of the sanction
	\item $j$ represents the relationship between the set of sender(s) for a particular sanction case and $i$
	\item $t$ the time period
\end{itemize}

%%%%%%%%%%%%%%%%%%%%%%

%%%%% Empirics %%%%%
\section{Results}
\label{Results} 

Table~\ref{table1} displays the results from our model. As expected, we find support for hypothesis one, which states that sanction cases with more proximate relationships between the sender countries will end sooner. 

% latex table generated in R 3.1.0 by xtable 1.7-3 package
% Sat Aug  2 12:32:14 2014
\begin{table}[ht]
\centering
{\normalsize
\begin{tabular}{lccc}
 Variable & Model 1 & Model 2 & Model 3 \\ 
  \hline
\hline
Compliance Reciprocity$_{j,t-1}$ &  &  & $0.212^{\ast\ast}$ \\ 
   &  &  & (0.063) \\ 
  Sanction Reciprocity$_{j,t-1}$ &  &  & $-0.103^{\ast\ast}$ \\ 
   &  &  & (0.033) \\ 
   \hline
Number of Senders$_{j,t}$ &  & $0.255^{\ast\ast}$ & $0.234^{\ast\ast}$ \\ 
   &  & (0.062) & (0.062) \\ 
  Distance$_{j,t}$ &  & $0.531^{\ast\ast}$ & $0.497^{\ast\ast}$ \\ 
   &  & (0.183) & (0.187) \\ 
  Trade$_{j,t}$ &  & $16.795^{\ast}$ & $16.106^{\ast}$ \\ 
   &  & (9.393) & (9.398) \\ 
  Ally$_{j,t}$ &  & 0.029 & 0.096 \\ 
   &  & (0.159) & (0.164) \\ 
   \hline
Polity$_{i,t-1}$ & $-0.029^{\ast}$ & -0.015 & -0.022 \\ 
   & (0.016) & (0.017) & (0.017) \\ 
  Ln(GDP per capita)$_{i,t-1}$ & -0.088 & -0.08 & -0.03 \\ 
   & (0.065) & (0.07) & (0.073) \\ 
  GDP Growth$_{i,t-1}$ & 0.011 & 0.02 & 0.015 \\ 
   & (0.018) & (0.019) & (0.019) \\ 
  Population$_{i,t-1}$ & $-0.172^{\ast\ast}$ & $-0.16^{\ast\ast}$ & $-0.102^{\ast\ast}$ \\ 
   & (0.039) & (0.043) & (0.05) \\ 
  Internal Conflict$_{i,t-1}$ & 0.001 & 0.004 & -0.001 \\ 
   & (0.017) & (0.017) & (0.017) \\ 
   \hline
n & 6517 & 6483 & 6483 \\ 
  Events & 191 & 190 & 190 \\ 
  Likelihood ratio test & 36.91 (0) & 59.55 (0) & 70.76 (0) \\ 
   \hline
\hline
\end{tabular}
}
\caption{Duration model with time varying covariates estimated using Cox Proportional Hazards. Standard errors in parentheses. $^{**}$ and $^{*}$ indicate significance at $p< 0.05 $ and $p< 0.10 $, respectively.} 
\label{tab:regResults}
\end{table}


\begin{figure}[ht]
	\centering
	\caption{Survival Probability by Number of Senders in a Sanction Case}
	% \includegraphics[width=1\textwidth]{nosSurv}
	\resizebox{1\textwidth}{!}{% Created by tikzDevice version 0.8.1 on 2015-05-16 17:34:47
% !TEX encoding = UTF-8 Unicode
\begin{tikzpicture}[x=1pt,y=1pt]
\definecolor{fillColor}{RGB}{255,255,255}
\path[use as bounding box,fill=fillColor,fill opacity=0.00] (0,0) rectangle (433.62,289.08);
\begin{scope}
\path[clip] (  0.00,  0.00) rectangle (433.62,289.08);
\definecolor{drawColor}{RGB}{0,0,0}

\path[draw=drawColor,line width= 0.4pt,line join=round,line cap=round] ( 49.20, 61.20) -- (408.42, 61.20);

\path[draw=drawColor,line width= 0.4pt,line join=round,line cap=round] ( 49.20, 61.20) -- ( 49.20, 55.20);

\path[draw=drawColor,line width= 0.4pt,line join=round,line cap=round] (109.07, 61.20) -- (109.07, 55.20);

\path[draw=drawColor,line width= 0.4pt,line join=round,line cap=round] (168.94, 61.20) -- (168.94, 55.20);

\path[draw=drawColor,line width= 0.4pt,line join=round,line cap=round] (228.81, 61.20) -- (228.81, 55.20);

\path[draw=drawColor,line width= 0.4pt,line join=round,line cap=round] (288.68, 61.20) -- (288.68, 55.20);

\path[draw=drawColor,line width= 0.4pt,line join=round,line cap=round] (348.55, 61.20) -- (348.55, 55.20);

\path[draw=drawColor,line width= 0.4pt,line join=round,line cap=round] (408.42, 61.20) -- (408.42, 55.20);

\node[text=drawColor,anchor=base,inner sep=0pt, outer sep=0pt, scale=  1.00] at ( 49.20, 39.60) {0};

\node[text=drawColor,anchor=base,inner sep=0pt, outer sep=0pt, scale=  1.00] at (109.07, 39.60) {5};

\node[text=drawColor,anchor=base,inner sep=0pt, outer sep=0pt, scale=  1.00] at (168.94, 39.60) {10};

\node[text=drawColor,anchor=base,inner sep=0pt, outer sep=0pt, scale=  1.00] at (228.81, 39.60) {15};

\node[text=drawColor,anchor=base,inner sep=0pt, outer sep=0pt, scale=  1.00] at (288.68, 39.60) {20};

\node[text=drawColor,anchor=base,inner sep=0pt, outer sep=0pt, scale=  1.00] at (348.55, 39.60) {25};

\node[text=drawColor,anchor=base,inner sep=0pt, outer sep=0pt, scale=  1.00] at (408.42, 39.60) {30};

\path[draw=drawColor,line width= 0.4pt,line join=round,line cap=round] ( 49.20, 67.82) -- ( 49.20,233.26);

\path[draw=drawColor,line width= 0.4pt,line join=round,line cap=round] ( 49.20, 67.82) -- ( 43.20, 67.82);

\path[draw=drawColor,line width= 0.4pt,line join=round,line cap=round] ( 49.20,109.18) -- ( 43.20,109.18);

\path[draw=drawColor,line width= 0.4pt,line join=round,line cap=round] ( 49.20,150.54) -- ( 43.20,150.54);

\path[draw=drawColor,line width= 0.4pt,line join=round,line cap=round] ( 49.20,191.90) -- ( 43.20,191.90);

\path[draw=drawColor,line width= 0.4pt,line join=round,line cap=round] ( 49.20,233.26) -- ( 43.20,233.26);

\node[text=drawColor,anchor=base east,inner sep=0pt, outer sep=0pt, scale=  1.00] at ( 37.20, 64.37) {0.2};

\node[text=drawColor,anchor=base east,inner sep=0pt, outer sep=0pt, scale=  1.00] at ( 37.20,105.74) {0.4};

\node[text=drawColor,anchor=base east,inner sep=0pt, outer sep=0pt, scale=  1.00] at ( 37.20,147.10) {0.6};

\node[text=drawColor,anchor=base east,inner sep=0pt, outer sep=0pt, scale=  1.00] at ( 37.20,188.46) {0.8};

\node[text=drawColor,anchor=base east,inner sep=0pt, outer sep=0pt, scale=  1.00] at ( 37.20,229.82) {1.0};
\end{scope}
\begin{scope}
\path[clip] (  0.00,  0.00) rectangle (433.62,289.08);
\definecolor{drawColor}{RGB}{0,0,0}

\node[text=drawColor,anchor=base,inner sep=0pt, outer sep=0pt, scale=  1.00] at (228.81, 15.60) {Time (Years)};

\node[text=drawColor,rotate= 90.00,anchor=base,inner sep=0pt, outer sep=0pt, scale=  1.00] at ( 10.80,150.54) {Survival Probability};
\end{scope}
\begin{scope}
\path[clip] ( 49.20, 61.20) rectangle (408.42,239.88);
\definecolor{drawColor}{RGB}{169,169,169}

\path[draw=drawColor,line width= 0.4pt,line join=round,line cap=round] ( 49.20,233.26) --
	( 61.17,233.26) --
	( 61.17,229.01) --
	( 73.15,229.01) --
	( 73.15,224.14) --
	( 85.12,224.14) --
	( 85.12,218.85) --
	( 97.10,218.85) --
	( 97.10,215.94) --
	(109.07,215.94) --
	(109.07,214.09) --
	(121.04,214.09) --
	(121.04,213.30) --
	(133.02,213.30) --
	(133.02,212.92) --
	(144.99,212.92) --
	(144.99,212.50) --
	(156.97,212.50) --
	(156.97,212.03) --
	(168.94,212.03) --
	(168.94,211.56) --
	(180.91,211.56) --
	(180.91,211.05) --
	(204.86,211.05) --
	(204.86,210.45) --
	(216.84,210.45) --
	(216.84,209.80) --
	(252.76,209.80) --
	(252.76,208.69) --
	(264.73,208.69) --
	(264.73,206.89) --
	(288.68,206.89) --
	(288.68,203.45) --
	(300.65,203.45) --
	(300.65,199.35) --
	(336.58,199.35) --
	(336.58,195.48) --
	(433.62,195.48);

\path[draw=drawColor,line width= 0.4pt,line join=round,line cap=round] ( 57.99,229.01) -- ( 64.36,229.01);

\path[draw=drawColor,line width= 0.4pt,line join=round,line cap=round] ( 61.17,225.83) -- ( 61.17,232.19);

\path[draw=drawColor,line width= 0.4pt,line join=round,line cap=round] ( 69.97,224.14) -- ( 76.33,224.14);

\path[draw=drawColor,line width= 0.4pt,line join=round,line cap=round] ( 73.15,220.95) -- ( 73.15,227.32);

\path[draw=drawColor,line width= 0.4pt,line join=round,line cap=round] ( 81.94,218.85) -- ( 88.30,218.85);

\path[draw=drawColor,line width= 0.4pt,line join=round,line cap=round] ( 85.12,215.67) -- ( 85.12,222.03);

\path[draw=drawColor,line width= 0.4pt,line join=round,line cap=round] ( 93.91,215.94) -- (100.28,215.94);

\path[draw=drawColor,line width= 0.4pt,line join=round,line cap=round] ( 97.10,212.76) -- ( 97.10,219.12);

\path[draw=drawColor,line width= 0.4pt,line join=round,line cap=round] (105.89,214.09) -- (112.25,214.09);

\path[draw=drawColor,line width= 0.4pt,line join=round,line cap=round] (109.07,210.91) -- (109.07,217.27);

\path[draw=drawColor,line width= 0.4pt,line join=round,line cap=round] (117.86,213.30) -- (124.23,213.30);

\path[draw=drawColor,line width= 0.4pt,line join=round,line cap=round] (121.04,210.12) -- (121.04,216.49);

\path[draw=drawColor,line width= 0.4pt,line join=round,line cap=round] (129.84,212.92) -- (136.20,212.92);

\path[draw=drawColor,line width= 0.4pt,line join=round,line cap=round] (133.02,209.74) -- (133.02,216.10);

\path[draw=drawColor,line width= 0.4pt,line join=round,line cap=round] (141.81,212.50) -- (148.17,212.50);

\path[draw=drawColor,line width= 0.4pt,line join=round,line cap=round] (144.99,209.32) -- (144.99,215.68);

\path[draw=drawColor,line width= 0.4pt,line join=round,line cap=round] (153.78,212.03) -- (160.15,212.03);

\path[draw=drawColor,line width= 0.4pt,line join=round,line cap=round] (156.97,208.85) -- (156.97,215.21);

\path[draw=drawColor,line width= 0.4pt,line join=round,line cap=round] (165.76,211.56) -- (172.12,211.56);

\path[draw=drawColor,line width= 0.4pt,line join=round,line cap=round] (168.94,208.38) -- (168.94,214.74);

\path[draw=drawColor,line width= 0.4pt,line join=round,line cap=round] (177.73,211.05) -- (184.10,211.05);

\path[draw=drawColor,line width= 0.4pt,line join=round,line cap=round] (180.91,207.87) -- (180.91,214.24);

\path[draw=drawColor,line width= 0.4pt,line join=round,line cap=round] (189.71,211.05) -- (196.07,211.05);

\path[draw=drawColor,line width= 0.4pt,line join=round,line cap=round] (192.89,207.87) -- (192.89,214.24);

\path[draw=drawColor,line width= 0.4pt,line join=round,line cap=round] (201.68,210.45) -- (208.04,210.45);

\path[draw=drawColor,line width= 0.4pt,line join=round,line cap=round] (204.86,207.27) -- (204.86,213.63);

\path[draw=drawColor,line width= 0.4pt,line join=round,line cap=round] (213.65,209.80) -- (220.02,209.80);

\path[draw=drawColor,line width= 0.4pt,line join=round,line cap=round] (216.84,206.62) -- (216.84,212.98);

\path[draw=drawColor,line width= 0.4pt,line join=round,line cap=round] (225.63,209.80) -- (231.99,209.80);

\path[draw=drawColor,line width= 0.4pt,line join=round,line cap=round] (228.81,206.62) -- (228.81,212.98);

\path[draw=drawColor,line width= 0.4pt,line join=round,line cap=round] (237.60,209.80) -- (243.97,209.80);

\path[draw=drawColor,line width= 0.4pt,line join=round,line cap=round] (240.78,206.62) -- (240.78,212.98);

\path[draw=drawColor,line width= 0.4pt,line join=round,line cap=round] (249.58,208.69) -- (255.94,208.69);

\path[draw=drawColor,line width= 0.4pt,line join=round,line cap=round] (252.76,205.51) -- (252.76,211.87);

\path[draw=drawColor,line width= 0.4pt,line join=round,line cap=round] (261.55,206.89) -- (267.91,206.89);

\path[draw=drawColor,line width= 0.4pt,line join=round,line cap=round] (264.73,203.71) -- (264.73,210.08);

\path[draw=drawColor,line width= 0.4pt,line join=round,line cap=round] (273.52,206.89) -- (279.89,206.89);

\path[draw=drawColor,line width= 0.4pt,line join=round,line cap=round] (276.71,203.71) -- (276.71,210.08);

\path[draw=drawColor,line width= 0.4pt,line join=round,line cap=round] (285.50,203.45) -- (291.86,203.45);

\path[draw=drawColor,line width= 0.4pt,line join=round,line cap=round] (288.68,200.27) -- (288.68,206.63);

\path[draw=drawColor,line width= 0.4pt,line join=round,line cap=round] (297.47,199.35) -- (303.84,199.35);

\path[draw=drawColor,line width= 0.4pt,line join=round,line cap=round] (300.65,196.17) -- (300.65,202.53);

\path[draw=drawColor,line width= 0.4pt,line join=round,line cap=round] (309.45,199.35) -- (315.81,199.35);

\path[draw=drawColor,line width= 0.4pt,line join=round,line cap=round] (312.63,196.17) -- (312.63,202.53);

\path[draw=drawColor,line width= 0.4pt,line join=round,line cap=round] (321.42,199.35) -- (327.78,199.35);

\path[draw=drawColor,line width= 0.4pt,line join=round,line cap=round] (324.60,196.17) -- (324.60,202.53);

\path[draw=drawColor,line width= 0.4pt,line join=round,line cap=round] (333.39,195.48) -- (339.76,195.48);

\path[draw=drawColor,line width= 0.4pt,line join=round,line cap=round] (336.58,192.30) -- (336.58,198.66);

\path[draw=drawColor,line width= 0.4pt,line join=round,line cap=round] (345.37,195.48) -- (351.73,195.48);

\path[draw=drawColor,line width= 0.4pt,line join=round,line cap=round] (348.55,192.30) -- (348.55,198.66);

\path[draw=drawColor,line width= 0.4pt,line join=round,line cap=round] (357.34,195.48) -- (363.71,195.48);

\path[draw=drawColor,line width= 0.4pt,line join=round,line cap=round] (360.52,192.30) -- (360.52,198.66);

\path[draw=drawColor,line width= 0.4pt,line join=round,line cap=round] (369.32,195.48) -- (375.68,195.48);

\path[draw=drawColor,line width= 0.4pt,line join=round,line cap=round] (372.50,192.30) -- (372.50,198.66);

\path[draw=drawColor,line width= 0.4pt,line join=round,line cap=round] (381.29,195.48) -- (387.65,195.48);

\path[draw=drawColor,line width= 0.4pt,line join=round,line cap=round] (384.47,192.30) -- (384.47,198.66);

\path[draw=drawColor,line width= 0.4pt,line join=round,line cap=round] (393.26,195.48) -- (399.63,195.48);

\path[draw=drawColor,line width= 0.4pt,line join=round,line cap=round] (396.45,192.30) -- (396.45,198.66);

\path[draw=drawColor,line width= 0.4pt,line join=round,line cap=round] (405.24,195.48) -- (411.60,195.48);

\path[draw=drawColor,line width= 0.4pt,line join=round,line cap=round] (408.42,192.30) -- (408.42,198.66);

\path[draw=drawColor,line width= 0.4pt,line join=round,line cap=round] (417.21,195.48) -- (423.58,195.48);

\path[draw=drawColor,line width= 0.4pt,line join=round,line cap=round] (420.39,192.30) -- (420.39,198.66);

\path[draw=drawColor,line width= 0.4pt,line join=round,line cap=round] (429.19,195.48) -- (433.62,195.48);

\path[draw=drawColor,line width= 0.4pt,line join=round,line cap=round] (432.37,192.30) -- (432.37,198.66);

\path[draw=drawColor,line width= 0.4pt,dash pattern=on 4pt off 4pt ,line join=round,line cap=round] ( 49.20,233.26) --
	( 61.17,233.26) --
	( 61.17,226.45) --
	( 73.15,226.45) --
	( 73.15,220.05) --
	( 85.12,220.05) --
	( 85.12,213.23) --
	( 97.10,213.23) --
	( 97.10,209.55) --
	(109.07,209.55) --
	(109.07,207.24) --
	(121.04,207.24) --
	(121.04,206.27) --
	(133.02,206.27) --
	(133.02,205.79) --
	(144.99,205.79) --
	(144.99,205.27) --
	(156.97,205.27) --
	(156.97,204.69) --
	(168.94,204.69) --
	(168.94,204.11) --
	(180.91,204.11) --
	(180.91,203.48) --
	(204.86,203.48) --
	(204.86,202.74) --
	(216.84,202.74) --
	(216.84,201.94) --
	(252.76,201.94) --
	(252.76,200.45) --
	(264.73,200.45) --
	(264.73,197.84) --
	(288.68,197.84) --
	(288.68,192.13) --
	(300.65,192.13) --
	(300.65,185.35) --
	(336.58,185.35) --
	(336.58,179.19) --
	(433.62,179.19);

\path[draw=drawColor,line width= 0.4pt,dash pattern=on 4pt off 4pt ,line join=round,line cap=round] ( 49.20,233.26) --
	( 61.17,233.26) --
	( 61.17,231.60) --
	( 73.15,231.60) --
	( 73.15,228.31) --
	( 85.12,228.31) --
	( 85.12,224.64) --
	( 97.10,224.64) --
	( 97.10,222.56) --
	(109.07,222.56) --
	(109.07,221.20) --
	(121.04,221.20) --
	(121.04,220.62) --
	(133.02,220.62) --
	(133.02,220.34) --
	(144.99,220.34) --
	(144.99,220.02) --
	(156.97,220.02) --
	(156.97,219.67) --
	(168.94,219.67) --
	(168.94,219.32) --
	(180.91,219.32) --
	(180.91,218.95) --
	(204.86,218.95) --
	(204.86,218.50) --
	(216.84,218.50) --
	(216.84,218.02) --
	(252.76,218.02) --
	(252.76,217.32) --
	(264.73,217.32) --
	(264.73,216.43) --
	(288.68,216.43) --
	(288.68,215.55) --
	(300.65,215.55) --
	(300.65,214.59) --
	(336.58,214.59) --
	(336.58,213.50) --
	(433.62,213.50);
\definecolor{drawColor}{RGB}{0,0,0}

\path[draw=drawColor,line width= 0.4pt,line join=round,line cap=round] ( 49.20,233.26) --
	( 61.17,233.26) --
	( 61.17,216.17) --
	( 73.15,216.17) --
	( 73.15,197.92) --
	( 85.12,197.92) --
	( 85.12,179.68) --
	( 97.10,179.68) --
	( 97.10,170.29) --
	(109.07,170.29) --
	(109.07,164.54) --
	(121.04,164.54) --
	(121.04,162.15) --
	(133.02,162.15) --
	(133.02,161.00) --
	(144.99,161.00) --
	(144.99,159.74) --
	(156.97,159.74) --
	(156.97,158.35) --
	(168.94,158.35) --
	(168.94,156.97) --
	(180.91,156.97) --
	(180.91,155.49) --
	(204.86,155.49) --
	(204.86,153.75) --
	(216.84,153.75) --
	(216.84,151.90) --
	(252.76,151.90) --
	(252.76,148.77) --
	(264.73,148.77) --
	(264.73,143.84) --
	(288.68,143.84) --
	(288.68,134.82) --
	(300.65,134.82) --
	(300.65,124.77) --
	(336.58,124.77) --
	(336.58,115.95) --
	(433.62,115.95);

\path[draw=drawColor,line width= 0.4pt,line join=round,line cap=round] ( 57.99,216.17) -- ( 64.36,216.17);

\path[draw=drawColor,line width= 0.4pt,line join=round,line cap=round] ( 61.17,212.99) -- ( 61.17,219.35);

\path[draw=drawColor,line width= 0.4pt,line join=round,line cap=round] ( 69.97,197.92) -- ( 76.33,197.92);

\path[draw=drawColor,line width= 0.4pt,line join=round,line cap=round] ( 73.15,194.74) -- ( 73.15,201.10);

\path[draw=drawColor,line width= 0.4pt,line join=round,line cap=round] ( 81.94,179.68) -- ( 88.30,179.68);

\path[draw=drawColor,line width= 0.4pt,line join=round,line cap=round] ( 85.12,176.50) -- ( 85.12,182.86);

\path[draw=drawColor,line width= 0.4pt,line join=round,line cap=round] ( 93.91,170.29) -- (100.28,170.29);

\path[draw=drawColor,line width= 0.4pt,line join=round,line cap=round] ( 97.10,167.10) -- ( 97.10,173.47);

\path[draw=drawColor,line width= 0.4pt,line join=round,line cap=round] (105.89,164.54) -- (112.25,164.54);

\path[draw=drawColor,line width= 0.4pt,line join=round,line cap=round] (109.07,161.36) -- (109.07,167.72);

\path[draw=drawColor,line width= 0.4pt,line join=round,line cap=round] (117.86,162.15) -- (124.23,162.15);

\path[draw=drawColor,line width= 0.4pt,line join=round,line cap=round] (121.04,158.97) -- (121.04,165.33);

\path[draw=drawColor,line width= 0.4pt,line join=round,line cap=round] (129.84,161.00) -- (136.20,161.00);

\path[draw=drawColor,line width= 0.4pt,line join=round,line cap=round] (133.02,157.82) -- (133.02,164.18);

\path[draw=drawColor,line width= 0.4pt,line join=round,line cap=round] (141.81,159.74) -- (148.17,159.74);

\path[draw=drawColor,line width= 0.4pt,line join=round,line cap=round] (144.99,156.55) -- (144.99,162.92);

\path[draw=drawColor,line width= 0.4pt,line join=round,line cap=round] (153.78,158.35) -- (160.15,158.35);

\path[draw=drawColor,line width= 0.4pt,line join=round,line cap=round] (156.97,155.17) -- (156.97,161.54);

\path[draw=drawColor,line width= 0.4pt,line join=round,line cap=round] (165.76,156.97) -- (172.12,156.97);

\path[draw=drawColor,line width= 0.4pt,line join=round,line cap=round] (168.94,153.79) -- (168.94,160.15);

\path[draw=drawColor,line width= 0.4pt,line join=round,line cap=round] (177.73,155.49) -- (184.10,155.49);

\path[draw=drawColor,line width= 0.4pt,line join=round,line cap=round] (180.91,152.31) -- (180.91,158.68);

\path[draw=drawColor,line width= 0.4pt,line join=round,line cap=round] (189.71,155.49) -- (196.07,155.49);

\path[draw=drawColor,line width= 0.4pt,line join=round,line cap=round] (192.89,152.31) -- (192.89,158.68);

\path[draw=drawColor,line width= 0.4pt,line join=round,line cap=round] (201.68,153.75) -- (208.04,153.75);

\path[draw=drawColor,line width= 0.4pt,line join=round,line cap=round] (204.86,150.57) -- (204.86,156.93);

\path[draw=drawColor,line width= 0.4pt,line join=round,line cap=round] (213.65,151.90) -- (220.02,151.90);

\path[draw=drawColor,line width= 0.4pt,line join=round,line cap=round] (216.84,148.72) -- (216.84,155.08);

\path[draw=drawColor,line width= 0.4pt,line join=round,line cap=round] (225.63,151.90) -- (231.99,151.90);

\path[draw=drawColor,line width= 0.4pt,line join=round,line cap=round] (228.81,148.72) -- (228.81,155.08);

\path[draw=drawColor,line width= 0.4pt,line join=round,line cap=round] (237.60,151.90) -- (243.97,151.90);

\path[draw=drawColor,line width= 0.4pt,line join=round,line cap=round] (240.78,148.72) -- (240.78,155.08);

\path[draw=drawColor,line width= 0.4pt,line join=round,line cap=round] (249.58,148.77) -- (255.94,148.77);

\path[draw=drawColor,line width= 0.4pt,line join=round,line cap=round] (252.76,145.59) -- (252.76,151.95);

\path[draw=drawColor,line width= 0.4pt,line join=round,line cap=round] (261.55,143.84) -- (267.91,143.84);

\path[draw=drawColor,line width= 0.4pt,line join=round,line cap=round] (264.73,140.66) -- (264.73,147.02);

\path[draw=drawColor,line width= 0.4pt,line join=round,line cap=round] (273.52,143.84) -- (279.89,143.84);

\path[draw=drawColor,line width= 0.4pt,line join=round,line cap=round] (276.71,140.66) -- (276.71,147.02);

\path[draw=drawColor,line width= 0.4pt,line join=round,line cap=round] (285.50,134.82) -- (291.86,134.82);

\path[draw=drawColor,line width= 0.4pt,line join=round,line cap=round] (288.68,131.64) -- (288.68,138.01);

\path[draw=drawColor,line width= 0.4pt,line join=round,line cap=round] (297.47,124.77) -- (303.84,124.77);

\path[draw=drawColor,line width= 0.4pt,line join=round,line cap=round] (300.65,121.59) -- (300.65,127.95);

\path[draw=drawColor,line width= 0.4pt,line join=round,line cap=round] (309.45,124.77) -- (315.81,124.77);

\path[draw=drawColor,line width= 0.4pt,line join=round,line cap=round] (312.63,121.59) -- (312.63,127.95);

\path[draw=drawColor,line width= 0.4pt,line join=round,line cap=round] (321.42,124.77) -- (327.78,124.77);

\path[draw=drawColor,line width= 0.4pt,line join=round,line cap=round] (324.60,121.59) -- (324.60,127.95);

\path[draw=drawColor,line width= 0.4pt,line join=round,line cap=round] (333.39,115.95) -- (339.76,115.95);

\path[draw=drawColor,line width= 0.4pt,line join=round,line cap=round] (336.58,112.77) -- (336.58,119.13);

\path[draw=drawColor,line width= 0.4pt,line join=round,line cap=round] (345.37,115.95) -- (351.73,115.95);

\path[draw=drawColor,line width= 0.4pt,line join=round,line cap=round] (348.55,112.77) -- (348.55,119.13);

\path[draw=drawColor,line width= 0.4pt,line join=round,line cap=round] (357.34,115.95) -- (363.71,115.95);

\path[draw=drawColor,line width= 0.4pt,line join=round,line cap=round] (360.52,112.77) -- (360.52,119.13);

\path[draw=drawColor,line width= 0.4pt,line join=round,line cap=round] (369.32,115.95) -- (375.68,115.95);

\path[draw=drawColor,line width= 0.4pt,line join=round,line cap=round] (372.50,112.77) -- (372.50,119.13);

\path[draw=drawColor,line width= 0.4pt,line join=round,line cap=round] (381.29,115.95) -- (387.65,115.95);

\path[draw=drawColor,line width= 0.4pt,line join=round,line cap=round] (384.47,112.77) -- (384.47,119.13);

\path[draw=drawColor,line width= 0.4pt,line join=round,line cap=round] (393.26,115.95) -- (399.63,115.95);

\path[draw=drawColor,line width= 0.4pt,line join=round,line cap=round] (396.45,112.77) -- (396.45,119.13);

\path[draw=drawColor,line width= 0.4pt,line join=round,line cap=round] (405.24,115.95) -- (411.60,115.95);

\path[draw=drawColor,line width= 0.4pt,line join=round,line cap=round] (408.42,112.77) -- (408.42,119.13);

\path[draw=drawColor,line width= 0.4pt,line join=round,line cap=round] (417.21,115.95) -- (423.58,115.95);

\path[draw=drawColor,line width= 0.4pt,line join=round,line cap=round] (420.39,112.77) -- (420.39,119.13);

\path[draw=drawColor,line width= 0.4pt,line join=round,line cap=round] (429.19,115.95) -- (433.62,115.95);

\path[draw=drawColor,line width= 0.4pt,line join=round,line cap=round] (432.37,112.77) -- (432.37,119.13);

\path[draw=drawColor,line width= 0.4pt,dash pattern=on 4pt off 4pt ,line join=round,line cap=round] ( 49.20,233.26) --
	( 61.17,233.26) --
	( 61.17,203.55) --
	( 73.15,203.55) --
	( 73.15,177.09) --
	( 85.12,177.09) --
	( 85.12,152.36) --
	( 97.10,152.36) --
	( 97.10,140.23) --
	(109.07,140.23) --
	(109.07,132.99) --
	(121.04,132.99) --
	(121.04,130.06) --
	(133.02,130.06) --
	(133.02,128.64) --
	(144.99,128.64) --
	(144.99,127.09) --
	(156.97,127.09) --
	(156.97,125.40) --
	(168.94,125.40) --
	(168.94,123.71) --
	(180.91,123.71) --
	(180.91,121.92) --
	(204.86,121.92) --
	(204.86,119.84) --
	(216.84,119.84) --
	(216.84,117.64) --
	(252.76,117.64) --
	(252.76,113.86) --
	(264.73,113.86) --
	(264.73,107.91) --
	(288.68,107.91) --
	(288.68, 97.06) --
	(300.65, 97.06) --
	(300.65, 86.04) --
	(336.58, 86.04) --
	(336.58, 77.65) --
	(433.62, 77.65);

\path[draw=drawColor,line width= 0.4pt,dash pattern=on 4pt off 4pt ,line join=round,line cap=round] ( 49.20,233.26) --
	( 61.17,233.26) --
	( 61.17,229.69) --
	( 73.15,229.69) --
	( 73.15,221.63) --
	( 85.12,221.63) --
	( 85.12,212.93) --
	( 97.10,212.93) --
	( 97.10,208.28) --
	(109.07,208.28) --
	(109.07,205.43) --
	(121.04,205.43) --
	(121.04,204.18) --
	(133.02,204.18) --
	(133.02,203.61) --
	(144.99,203.61) --
	(144.99,202.96) --
	(156.97,202.96) --
	(156.97,202.29) --
	(168.94,202.29) --
	(168.94,201.60) --
	(180.91,201.60) --
	(180.91,200.88) --
	(204.86,200.88) --
	(204.86,199.97) --
	(216.84,199.97) --
	(216.84,199.03) --
	(252.76,199.03) --
	(252.76,197.62) --
	(264.73,197.62) --
	(264.73,195.63) --
	(288.68,195.63) --
	(288.68,192.77) --
	(300.65,192.77) --
	(300.65,188.69) --
	(336.58,188.69) --
	(336.58,182.91) --
	(433.62,182.91);
\definecolor{drawColor}{RGB}{169,169,169}

\path[draw=drawColor,line width= 0.4pt,line join=round,line cap=round] (315.46,227.88) -- (333.46,227.88);
\definecolor{drawColor}{RGB}{0,0,0}

\path[draw=drawColor,line width= 0.4pt,line join=round,line cap=round] (315.46,215.88) -- (333.46,215.88);

\node[text=drawColor,anchor=base west,inner sep=0pt, outer sep=0pt, scale=  1.00] at (342.46,224.44) {Few Senders};

\node[text=drawColor,anchor=base west,inner sep=0pt, outer sep=0pt, scale=  1.00] at (342.46,212.44) {Many Senders};
\end{scope}
\end{tikzpicture}
}
\end{figure}

\begin{figure}[ht]
	\centering
	% \includegraphics[width=1\textwidth]{oNet}
	\resizebox{1\textwidth}{!}{% Created by tikzDevice version 0.7.0 on 2014-08-02 12:32:24
% !TEX encoding = UTF-8 Unicode
\begin{tikzpicture}[x=1pt,y=1pt]
\definecolor[named]{fillColor}{rgb}{1.00,1.00,1.00}
\path[use as bounding box,fill=fillColor,fill opacity=0.00] (0,0) rectangle (578.16,216.81);
\begin{scope}
\path[clip] (  0.00,  0.00) rectangle (578.16,216.81);
\definecolor[named]{drawColor}{rgb}{0.00,0.00,0.00}

\path[draw=drawColor,line width= 0.4pt,line join=round,line cap=round] ( 49.20, 61.20) -- (263.88, 61.20);

\path[draw=drawColor,line width= 0.4pt,line join=round,line cap=round] ( 49.20, 61.20) -- ( 49.20, 55.20);

\path[draw=drawColor,line width= 0.4pt,line join=round,line cap=round] ( 84.98, 61.20) -- ( 84.98, 55.20);

\path[draw=drawColor,line width= 0.4pt,line join=round,line cap=round] (120.76, 61.20) -- (120.76, 55.20);

\path[draw=drawColor,line width= 0.4pt,line join=round,line cap=round] (156.54, 61.20) -- (156.54, 55.20);

\path[draw=drawColor,line width= 0.4pt,line join=round,line cap=round] (192.32, 61.20) -- (192.32, 55.20);

\path[draw=drawColor,line width= 0.4pt,line join=round,line cap=round] (228.10, 61.20) -- (228.10, 55.20);

\path[draw=drawColor,line width= 0.4pt,line join=round,line cap=round] (263.88, 61.20) -- (263.88, 55.20);

\node[text=drawColor,anchor=base,inner sep=0pt, outer sep=0pt, scale=  1.00] at ( 49.20, 39.60) {0};

\node[text=drawColor,anchor=base,inner sep=0pt, outer sep=0pt, scale=  1.00] at ( 84.98, 39.60) {5};

\node[text=drawColor,anchor=base,inner sep=0pt, outer sep=0pt, scale=  1.00] at (120.76, 39.60) {10};

\node[text=drawColor,anchor=base,inner sep=0pt, outer sep=0pt, scale=  1.00] at (156.54, 39.60) {15};

\node[text=drawColor,anchor=base,inner sep=0pt, outer sep=0pt, scale=  1.00] at (192.32, 39.60) {20};

\node[text=drawColor,anchor=base,inner sep=0pt, outer sep=0pt, scale=  1.00] at (228.10, 39.60) {25};

\node[text=drawColor,anchor=base,inner sep=0pt, outer sep=0pt, scale=  1.00] at (263.88, 39.60) {30};

\path[draw=drawColor,line width= 0.4pt,line join=round,line cap=round] ( 49.20, 65.14) -- ( 49.20,163.67);

\path[draw=drawColor,line width= 0.4pt,line join=round,line cap=round] ( 49.20, 65.14) -- ( 43.20, 65.14);

\path[draw=drawColor,line width= 0.4pt,line join=round,line cap=round] ( 49.20, 84.85) -- ( 43.20, 84.85);

\path[draw=drawColor,line width= 0.4pt,line join=round,line cap=round] ( 49.20,104.55) -- ( 43.20,104.55);

\path[draw=drawColor,line width= 0.4pt,line join=round,line cap=round] ( 49.20,124.26) -- ( 43.20,124.26);

\path[draw=drawColor,line width= 0.4pt,line join=round,line cap=round] ( 49.20,143.96) -- ( 43.20,143.96);

\path[draw=drawColor,line width= 0.4pt,line join=round,line cap=round] ( 49.20,163.67) -- ( 43.20,163.67);

\node[text=drawColor,anchor=base east,inner sep=0pt, outer sep=0pt, scale=  1.00] at ( 37.20, 61.70) {0.0};

\node[text=drawColor,anchor=base east,inner sep=0pt, outer sep=0pt, scale=  1.00] at ( 37.20, 81.40) {0.2};

\node[text=drawColor,anchor=base east,inner sep=0pt, outer sep=0pt, scale=  1.00] at ( 37.20,101.11) {0.4};

\node[text=drawColor,anchor=base east,inner sep=0pt, outer sep=0pt, scale=  1.00] at ( 37.20,120.81) {0.6};

\node[text=drawColor,anchor=base east,inner sep=0pt, outer sep=0pt, scale=  1.00] at ( 37.20,140.52) {0.8};

\node[text=drawColor,anchor=base east,inner sep=0pt, outer sep=0pt, scale=  1.00] at ( 37.20,160.23) {1.0};

\path[draw=drawColor,line width= 0.4pt,line join=round,line cap=round] ( 49.20, 61.20) --
	(263.88, 61.20) --
	(263.88,167.61) --
	( 49.20,167.61) --
	( 49.20, 61.20);
\end{scope}
\begin{scope}
\path[clip] (  0.00,  0.00) rectangle (289.08,216.81);
\definecolor[named]{drawColor}{rgb}{0.00,0.00,0.00}

\node[text=drawColor,anchor=base,inner sep=0pt, outer sep=0pt, scale=  1.20] at (156.54,188.07) {\bfseries Distance};
\end{scope}
\begin{scope}
\path[clip] ( 49.20, 61.20) rectangle (263.88,167.61);
\definecolor[named]{drawColor}{rgb}{0.66,0.66,0.66}

\path[draw=drawColor,line width= 0.4pt,line join=round,line cap=round] ( 49.20,163.67) --
	( 56.36,163.67) --
	( 56.36,154.96) --
	( 63.51,154.96) --
	( 63.51,148.45) --
	( 70.67,148.45) --
	( 70.67,143.70) --
	( 77.82,143.70) --
	( 77.82,142.09) --
	( 84.98,142.09) --
	( 84.98,140.90) --
	( 92.14,140.90) --
	( 92.14,140.38) --
	( 99.29,140.38) --
	( 99.29,139.68) --
	(106.45,139.68) --
	(106.45,139.31) --
	(113.60,139.31) --
	(113.60,138.91) --
	(120.76,138.91) --
	(120.76,138.10) --
	(127.92,138.10) --
	(127.92,137.88) --
	(142.23,137.88) --
	(142.23,137.64) --
	(149.38,137.64) --
	(149.38,137.39) --
	(170.85,137.39) --
	(170.85,137.02) --
	(178.01,137.02) --
	(178.01,136.50) --
	(192.32,136.50) --
	(192.32,135.79) --
	(199.48,135.79) --
	(199.48,134.89) --
	(220.94,134.89) --
	(220.94,133.83) --
	(492.87,133.83) --
	(492.87,133.83);

\path[draw=drawColor,line width= 0.4pt,line join=round,line cap=round] ( 53.17,154.96) -- ( 59.54,154.96);

\path[draw=drawColor,line width= 0.4pt,line join=round,line cap=round] ( 56.36,151.77) -- ( 56.36,158.14);

\path[draw=drawColor,line width= 0.4pt,line join=round,line cap=round] ( 60.33,148.45) -- ( 66.69,148.45);

\path[draw=drawColor,line width= 0.4pt,line join=round,line cap=round] ( 63.51,145.27) -- ( 63.51,151.63);

\path[draw=drawColor,line width= 0.4pt,line join=round,line cap=round] ( 67.49,143.70) -- ( 73.85,143.70);

\path[draw=drawColor,line width= 0.4pt,line join=round,line cap=round] ( 70.67,140.52) -- ( 70.67,146.88);

\path[draw=drawColor,line width= 0.4pt,line join=round,line cap=round] ( 74.64,142.09) -- ( 81.01,142.09);

\path[draw=drawColor,line width= 0.4pt,line join=round,line cap=round] ( 77.82,138.91) -- ( 77.82,145.27);

\path[draw=drawColor,line width= 0.4pt,line join=round,line cap=round] ( 81.80,140.90) -- ( 88.16,140.90);

\path[draw=drawColor,line width= 0.4pt,line join=round,line cap=round] ( 84.98,137.72) -- ( 84.98,144.08);

\path[draw=drawColor,line width= 0.4pt,line join=round,line cap=round] ( 88.95,140.38) -- ( 95.32,140.38);

\path[draw=drawColor,line width= 0.4pt,line join=round,line cap=round] ( 92.14,137.20) -- ( 92.14,143.57);

\path[draw=drawColor,line width= 0.4pt,line join=round,line cap=round] ( 96.11,139.68) -- (102.47,139.68);

\path[draw=drawColor,line width= 0.4pt,line join=round,line cap=round] ( 99.29,136.50) -- ( 99.29,142.86);

\path[draw=drawColor,line width= 0.4pt,line join=round,line cap=round] (103.27,139.31) -- (109.63,139.31);

\path[draw=drawColor,line width= 0.4pt,line join=round,line cap=round] (106.45,136.13) -- (106.45,142.49);

\path[draw=drawColor,line width= 0.4pt,line join=round,line cap=round] (110.42,138.91) -- (116.79,138.91);

\path[draw=drawColor,line width= 0.4pt,line join=round,line cap=round] (113.60,135.73) -- (113.60,142.10);

\path[draw=drawColor,line width= 0.4pt,line join=round,line cap=round] (117.58,138.10) -- (123.94,138.10);

\path[draw=drawColor,line width= 0.4pt,line join=round,line cap=round] (120.76,134.92) -- (120.76,141.28);

\path[draw=drawColor,line width= 0.4pt,line join=round,line cap=round] (124.73,137.88) -- (131.10,137.88);

\path[draw=drawColor,line width= 0.4pt,line join=round,line cap=round] (127.92,134.70) -- (127.92,141.06);

\path[draw=drawColor,line width= 0.4pt,line join=round,line cap=round] (131.89,137.88) -- (138.25,137.88);

\path[draw=drawColor,line width= 0.4pt,line join=round,line cap=round] (135.07,134.70) -- (135.07,141.06);

\path[draw=drawColor,line width= 0.4pt,line join=round,line cap=round] (139.05,137.64) -- (145.41,137.64);

\path[draw=drawColor,line width= 0.4pt,line join=round,line cap=round] (142.23,134.46) -- (142.23,140.83);

\path[draw=drawColor,line width= 0.4pt,line join=round,line cap=round] (146.20,137.39) -- (152.57,137.39);

\path[draw=drawColor,line width= 0.4pt,line join=round,line cap=round] (149.38,134.21) -- (149.38,140.57);

\path[draw=drawColor,line width= 0.4pt,line join=round,line cap=round] (153.36,137.39) -- (159.72,137.39);

\path[draw=drawColor,line width= 0.4pt,line join=round,line cap=round] (156.54,134.21) -- (156.54,140.57);

\path[draw=drawColor,line width= 0.4pt,line join=round,line cap=round] (160.51,137.39) -- (166.88,137.39);

\path[draw=drawColor,line width= 0.4pt,line join=round,line cap=round] (163.70,134.21) -- (163.70,140.57);

\path[draw=drawColor,line width= 0.4pt,line join=round,line cap=round] (167.67,137.02) -- (174.03,137.02);

\path[draw=drawColor,line width= 0.4pt,line join=round,line cap=round] (170.85,133.83) -- (170.85,140.20);

\path[draw=drawColor,line width= 0.4pt,line join=round,line cap=round] (174.83,136.50) -- (181.19,136.50);

\path[draw=drawColor,line width= 0.4pt,line join=round,line cap=round] (178.01,133.32) -- (178.01,139.68);

\path[draw=drawColor,line width= 0.4pt,line join=round,line cap=round] (181.98,136.50) -- (188.35,136.50);

\path[draw=drawColor,line width= 0.4pt,line join=round,line cap=round] (185.16,133.32) -- (185.16,139.68);

\path[draw=drawColor,line width= 0.4pt,line join=round,line cap=round] (189.14,135.79) -- (195.50,135.79);

\path[draw=drawColor,line width= 0.4pt,line join=round,line cap=round] (192.32,132.61) -- (192.32,138.97);

\path[draw=drawColor,line width= 0.4pt,line join=round,line cap=round] (196.29,134.89) -- (202.66,134.89);

\path[draw=drawColor,line width= 0.4pt,line join=round,line cap=round] (199.48,131.71) -- (199.48,138.07);

\path[draw=drawColor,line width= 0.4pt,line join=round,line cap=round] (203.45,134.89) -- (209.81,134.89);

\path[draw=drawColor,line width= 0.4pt,line join=round,line cap=round] (206.63,131.71) -- (206.63,138.07);

\path[draw=drawColor,line width= 0.4pt,line join=round,line cap=round] (210.61,134.89) -- (216.97,134.89);

\path[draw=drawColor,line width= 0.4pt,line join=round,line cap=round] (213.79,131.71) -- (213.79,138.07);

\path[draw=drawColor,line width= 0.4pt,line join=round,line cap=round] (217.76,133.83) -- (224.13,133.83);

\path[draw=drawColor,line width= 0.4pt,line join=round,line cap=round] (220.94,130.65) -- (220.94,137.01);

\path[draw=drawColor,line width= 0.4pt,line join=round,line cap=round] (224.92,133.83) -- (231.28,133.83);

\path[draw=drawColor,line width= 0.4pt,line join=round,line cap=round] (228.10,130.65) -- (228.10,137.01);

\path[draw=drawColor,line width= 0.4pt,line join=round,line cap=round] (232.07,133.83) -- (238.44,133.83);

\path[draw=drawColor,line width= 0.4pt,line join=round,line cap=round] (235.26,130.65) -- (235.26,137.01);

\path[draw=drawColor,line width= 0.4pt,line join=round,line cap=round] (239.23,133.83) -- (245.59,133.83);

\path[draw=drawColor,line width= 0.4pt,line join=round,line cap=round] (242.41,130.65) -- (242.41,137.01);

\path[draw=drawColor,line width= 0.4pt,line join=round,line cap=round] (246.39,133.83) -- (252.75,133.83);

\path[draw=drawColor,line width= 0.4pt,line join=round,line cap=round] (249.57,130.65) -- (249.57,137.01);

\path[draw=drawColor,line width= 0.4pt,line join=round,line cap=round] (253.54,133.83) -- (259.91,133.83);

\path[draw=drawColor,line width= 0.4pt,line join=round,line cap=round] (256.72,130.65) -- (256.72,137.01);

\path[draw=drawColor,line width= 0.4pt,line join=round,line cap=round] (260.70,133.83) -- (267.06,133.83);

\path[draw=drawColor,line width= 0.4pt,line join=round,line cap=round] (263.88,130.65) -- (263.88,137.01);

\path[draw=drawColor,line width= 0.4pt,line join=round,line cap=round] (267.85,133.83) -- (274.22,133.83);

\path[draw=drawColor,line width= 0.4pt,line join=round,line cap=round] (271.04,130.65) -- (271.04,137.01);

\path[draw=drawColor,line width= 0.4pt,line join=round,line cap=round] (275.01,133.83) -- (281.37,133.83);

\path[draw=drawColor,line width= 0.4pt,line join=round,line cap=round] (278.19,130.65) -- (278.19,137.01);

\path[draw=drawColor,line width= 0.4pt,line join=round,line cap=round] (282.17,133.83) -- (288.53,133.83);

\path[draw=drawColor,line width= 0.4pt,line join=round,line cap=round] (285.35,130.65) -- (285.35,137.01);

\path[draw=drawColor,line width= 0.4pt,line join=round,line cap=round] (289.32,133.83) -- (295.69,133.83);

\path[draw=drawColor,line width= 0.4pt,line join=round,line cap=round] (292.50,130.65) -- (292.50,137.01);

\path[draw=drawColor,line width= 0.4pt,line join=round,line cap=round] (296.48,133.83) -- (302.84,133.83);

\path[draw=drawColor,line width= 0.4pt,line join=round,line cap=round] (299.66,130.65) -- (299.66,137.01);

\path[draw=drawColor,line width= 0.4pt,line join=round,line cap=round] (303.63,133.83) -- (310.00,133.83);

\path[draw=drawColor,line width= 0.4pt,line join=round,line cap=round] (306.82,130.65) -- (306.82,137.01);

\path[draw=drawColor,line width= 0.4pt,line join=round,line cap=round] (310.79,133.83) -- (317.15,133.83);

\path[draw=drawColor,line width= 0.4pt,line join=round,line cap=round] (313.97,130.65) -- (313.97,137.01);

\path[draw=drawColor,line width= 0.4pt,line join=round,line cap=round] (317.95,133.83) -- (324.31,133.83);

\path[draw=drawColor,line width= 0.4pt,line join=round,line cap=round] (321.13,130.65) -- (321.13,137.01);

\path[draw=drawColor,line width= 0.4pt,line join=round,line cap=round] (325.10,133.83) -- (331.47,133.83);

\path[draw=drawColor,line width= 0.4pt,line join=round,line cap=round] (328.28,130.65) -- (328.28,137.01);

\path[draw=drawColor,line width= 0.4pt,line join=round,line cap=round] (332.26,133.83) -- (338.62,133.83);

\path[draw=drawColor,line width= 0.4pt,line join=round,line cap=round] (335.44,130.65) -- (335.44,137.01);

\path[draw=drawColor,line width= 0.4pt,line join=round,line cap=round] (339.41,133.83) -- (345.78,133.83);

\path[draw=drawColor,line width= 0.4pt,line join=round,line cap=round] (342.60,130.65) -- (342.60,137.01);

\path[draw=drawColor,line width= 0.4pt,line join=round,line cap=round] (346.57,133.83) -- (352.93,133.83);

\path[draw=drawColor,line width= 0.4pt,line join=round,line cap=round] (349.75,130.65) -- (349.75,137.01);

\path[draw=drawColor,line width= 0.4pt,line join=round,line cap=round] (353.73,133.83) -- (360.09,133.83);

\path[draw=drawColor,line width= 0.4pt,line join=round,line cap=round] (356.91,130.65) -- (356.91,137.01);

\path[draw=drawColor,line width= 0.4pt,line join=round,line cap=round] (360.88,133.83) -- (367.25,133.83);

\path[draw=drawColor,line width= 0.4pt,line join=round,line cap=round] (364.06,130.65) -- (364.06,137.01);

\path[draw=drawColor,line width= 0.4pt,line join=round,line cap=round] (368.04,133.83) -- (374.40,133.83);

\path[draw=drawColor,line width= 0.4pt,line join=round,line cap=round] (371.22,130.65) -- (371.22,137.01);

\path[draw=drawColor,line width= 0.4pt,line join=round,line cap=round] (375.19,133.83) -- (381.56,133.83);

\path[draw=drawColor,line width= 0.4pt,line join=round,line cap=round] (378.38,130.65) -- (378.38,137.01);

\path[draw=drawColor,line width= 0.4pt,line join=round,line cap=round] (382.35,133.83) -- (388.71,133.83);

\path[draw=drawColor,line width= 0.4pt,line join=round,line cap=round] (385.53,130.65) -- (385.53,137.01);

\path[draw=drawColor,line width= 0.4pt,line join=round,line cap=round] (389.51,133.83) -- (395.87,133.83);

\path[draw=drawColor,line width= 0.4pt,line join=round,line cap=round] (392.69,130.65) -- (392.69,137.01);

\path[draw=drawColor,line width= 0.4pt,line join=round,line cap=round] (396.66,133.83) -- (403.03,133.83);

\path[draw=drawColor,line width= 0.4pt,line join=round,line cap=round] (399.84,130.65) -- (399.84,137.01);

\path[draw=drawColor,line width= 0.4pt,line join=round,line cap=round] (403.82,133.83) -- (410.18,133.83);

\path[draw=drawColor,line width= 0.4pt,line join=round,line cap=round] (407.00,130.65) -- (407.00,137.01);

\path[draw=drawColor,line width= 0.4pt,line join=round,line cap=round] (410.97,133.83) -- (417.34,133.83);

\path[draw=drawColor,line width= 0.4pt,line join=round,line cap=round] (414.16,130.65) -- (414.16,137.01);

\path[draw=drawColor,line width= 0.4pt,line join=round,line cap=round] (418.13,133.83) -- (424.49,133.83);

\path[draw=drawColor,line width= 0.4pt,line join=round,line cap=round] (421.31,130.65) -- (421.31,137.01);

\path[draw=drawColor,line width= 0.4pt,line join=round,line cap=round] (425.29,133.83) -- (431.65,133.83);

\path[draw=drawColor,line width= 0.4pt,line join=round,line cap=round] (428.47,130.65) -- (428.47,137.01);

\path[draw=drawColor,line width= 0.4pt,line join=round,line cap=round] (432.44,133.83) -- (438.81,133.83);

\path[draw=drawColor,line width= 0.4pt,line join=round,line cap=round] (435.62,130.65) -- (435.62,137.01);

\path[draw=drawColor,line width= 0.4pt,line join=round,line cap=round] (439.60,133.83) -- (445.96,133.83);

\path[draw=drawColor,line width= 0.4pt,line join=round,line cap=round] (442.78,130.65) -- (442.78,137.01);

\path[draw=drawColor,line width= 0.4pt,line join=round,line cap=round] (446.75,133.83) -- (453.12,133.83);

\path[draw=drawColor,line width= 0.4pt,line join=round,line cap=round] (449.94,130.65) -- (449.94,137.01);

\path[draw=drawColor,line width= 0.4pt,line join=round,line cap=round] (453.91,133.83) -- (460.27,133.83);

\path[draw=drawColor,line width= 0.4pt,line join=round,line cap=round] (457.09,130.65) -- (457.09,137.01);

\path[draw=drawColor,line width= 0.4pt,line join=round,line cap=round] (461.07,133.83) -- (467.43,133.83);

\path[draw=drawColor,line width= 0.4pt,line join=round,line cap=round] (464.25,130.65) -- (464.25,137.01);

\path[draw=drawColor,line width= 0.4pt,line join=round,line cap=round] (468.22,133.83) -- (474.59,133.83);

\path[draw=drawColor,line width= 0.4pt,line join=round,line cap=round] (471.40,130.65) -- (471.40,137.01);

\path[draw=drawColor,line width= 0.4pt,line join=round,line cap=round] (475.38,133.83) -- (481.74,133.83);

\path[draw=drawColor,line width= 0.4pt,line join=round,line cap=round] (478.56,130.65) -- (478.56,137.01);

\path[draw=drawColor,line width= 0.4pt,line join=round,line cap=round] (482.53,133.83) -- (488.90,133.83);

\path[draw=drawColor,line width= 0.4pt,line join=round,line cap=round] (485.72,130.65) -- (485.72,137.01);

\path[draw=drawColor,line width= 0.4pt,line join=round,line cap=round] (489.69,133.83) -- (496.05,133.83);

\path[draw=drawColor,line width= 0.4pt,line join=round,line cap=round] (492.87,130.65) -- (492.87,137.01);
\definecolor[named]{drawColor}{rgb}{0.00,0.00,0.00}

\path[draw=drawColor,line width= 0.4pt,line join=round,line cap=round] ( 49.20,163.67) --
	( 56.36,163.67) --
	( 56.36,149.76) --
	( 63.51,149.76) --
	( 63.51,139.92) --
	( 70.67,139.92) --
	( 70.67,133.05) --
	( 77.82,133.05) --
	( 77.82,130.78) --
	( 84.98,130.78) --
	( 84.98,129.12) --
	( 92.14,129.12) --
	( 92.14,128.40) --
	( 99.29,128.40) --
	( 99.29,127.43) --
	(106.45,127.43) --
	(106.45,126.92) --
	(113.60,126.92) --
	(113.60,126.38) --
	(120.76,126.38) --
	(120.76,125.27) --
	(127.92,125.27) --
	(127.92,124.98) --
	(142.23,124.98) --
	(142.23,124.66) --
	(149.38,124.66) --
	(149.38,124.31) --
	(170.85,124.31) --
	(170.85,123.81) --
	(178.01,123.81) --
	(178.01,123.12) --
	(192.32,123.12) --
	(192.32,122.17) --
	(199.48,122.17) --
	(199.48,120.99) --
	(220.94,120.99) --
	(220.94,119.60) --
	(492.87,119.60) --
	(492.87,119.60);

\path[draw=drawColor,line width= 0.4pt,line join=round,line cap=round] ( 53.17,149.76) -- ( 59.54,149.76);

\path[draw=drawColor,line width= 0.4pt,line join=round,line cap=round] ( 56.36,146.58) -- ( 56.36,152.94);

\path[draw=drawColor,line width= 0.4pt,line join=round,line cap=round] ( 60.33,139.92) -- ( 66.69,139.92);

\path[draw=drawColor,line width= 0.4pt,line join=round,line cap=round] ( 63.51,136.74) -- ( 63.51,143.10);

\path[draw=drawColor,line width= 0.4pt,line join=round,line cap=round] ( 67.49,133.05) -- ( 73.85,133.05);

\path[draw=drawColor,line width= 0.4pt,line join=round,line cap=round] ( 70.67,129.86) -- ( 70.67,136.23);

\path[draw=drawColor,line width= 0.4pt,line join=round,line cap=round] ( 74.64,130.78) -- ( 81.01,130.78);

\path[draw=drawColor,line width= 0.4pt,line join=round,line cap=round] ( 77.82,127.59) -- ( 77.82,133.96);

\path[draw=drawColor,line width= 0.4pt,line join=round,line cap=round] ( 81.80,129.12) -- ( 88.16,129.12);

\path[draw=drawColor,line width= 0.4pt,line join=round,line cap=round] ( 84.98,125.93) -- ( 84.98,132.30);

\path[draw=drawColor,line width= 0.4pt,line join=round,line cap=round] ( 88.95,128.40) -- ( 95.32,128.40);

\path[draw=drawColor,line width= 0.4pt,line join=round,line cap=round] ( 92.14,125.22) -- ( 92.14,131.58);

\path[draw=drawColor,line width= 0.4pt,line join=round,line cap=round] ( 96.11,127.43) -- (102.47,127.43);

\path[draw=drawColor,line width= 0.4pt,line join=round,line cap=round] ( 99.29,124.25) -- ( 99.29,130.61);

\path[draw=drawColor,line width= 0.4pt,line join=round,line cap=round] (103.27,126.92) -- (109.63,126.92);

\path[draw=drawColor,line width= 0.4pt,line join=round,line cap=round] (106.45,123.74) -- (106.45,130.10);

\path[draw=drawColor,line width= 0.4pt,line join=round,line cap=round] (110.42,126.38) -- (116.79,126.38);

\path[draw=drawColor,line width= 0.4pt,line join=round,line cap=round] (113.60,123.20) -- (113.60,129.56);

\path[draw=drawColor,line width= 0.4pt,line join=round,line cap=round] (117.58,125.27) -- (123.94,125.27);

\path[draw=drawColor,line width= 0.4pt,line join=round,line cap=round] (120.76,122.09) -- (120.76,128.46);

\path[draw=drawColor,line width= 0.4pt,line join=round,line cap=round] (124.73,124.98) -- (131.10,124.98);

\path[draw=drawColor,line width= 0.4pt,line join=round,line cap=round] (127.92,121.79) -- (127.92,128.16);

\path[draw=drawColor,line width= 0.4pt,line join=round,line cap=round] (131.89,124.98) -- (138.25,124.98);

\path[draw=drawColor,line width= 0.4pt,line join=round,line cap=round] (135.07,121.79) -- (135.07,128.16);

\path[draw=drawColor,line width= 0.4pt,line join=round,line cap=round] (139.05,124.66) -- (145.41,124.66);

\path[draw=drawColor,line width= 0.4pt,line join=round,line cap=round] (142.23,121.48) -- (142.23,127.84);

\path[draw=drawColor,line width= 0.4pt,line join=round,line cap=round] (146.20,124.31) -- (152.57,124.31);

\path[draw=drawColor,line width= 0.4pt,line join=round,line cap=round] (149.38,121.13) -- (149.38,127.49);

\path[draw=drawColor,line width= 0.4pt,line join=round,line cap=round] (153.36,124.31) -- (159.72,124.31);

\path[draw=drawColor,line width= 0.4pt,line join=round,line cap=round] (156.54,121.13) -- (156.54,127.49);

\path[draw=drawColor,line width= 0.4pt,line join=round,line cap=round] (160.51,124.31) -- (166.88,124.31);

\path[draw=drawColor,line width= 0.4pt,line join=round,line cap=round] (163.70,121.13) -- (163.70,127.49);

\path[draw=drawColor,line width= 0.4pt,line join=round,line cap=round] (167.67,123.81) -- (174.03,123.81);

\path[draw=drawColor,line width= 0.4pt,line join=round,line cap=round] (170.85,120.63) -- (170.85,126.99);

\path[draw=drawColor,line width= 0.4pt,line join=round,line cap=round] (174.83,123.12) -- (181.19,123.12);

\path[draw=drawColor,line width= 0.4pt,line join=round,line cap=round] (178.01,119.94) -- (178.01,126.30);

\path[draw=drawColor,line width= 0.4pt,line join=round,line cap=round] (181.98,123.12) -- (188.35,123.12);

\path[draw=drawColor,line width= 0.4pt,line join=round,line cap=round] (185.16,119.94) -- (185.16,126.30);

\path[draw=drawColor,line width= 0.4pt,line join=round,line cap=round] (189.14,122.17) -- (195.50,122.17);

\path[draw=drawColor,line width= 0.4pt,line join=round,line cap=round] (192.32,118.99) -- (192.32,125.36);

\path[draw=drawColor,line width= 0.4pt,line join=round,line cap=round] (196.29,120.99) -- (202.66,120.99);

\path[draw=drawColor,line width= 0.4pt,line join=round,line cap=round] (199.48,117.81) -- (199.48,124.17);

\path[draw=drawColor,line width= 0.4pt,line join=round,line cap=round] (203.45,120.99) -- (209.81,120.99);

\path[draw=drawColor,line width= 0.4pt,line join=round,line cap=round] (206.63,117.81) -- (206.63,124.17);

\path[draw=drawColor,line width= 0.4pt,line join=round,line cap=round] (210.61,120.99) -- (216.97,120.99);

\path[draw=drawColor,line width= 0.4pt,line join=round,line cap=round] (213.79,117.81) -- (213.79,124.17);

\path[draw=drawColor,line width= 0.4pt,line join=round,line cap=round] (217.76,119.60) -- (224.13,119.60);

\path[draw=drawColor,line width= 0.4pt,line join=round,line cap=round] (220.94,116.42) -- (220.94,122.78);

\path[draw=drawColor,line width= 0.4pt,line join=round,line cap=round] (224.92,119.60) -- (231.28,119.60);

\path[draw=drawColor,line width= 0.4pt,line join=round,line cap=round] (228.10,116.42) -- (228.10,122.78);

\path[draw=drawColor,line width= 0.4pt,line join=round,line cap=round] (232.07,119.60) -- (238.44,119.60);

\path[draw=drawColor,line width= 0.4pt,line join=round,line cap=round] (235.26,116.42) -- (235.26,122.78);

\path[draw=drawColor,line width= 0.4pt,line join=round,line cap=round] (239.23,119.60) -- (245.59,119.60);

\path[draw=drawColor,line width= 0.4pt,line join=round,line cap=round] (242.41,116.42) -- (242.41,122.78);

\path[draw=drawColor,line width= 0.4pt,line join=round,line cap=round] (246.39,119.60) -- (252.75,119.60);

\path[draw=drawColor,line width= 0.4pt,line join=round,line cap=round] (249.57,116.42) -- (249.57,122.78);

\path[draw=drawColor,line width= 0.4pt,line join=round,line cap=round] (253.54,119.60) -- (259.91,119.60);

\path[draw=drawColor,line width= 0.4pt,line join=round,line cap=round] (256.72,116.42) -- (256.72,122.78);

\path[draw=drawColor,line width= 0.4pt,line join=round,line cap=round] (260.70,119.60) -- (267.06,119.60);

\path[draw=drawColor,line width= 0.4pt,line join=round,line cap=round] (263.88,116.42) -- (263.88,122.78);

\path[draw=drawColor,line width= 0.4pt,line join=round,line cap=round] (267.85,119.60) -- (274.22,119.60);

\path[draw=drawColor,line width= 0.4pt,line join=round,line cap=round] (271.04,116.42) -- (271.04,122.78);

\path[draw=drawColor,line width= 0.4pt,line join=round,line cap=round] (275.01,119.60) -- (281.37,119.60);

\path[draw=drawColor,line width= 0.4pt,line join=round,line cap=round] (278.19,116.42) -- (278.19,122.78);

\path[draw=drawColor,line width= 0.4pt,line join=round,line cap=round] (282.17,119.60) -- (288.53,119.60);

\path[draw=drawColor,line width= 0.4pt,line join=round,line cap=round] (285.35,116.42) -- (285.35,122.78);

\path[draw=drawColor,line width= 0.4pt,line join=round,line cap=round] (289.32,119.60) -- (295.69,119.60);

\path[draw=drawColor,line width= 0.4pt,line join=round,line cap=round] (292.50,116.42) -- (292.50,122.78);

\path[draw=drawColor,line width= 0.4pt,line join=round,line cap=round] (296.48,119.60) -- (302.84,119.60);

\path[draw=drawColor,line width= 0.4pt,line join=round,line cap=round] (299.66,116.42) -- (299.66,122.78);

\path[draw=drawColor,line width= 0.4pt,line join=round,line cap=round] (303.63,119.60) -- (310.00,119.60);

\path[draw=drawColor,line width= 0.4pt,line join=round,line cap=round] (306.82,116.42) -- (306.82,122.78);

\path[draw=drawColor,line width= 0.4pt,line join=round,line cap=round] (310.79,119.60) -- (317.15,119.60);

\path[draw=drawColor,line width= 0.4pt,line join=round,line cap=round] (313.97,116.42) -- (313.97,122.78);

\path[draw=drawColor,line width= 0.4pt,line join=round,line cap=round] (317.95,119.60) -- (324.31,119.60);

\path[draw=drawColor,line width= 0.4pt,line join=round,line cap=round] (321.13,116.42) -- (321.13,122.78);

\path[draw=drawColor,line width= 0.4pt,line join=round,line cap=round] (325.10,119.60) -- (331.47,119.60);

\path[draw=drawColor,line width= 0.4pt,line join=round,line cap=round] (328.28,116.42) -- (328.28,122.78);

\path[draw=drawColor,line width= 0.4pt,line join=round,line cap=round] (332.26,119.60) -- (338.62,119.60);

\path[draw=drawColor,line width= 0.4pt,line join=round,line cap=round] (335.44,116.42) -- (335.44,122.78);

\path[draw=drawColor,line width= 0.4pt,line join=round,line cap=round] (339.41,119.60) -- (345.78,119.60);

\path[draw=drawColor,line width= 0.4pt,line join=round,line cap=round] (342.60,116.42) -- (342.60,122.78);

\path[draw=drawColor,line width= 0.4pt,line join=round,line cap=round] (346.57,119.60) -- (352.93,119.60);

\path[draw=drawColor,line width= 0.4pt,line join=round,line cap=round] (349.75,116.42) -- (349.75,122.78);

\path[draw=drawColor,line width= 0.4pt,line join=round,line cap=round] (353.73,119.60) -- (360.09,119.60);

\path[draw=drawColor,line width= 0.4pt,line join=round,line cap=round] (356.91,116.42) -- (356.91,122.78);

\path[draw=drawColor,line width= 0.4pt,line join=round,line cap=round] (360.88,119.60) -- (367.25,119.60);

\path[draw=drawColor,line width= 0.4pt,line join=round,line cap=round] (364.06,116.42) -- (364.06,122.78);

\path[draw=drawColor,line width= 0.4pt,line join=round,line cap=round] (368.04,119.60) -- (374.40,119.60);

\path[draw=drawColor,line width= 0.4pt,line join=round,line cap=round] (371.22,116.42) -- (371.22,122.78);

\path[draw=drawColor,line width= 0.4pt,line join=round,line cap=round] (375.19,119.60) -- (381.56,119.60);

\path[draw=drawColor,line width= 0.4pt,line join=round,line cap=round] (378.38,116.42) -- (378.38,122.78);

\path[draw=drawColor,line width= 0.4pt,line join=round,line cap=round] (382.35,119.60) -- (388.71,119.60);

\path[draw=drawColor,line width= 0.4pt,line join=round,line cap=round] (385.53,116.42) -- (385.53,122.78);

\path[draw=drawColor,line width= 0.4pt,line join=round,line cap=round] (389.51,119.60) -- (395.87,119.60);

\path[draw=drawColor,line width= 0.4pt,line join=round,line cap=round] (392.69,116.42) -- (392.69,122.78);

\path[draw=drawColor,line width= 0.4pt,line join=round,line cap=round] (396.66,119.60) -- (403.03,119.60);

\path[draw=drawColor,line width= 0.4pt,line join=round,line cap=round] (399.84,116.42) -- (399.84,122.78);

\path[draw=drawColor,line width= 0.4pt,line join=round,line cap=round] (403.82,119.60) -- (410.18,119.60);

\path[draw=drawColor,line width= 0.4pt,line join=round,line cap=round] (407.00,116.42) -- (407.00,122.78);

\path[draw=drawColor,line width= 0.4pt,line join=round,line cap=round] (410.97,119.60) -- (417.34,119.60);

\path[draw=drawColor,line width= 0.4pt,line join=round,line cap=round] (414.16,116.42) -- (414.16,122.78);

\path[draw=drawColor,line width= 0.4pt,line join=round,line cap=round] (418.13,119.60) -- (424.49,119.60);

\path[draw=drawColor,line width= 0.4pt,line join=round,line cap=round] (421.31,116.42) -- (421.31,122.78);

\path[draw=drawColor,line width= 0.4pt,line join=round,line cap=round] (425.29,119.60) -- (431.65,119.60);

\path[draw=drawColor,line width= 0.4pt,line join=round,line cap=round] (428.47,116.42) -- (428.47,122.78);

\path[draw=drawColor,line width= 0.4pt,line join=round,line cap=round] (432.44,119.60) -- (438.81,119.60);

\path[draw=drawColor,line width= 0.4pt,line join=round,line cap=round] (435.62,116.42) -- (435.62,122.78);

\path[draw=drawColor,line width= 0.4pt,line join=round,line cap=round] (439.60,119.60) -- (445.96,119.60);

\path[draw=drawColor,line width= 0.4pt,line join=round,line cap=round] (442.78,116.42) -- (442.78,122.78);

\path[draw=drawColor,line width= 0.4pt,line join=round,line cap=round] (446.75,119.60) -- (453.12,119.60);

\path[draw=drawColor,line width= 0.4pt,line join=round,line cap=round] (449.94,116.42) -- (449.94,122.78);

\path[draw=drawColor,line width= 0.4pt,line join=round,line cap=round] (453.91,119.60) -- (460.27,119.60);

\path[draw=drawColor,line width= 0.4pt,line join=round,line cap=round] (457.09,116.42) -- (457.09,122.78);

\path[draw=drawColor,line width= 0.4pt,line join=round,line cap=round] (461.07,119.60) -- (467.43,119.60);

\path[draw=drawColor,line width= 0.4pt,line join=round,line cap=round] (464.25,116.42) -- (464.25,122.78);

\path[draw=drawColor,line width= 0.4pt,line join=round,line cap=round] (468.22,119.60) -- (474.59,119.60);

\path[draw=drawColor,line width= 0.4pt,line join=round,line cap=round] (471.40,116.42) -- (471.40,122.78);

\path[draw=drawColor,line width= 0.4pt,line join=round,line cap=round] (475.38,119.60) -- (481.74,119.60);

\path[draw=drawColor,line width= 0.4pt,line join=round,line cap=round] (478.56,116.42) -- (478.56,122.78);

\path[draw=drawColor,line width= 0.4pt,line join=round,line cap=round] (482.53,119.60) -- (488.90,119.60);

\path[draw=drawColor,line width= 0.4pt,line join=round,line cap=round] (485.72,116.42) -- (485.72,122.78);

\path[draw=drawColor,line width= 0.4pt,line join=round,line cap=round] (489.69,119.60) -- (496.05,119.60);

\path[draw=drawColor,line width= 0.4pt,line join=round,line cap=round] (492.87,116.42) -- (492.87,122.78);
\end{scope}
\begin{scope}
\path[clip] (  0.00,  0.00) rectangle (289.08,216.81);
\definecolor[named]{drawColor}{rgb}{0.00,0.00,0.00}

\node[text=drawColor,rotate= 90.00,anchor=base,inner sep=0pt, outer sep=0pt, scale=  1.00] at ( 10.80,114.41) {Survival Prob.};

\node[text=drawColor,anchor=base,inner sep=0pt, outer sep=0pt, scale=  1.00] at (156.54, 15.60) {Time (Years)};
\end{scope}
\begin{scope}
\path[clip] (  0.00,  0.00) rectangle (578.16,216.81);
\definecolor[named]{drawColor}{rgb}{0.00,0.00,0.00}

\path[draw=drawColor,line width= 0.4pt,line join=round,line cap=round] (338.28, 61.20) -- (552.96, 61.20);

\path[draw=drawColor,line width= 0.4pt,line join=round,line cap=round] (338.28, 61.20) -- (338.28, 55.20);

\path[draw=drawColor,line width= 0.4pt,line join=round,line cap=round] (374.06, 61.20) -- (374.06, 55.20);

\path[draw=drawColor,line width= 0.4pt,line join=round,line cap=round] (409.84, 61.20) -- (409.84, 55.20);

\path[draw=drawColor,line width= 0.4pt,line join=round,line cap=round] (445.62, 61.20) -- (445.62, 55.20);

\path[draw=drawColor,line width= 0.4pt,line join=round,line cap=round] (481.40, 61.20) -- (481.40, 55.20);

\path[draw=drawColor,line width= 0.4pt,line join=round,line cap=round] (517.18, 61.20) -- (517.18, 55.20);

\path[draw=drawColor,line width= 0.4pt,line join=round,line cap=round] (552.96, 61.20) -- (552.96, 55.20);

\node[text=drawColor,anchor=base,inner sep=0pt, outer sep=0pt, scale=  1.00] at (338.28, 39.60) {0};

\node[text=drawColor,anchor=base,inner sep=0pt, outer sep=0pt, scale=  1.00] at (374.06, 39.60) {5};

\node[text=drawColor,anchor=base,inner sep=0pt, outer sep=0pt, scale=  1.00] at (409.84, 39.60) {10};

\node[text=drawColor,anchor=base,inner sep=0pt, outer sep=0pt, scale=  1.00] at (445.62, 39.60) {15};

\node[text=drawColor,anchor=base,inner sep=0pt, outer sep=0pt, scale=  1.00] at (481.40, 39.60) {20};

\node[text=drawColor,anchor=base,inner sep=0pt, outer sep=0pt, scale=  1.00] at (517.18, 39.60) {25};

\node[text=drawColor,anchor=base,inner sep=0pt, outer sep=0pt, scale=  1.00] at (552.96, 39.60) {30};

\path[draw=drawColor,line width= 0.4pt,line join=round,line cap=round] (338.28, 65.14) -- (338.28,163.67);

\path[draw=drawColor,line width= 0.4pt,line join=round,line cap=round] (338.28, 65.14) -- (332.28, 65.14);

\path[draw=drawColor,line width= 0.4pt,line join=round,line cap=round] (338.28, 84.85) -- (332.28, 84.85);

\path[draw=drawColor,line width= 0.4pt,line join=round,line cap=round] (338.28,104.55) -- (332.28,104.55);

\path[draw=drawColor,line width= 0.4pt,line join=round,line cap=round] (338.28,124.26) -- (332.28,124.26);

\path[draw=drawColor,line width= 0.4pt,line join=round,line cap=round] (338.28,143.96) -- (332.28,143.96);

\path[draw=drawColor,line width= 0.4pt,line join=round,line cap=round] (338.28,163.67) -- (332.28,163.67);

\node[text=drawColor,anchor=base east,inner sep=0pt, outer sep=0pt, scale=  1.00] at (326.28, 61.70) {0.0};

\node[text=drawColor,anchor=base east,inner sep=0pt, outer sep=0pt, scale=  1.00] at (326.28, 81.40) {0.2};

\node[text=drawColor,anchor=base east,inner sep=0pt, outer sep=0pt, scale=  1.00] at (326.28,101.11) {0.4};

\node[text=drawColor,anchor=base east,inner sep=0pt, outer sep=0pt, scale=  1.00] at (326.28,120.81) {0.6};

\node[text=drawColor,anchor=base east,inner sep=0pt, outer sep=0pt, scale=  1.00] at (326.28,140.52) {0.8};

\node[text=drawColor,anchor=base east,inner sep=0pt, outer sep=0pt, scale=  1.00] at (326.28,160.23) {1.0};

\path[draw=drawColor,line width= 0.4pt,line join=round,line cap=round] (338.28, 61.20) --
	(552.96, 61.20) --
	(552.96,167.61) --
	(338.28,167.61) --
	(338.28, 61.20);
\end{scope}
\begin{scope}
\path[clip] (289.08,  0.00) rectangle (578.16,216.81);
\definecolor[named]{drawColor}{rgb}{0.00,0.00,0.00}

\node[text=drawColor,anchor=base,inner sep=0pt, outer sep=0pt, scale=  1.20] at (445.62,188.07) {\bfseries Trade};
\end{scope}
\begin{scope}
\path[clip] (338.28, 61.20) rectangle (552.96,167.61);
\definecolor[named]{drawColor}{rgb}{0.66,0.66,0.66}

\path[draw=drawColor,line width= 0.4pt,line join=round,line cap=round] (338.28,163.67) --
	(345.44,163.67) --
	(345.44,155.51) --
	(352.59,155.51) --
	(352.59,149.38) --
	(359.75,149.38) --
	(359.75,144.89) --
	(366.90,144.89) --
	(366.90,143.36) --
	(374.06,143.36) --
	(374.06,142.23) --
	(381.22,142.23) --
	(381.22,141.74) --
	(388.37,141.74) --
	(388.37,141.07) --
	(395.53,141.07) --
	(395.53,140.72) --
	(402.68,140.72) --
	(402.68,140.34) --
	(409.84,140.34) --
	(409.84,139.56) --
	(417.00,139.56) --
	(417.00,139.35) --
	(431.31,139.35) --
	(431.31,139.13) --
	(438.46,139.13) --
	(438.46,138.88) --
	(459.93,138.88) --
	(459.93,138.53) --
	(467.09,138.53) --
	(467.09,138.04) --
	(481.40,138.04) --
	(481.40,137.36) --
	(488.56,137.36) --
	(488.56,136.50) --
	(510.02,136.50) --
	(510.02,135.49) --
	(578.16,135.49);

\path[draw=drawColor,line width= 0.4pt,line join=round,line cap=round] (342.25,155.51) -- (348.62,155.51);

\path[draw=drawColor,line width= 0.4pt,line join=round,line cap=round] (345.44,152.32) -- (345.44,158.69);

\path[draw=drawColor,line width= 0.4pt,line join=round,line cap=round] (349.41,149.38) -- (355.77,149.38);

\path[draw=drawColor,line width= 0.4pt,line join=round,line cap=round] (352.59,146.20) -- (352.59,152.56);

\path[draw=drawColor,line width= 0.4pt,line join=round,line cap=round] (356.57,144.89) -- (362.93,144.89);

\path[draw=drawColor,line width= 0.4pt,line join=round,line cap=round] (359.75,141.70) -- (359.75,148.07);

\path[draw=drawColor,line width= 0.4pt,line join=round,line cap=round] (363.72,143.36) -- (370.09,143.36);

\path[draw=drawColor,line width= 0.4pt,line join=round,line cap=round] (366.90,140.18) -- (366.90,146.54);

\path[draw=drawColor,line width= 0.4pt,line join=round,line cap=round] (370.88,142.23) -- (377.24,142.23);

\path[draw=drawColor,line width= 0.4pt,line join=round,line cap=round] (374.06,139.05) -- (374.06,145.41);

\path[draw=drawColor,line width= 0.4pt,line join=round,line cap=round] (378.03,141.74) -- (384.40,141.74);

\path[draw=drawColor,line width= 0.4pt,line join=round,line cap=round] (381.22,138.56) -- (381.22,144.92);

\path[draw=drawColor,line width= 0.4pt,line join=round,line cap=round] (385.19,141.07) -- (391.55,141.07);

\path[draw=drawColor,line width= 0.4pt,line join=round,line cap=round] (388.37,137.89) -- (388.37,144.25);

\path[draw=drawColor,line width= 0.4pt,line join=round,line cap=round] (392.35,140.72) -- (398.71,140.72);

\path[draw=drawColor,line width= 0.4pt,line join=round,line cap=round] (395.53,137.53) -- (395.53,143.90);

\path[draw=drawColor,line width= 0.4pt,line join=round,line cap=round] (399.50,140.34) -- (405.87,140.34);

\path[draw=drawColor,line width= 0.4pt,line join=round,line cap=round] (402.68,137.16) -- (402.68,143.52);

\path[draw=drawColor,line width= 0.4pt,line join=round,line cap=round] (406.66,139.56) -- (413.02,139.56);

\path[draw=drawColor,line width= 0.4pt,line join=round,line cap=round] (409.84,136.38) -- (409.84,142.74);

\path[draw=drawColor,line width= 0.4pt,line join=round,line cap=round] (413.81,139.35) -- (420.18,139.35);

\path[draw=drawColor,line width= 0.4pt,line join=round,line cap=round] (417.00,136.17) -- (417.00,142.53);

\path[draw=drawColor,line width= 0.4pt,line join=round,line cap=round] (420.97,139.35) -- (427.33,139.35);

\path[draw=drawColor,line width= 0.4pt,line join=round,line cap=round] (424.15,136.17) -- (424.15,142.53);

\path[draw=drawColor,line width= 0.4pt,line join=round,line cap=round] (428.13,139.13) -- (434.49,139.13);

\path[draw=drawColor,line width= 0.4pt,line join=round,line cap=round] (431.31,135.95) -- (431.31,142.31);

\path[draw=drawColor,line width= 0.4pt,line join=round,line cap=round] (435.28,138.88) -- (441.65,138.88);

\path[draw=drawColor,line width= 0.4pt,line join=round,line cap=round] (438.46,135.70) -- (438.46,142.07);

\path[draw=drawColor,line width= 0.4pt,line join=round,line cap=round] (442.44,138.88) -- (448.80,138.88);

\path[draw=drawColor,line width= 0.4pt,line join=round,line cap=round] (445.62,135.70) -- (445.62,142.07);

\path[draw=drawColor,line width= 0.4pt,line join=round,line cap=round] (449.59,138.88) -- (455.96,138.88);

\path[draw=drawColor,line width= 0.4pt,line join=round,line cap=round] (452.78,135.70) -- (452.78,142.07);

\path[draw=drawColor,line width= 0.4pt,line join=round,line cap=round] (456.75,138.53) -- (463.11,138.53);

\path[draw=drawColor,line width= 0.4pt,line join=round,line cap=round] (459.93,135.35) -- (459.93,141.71);

\path[draw=drawColor,line width= 0.4pt,line join=round,line cap=round] (463.91,138.04) -- (470.27,138.04);

\path[draw=drawColor,line width= 0.4pt,line join=round,line cap=round] (467.09,134.86) -- (467.09,141.22);

\path[draw=drawColor,line width= 0.4pt,line join=round,line cap=round] (471.06,138.04) -- (477.43,138.04);

\path[draw=drawColor,line width= 0.4pt,line join=round,line cap=round] (474.24,134.86) -- (474.24,141.22);

\path[draw=drawColor,line width= 0.4pt,line join=round,line cap=round] (478.22,137.36) -- (484.58,137.36);

\path[draw=drawColor,line width= 0.4pt,line join=round,line cap=round] (481.40,134.18) -- (481.40,140.54);

\path[draw=drawColor,line width= 0.4pt,line join=round,line cap=round] (485.37,136.50) -- (491.74,136.50);

\path[draw=drawColor,line width= 0.4pt,line join=round,line cap=round] (488.56,133.32) -- (488.56,139.68);

\path[draw=drawColor,line width= 0.4pt,line join=round,line cap=round] (492.53,136.50) -- (498.89,136.50);

\path[draw=drawColor,line width= 0.4pt,line join=round,line cap=round] (495.71,133.32) -- (495.71,139.68);

\path[draw=drawColor,line width= 0.4pt,line join=round,line cap=round] (499.69,136.50) -- (506.05,136.50);

\path[draw=drawColor,line width= 0.4pt,line join=round,line cap=round] (502.87,133.32) -- (502.87,139.68);

\path[draw=drawColor,line width= 0.4pt,line join=round,line cap=round] (506.84,135.49) -- (513.21,135.49);

\path[draw=drawColor,line width= 0.4pt,line join=round,line cap=round] (510.02,132.31) -- (510.02,138.67);

\path[draw=drawColor,line width= 0.4pt,line join=round,line cap=round] (514.00,135.49) -- (520.36,135.49);

\path[draw=drawColor,line width= 0.4pt,line join=round,line cap=round] (517.18,132.31) -- (517.18,138.67);

\path[draw=drawColor,line width= 0.4pt,line join=round,line cap=round] (521.15,135.49) -- (527.52,135.49);

\path[draw=drawColor,line width= 0.4pt,line join=round,line cap=round] (524.34,132.31) -- (524.34,138.67);

\path[draw=drawColor,line width= 0.4pt,line join=round,line cap=round] (528.31,135.49) -- (534.67,135.49);

\path[draw=drawColor,line width= 0.4pt,line join=round,line cap=round] (531.49,132.31) -- (531.49,138.67);

\path[draw=drawColor,line width= 0.4pt,line join=round,line cap=round] (535.47,135.49) -- (541.83,135.49);

\path[draw=drawColor,line width= 0.4pt,line join=round,line cap=round] (538.65,132.31) -- (538.65,138.67);

\path[draw=drawColor,line width= 0.4pt,line join=round,line cap=round] (542.62,135.49) -- (548.99,135.49);

\path[draw=drawColor,line width= 0.4pt,line join=round,line cap=round] (545.80,132.31) -- (545.80,138.67);

\path[draw=drawColor,line width= 0.4pt,line join=round,line cap=round] (549.78,135.49) -- (556.14,135.49);

\path[draw=drawColor,line width= 0.4pt,line join=round,line cap=round] (552.96,132.31) -- (552.96,138.67);

\path[draw=drawColor,line width= 0.4pt,line join=round,line cap=round] (556.93,135.49) -- (563.30,135.49);

\path[draw=drawColor,line width= 0.4pt,line join=round,line cap=round] (560.12,132.31) -- (560.12,138.67);

\path[draw=drawColor,line width= 0.4pt,line join=round,line cap=round] (564.09,135.49) -- (570.45,135.49);

\path[draw=drawColor,line width= 0.4pt,line join=round,line cap=round] (567.27,132.31) -- (567.27,138.67);

\path[draw=drawColor,line width= 0.4pt,line join=round,line cap=round] (571.25,135.49) -- (577.61,135.49);

\path[draw=drawColor,line width= 0.4pt,line join=round,line cap=round] (574.43,132.31) -- (574.43,138.67);
\definecolor[named]{drawColor}{rgb}{0.00,0.00,0.00}

\path[draw=drawColor,line width= 0.4pt,line join=round,line cap=round] (338.28,163.67) --
	(345.44,163.67) --
	(345.44,141.48) --
	(352.59,141.48) --
	(352.59,127.19) --
	(359.75,127.19) --
	(359.75,117.92) --
	(366.90,117.92) --
	(366.90,115.00) --
	(374.06,115.00) --
	(374.06,112.90) --
	(381.22,112.90) --
	(381.22,112.01) --
	(388.37,112.01) --
	(388.37,110.81) --
	(395.53,110.81) --
	(395.53,110.19) --
	(402.68,110.19) --
	(402.68,109.53) --
	(409.84,109.53) --
	(409.84,108.19) --
	(417.00,108.19) --
	(417.00,107.83) --
	(431.31,107.83) --
	(431.31,107.45) --
	(438.46,107.45) --
	(438.46,107.04) --
	(459.93,107.04) --
	(459.93,106.45) --
	(467.09,106.45) --
	(467.09,105.64) --
	(481.40,105.64) --
	(481.40,104.53) --
	(488.56,104.53) --
	(488.56,103.17) --
	(510.02,103.17) --
	(510.02,101.60) --
	(578.16,101.60);

\path[draw=drawColor,line width= 0.4pt,line join=round,line cap=round] (342.25,141.48) -- (348.62,141.48);

\path[draw=drawColor,line width= 0.4pt,line join=round,line cap=round] (345.44,138.30) -- (345.44,144.66);

\path[draw=drawColor,line width= 0.4pt,line join=round,line cap=round] (349.41,127.19) -- (355.77,127.19);

\path[draw=drawColor,line width= 0.4pt,line join=round,line cap=round] (352.59,124.01) -- (352.59,130.37);

\path[draw=drawColor,line width= 0.4pt,line join=round,line cap=round] (356.57,117.92) -- (362.93,117.92);

\path[draw=drawColor,line width= 0.4pt,line join=round,line cap=round] (359.75,114.74) -- (359.75,121.10);

\path[draw=drawColor,line width= 0.4pt,line join=round,line cap=round] (363.72,115.00) -- (370.09,115.00);

\path[draw=drawColor,line width= 0.4pt,line join=round,line cap=round] (366.90,111.81) -- (366.90,118.18);

\path[draw=drawColor,line width= 0.4pt,line join=round,line cap=round] (370.88,112.90) -- (377.24,112.90);

\path[draw=drawColor,line width= 0.4pt,line join=round,line cap=round] (374.06,109.72) -- (374.06,116.08);

\path[draw=drawColor,line width= 0.4pt,line join=round,line cap=round] (378.03,112.01) -- (384.40,112.01);

\path[draw=drawColor,line width= 0.4pt,line join=round,line cap=round] (381.22,108.82) -- (381.22,115.19);

\path[draw=drawColor,line width= 0.4pt,line join=round,line cap=round] (385.19,110.81) -- (391.55,110.81);

\path[draw=drawColor,line width= 0.4pt,line join=round,line cap=round] (388.37,107.63) -- (388.37,113.99);

\path[draw=drawColor,line width= 0.4pt,line join=round,line cap=round] (392.35,110.19) -- (398.71,110.19);

\path[draw=drawColor,line width= 0.4pt,line join=round,line cap=round] (395.53,107.00) -- (395.53,113.37);

\path[draw=drawColor,line width= 0.4pt,line join=round,line cap=round] (399.50,109.53) -- (405.87,109.53);

\path[draw=drawColor,line width= 0.4pt,line join=round,line cap=round] (402.68,106.34) -- (402.68,112.71);

\path[draw=drawColor,line width= 0.4pt,line join=round,line cap=round] (406.66,108.19) -- (413.02,108.19);

\path[draw=drawColor,line width= 0.4pt,line join=round,line cap=round] (409.84,105.01) -- (409.84,111.37);

\path[draw=drawColor,line width= 0.4pt,line join=round,line cap=round] (413.81,107.83) -- (420.18,107.83);

\path[draw=drawColor,line width= 0.4pt,line join=round,line cap=round] (417.00,104.65) -- (417.00,111.01);

\path[draw=drawColor,line width= 0.4pt,line join=round,line cap=round] (420.97,107.83) -- (427.33,107.83);

\path[draw=drawColor,line width= 0.4pt,line join=round,line cap=round] (424.15,104.65) -- (424.15,111.01);

\path[draw=drawColor,line width= 0.4pt,line join=round,line cap=round] (428.13,107.45) -- (434.49,107.45);

\path[draw=drawColor,line width= 0.4pt,line join=round,line cap=round] (431.31,104.27) -- (431.31,110.63);

\path[draw=drawColor,line width= 0.4pt,line join=round,line cap=round] (435.28,107.04) -- (441.65,107.04);

\path[draw=drawColor,line width= 0.4pt,line join=round,line cap=round] (438.46,103.86) -- (438.46,110.22);

\path[draw=drawColor,line width= 0.4pt,line join=round,line cap=round] (442.44,107.04) -- (448.80,107.04);

\path[draw=drawColor,line width= 0.4pt,line join=round,line cap=round] (445.62,103.86) -- (445.62,110.22);

\path[draw=drawColor,line width= 0.4pt,line join=round,line cap=round] (449.59,107.04) -- (455.96,107.04);

\path[draw=drawColor,line width= 0.4pt,line join=round,line cap=round] (452.78,103.86) -- (452.78,110.22);

\path[draw=drawColor,line width= 0.4pt,line join=round,line cap=round] (456.75,106.45) -- (463.11,106.45);

\path[draw=drawColor,line width= 0.4pt,line join=round,line cap=round] (459.93,103.26) -- (459.93,109.63);

\path[draw=drawColor,line width= 0.4pt,line join=round,line cap=round] (463.91,105.64) -- (470.27,105.64);

\path[draw=drawColor,line width= 0.4pt,line join=round,line cap=round] (467.09,102.45) -- (467.09,108.82);

\path[draw=drawColor,line width= 0.4pt,line join=round,line cap=round] (471.06,105.64) -- (477.43,105.64);

\path[draw=drawColor,line width= 0.4pt,line join=round,line cap=round] (474.24,102.45) -- (474.24,108.82);

\path[draw=drawColor,line width= 0.4pt,line join=round,line cap=round] (478.22,104.53) -- (484.58,104.53);

\path[draw=drawColor,line width= 0.4pt,line join=round,line cap=round] (481.40,101.35) -- (481.40,107.71);

\path[draw=drawColor,line width= 0.4pt,line join=round,line cap=round] (485.37,103.17) -- (491.74,103.17);

\path[draw=drawColor,line width= 0.4pt,line join=round,line cap=round] (488.56, 99.99) -- (488.56,106.35);

\path[draw=drawColor,line width= 0.4pt,line join=round,line cap=round] (492.53,103.17) -- (498.89,103.17);

\path[draw=drawColor,line width= 0.4pt,line join=round,line cap=round] (495.71, 99.99) -- (495.71,106.35);

\path[draw=drawColor,line width= 0.4pt,line join=round,line cap=round] (499.69,103.17) -- (506.05,103.17);

\path[draw=drawColor,line width= 0.4pt,line join=round,line cap=round] (502.87, 99.99) -- (502.87,106.35);

\path[draw=drawColor,line width= 0.4pt,line join=round,line cap=round] (506.84,101.60) -- (513.21,101.60);

\path[draw=drawColor,line width= 0.4pt,line join=round,line cap=round] (510.02, 98.41) -- (510.02,104.78);

\path[draw=drawColor,line width= 0.4pt,line join=round,line cap=round] (514.00,101.60) -- (520.36,101.60);

\path[draw=drawColor,line width= 0.4pt,line join=round,line cap=round] (517.18, 98.41) -- (517.18,104.78);

\path[draw=drawColor,line width= 0.4pt,line join=round,line cap=round] (521.15,101.60) -- (527.52,101.60);

\path[draw=drawColor,line width= 0.4pt,line join=round,line cap=round] (524.34, 98.41) -- (524.34,104.78);

\path[draw=drawColor,line width= 0.4pt,line join=round,line cap=round] (528.31,101.60) -- (534.67,101.60);

\path[draw=drawColor,line width= 0.4pt,line join=round,line cap=round] (531.49, 98.41) -- (531.49,104.78);

\path[draw=drawColor,line width= 0.4pt,line join=round,line cap=round] (535.47,101.60) -- (541.83,101.60);

\path[draw=drawColor,line width= 0.4pt,line join=round,line cap=round] (538.65, 98.41) -- (538.65,104.78);

\path[draw=drawColor,line width= 0.4pt,line join=round,line cap=round] (542.62,101.60) -- (548.99,101.60);

\path[draw=drawColor,line width= 0.4pt,line join=round,line cap=round] (545.80, 98.41) -- (545.80,104.78);

\path[draw=drawColor,line width= 0.4pt,line join=round,line cap=round] (549.78,101.60) -- (556.14,101.60);

\path[draw=drawColor,line width= 0.4pt,line join=round,line cap=round] (552.96, 98.41) -- (552.96,104.78);

\path[draw=drawColor,line width= 0.4pt,line join=round,line cap=round] (556.93,101.60) -- (563.30,101.60);

\path[draw=drawColor,line width= 0.4pt,line join=round,line cap=round] (560.12, 98.41) -- (560.12,104.78);

\path[draw=drawColor,line width= 0.4pt,line join=round,line cap=round] (564.09,101.60) -- (570.45,101.60);

\path[draw=drawColor,line width= 0.4pt,line join=round,line cap=round] (567.27, 98.41) -- (567.27,104.78);

\path[draw=drawColor,line width= 0.4pt,line join=round,line cap=round] (571.25,101.60) -- (577.61,101.60);

\path[draw=drawColor,line width= 0.4pt,line join=round,line cap=round] (574.43, 98.41) -- (574.43,104.78);
\end{scope}
\begin{scope}
\path[clip] (289.08,  0.00) rectangle (578.16,216.81);
\definecolor[named]{drawColor}{rgb}{0.00,0.00,0.00}

\node[text=drawColor,anchor=base,inner sep=0pt, outer sep=0pt, scale=  1.00] at (445.62, 15.60) {Time (Years)};
\end{scope}
\end{tikzpicture}
}	
\end{figure}

\begin{figure}[ht]
	\centering
	% \includegraphics[width=1\textwidth]{oNet}
	\resizebox{1\textwidth}{!}{% Created by tikzDevice version 0.7.0 on 2015-05-17 06:40:04
% !TEX encoding = UTF-8 Unicode
\begin{tikzpicture}[x=1pt,y=1pt]
\definecolor[named]{fillColor}{rgb}{1.00,1.00,1.00}
\path[use as bounding box,fill=fillColor,fill opacity=0.00] (0,0) rectangle (578.16,216.81);
\begin{scope}
\path[clip] (  0.00,  0.00) rectangle (578.16,216.81);
\definecolor[named]{drawColor}{rgb}{0.00,0.00,0.00}

\path[draw=drawColor,line width= 0.4pt,line join=round,line cap=round] ( 49.20, 61.20) -- (263.88, 61.20);

\path[draw=drawColor,line width= 0.4pt,line join=round,line cap=round] ( 49.20, 61.20) -- ( 49.20, 55.20);

\path[draw=drawColor,line width= 0.4pt,line join=round,line cap=round] ( 84.98, 61.20) -- ( 84.98, 55.20);

\path[draw=drawColor,line width= 0.4pt,line join=round,line cap=round] (120.76, 61.20) -- (120.76, 55.20);

\path[draw=drawColor,line width= 0.4pt,line join=round,line cap=round] (156.54, 61.20) -- (156.54, 55.20);

\path[draw=drawColor,line width= 0.4pt,line join=round,line cap=round] (192.32, 61.20) -- (192.32, 55.20);

\path[draw=drawColor,line width= 0.4pt,line join=round,line cap=round] (228.10, 61.20) -- (228.10, 55.20);

\path[draw=drawColor,line width= 0.4pt,line join=round,line cap=round] (263.88, 61.20) -- (263.88, 55.20);

\node[text=drawColor,anchor=base,inner sep=0pt, outer sep=0pt, scale=  1.00] at ( 49.20, 39.60) {0};

\node[text=drawColor,anchor=base,inner sep=0pt, outer sep=0pt, scale=  1.00] at ( 84.98, 39.60) {5};

\node[text=drawColor,anchor=base,inner sep=0pt, outer sep=0pt, scale=  1.00] at (120.76, 39.60) {10};

\node[text=drawColor,anchor=base,inner sep=0pt, outer sep=0pt, scale=  1.00] at (156.54, 39.60) {15};

\node[text=drawColor,anchor=base,inner sep=0pt, outer sep=0pt, scale=  1.00] at (192.32, 39.60) {20};

\node[text=drawColor,anchor=base,inner sep=0pt, outer sep=0pt, scale=  1.00] at (228.10, 39.60) {25};

\node[text=drawColor,anchor=base,inner sep=0pt, outer sep=0pt, scale=  1.00] at (263.88, 39.60) {30};

\path[draw=drawColor,line width= 0.4pt,line join=round,line cap=round] ( 49.20, 65.14) -- ( 49.20,163.67);

\path[draw=drawColor,line width= 0.4pt,line join=round,line cap=round] ( 49.20, 65.14) -- ( 43.20, 65.14);

\path[draw=drawColor,line width= 0.4pt,line join=round,line cap=round] ( 49.20, 84.85) -- ( 43.20, 84.85);

\path[draw=drawColor,line width= 0.4pt,line join=round,line cap=round] ( 49.20,104.55) -- ( 43.20,104.55);

\path[draw=drawColor,line width= 0.4pt,line join=round,line cap=round] ( 49.20,124.26) -- ( 43.20,124.26);

\path[draw=drawColor,line width= 0.4pt,line join=round,line cap=round] ( 49.20,143.96) -- ( 43.20,143.96);

\path[draw=drawColor,line width= 0.4pt,line join=round,line cap=round] ( 49.20,163.67) -- ( 43.20,163.67);

\node[text=drawColor,anchor=base east,inner sep=0pt, outer sep=0pt, scale=  1.00] at ( 37.20, 61.70) {0.0};

\node[text=drawColor,anchor=base east,inner sep=0pt, outer sep=0pt, scale=  1.00] at ( 37.20, 81.40) {0.2};

\node[text=drawColor,anchor=base east,inner sep=0pt, outer sep=0pt, scale=  1.00] at ( 37.20,101.11) {0.4};

\node[text=drawColor,anchor=base east,inner sep=0pt, outer sep=0pt, scale=  1.00] at ( 37.20,120.81) {0.6};

\node[text=drawColor,anchor=base east,inner sep=0pt, outer sep=0pt, scale=  1.00] at ( 37.20,140.52) {0.8};

\node[text=drawColor,anchor=base east,inner sep=0pt, outer sep=0pt, scale=  1.00] at ( 37.20,160.23) {1.0};

\path[draw=drawColor,line width= 0.4pt,line join=round,line cap=round] ( 49.20, 61.20) --
	(263.88, 61.20) --
	(263.88,167.61) --
	( 49.20,167.61) --
	( 49.20, 61.20);
\end{scope}
\begin{scope}
\path[clip] (  0.00,  0.00) rectangle (289.08,216.81);
\definecolor[named]{drawColor}{rgb}{0.00,0.00,0.00}

\node[text=drawColor,anchor=base,inner sep=0pt, outer sep=0pt, scale=  1.20] at (156.54,188.07) {\bfseries Compliance Reciprocity};
\end{scope}
\begin{scope}
\path[clip] ( 49.20, 61.20) rectangle (263.88,167.61);
\definecolor[named]{drawColor}{rgb}{0.66,0.66,0.66}

\path[draw=drawColor,line width= 0.4pt,line join=round,line cap=round] ( 49.20,163.67) --
	( 56.36,163.67) --
	( 56.36,162.71) --
	( 63.51,162.71) --
	( 63.51,161.62) --
	( 70.67,161.62) --
	( 70.67,160.49) --
	( 77.82,160.49) --
	( 77.82,159.82) --
	( 84.98,159.82) --
	( 84.98,159.56) --
	( 92.14,159.56) --
	( 92.14,159.47) --
	( 99.29,159.47) --
	( 99.29,159.37) --
	(106.45,159.37) --
	(106.45,159.26) --
	(113.60,159.26) --
	(113.60,159.13) --
	(120.76,159.13) --
	(120.76,159.00) --
	(142.23,159.00) --
	(142.23,158.78) --
	(149.38,158.78) --
	(149.38,158.47) --
	(178.01,158.47) --
	(178.01,157.71) --
	(192.32,157.71) --
	(192.32,156.73) --
	(199.48,156.73) --
	(199.48,155.74) --
	(220.94,155.74) --
	(220.94,154.76) --
	(464.25,154.76) --
	(464.25,154.76);

\path[draw=drawColor,line width= 0.4pt,line join=round,line cap=round] ( 53.17,162.71) -- ( 59.54,162.71);

\path[draw=drawColor,line width= 0.4pt,line join=round,line cap=round] ( 56.36,159.53) -- ( 56.36,165.90);

\path[draw=drawColor,line width= 0.4pt,line join=round,line cap=round] ( 60.33,161.62) -- ( 66.69,161.62);

\path[draw=drawColor,line width= 0.4pt,line join=round,line cap=round] ( 63.51,158.44) -- ( 63.51,164.80);

\path[draw=drawColor,line width= 0.4pt,line join=round,line cap=round] ( 67.49,160.49) -- ( 73.85,160.49);

\path[draw=drawColor,line width= 0.4pt,line join=round,line cap=round] ( 70.67,157.30) -- ( 70.67,163.67);

\path[draw=drawColor,line width= 0.4pt,line join=round,line cap=round] ( 74.64,159.82) -- ( 81.01,159.82);

\path[draw=drawColor,line width= 0.4pt,line join=round,line cap=round] ( 77.82,156.64) -- ( 77.82,163.01);

\path[draw=drawColor,line width= 0.4pt,line join=round,line cap=round] ( 81.80,159.56) -- ( 88.16,159.56);

\path[draw=drawColor,line width= 0.4pt,line join=round,line cap=round] ( 84.98,156.38) -- ( 84.98,162.74);

\path[draw=drawColor,line width= 0.4pt,line join=round,line cap=round] ( 88.95,159.47) -- ( 95.32,159.47);

\path[draw=drawColor,line width= 0.4pt,line join=round,line cap=round] ( 92.14,156.28) -- ( 92.14,162.65);

\path[draw=drawColor,line width= 0.4pt,line join=round,line cap=round] ( 96.11,159.37) -- (102.47,159.37);

\path[draw=drawColor,line width= 0.4pt,line join=round,line cap=round] ( 99.29,156.19) -- ( 99.29,162.55);

\path[draw=drawColor,line width= 0.4pt,line join=round,line cap=round] (103.27,159.26) -- (109.63,159.26);

\path[draw=drawColor,line width= 0.4pt,line join=round,line cap=round] (106.45,156.08) -- (106.45,162.44);

\path[draw=drawColor,line width= 0.4pt,line join=round,line cap=round] (110.42,159.13) -- (116.79,159.13);

\path[draw=drawColor,line width= 0.4pt,line join=round,line cap=round] (113.60,155.95) -- (113.60,162.31);

\path[draw=drawColor,line width= 0.4pt,line join=round,line cap=round] (117.58,159.00) -- (123.94,159.00);

\path[draw=drawColor,line width= 0.4pt,line join=round,line cap=round] (120.76,155.82) -- (120.76,162.18);

\path[draw=drawColor,line width= 0.4pt,line join=round,line cap=round] (124.73,159.00) -- (131.10,159.00);

\path[draw=drawColor,line width= 0.4pt,line join=round,line cap=round] (127.92,155.82) -- (127.92,162.18);

\path[draw=drawColor,line width= 0.4pt,line join=round,line cap=round] (131.89,159.00) -- (138.25,159.00);

\path[draw=drawColor,line width= 0.4pt,line join=round,line cap=round] (135.07,155.82) -- (135.07,162.18);

\path[draw=drawColor,line width= 0.4pt,line join=round,line cap=round] (139.05,158.78) -- (145.41,158.78);

\path[draw=drawColor,line width= 0.4pt,line join=round,line cap=round] (142.23,155.60) -- (142.23,161.97);

\path[draw=drawColor,line width= 0.4pt,line join=round,line cap=round] (146.20,158.47) -- (152.57,158.47);

\path[draw=drawColor,line width= 0.4pt,line join=round,line cap=round] (149.38,155.29) -- (149.38,161.65);

\path[draw=drawColor,line width= 0.4pt,line join=round,line cap=round] (153.36,158.47) -- (159.72,158.47);

\path[draw=drawColor,line width= 0.4pt,line join=round,line cap=round] (156.54,155.29) -- (156.54,161.65);

\path[draw=drawColor,line width= 0.4pt,line join=round,line cap=round] (160.51,158.47) -- (166.88,158.47);

\path[draw=drawColor,line width= 0.4pt,line join=round,line cap=round] (163.70,155.29) -- (163.70,161.65);

\path[draw=drawColor,line width= 0.4pt,line join=round,line cap=round] (167.67,158.47) -- (174.03,158.47);

\path[draw=drawColor,line width= 0.4pt,line join=round,line cap=round] (170.85,155.29) -- (170.85,161.65);

\path[draw=drawColor,line width= 0.4pt,line join=round,line cap=round] (174.83,157.71) -- (181.19,157.71);

\path[draw=drawColor,line width= 0.4pt,line join=round,line cap=round] (178.01,154.52) -- (178.01,160.89);

\path[draw=drawColor,line width= 0.4pt,line join=round,line cap=round] (181.98,157.71) -- (188.35,157.71);

\path[draw=drawColor,line width= 0.4pt,line join=round,line cap=round] (185.16,154.52) -- (185.16,160.89);

\path[draw=drawColor,line width= 0.4pt,line join=round,line cap=round] (189.14,156.73) -- (195.50,156.73);

\path[draw=drawColor,line width= 0.4pt,line join=round,line cap=round] (192.32,153.55) -- (192.32,159.91);

\path[draw=drawColor,line width= 0.4pt,line join=round,line cap=round] (196.29,155.74) -- (202.66,155.74);

\path[draw=drawColor,line width= 0.4pt,line join=round,line cap=round] (199.48,152.56) -- (199.48,158.92);

\path[draw=drawColor,line width= 0.4pt,line join=round,line cap=round] (203.45,155.74) -- (209.81,155.74);

\path[draw=drawColor,line width= 0.4pt,line join=round,line cap=round] (206.63,152.56) -- (206.63,158.92);

\path[draw=drawColor,line width= 0.4pt,line join=round,line cap=round] (210.61,155.74) -- (216.97,155.74);

\path[draw=drawColor,line width= 0.4pt,line join=round,line cap=round] (213.79,152.56) -- (213.79,158.92);

\path[draw=drawColor,line width= 0.4pt,line join=round,line cap=round] (217.76,154.76) -- (224.13,154.76);

\path[draw=drawColor,line width= 0.4pt,line join=round,line cap=round] (220.94,151.58) -- (220.94,157.95);

\path[draw=drawColor,line width= 0.4pt,line join=round,line cap=round] (224.92,154.76) -- (231.28,154.76);

\path[draw=drawColor,line width= 0.4pt,line join=round,line cap=round] (228.10,151.58) -- (228.10,157.95);

\path[draw=drawColor,line width= 0.4pt,line join=round,line cap=round] (232.07,154.76) -- (238.44,154.76);

\path[draw=drawColor,line width= 0.4pt,line join=round,line cap=round] (235.26,151.58) -- (235.26,157.95);

\path[draw=drawColor,line width= 0.4pt,line join=round,line cap=round] (239.23,154.76) -- (245.59,154.76);

\path[draw=drawColor,line width= 0.4pt,line join=round,line cap=round] (242.41,151.58) -- (242.41,157.95);

\path[draw=drawColor,line width= 0.4pt,line join=round,line cap=round] (246.39,154.76) -- (252.75,154.76);

\path[draw=drawColor,line width= 0.4pt,line join=round,line cap=round] (249.57,151.58) -- (249.57,157.95);

\path[draw=drawColor,line width= 0.4pt,line join=round,line cap=round] (253.54,154.76) -- (259.91,154.76);

\path[draw=drawColor,line width= 0.4pt,line join=round,line cap=round] (256.72,151.58) -- (256.72,157.95);

\path[draw=drawColor,line width= 0.4pt,line join=round,line cap=round] (260.70,154.76) -- (267.06,154.76);

\path[draw=drawColor,line width= 0.4pt,line join=round,line cap=round] (263.88,151.58) -- (263.88,157.95);

\path[draw=drawColor,line width= 0.4pt,line join=round,line cap=round] (267.85,154.76) -- (274.22,154.76);

\path[draw=drawColor,line width= 0.4pt,line join=round,line cap=round] (271.04,151.58) -- (271.04,157.95);

\path[draw=drawColor,line width= 0.4pt,line join=round,line cap=round] (275.01,154.76) -- (281.37,154.76);

\path[draw=drawColor,line width= 0.4pt,line join=round,line cap=round] (278.19,151.58) -- (278.19,157.95);

\path[draw=drawColor,line width= 0.4pt,line join=round,line cap=round] (282.17,154.76) -- (288.53,154.76);

\path[draw=drawColor,line width= 0.4pt,line join=round,line cap=round] (285.35,151.58) -- (285.35,157.95);

\path[draw=drawColor,line width= 0.4pt,line join=round,line cap=round] (289.32,154.76) -- (295.69,154.76);

\path[draw=drawColor,line width= 0.4pt,line join=round,line cap=round] (292.50,151.58) -- (292.50,157.95);

\path[draw=drawColor,line width= 0.4pt,line join=round,line cap=round] (296.48,154.76) -- (302.84,154.76);

\path[draw=drawColor,line width= 0.4pt,line join=round,line cap=round] (299.66,151.58) -- (299.66,157.95);

\path[draw=drawColor,line width= 0.4pt,line join=round,line cap=round] (303.63,154.76) -- (310.00,154.76);

\path[draw=drawColor,line width= 0.4pt,line join=round,line cap=round] (306.82,151.58) -- (306.82,157.95);

\path[draw=drawColor,line width= 0.4pt,line join=round,line cap=round] (310.79,154.76) -- (317.15,154.76);

\path[draw=drawColor,line width= 0.4pt,line join=round,line cap=round] (313.97,151.58) -- (313.97,157.95);

\path[draw=drawColor,line width= 0.4pt,line join=round,line cap=round] (317.95,154.76) -- (324.31,154.76);

\path[draw=drawColor,line width= 0.4pt,line join=round,line cap=round] (321.13,151.58) -- (321.13,157.95);

\path[draw=drawColor,line width= 0.4pt,line join=round,line cap=round] (325.10,154.76) -- (331.47,154.76);

\path[draw=drawColor,line width= 0.4pt,line join=round,line cap=round] (328.28,151.58) -- (328.28,157.95);

\path[draw=drawColor,line width= 0.4pt,line join=round,line cap=round] (332.26,154.76) -- (338.62,154.76);

\path[draw=drawColor,line width= 0.4pt,line join=round,line cap=round] (335.44,151.58) -- (335.44,157.95);

\path[draw=drawColor,line width= 0.4pt,line join=round,line cap=round] (339.41,154.76) -- (345.78,154.76);

\path[draw=drawColor,line width= 0.4pt,line join=round,line cap=round] (342.60,151.58) -- (342.60,157.95);

\path[draw=drawColor,line width= 0.4pt,line join=round,line cap=round] (346.57,154.76) -- (352.93,154.76);

\path[draw=drawColor,line width= 0.4pt,line join=round,line cap=round] (349.75,151.58) -- (349.75,157.95);

\path[draw=drawColor,line width= 0.4pt,line join=round,line cap=round] (353.73,154.76) -- (360.09,154.76);

\path[draw=drawColor,line width= 0.4pt,line join=round,line cap=round] (356.91,151.58) -- (356.91,157.95);

\path[draw=drawColor,line width= 0.4pt,line join=round,line cap=round] (360.88,154.76) -- (367.25,154.76);

\path[draw=drawColor,line width= 0.4pt,line join=round,line cap=round] (364.06,151.58) -- (364.06,157.95);

\path[draw=drawColor,line width= 0.4pt,line join=round,line cap=round] (368.04,154.76) -- (374.40,154.76);

\path[draw=drawColor,line width= 0.4pt,line join=round,line cap=round] (371.22,151.58) -- (371.22,157.95);

\path[draw=drawColor,line width= 0.4pt,line join=round,line cap=round] (375.19,154.76) -- (381.56,154.76);

\path[draw=drawColor,line width= 0.4pt,line join=round,line cap=round] (378.38,151.58) -- (378.38,157.95);

\path[draw=drawColor,line width= 0.4pt,line join=round,line cap=round] (382.35,154.76) -- (388.71,154.76);

\path[draw=drawColor,line width= 0.4pt,line join=round,line cap=round] (385.53,151.58) -- (385.53,157.95);

\path[draw=drawColor,line width= 0.4pt,line join=round,line cap=round] (389.51,154.76) -- (395.87,154.76);

\path[draw=drawColor,line width= 0.4pt,line join=round,line cap=round] (392.69,151.58) -- (392.69,157.95);

\path[draw=drawColor,line width= 0.4pt,line join=round,line cap=round] (396.66,154.76) -- (403.03,154.76);

\path[draw=drawColor,line width= 0.4pt,line join=round,line cap=round] (399.84,151.58) -- (399.84,157.95);

\path[draw=drawColor,line width= 0.4pt,line join=round,line cap=round] (403.82,154.76) -- (410.18,154.76);

\path[draw=drawColor,line width= 0.4pt,line join=round,line cap=round] (407.00,151.58) -- (407.00,157.95);

\path[draw=drawColor,line width= 0.4pt,line join=round,line cap=round] (410.97,154.76) -- (417.34,154.76);

\path[draw=drawColor,line width= 0.4pt,line join=round,line cap=round] (414.16,151.58) -- (414.16,157.95);

\path[draw=drawColor,line width= 0.4pt,line join=round,line cap=round] (418.13,154.76) -- (424.49,154.76);

\path[draw=drawColor,line width= 0.4pt,line join=round,line cap=round] (421.31,151.58) -- (421.31,157.95);

\path[draw=drawColor,line width= 0.4pt,line join=round,line cap=round] (425.29,154.76) -- (431.65,154.76);

\path[draw=drawColor,line width= 0.4pt,line join=round,line cap=round] (428.47,151.58) -- (428.47,157.95);

\path[draw=drawColor,line width= 0.4pt,line join=round,line cap=round] (432.44,154.76) -- (438.81,154.76);

\path[draw=drawColor,line width= 0.4pt,line join=round,line cap=round] (435.62,151.58) -- (435.62,157.95);

\path[draw=drawColor,line width= 0.4pt,line join=round,line cap=round] (439.60,154.76) -- (445.96,154.76);

\path[draw=drawColor,line width= 0.4pt,line join=round,line cap=round] (442.78,151.58) -- (442.78,157.95);

\path[draw=drawColor,line width= 0.4pt,line join=round,line cap=round] (446.75,154.76) -- (453.12,154.76);

\path[draw=drawColor,line width= 0.4pt,line join=round,line cap=round] (449.94,151.58) -- (449.94,157.95);

\path[draw=drawColor,line width= 0.4pt,line join=round,line cap=round] (453.91,154.76) -- (460.27,154.76);

\path[draw=drawColor,line width= 0.4pt,line join=round,line cap=round] (457.09,151.58) -- (457.09,157.95);

\path[draw=drawColor,line width= 0.4pt,line join=round,line cap=round] (461.07,154.76) -- (467.43,154.76);

\path[draw=drawColor,line width= 0.4pt,line join=round,line cap=round] (464.25,151.58) -- (464.25,157.95);
\definecolor[named]{drawColor}{rgb}{0.00,0.00,0.00}

\path[draw=drawColor,line width= 0.4pt,line join=round,line cap=round] ( 49.20,163.67) --
	( 56.36,163.67) --
	( 56.36,156.34) --
	( 63.51,156.34) --
	( 63.51,148.53) --
	( 70.67,148.53) --
	( 70.67,141.08) --
	( 77.82,141.08) --
	( 77.82,136.99) --
	( 84.98,136.99) --
	( 84.98,135.42) --
	( 92.14,135.42) --
	( 92.14,134.87) --
	( 99.29,134.87) --
	( 99.29,134.31) --
	(106.45,134.31) --
	(106.45,133.67) --
	(113.60,133.67) --
	(113.60,132.94) --
	(120.76,132.94) --
	(120.76,132.18) --
	(142.23,132.18) --
	(142.23,130.97) --
	(149.38,130.97) --
	(149.38,129.25) --
	(178.01,129.25) --
	(178.01,125.19) --
	(192.32,125.19) --
	(192.32,120.36) --
	(199.48,120.36) --
	(199.48,115.80) --
	(220.94,115.80) --
	(220.94,111.63) --
	(464.25,111.63) --
	(464.25,111.63);

\path[draw=drawColor,line width= 0.4pt,line join=round,line cap=round] ( 53.17,156.34) -- ( 59.54,156.34);

\path[draw=drawColor,line width= 0.4pt,line join=round,line cap=round] ( 56.36,153.16) -- ( 56.36,159.52);

\path[draw=drawColor,line width= 0.4pt,line join=round,line cap=round] ( 60.33,148.53) -- ( 66.69,148.53);

\path[draw=drawColor,line width= 0.4pt,line join=round,line cap=round] ( 63.51,145.35) -- ( 63.51,151.71);

\path[draw=drawColor,line width= 0.4pt,line join=round,line cap=round] ( 67.49,141.08) -- ( 73.85,141.08);

\path[draw=drawColor,line width= 0.4pt,line join=round,line cap=round] ( 70.67,137.90) -- ( 70.67,144.26);

\path[draw=drawColor,line width= 0.4pt,line join=round,line cap=round] ( 74.64,136.99) -- ( 81.01,136.99);

\path[draw=drawColor,line width= 0.4pt,line join=round,line cap=round] ( 77.82,133.81) -- ( 77.82,140.17);

\path[draw=drawColor,line width= 0.4pt,line join=round,line cap=round] ( 81.80,135.42) -- ( 88.16,135.42);

\path[draw=drawColor,line width= 0.4pt,line join=round,line cap=round] ( 84.98,132.24) -- ( 84.98,138.60);

\path[draw=drawColor,line width= 0.4pt,line join=round,line cap=round] ( 88.95,134.87) -- ( 95.32,134.87);

\path[draw=drawColor,line width= 0.4pt,line join=round,line cap=round] ( 92.14,131.68) -- ( 92.14,138.05);

\path[draw=drawColor,line width= 0.4pt,line join=round,line cap=round] ( 96.11,134.31) -- (102.47,134.31);

\path[draw=drawColor,line width= 0.4pt,line join=round,line cap=round] ( 99.29,131.13) -- ( 99.29,137.49);

\path[draw=drawColor,line width= 0.4pt,line join=round,line cap=round] (103.27,133.67) -- (109.63,133.67);

\path[draw=drawColor,line width= 0.4pt,line join=round,line cap=round] (106.45,130.49) -- (106.45,136.85);

\path[draw=drawColor,line width= 0.4pt,line join=round,line cap=round] (110.42,132.94) -- (116.79,132.94);

\path[draw=drawColor,line width= 0.4pt,line join=round,line cap=round] (113.60,129.75) -- (113.60,136.12);

\path[draw=drawColor,line width= 0.4pt,line join=round,line cap=round] (117.58,132.18) -- (123.94,132.18);

\path[draw=drawColor,line width= 0.4pt,line join=round,line cap=round] (120.76,128.99) -- (120.76,135.36);

\path[draw=drawColor,line width= 0.4pt,line join=round,line cap=round] (124.73,132.18) -- (131.10,132.18);

\path[draw=drawColor,line width= 0.4pt,line join=round,line cap=round] (127.92,128.99) -- (127.92,135.36);

\path[draw=drawColor,line width= 0.4pt,line join=round,line cap=round] (131.89,132.18) -- (138.25,132.18);

\path[draw=drawColor,line width= 0.4pt,line join=round,line cap=round] (135.07,128.99) -- (135.07,135.36);

\path[draw=drawColor,line width= 0.4pt,line join=round,line cap=round] (139.05,130.97) -- (145.41,130.97);

\path[draw=drawColor,line width= 0.4pt,line join=round,line cap=round] (142.23,127.79) -- (142.23,134.15);

\path[draw=drawColor,line width= 0.4pt,line join=round,line cap=round] (146.20,129.25) -- (152.57,129.25);

\path[draw=drawColor,line width= 0.4pt,line join=round,line cap=round] (149.38,126.06) -- (149.38,132.43);

\path[draw=drawColor,line width= 0.4pt,line join=round,line cap=round] (153.36,129.25) -- (159.72,129.25);

\path[draw=drawColor,line width= 0.4pt,line join=round,line cap=round] (156.54,126.06) -- (156.54,132.43);

\path[draw=drawColor,line width= 0.4pt,line join=round,line cap=round] (160.51,129.25) -- (166.88,129.25);

\path[draw=drawColor,line width= 0.4pt,line join=round,line cap=round] (163.70,126.06) -- (163.70,132.43);

\path[draw=drawColor,line width= 0.4pt,line join=round,line cap=round] (167.67,129.25) -- (174.03,129.25);

\path[draw=drawColor,line width= 0.4pt,line join=round,line cap=round] (170.85,126.06) -- (170.85,132.43);

\path[draw=drawColor,line width= 0.4pt,line join=round,line cap=round] (174.83,125.19) -- (181.19,125.19);

\path[draw=drawColor,line width= 0.4pt,line join=round,line cap=round] (178.01,122.01) -- (178.01,128.38);

\path[draw=drawColor,line width= 0.4pt,line join=round,line cap=round] (181.98,125.19) -- (188.35,125.19);

\path[draw=drawColor,line width= 0.4pt,line join=round,line cap=round] (185.16,122.01) -- (185.16,128.38);

\path[draw=drawColor,line width= 0.4pt,line join=round,line cap=round] (189.14,120.36) -- (195.50,120.36);

\path[draw=drawColor,line width= 0.4pt,line join=round,line cap=round] (192.32,117.18) -- (192.32,123.54);

\path[draw=drawColor,line width= 0.4pt,line join=round,line cap=round] (196.29,115.80) -- (202.66,115.80);

\path[draw=drawColor,line width= 0.4pt,line join=round,line cap=round] (199.48,112.62) -- (199.48,118.99);

\path[draw=drawColor,line width= 0.4pt,line join=round,line cap=round] (203.45,115.80) -- (209.81,115.80);

\path[draw=drawColor,line width= 0.4pt,line join=round,line cap=round] (206.63,112.62) -- (206.63,118.99);

\path[draw=drawColor,line width= 0.4pt,line join=round,line cap=round] (210.61,115.80) -- (216.97,115.80);

\path[draw=drawColor,line width= 0.4pt,line join=round,line cap=round] (213.79,112.62) -- (213.79,118.99);

\path[draw=drawColor,line width= 0.4pt,line join=round,line cap=round] (217.76,111.63) -- (224.13,111.63);

\path[draw=drawColor,line width= 0.4pt,line join=round,line cap=round] (220.94,108.44) -- (220.94,114.81);

\path[draw=drawColor,line width= 0.4pt,line join=round,line cap=round] (224.92,111.63) -- (231.28,111.63);

\path[draw=drawColor,line width= 0.4pt,line join=round,line cap=round] (228.10,108.44) -- (228.10,114.81);

\path[draw=drawColor,line width= 0.4pt,line join=round,line cap=round] (232.07,111.63) -- (238.44,111.63);

\path[draw=drawColor,line width= 0.4pt,line join=round,line cap=round] (235.26,108.44) -- (235.26,114.81);

\path[draw=drawColor,line width= 0.4pt,line join=round,line cap=round] (239.23,111.63) -- (245.59,111.63);

\path[draw=drawColor,line width= 0.4pt,line join=round,line cap=round] (242.41,108.44) -- (242.41,114.81);

\path[draw=drawColor,line width= 0.4pt,line join=round,line cap=round] (246.39,111.63) -- (252.75,111.63);

\path[draw=drawColor,line width= 0.4pt,line join=round,line cap=round] (249.57,108.44) -- (249.57,114.81);

\path[draw=drawColor,line width= 0.4pt,line join=round,line cap=round] (253.54,111.63) -- (259.91,111.63);

\path[draw=drawColor,line width= 0.4pt,line join=round,line cap=round] (256.72,108.44) -- (256.72,114.81);

\path[draw=drawColor,line width= 0.4pt,line join=round,line cap=round] (260.70,111.63) -- (267.06,111.63);

\path[draw=drawColor,line width= 0.4pt,line join=round,line cap=round] (263.88,108.44) -- (263.88,114.81);

\path[draw=drawColor,line width= 0.4pt,line join=round,line cap=round] (267.85,111.63) -- (274.22,111.63);

\path[draw=drawColor,line width= 0.4pt,line join=round,line cap=round] (271.04,108.44) -- (271.04,114.81);

\path[draw=drawColor,line width= 0.4pt,line join=round,line cap=round] (275.01,111.63) -- (281.37,111.63);

\path[draw=drawColor,line width= 0.4pt,line join=round,line cap=round] (278.19,108.44) -- (278.19,114.81);

\path[draw=drawColor,line width= 0.4pt,line join=round,line cap=round] (282.17,111.63) -- (288.53,111.63);

\path[draw=drawColor,line width= 0.4pt,line join=round,line cap=round] (285.35,108.44) -- (285.35,114.81);

\path[draw=drawColor,line width= 0.4pt,line join=round,line cap=round] (289.32,111.63) -- (295.69,111.63);

\path[draw=drawColor,line width= 0.4pt,line join=round,line cap=round] (292.50,108.44) -- (292.50,114.81);

\path[draw=drawColor,line width= 0.4pt,line join=round,line cap=round] (296.48,111.63) -- (302.84,111.63);

\path[draw=drawColor,line width= 0.4pt,line join=round,line cap=round] (299.66,108.44) -- (299.66,114.81);

\path[draw=drawColor,line width= 0.4pt,line join=round,line cap=round] (303.63,111.63) -- (310.00,111.63);

\path[draw=drawColor,line width= 0.4pt,line join=round,line cap=round] (306.82,108.44) -- (306.82,114.81);

\path[draw=drawColor,line width= 0.4pt,line join=round,line cap=round] (310.79,111.63) -- (317.15,111.63);

\path[draw=drawColor,line width= 0.4pt,line join=round,line cap=round] (313.97,108.44) -- (313.97,114.81);

\path[draw=drawColor,line width= 0.4pt,line join=round,line cap=round] (317.95,111.63) -- (324.31,111.63);

\path[draw=drawColor,line width= 0.4pt,line join=round,line cap=round] (321.13,108.44) -- (321.13,114.81);

\path[draw=drawColor,line width= 0.4pt,line join=round,line cap=round] (325.10,111.63) -- (331.47,111.63);

\path[draw=drawColor,line width= 0.4pt,line join=round,line cap=round] (328.28,108.44) -- (328.28,114.81);

\path[draw=drawColor,line width= 0.4pt,line join=round,line cap=round] (332.26,111.63) -- (338.62,111.63);

\path[draw=drawColor,line width= 0.4pt,line join=round,line cap=round] (335.44,108.44) -- (335.44,114.81);

\path[draw=drawColor,line width= 0.4pt,line join=round,line cap=round] (339.41,111.63) -- (345.78,111.63);

\path[draw=drawColor,line width= 0.4pt,line join=round,line cap=round] (342.60,108.44) -- (342.60,114.81);

\path[draw=drawColor,line width= 0.4pt,line join=round,line cap=round] (346.57,111.63) -- (352.93,111.63);

\path[draw=drawColor,line width= 0.4pt,line join=round,line cap=round] (349.75,108.44) -- (349.75,114.81);

\path[draw=drawColor,line width= 0.4pt,line join=round,line cap=round] (353.73,111.63) -- (360.09,111.63);

\path[draw=drawColor,line width= 0.4pt,line join=round,line cap=round] (356.91,108.44) -- (356.91,114.81);

\path[draw=drawColor,line width= 0.4pt,line join=round,line cap=round] (360.88,111.63) -- (367.25,111.63);

\path[draw=drawColor,line width= 0.4pt,line join=round,line cap=round] (364.06,108.44) -- (364.06,114.81);

\path[draw=drawColor,line width= 0.4pt,line join=round,line cap=round] (368.04,111.63) -- (374.40,111.63);

\path[draw=drawColor,line width= 0.4pt,line join=round,line cap=round] (371.22,108.44) -- (371.22,114.81);

\path[draw=drawColor,line width= 0.4pt,line join=round,line cap=round] (375.19,111.63) -- (381.56,111.63);

\path[draw=drawColor,line width= 0.4pt,line join=round,line cap=round] (378.38,108.44) -- (378.38,114.81);

\path[draw=drawColor,line width= 0.4pt,line join=round,line cap=round] (382.35,111.63) -- (388.71,111.63);

\path[draw=drawColor,line width= 0.4pt,line join=round,line cap=round] (385.53,108.44) -- (385.53,114.81);

\path[draw=drawColor,line width= 0.4pt,line join=round,line cap=round] (389.51,111.63) -- (395.87,111.63);

\path[draw=drawColor,line width= 0.4pt,line join=round,line cap=round] (392.69,108.44) -- (392.69,114.81);

\path[draw=drawColor,line width= 0.4pt,line join=round,line cap=round] (396.66,111.63) -- (403.03,111.63);

\path[draw=drawColor,line width= 0.4pt,line join=round,line cap=round] (399.84,108.44) -- (399.84,114.81);

\path[draw=drawColor,line width= 0.4pt,line join=round,line cap=round] (403.82,111.63) -- (410.18,111.63);

\path[draw=drawColor,line width= 0.4pt,line join=round,line cap=round] (407.00,108.44) -- (407.00,114.81);

\path[draw=drawColor,line width= 0.4pt,line join=round,line cap=round] (410.97,111.63) -- (417.34,111.63);

\path[draw=drawColor,line width= 0.4pt,line join=round,line cap=round] (414.16,108.44) -- (414.16,114.81);

\path[draw=drawColor,line width= 0.4pt,line join=round,line cap=round] (418.13,111.63) -- (424.49,111.63);

\path[draw=drawColor,line width= 0.4pt,line join=round,line cap=round] (421.31,108.44) -- (421.31,114.81);

\path[draw=drawColor,line width= 0.4pt,line join=round,line cap=round] (425.29,111.63) -- (431.65,111.63);

\path[draw=drawColor,line width= 0.4pt,line join=round,line cap=round] (428.47,108.44) -- (428.47,114.81);

\path[draw=drawColor,line width= 0.4pt,line join=round,line cap=round] (432.44,111.63) -- (438.81,111.63);

\path[draw=drawColor,line width= 0.4pt,line join=round,line cap=round] (435.62,108.44) -- (435.62,114.81);

\path[draw=drawColor,line width= 0.4pt,line join=round,line cap=round] (439.60,111.63) -- (445.96,111.63);

\path[draw=drawColor,line width= 0.4pt,line join=round,line cap=round] (442.78,108.44) -- (442.78,114.81);

\path[draw=drawColor,line width= 0.4pt,line join=round,line cap=round] (446.75,111.63) -- (453.12,111.63);

\path[draw=drawColor,line width= 0.4pt,line join=round,line cap=round] (449.94,108.44) -- (449.94,114.81);

\path[draw=drawColor,line width= 0.4pt,line join=round,line cap=round] (453.91,111.63) -- (460.27,111.63);

\path[draw=drawColor,line width= 0.4pt,line join=round,line cap=round] (457.09,108.44) -- (457.09,114.81);

\path[draw=drawColor,line width= 0.4pt,line join=round,line cap=round] (461.07,111.63) -- (467.43,111.63);

\path[draw=drawColor,line width= 0.4pt,line join=round,line cap=round] (464.25,108.44) -- (464.25,114.81);
\end{scope}
\begin{scope}
\path[clip] (  0.00,  0.00) rectangle (289.08,216.81);
\definecolor[named]{drawColor}{rgb}{0.00,0.00,0.00}

\node[text=drawColor,rotate= 90.00,anchor=base,inner sep=0pt, outer sep=0pt, scale=  1.00] at ( 10.80,114.41) {Survival Prob.};

\node[text=drawColor,anchor=base,inner sep=0pt, outer sep=0pt, scale=  1.00] at (156.54, 15.60) {Time (Years)};
\end{scope}
\begin{scope}
\path[clip] (  0.00,  0.00) rectangle (578.16,216.81);
\definecolor[named]{drawColor}{rgb}{0.00,0.00,0.00}

\path[draw=drawColor,line width= 0.4pt,line join=round,line cap=round] (338.28, 61.20) -- (552.96, 61.20);

\path[draw=drawColor,line width= 0.4pt,line join=round,line cap=round] (338.28, 61.20) -- (338.28, 55.20);

\path[draw=drawColor,line width= 0.4pt,line join=round,line cap=round] (374.06, 61.20) -- (374.06, 55.20);

\path[draw=drawColor,line width= 0.4pt,line join=round,line cap=round] (409.84, 61.20) -- (409.84, 55.20);

\path[draw=drawColor,line width= 0.4pt,line join=round,line cap=round] (445.62, 61.20) -- (445.62, 55.20);

\path[draw=drawColor,line width= 0.4pt,line join=round,line cap=round] (481.40, 61.20) -- (481.40, 55.20);

\path[draw=drawColor,line width= 0.4pt,line join=round,line cap=round] (517.18, 61.20) -- (517.18, 55.20);

\path[draw=drawColor,line width= 0.4pt,line join=round,line cap=round] (552.96, 61.20) -- (552.96, 55.20);

\node[text=drawColor,anchor=base,inner sep=0pt, outer sep=0pt, scale=  1.00] at (338.28, 39.60) {0};

\node[text=drawColor,anchor=base,inner sep=0pt, outer sep=0pt, scale=  1.00] at (374.06, 39.60) {5};

\node[text=drawColor,anchor=base,inner sep=0pt, outer sep=0pt, scale=  1.00] at (409.84, 39.60) {10};

\node[text=drawColor,anchor=base,inner sep=0pt, outer sep=0pt, scale=  1.00] at (445.62, 39.60) {15};

\node[text=drawColor,anchor=base,inner sep=0pt, outer sep=0pt, scale=  1.00] at (481.40, 39.60) {20};

\node[text=drawColor,anchor=base,inner sep=0pt, outer sep=0pt, scale=  1.00] at (517.18, 39.60) {25};

\node[text=drawColor,anchor=base,inner sep=0pt, outer sep=0pt, scale=  1.00] at (552.96, 39.60) {30};

\path[draw=drawColor,line width= 0.4pt,line join=round,line cap=round] (338.28, 65.14) -- (338.28,163.67);

\path[draw=drawColor,line width= 0.4pt,line join=round,line cap=round] (338.28, 65.14) -- (332.28, 65.14);

\path[draw=drawColor,line width= 0.4pt,line join=round,line cap=round] (338.28, 84.85) -- (332.28, 84.85);

\path[draw=drawColor,line width= 0.4pt,line join=round,line cap=round] (338.28,104.55) -- (332.28,104.55);

\path[draw=drawColor,line width= 0.4pt,line join=round,line cap=round] (338.28,124.26) -- (332.28,124.26);

\path[draw=drawColor,line width= 0.4pt,line join=round,line cap=round] (338.28,143.96) -- (332.28,143.96);

\path[draw=drawColor,line width= 0.4pt,line join=round,line cap=round] (338.28,163.67) -- (332.28,163.67);

\node[text=drawColor,anchor=base east,inner sep=0pt, outer sep=0pt, scale=  1.00] at (326.28, 61.70) {0.0};

\node[text=drawColor,anchor=base east,inner sep=0pt, outer sep=0pt, scale=  1.00] at (326.28, 81.40) {0.2};

\node[text=drawColor,anchor=base east,inner sep=0pt, outer sep=0pt, scale=  1.00] at (326.28,101.11) {0.4};

\node[text=drawColor,anchor=base east,inner sep=0pt, outer sep=0pt, scale=  1.00] at (326.28,120.81) {0.6};

\node[text=drawColor,anchor=base east,inner sep=0pt, outer sep=0pt, scale=  1.00] at (326.28,140.52) {0.8};

\node[text=drawColor,anchor=base east,inner sep=0pt, outer sep=0pt, scale=  1.00] at (326.28,160.23) {1.0};

\path[draw=drawColor,line width= 0.4pt,line join=round,line cap=round] (338.28, 61.20) --
	(552.96, 61.20) --
	(552.96,167.61) --
	(338.28,167.61) --
	(338.28, 61.20);
\end{scope}
\begin{scope}
\path[clip] (289.08,  0.00) rectangle (578.16,216.81);
\definecolor[named]{drawColor}{rgb}{0.00,0.00,0.00}

\node[text=drawColor,anchor=base,inner sep=0pt, outer sep=0pt, scale=  1.20] at (445.62,188.07) {\bfseries Sanction Reciprocity};
\end{scope}
\begin{scope}
\path[clip] (338.28, 61.20) rectangle (552.96,167.61);
\definecolor[named]{drawColor}{rgb}{0.66,0.66,0.66}

\path[draw=drawColor,line width= 0.4pt,line join=round,line cap=round] (338.28,163.67) --
	(345.44,163.67) --
	(345.44,155.21) --
	(352.59,155.21) --
	(352.59,146.32) --
	(359.75,146.32) --
	(359.75,137.96) --
	(366.90,137.96) --
	(366.90,133.42) --
	(374.06,133.42) --
	(374.06,131.69) --
	(381.22,131.69) --
	(381.22,131.08) --
	(388.37,131.08) --
	(388.37,130.47) --
	(395.53,130.47) --
	(395.53,129.77) --
	(402.68,129.77) --
	(402.68,128.97) --
	(409.84,128.97) --
	(409.84,128.14) --
	(431.31,128.14) --
	(431.31,126.82) --
	(438.46,126.82) --
	(438.46,124.95) --
	(467.09,124.95) --
	(467.09,120.58) --
	(481.40,120.58) --
	(481.40,115.43) --
	(488.56,115.43) --
	(488.56,110.65) --
	(510.02,110.65) --
	(510.02,106.32) --
	(578.16,106.32);

\path[draw=drawColor,line width= 0.4pt,line join=round,line cap=round] (342.25,155.21) -- (348.62,155.21);

\path[draw=drawColor,line width= 0.4pt,line join=round,line cap=round] (345.44,152.03) -- (345.44,158.39);

\path[draw=drawColor,line width= 0.4pt,line join=round,line cap=round] (349.41,146.32) -- (355.77,146.32);

\path[draw=drawColor,line width= 0.4pt,line join=round,line cap=round] (352.59,143.14) -- (352.59,149.50);

\path[draw=drawColor,line width= 0.4pt,line join=round,line cap=round] (356.57,137.96) -- (362.93,137.96);

\path[draw=drawColor,line width= 0.4pt,line join=round,line cap=round] (359.75,134.78) -- (359.75,141.14);

\path[draw=drawColor,line width= 0.4pt,line join=round,line cap=round] (363.72,133.42) -- (370.09,133.42);

\path[draw=drawColor,line width= 0.4pt,line join=round,line cap=round] (366.90,130.24) -- (366.90,136.61);

\path[draw=drawColor,line width= 0.4pt,line join=round,line cap=round] (370.88,131.69) -- (377.24,131.69);

\path[draw=drawColor,line width= 0.4pt,line join=round,line cap=round] (374.06,128.51) -- (374.06,134.88);

\path[draw=drawColor,line width= 0.4pt,line join=round,line cap=round] (378.03,131.08) -- (384.40,131.08);

\path[draw=drawColor,line width= 0.4pt,line join=round,line cap=round] (381.22,127.90) -- (381.22,134.26);

\path[draw=drawColor,line width= 0.4pt,line join=round,line cap=round] (385.19,130.47) -- (391.55,130.47);

\path[draw=drawColor,line width= 0.4pt,line join=round,line cap=round] (388.37,127.29) -- (388.37,133.65);

\path[draw=drawColor,line width= 0.4pt,line join=round,line cap=round] (392.35,129.77) -- (398.71,129.77);

\path[draw=drawColor,line width= 0.4pt,line join=round,line cap=round] (395.53,126.59) -- (395.53,132.95);

\path[draw=drawColor,line width= 0.4pt,line join=round,line cap=round] (399.50,128.97) -- (405.87,128.97);

\path[draw=drawColor,line width= 0.4pt,line join=round,line cap=round] (402.68,125.78) -- (402.68,132.15);

\path[draw=drawColor,line width= 0.4pt,line join=round,line cap=round] (406.66,128.14) -- (413.02,128.14);

\path[draw=drawColor,line width= 0.4pt,line join=round,line cap=round] (409.84,124.96) -- (409.84,131.32);

\path[draw=drawColor,line width= 0.4pt,line join=round,line cap=round] (413.81,128.14) -- (420.18,128.14);

\path[draw=drawColor,line width= 0.4pt,line join=round,line cap=round] (417.00,124.96) -- (417.00,131.32);

\path[draw=drawColor,line width= 0.4pt,line join=round,line cap=round] (420.97,128.14) -- (427.33,128.14);

\path[draw=drawColor,line width= 0.4pt,line join=round,line cap=round] (424.15,124.96) -- (424.15,131.32);

\path[draw=drawColor,line width= 0.4pt,line join=round,line cap=round] (428.13,126.82) -- (434.49,126.82);

\path[draw=drawColor,line width= 0.4pt,line join=round,line cap=round] (431.31,123.64) -- (431.31,130.01);

\path[draw=drawColor,line width= 0.4pt,line join=round,line cap=round] (435.28,124.95) -- (441.65,124.95);

\path[draw=drawColor,line width= 0.4pt,line join=round,line cap=round] (438.46,121.77) -- (438.46,128.13);

\path[draw=drawColor,line width= 0.4pt,line join=round,line cap=round] (442.44,124.95) -- (448.80,124.95);

\path[draw=drawColor,line width= 0.4pt,line join=round,line cap=round] (445.62,121.77) -- (445.62,128.13);

\path[draw=drawColor,line width= 0.4pt,line join=round,line cap=round] (449.59,124.95) -- (455.96,124.95);

\path[draw=drawColor,line width= 0.4pt,line join=round,line cap=round] (452.78,121.77) -- (452.78,128.13);

\path[draw=drawColor,line width= 0.4pt,line join=round,line cap=round] (456.75,124.95) -- (463.11,124.95);

\path[draw=drawColor,line width= 0.4pt,line join=round,line cap=round] (459.93,121.77) -- (459.93,128.13);

\path[draw=drawColor,line width= 0.4pt,line join=round,line cap=round] (463.91,120.58) -- (470.27,120.58);

\path[draw=drawColor,line width= 0.4pt,line join=round,line cap=round] (467.09,117.40) -- (467.09,123.77);

\path[draw=drawColor,line width= 0.4pt,line join=round,line cap=round] (471.06,120.58) -- (477.43,120.58);

\path[draw=drawColor,line width= 0.4pt,line join=round,line cap=round] (474.24,117.40) -- (474.24,123.77);

\path[draw=drawColor,line width= 0.4pt,line join=round,line cap=round] (478.22,115.43) -- (484.58,115.43);

\path[draw=drawColor,line width= 0.4pt,line join=round,line cap=round] (481.40,112.25) -- (481.40,118.62);

\path[draw=drawColor,line width= 0.4pt,line join=round,line cap=round] (485.37,110.65) -- (491.74,110.65);

\path[draw=drawColor,line width= 0.4pt,line join=round,line cap=round] (488.56,107.47) -- (488.56,113.83);

\path[draw=drawColor,line width= 0.4pt,line join=round,line cap=round] (492.53,110.65) -- (498.89,110.65);

\path[draw=drawColor,line width= 0.4pt,line join=round,line cap=round] (495.71,107.47) -- (495.71,113.83);

\path[draw=drawColor,line width= 0.4pt,line join=round,line cap=round] (499.69,110.65) -- (506.05,110.65);

\path[draw=drawColor,line width= 0.4pt,line join=round,line cap=round] (502.87,107.47) -- (502.87,113.83);

\path[draw=drawColor,line width= 0.4pt,line join=round,line cap=round] (506.84,106.32) -- (513.21,106.32);

\path[draw=drawColor,line width= 0.4pt,line join=round,line cap=round] (510.02,103.14) -- (510.02,109.50);

\path[draw=drawColor,line width= 0.4pt,line join=round,line cap=round] (514.00,106.32) -- (520.36,106.32);

\path[draw=drawColor,line width= 0.4pt,line join=round,line cap=round] (517.18,103.14) -- (517.18,109.50);

\path[draw=drawColor,line width= 0.4pt,line join=round,line cap=round] (521.15,106.32) -- (527.52,106.32);

\path[draw=drawColor,line width= 0.4pt,line join=round,line cap=round] (524.34,103.14) -- (524.34,109.50);

\path[draw=drawColor,line width= 0.4pt,line join=round,line cap=round] (528.31,106.32) -- (534.67,106.32);

\path[draw=drawColor,line width= 0.4pt,line join=round,line cap=round] (531.49,103.14) -- (531.49,109.50);

\path[draw=drawColor,line width= 0.4pt,line join=round,line cap=round] (535.47,106.32) -- (541.83,106.32);

\path[draw=drawColor,line width= 0.4pt,line join=round,line cap=round] (538.65,103.14) -- (538.65,109.50);

\path[draw=drawColor,line width= 0.4pt,line join=round,line cap=round] (542.62,106.32) -- (548.99,106.32);

\path[draw=drawColor,line width= 0.4pt,line join=round,line cap=round] (545.80,103.14) -- (545.80,109.50);

\path[draw=drawColor,line width= 0.4pt,line join=round,line cap=round] (549.78,106.32) -- (556.14,106.32);

\path[draw=drawColor,line width= 0.4pt,line join=round,line cap=round] (552.96,103.14) -- (552.96,109.50);

\path[draw=drawColor,line width= 0.4pt,line join=round,line cap=round] (556.93,106.32) -- (563.30,106.32);

\path[draw=drawColor,line width= 0.4pt,line join=round,line cap=round] (560.12,103.14) -- (560.12,109.50);

\path[draw=drawColor,line width= 0.4pt,line join=round,line cap=round] (564.09,106.32) -- (570.45,106.32);

\path[draw=drawColor,line width= 0.4pt,line join=round,line cap=round] (567.27,103.14) -- (567.27,109.50);

\path[draw=drawColor,line width= 0.4pt,line join=round,line cap=round] (571.25,106.32) -- (577.61,106.32);

\path[draw=drawColor,line width= 0.4pt,line join=round,line cap=round] (574.43,103.14) -- (574.43,109.50);
\definecolor[named]{drawColor}{rgb}{0.00,0.00,0.00}

\path[draw=drawColor,line width= 0.4pt,line join=round,line cap=round] (338.28,163.67) --
	(345.44,163.67) --
	(345.44,162.94) --
	(352.59,162.94) --
	(352.59,162.10) --
	(359.75,162.10) --
	(359.75,161.23) --
	(366.90,161.23) --
	(366.90,160.72) --
	(374.06,160.72) --
	(374.06,160.52) --
	(381.22,160.52) --
	(381.22,160.45) --
	(388.37,160.45) --
	(388.37,160.37) --
	(395.53,160.37) --
	(395.53,160.29) --
	(402.68,160.29) --
	(402.68,160.19) --
	(409.84,160.19) --
	(409.84,160.09) --
	(431.31,160.09) --
	(431.31,159.92) --
	(438.46,159.92) --
	(438.46,159.68) --
	(467.09,159.68) --
	(467.09,159.09) --
	(481.40,159.09) --
	(481.40,158.33) --
	(488.56,158.33) --
	(488.56,157.57) --
	(510.02,157.57) --
	(510.02,156.81) --
	(578.16,156.81);

\path[draw=drawColor,line width= 0.4pt,line join=round,line cap=round] (342.25,162.94) -- (348.62,162.94);

\path[draw=drawColor,line width= 0.4pt,line join=round,line cap=round] (345.44,159.76) -- (345.44,166.12);

\path[draw=drawColor,line width= 0.4pt,line join=round,line cap=round] (349.41,162.10) -- (355.77,162.10);

\path[draw=drawColor,line width= 0.4pt,line join=round,line cap=round] (352.59,158.92) -- (352.59,165.28);

\path[draw=drawColor,line width= 0.4pt,line join=round,line cap=round] (356.57,161.23) -- (362.93,161.23);

\path[draw=drawColor,line width= 0.4pt,line join=round,line cap=round] (359.75,158.05) -- (359.75,164.42);

\path[draw=drawColor,line width= 0.4pt,line join=round,line cap=round] (363.72,160.72) -- (370.09,160.72);

\path[draw=drawColor,line width= 0.4pt,line join=round,line cap=round] (366.90,157.54) -- (366.90,163.91);

\path[draw=drawColor,line width= 0.4pt,line join=round,line cap=round] (370.88,160.52) -- (377.24,160.52);

\path[draw=drawColor,line width= 0.4pt,line join=round,line cap=round] (374.06,157.34) -- (374.06,163.70);

\path[draw=drawColor,line width= 0.4pt,line join=round,line cap=round] (378.03,160.45) -- (384.40,160.45);

\path[draw=drawColor,line width= 0.4pt,line join=round,line cap=round] (381.22,157.27) -- (381.22,163.63);

\path[draw=drawColor,line width= 0.4pt,line join=round,line cap=round] (385.19,160.37) -- (391.55,160.37);

\path[draw=drawColor,line width= 0.4pt,line join=round,line cap=round] (388.37,157.19) -- (388.37,163.56);

\path[draw=drawColor,line width= 0.4pt,line join=round,line cap=round] (392.35,160.29) -- (398.71,160.29);

\path[draw=drawColor,line width= 0.4pt,line join=round,line cap=round] (395.53,157.11) -- (395.53,163.47);

\path[draw=drawColor,line width= 0.4pt,line join=round,line cap=round] (399.50,160.19) -- (405.87,160.19);

\path[draw=drawColor,line width= 0.4pt,line join=round,line cap=round] (402.68,157.01) -- (402.68,163.37);

\path[draw=drawColor,line width= 0.4pt,line join=round,line cap=round] (406.66,160.09) -- (413.02,160.09);

\path[draw=drawColor,line width= 0.4pt,line join=round,line cap=round] (409.84,156.91) -- (409.84,163.27);

\path[draw=drawColor,line width= 0.4pt,line join=round,line cap=round] (413.81,160.09) -- (420.18,160.09);

\path[draw=drawColor,line width= 0.4pt,line join=round,line cap=round] (417.00,156.91) -- (417.00,163.27);

\path[draw=drawColor,line width= 0.4pt,line join=round,line cap=round] (420.97,160.09) -- (427.33,160.09);

\path[draw=drawColor,line width= 0.4pt,line join=round,line cap=round] (424.15,156.91) -- (424.15,163.27);

\path[draw=drawColor,line width= 0.4pt,line join=round,line cap=round] (428.13,159.92) -- (434.49,159.92);

\path[draw=drawColor,line width= 0.4pt,line join=round,line cap=round] (431.31,156.74) -- (431.31,163.10);

\path[draw=drawColor,line width= 0.4pt,line join=round,line cap=round] (435.28,159.68) -- (441.65,159.68);

\path[draw=drawColor,line width= 0.4pt,line join=round,line cap=round] (438.46,156.50) -- (438.46,162.86);

\path[draw=drawColor,line width= 0.4pt,line join=round,line cap=round] (442.44,159.68) -- (448.80,159.68);

\path[draw=drawColor,line width= 0.4pt,line join=round,line cap=round] (445.62,156.50) -- (445.62,162.86);

\path[draw=drawColor,line width= 0.4pt,line join=round,line cap=round] (449.59,159.68) -- (455.96,159.68);

\path[draw=drawColor,line width= 0.4pt,line join=round,line cap=round] (452.78,156.50) -- (452.78,162.86);

\path[draw=drawColor,line width= 0.4pt,line join=round,line cap=round] (456.75,159.68) -- (463.11,159.68);

\path[draw=drawColor,line width= 0.4pt,line join=round,line cap=round] (459.93,156.50) -- (459.93,162.86);

\path[draw=drawColor,line width= 0.4pt,line join=round,line cap=round] (463.91,159.09) -- (470.27,159.09);

\path[draw=drawColor,line width= 0.4pt,line join=round,line cap=round] (467.09,155.91) -- (467.09,162.27);

\path[draw=drawColor,line width= 0.4pt,line join=round,line cap=round] (471.06,159.09) -- (477.43,159.09);

\path[draw=drawColor,line width= 0.4pt,line join=round,line cap=round] (474.24,155.91) -- (474.24,162.27);

\path[draw=drawColor,line width= 0.4pt,line join=round,line cap=round] (478.22,158.33) -- (484.58,158.33);

\path[draw=drawColor,line width= 0.4pt,line join=round,line cap=round] (481.40,155.15) -- (481.40,161.52);

\path[draw=drawColor,line width= 0.4pt,line join=round,line cap=round] (485.37,157.57) -- (491.74,157.57);

\path[draw=drawColor,line width= 0.4pt,line join=round,line cap=round] (488.56,154.38) -- (488.56,160.75);

\path[draw=drawColor,line width= 0.4pt,line join=round,line cap=round] (492.53,157.57) -- (498.89,157.57);

\path[draw=drawColor,line width= 0.4pt,line join=round,line cap=round] (495.71,154.38) -- (495.71,160.75);

\path[draw=drawColor,line width= 0.4pt,line join=round,line cap=round] (499.69,157.57) -- (506.05,157.57);

\path[draw=drawColor,line width= 0.4pt,line join=round,line cap=round] (502.87,154.38) -- (502.87,160.75);

\path[draw=drawColor,line width= 0.4pt,line join=round,line cap=round] (506.84,156.81) -- (513.21,156.81);

\path[draw=drawColor,line width= 0.4pt,line join=round,line cap=round] (510.02,153.62) -- (510.02,159.99);

\path[draw=drawColor,line width= 0.4pt,line join=round,line cap=round] (514.00,156.81) -- (520.36,156.81);

\path[draw=drawColor,line width= 0.4pt,line join=round,line cap=round] (517.18,153.62) -- (517.18,159.99);

\path[draw=drawColor,line width= 0.4pt,line join=round,line cap=round] (521.15,156.81) -- (527.52,156.81);

\path[draw=drawColor,line width= 0.4pt,line join=round,line cap=round] (524.34,153.62) -- (524.34,159.99);

\path[draw=drawColor,line width= 0.4pt,line join=round,line cap=round] (528.31,156.81) -- (534.67,156.81);

\path[draw=drawColor,line width= 0.4pt,line join=round,line cap=round] (531.49,153.62) -- (531.49,159.99);

\path[draw=drawColor,line width= 0.4pt,line join=round,line cap=round] (535.47,156.81) -- (541.83,156.81);

\path[draw=drawColor,line width= 0.4pt,line join=round,line cap=round] (538.65,153.62) -- (538.65,159.99);

\path[draw=drawColor,line width= 0.4pt,line join=round,line cap=round] (542.62,156.81) -- (548.99,156.81);

\path[draw=drawColor,line width= 0.4pt,line join=round,line cap=round] (545.80,153.62) -- (545.80,159.99);

\path[draw=drawColor,line width= 0.4pt,line join=round,line cap=round] (549.78,156.81) -- (556.14,156.81);

\path[draw=drawColor,line width= 0.4pt,line join=round,line cap=round] (552.96,153.62) -- (552.96,159.99);

\path[draw=drawColor,line width= 0.4pt,line join=round,line cap=round] (556.93,156.81) -- (563.30,156.81);

\path[draw=drawColor,line width= 0.4pt,line join=round,line cap=round] (560.12,153.62) -- (560.12,159.99);

\path[draw=drawColor,line width= 0.4pt,line join=round,line cap=round] (564.09,156.81) -- (570.45,156.81);

\path[draw=drawColor,line width= 0.4pt,line join=round,line cap=round] (567.27,153.62) -- (567.27,159.99);

\path[draw=drawColor,line width= 0.4pt,line join=round,line cap=round] (571.25,156.81) -- (577.61,156.81);

\path[draw=drawColor,line width= 0.4pt,line join=round,line cap=round] (574.43,153.62) -- (574.43,159.99);
\end{scope}
\begin{scope}
\path[clip] (289.08,  0.00) rectangle (578.16,216.81);
\definecolor[named]{drawColor}{rgb}{0.00,0.00,0.00}

\node[text=drawColor,anchor=base,inner sep=0pt, outer sep=0pt, scale=  1.00] at (445.62, 15.60) {Time (Years)};
\end{scope}
\end{tikzpicture}
}	
\end{figure}


%%%%%%%%%%%%%%%%%%%%%%

%%%%% Conclusion %%%%%
\section*{Conclusion}
\label{conclusion}

In this paper we have shown that network variables do indeed play a substantive role in predicting sanction compliance. This indicates to us that the extant literature's focus on employing models that just capture the situation in the sanctioned country omits important information when it comes to understanding compliance. 


\newpage
\section*{Appendix}
\label{appendix}

\subsection*{Imputation Procedure}

The copula based approach developed by \citet{hoff:2007} is estimated through a Markov chain Monte Carlo (MCMC) algorithm. We run the MCMC for 6,000 imputations, saving every sixth imputation, using the \texttt{sbgcop.mcmc} function in the \texttt{sbgcop} package in $\mathcal{R}$. To account for time trends and obtain better performance from this imputaiton procedure, we create five lags of each variable, except for polity, prior to imputation. Every imputation of the MCMC leads to the creation of one dataset with all missing values imputed. Running this algorithm on our dataset then produces a total of 1,000 imputed datasets. Results across these 1,000 imputed datasets are then averaged, thereby accounting for a portion of the uncertainty in the imputed values. We then used the average of the results from these 1,000 imputed datasets to generate the regression estimates in table \ref{tab:regResults}. 

Regression results on the unimputed dataset are shown in table \ref{tab:regResultsNoImp}. The results, particularly for our reciprocity variables, are nearly identical. 

% latex table generated in R 3.0.2 by xtable 1.7-3 package
% Wed Jun 25 01:22:06 2014
\begin{table}[ht]
\centering
{\normalsize
\begin{tabular}{lccc}
 Variable & Model 1 & Model 2 & Model 3 \\ 
  \hline
\hline
Compliance Reciprocity$_{j,t-1}$ &  &  & $0.293^{\ast\ast}$ \\ 
   &  &  & (0.068) \\ 
  Sanction Reciprocity$_{j,t-1}$ &  &  & $-0.13^{\ast\ast}$ \\ 
   &  &  & (0.033) \\ 
   \hline
Number of Senders$_{j,t}$ &  & $0.358^{\ast\ast}$ & $0.33^{\ast\ast}$ \\ 
   &  & (0.065) & (0.067) \\ 
  Distance$_{j,t}$ &  & $0.438^{\ast\ast}$ & $0.408^{\ast\ast}$ \\ 
   &  & (0.198) & (0.203) \\ 
  Trade$_{j,t}$ &  & $97.145^{\ast\ast}$ & $102.839^{\ast\ast}$ \\ 
   &  & (24.315) & (24.462) \\ 
  Ally$_{j,t}$ &  & 0.025 & 0.056 \\ 
   &  & (0.196) & (0.193) \\ 
   \hline
Polity$_{i,t-1}$ & $-0.013^{\ast}$ & -0.007 & -0.008 \\ 
   & (0.007) & (0.008) & (0.008) \\ 
  Ln(GDP per capita)$_{i,t-1}$ & $-0.126^{\ast\ast}$ & -0.019 & -0.005 \\ 
   & (0.06) & (0.075) & (0.078) \\ 
  GDP Growth$_{i,t-1}$ & 0.024 & $0.042^{\ast\ast}$ & 0.032 \\ 
   & (0.02) & (0.021) & (0.021) \\ 
  Population$_{i,t-1}$ & -0.072 & 0.041 & $0.137^{\ast}$ \\ 
   & (0.054) & (0.064) & (0.07) \\ 
  Internal Conflict$_{i,t-1}$ & -0.004 & 0 & -0.011 \\ 
   & (0.017) & (0.018) & (0.018) \\ 
   \hline
n & 6140 & 6106 & 6106 \\ 
  Events & 160 & 159 & 159 \\ 
  Likelihood ratio test & 17.31 (0) & 59.16 (0) & 77.77 (0) \\ 
   \hline
\hline
\end{tabular}
}
\caption{Duration model on unimputed data with time varying covariates estimated using Cox Proportional Hazards. Standard errors in parentheses. $^{**}$ and $^{*}$ indicate significance at $p< 0.05 $ and $p< 0.10 $, respectively.} 
\label{tab:regResultsNoImp}
\end{table}

\FloatBarrier

\newpage
%%%%%%%%%%%%%%%%%%%%%%

\newpage
\bibliographystyle{apsr}
\bibliography{magRefs.bib}

\end{document} 