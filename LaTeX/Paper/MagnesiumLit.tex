\subsection*{When do Sanctions End?}
\label{lit}

Previous work on the duration of sanctions, or when and why a target state will decide to comply with a particular sanction, has focused on the role of domestic factors. \cite{marinov2005} argues that sanctions ``work'' by destabilizing the leaders of the governments that sanctions punish. This focus on internal state conditions echoes other work which suggests that sanction outcomes are dependent on domestic stability and domestic institutions. For example, if a regime is already experiencing a high level of internal conflict, such as protests or violent clashes, the onset of an economic sanction restricting trade would further weaken the regime. This heightens the cost of resistance against the sanction \citep{dashti1997}. 

Similarly, \citep{dorussen2001} suggest that domestic support determines the duration (or ``ending'') of sanctions whereby when the target state's domestic constituency supports resistance against the sanction, the leader has greater incentive to not comply with the sanction, which effectively increases the sanction's duration. Further supporting the idea that domestic institutions condition whether and when states comply with sanctions, \cite{lektzian2007} argues that because of differing institutional incentives, economic sanctions are more likely to succeed against nondemocratic regimes than democratic ones. While all of these studies present empirical evidence for the general claim that domestic factors condition sanction outcome, none of them account for third party effects, or network level dependencies that influence behavior between states over time. 

Clearly, domestic conditions seem to matter for predicting sanction compliance. Yet, another realm of conditions also matters. Research has shown that it is important to consider whether the group of sanctioners for each sanction case have specific types of relationships with the target state, i.e. are critical trade partners, allies, or neighbors.\textcolor{red}{CITE} Each relationship between the sanctioner and the sanctioned takes on a slightly different influence dependent on these factors. If a neighboring state is greatly dissatisfied with the target's behavior, this conflict of interest could have more serious repercussions than a sanctioner who is geographically removed from the target. These types of external factors are housed within the groups of sanctioners for each and every sanction case. 

Research on compliance has historically included both these domestic state level and intrastate level variables within a logit or probit-estimation approach. However, research has recently demonstrated that a duration modeling approach more accurately captures the important time-variant dynamics relevant to understanding the sanction process and for testing those theories. \cite{bolks2000} point out that a duration-modeling approach is able to include variables that fluctuate throughout the tenure of an individual sanction case. Clearly, if the goal of research is to understand and predict when a target state is likely to comply to a sanction, then researchers have clear incentives to include time-variant data. Using a duration modeling approach allows for the assessment of whether over time a specific factor, such as political instability or regime type, increases or decreases the probability that a target country will comply with a sanction.

\cite{mcgillivray2004} employ a hazard model to analyze a data set of 47 sanctions cases. They find that leadership change does strongly influence the duration of sanctions, but only in the case of non-democratic states. Similarly, \cite{bolks2000} consider the determinants of economic sanction duration using a duration model approach. These authors also look inside the target state to define domestic conditions that influence sanction outcome. They suggest that the ``decision-making'' environment can either hinder or help the leader take countermeasures against the sanction. This ``decision-making'' environment is affected by factors such as a lack of coordination between government actors and local instability. 


While it is intuitive to many researchers that trade dependence between target and sender states likely influences the duration of economic sanctions, and that domestic conditions influence the target's behavior, these previous approaches fail to account for the evolution of interaction between states over time. Interactions between actors over time can be essentially captured by the network concept, allowing for a deeper understanding of how the history of sanctions and compliance between states influences future sanction cases. By avoiding these network attributes, researchers miss a wealth of structural information that is critical to understanding the ebb and flow of international cooperation and conflict. The insight that the international system is inherently a network and must be studied as such, is by no means original to this project, but has gained increasing support in the literature; most prominent is the work on trade networks \citep{hoff2004modeling} , conflict \citep{dorff2013}, alliances \citep{warren2010geometry} and intragovernmental organizations \citep{cao2009networks,greenhill2010norm}. 

Furthermore, current duration approaches are unable to account for the history of dependencies between countries over time, and thus ignore previous cases of compliance and sanction interdependence between target and sanctioning states.  Over time, complex interdependencies likely emerge and drive behavior between states, where if country \textit{i} complies often to country \textit{j}, country \textit{j} might also be more likely to comply to country \textit{i}. This process is typically known as reciprocity, and is one of the network attributes we account for in our analysis below. Importantly, \citet{cranmer2014reciprocity} also argue that the sanction literature has not yet accounted for network dynamics. In their work they model the sanction network itself, and demonstrate that onset of sanction cases are best predicted by modeling the way in which the network complex interdependencies, such as reciprocity, evolve over time and influence the future decisions made by states. Critical concepts like these are currently ignored in the research on sanction compliance. This paper aims to fill this gap. 


%%%%%%%%%%%%%%%
% NOTES ON THINGS TO ADD
%%%%%%%%%%%%%%%

%do we only look at compliance (not just a sender ``quitting''?)
%Things we don't account for : threats and sanction "issues" driving compliance (see And and Peksen 2007)
%We also need to address why we want to study compliance/duration (mcgillivray and stam, as well as Bolks, have more on this)