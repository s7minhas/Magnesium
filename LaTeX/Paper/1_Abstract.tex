In this article, we address the long-debated question of when and why states comply with sanctions. While the literature remains indeterminate as to whether the key mechanisms driving sanction compliance are tied to interstate relations, intrastate constraints, or a dynamic combination of the two, our theoretical framework and methodological approach offer a critical test of these current arguments yet incorporates new insights drawn from network theory to explain sanction outcomes. We argue that an influential network concept, reciprocity, has largely been overlooked by researchers of international sanctions. Underlying the exogenous
relational dimensions explored in the literature is the argument that there exists a given set of countries from whom the sending of sanctions are more consequential, and will thus be complied to more quickly, than others. However, by ignoring reciprocity the extant analyses of these linkages disregards the history of interactions between two countries. We argue that reciprocal interactions provide a fundamental refection of the strategic environment that shapes state's compliance behavior. We present a model that incorporates this interdependent nature of the international system by including measures of reciprocity within a duration modeling framework.\\

\noindent \textit{Word Count: 8,668}

%maybe add one final "we find that.." type sentence? 