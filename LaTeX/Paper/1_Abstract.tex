In this article we address the long-debated question of when and why states comply with sanctions. While the literature remains indeterminate as to whether the key mechanisms driving sanction compliance are tied to interstate relations, intrastate constraints, or a dynamic combination of the two, our theoretical framework and methodological approach provide a novel perspective that incorporates insights drawn from network theory to explain the time until countries comply. Specifically, we argue that reciprocity, a concept with deep roots in both network theory and international relations, has largely been overlooked in the study of sanction compliance. Though often ignored, this concept captures an essential aspect of how cooperation is fostered in the international system, and allows us to better analyze the strategic environment underlying sanctioning behavior. Given the theoretical importance of reciprocity in understanding interstate relations, we provide an approach that integrates estimations of this type of network interdependency into extant frameworks for modeling the time until countries comply with sanctions. Our results highlight that reciprocity not only has a substantive effect in explaining the duration of sanctions, but that models excluding this concept from their specifications do notably worse in terms of their predictive performance.\\

\noindent \textit{Word Count: 9,605}