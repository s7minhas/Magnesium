This article explores when and why states comply with sanctions. Previous literature suggests a duration modeling approach is needed to adequately capture the time it takes for a sanction to ``work.'' This approach, however, has failed to carefully account for important dynamics relevant to the theoretical and empirical investigation of sanction outcomes. Namely, present sanction studies do not incorporate the network effects intrinsic to international sanction processes. We argue that a key network measure, reciprocity, is a previously overlooked component of the strategic environment between states and influences their willingness to comply to sanctions. We present a model that incorporates this interdependent nature of the international system by including measures of reciprocity within a duration modeling framework. In addition, we are able to test whether conditions the literature claims as critical for predicting sanction compliance, such as domestic institutions, remain influential once network dynamics are adequately modeled. In doing so, we test key hypotheses from the literature regarding the role of domestic conditions, intrastate relationships, and our own new hypotheses on the effect of reciprocity on sanction compliance. \\

\noindent \textit{Word Count: 8,668}