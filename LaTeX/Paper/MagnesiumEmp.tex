\section*{Data and Analysis}
\label{empirics}

To test the effects of network pressures on sanction compliance we use the Threat and Imposition of Sanctions (TIES) Database developed by \citet{morgan2009threat}. This database includes over 1,400 sanction case threats and initiations from 1945 to 2013.\footnote{Only sanction cases initiated by 2005 are included but outcomes for cases are recorded until 2013.} Our focus here is restricted to sanctions that are prompted as the result of economic issues such as expropriation, trade practices, and implementation of economic reforms. 

Restricting our analysis to threats or sanctions stemming from these issues during the period of 1960 to 2005 leaves us with over 800 cases. Our unit of analysis is the case-year, providing us with a total of 5,303 observations. For each case in the TIES database a final outcome is recorded to describe how and if the case has been resolved. The purpose of our analysis is to assess the time until a state complies to a threat or sanction. We consider a case to have been resolved by compliance if the target state completely or partially acquiesces to the demands of the sanction senders or negotiates a settlement.
	
In using this definition of compliance, approximately 37\% of cases in our dataset end with a state complying by 2013 while 40\% remain ongoing. The remaining 23\% of cases were terminated for other reasons show below in Table \ref{tab:termCases}.

\begin{table}[ht]
	\centering
	\begin{tabular}{lc}
		\hline\hline
		Outcome & Frequency \\
		\hline
		Capitulation by Sender in Threat Stage & 77 \\
		Stalemate in the Threat Stage & 11 \\		
		Capitulation by Sender After Imposition & 58 \\
		Stalemate after Sanctions Imposition & 38 \\
		\hline\hline
	\end{tabular}
	\caption{Outcomes of threat and sanction cases no longer ongoing where compliance was not achieved.}
	\label{tab:termCases}	
\end{table}

\subsection*{Modeling Approach} 

Next we discuss our modeling approach. To estimate the effect of network pressures on the ability of a threatened or sanctioned states to resist compliance, we use Cox proportional hazard (PH) models of the length of threat or sanction periods. Specifically, the dependent variable, sanction spell, is the number of years that a state has not complied to a threat or sanction at time $t$. We model the expected length of sanction spells as a function of a baseline hazard rate and a set of covariates that shift the baseline hazard. The Cox PH specification that we employ is:

\begin{center}
	$\log h_{i}(t | \boldsymbol{X}_{i}) \; = \; h_{0}(t) \times \exp(\boldsymbol{X}_{i} \beta)$,
\end{center}

where the log-hazard rate of compliance in a sanction case, $i$, conditional on having not complied for $t$ years is a function of a common baseline hazard $h_{0}(t)$ and covariates $\boldsymbol{X}$. In employing this approach, we assume no specific functional form for the baseline hazard and instead estimate it non-parametrically from the data.\footnote{To ensure against bias in our parameter estimates we included a vector of case-level shared frailties to account for variations in unit-specific factors. We found similar results with and without the shared frailties, so we report results without the inclusion of this additional term.}  The covariates $\boldsymbol{X}$ operate multiplicatevely on the hazard rate, shifting the expected risk of compliance up or down depending on the value of $\beta$.\footnote{\cite{crespo2013political}}

Providing no specific functional form for the baseline hazard necessitates testing the proportional hazard assumption. \citet{keele2010proportionally} notes that not inspecting this assumption in the covariates can lead to severely biased parameter estimates. To address this issue, we first fit smoothing splines for all continuous covariates. After ascertaining that none of the continuous covariates in our model required modeling with splines, we carried out tests of non-proportionality. For those covariates where the non-proportional effects assumption does not hold, we include interactions between the covariate and spell duration (log scale). The only covariate showing evidence of non-proportionality is the average similarity of religious profiles.

We also imputed missing values to avoid excluding instances of compliance. If we employed list-wise deletion, we would lose over 438 country-year observations, 32 of which contained instances in which a state complied to a sanction. Previous research has already highlighted how simply deleting missing observations can lead to biased results.\footnote{e.g., see \citealp{rubin1976inference,honaker2010missing}} To impute missing values, we use a copula based approach developed by \citet{hoff:2007}. Details on our imputation process and results based on the original unimputed dataset, which are nearly identical, can be found in the \nameref{appendix}. 

In addition to including our two key measures of reciprocity, discussed in the previous section, we control for a variety of other network-related components of the sanction network. In doing so, we also account for key conditions at the sanction-case level, not just for the sanction-year. To demonstrate, we deconstruct the broader yearly network example of 1984, to zoom in and narrow our focus to observe the individual sanction case of South Africa in 1984. In this case, South Africa is the target of multiple sanctions, as shown in Figure \ref{fig:saneti}. Because of this, we construct a sanction network that represents each sanction South Africa faces during this year. As one can easily see, in most cases during 1984, South Africa faces more than one sanctioner, and these sanctioners vary across each sanction case. For example in the first network graph, in the top left of Figure \ref{fig:saneti}, we see that South Africa faces a sanction from India, Pakistan, and Jamaica. Yet in the top right network graph, we see that South Africa also faces a sanction from Canada, Sweden, the USA, Finland, and Australia. 

By the 1980s, the South African apartheid regime had been in power for over thirty years. The international community moved to sanction the apartheid regime in hopes to end the violence and delegitimize the regime. As unrest intensified international action became inevitable and multilateral economic sanctions were initiated. As \citet{kinne2013dependent} points out, the aim of delegitimizing a regime is a cooperative act between multiple actors, whereby its effectiveness is dependent on coordinated consent and action. To diplomatically exclude South Africa, successful coordination and cooperation amongst sanctioner states was key.\footnote{\cite{kinne2013dependent,christopher1994pattern}} Critical coordination was achieved through this group of state sanctioners. By also looking at the sanction-case network, we are able to capture these state-level to test whether characteristics among sanctioners effect sanction compliance. For example, we might expect that a sanction from Pakistan, India, and Jamaica has substantially different implications than one from largely ``western'' sanctioners such as Sweden, Canada, USA, Finland and Australia. We conceptualize these relationships as composed of ``pressures'' which likely influences the behavior of the target state. It is intuitive that the number of senders should influence the willingness of the target state to comply because as the number of senders increases, the more constraints through multiple relationships the target faces. 

%\begin{figure}[ht]
%	\centering
%	\includegraphics[width=1\textwidth]{saneti}
%	\caption{Network for each sanction case that South Africa faced in 1984.}
%	\label{fig:saneti}
%\end{figure}
%\FloatBarrier

We also expect that these relationships must be meaningful not just plentiful. Just as one would imagine that a person is less swayed by the demands of 10 strangers than the demands of a few close friends, we conceptualize senders as most influential when they interact with the target state on a number of dimensions.  Thus, for each sanction case we determine the number of senders but we also measure other dimensions of each sender's relationship to the target. In addition, we calculate and control for the average number of other sanctions being sent by the senders of each particular sanction case.
%
%\begin{quote}
%\textbf{H3}: As the number of sender states increases for any given sanction case, the time to compliance will decrease. 
%\end{quote}

 We now describe exactly what is meant behind our concept of ``proximity.'' In figure \ref{fig:saneti}, we show the six sanction cases faced by South Africa in 1984.\footnote{Data for sanction cases are from \citet{morgan2009threat} and is explained further in the following section.} For the most part, each sanction case involves a variety of actors with whom South Africa has differing cultural, geographic, diplomatic, and economic relationships. Within any individual sanction case we hypothesize (\textbf{H4}) that the proximity, (i.e. the ways in which the sender and target interact on a number of dimensions) of relationships between sender(s) of a sanction and a receiver influence whether a target state complies. We construct a number of covariates to test this idea that the normative closeness, or general ``proximity'' between sender(s) and receiver(s) increases sanction compliance. 

We focus on five key measures of the ``proximate'' nature of relationships. First, we measure the average distance between sender(s) and receiver.\footnote{To construct this measure we use the minimum distance between countries from the Cshapes Dataset \citep{weidmann2010geography}.} We utilize the Correlates of War (COW) data to construct the remaining four variables. Our second covariate relating to proximity is trade, which we measure as the total share of the receiver's trade in that year accounted for by sender states. Last, we measure alliances as the proportion of sender(s) that are allied with the receiver. 

%\begin{quote}
%	\textbf{H4}: Sanction cases where relationships between sender(s) and receiver(s) are more proximate will be more quickly resolved.
%\end{quote}

%%%%%%
% SM NOTE: commented this out for now since we dont include this variable in the current set of models. also it doesnt load in the current set of models. i'm thinking that we might just want to dump this hypothesis and the corresponding figure (which i think you had suggested already).

% Thus far we have explained how the relationships between sanctioners and target states matter for predicting compliance. This view has focused on the individual sanction case. However, the six separate sanctions that South Africa faced in 1984 can also be thought of within the context of a yearly sanction network. In figure \ref{fig:sanet}, we aggregate the six sanction networks into one where each separate sanction is denoted by a differing color. Here we hypothesize that states under the pressure of a multitude of sanctions will more quickly resolve sanction cases than those facing only a few.

% \begin{quote}
% 	\textbf{H3}: States facing the pressure of a multitude of sanctions will more quickly resolve any one of those sanction cases.
% \end{quote}

% \begin{figure}[ht]
% 	\centering
% 	\includegraphics[width=0.75\textwidth]{sanet}
% 	\caption{All sanctions faced by South Africa in 1984 collapsed to one network}
% 	\label{fig:sanet}
% \end{figure}
% \FloatBarrier
%%%%%%%%

Finally, we include a number of covariates to account for domestic explanations of sanction compliance in the extant literature. First is a measure of the target states' domestic institutions from the Polity IV data.\footnote{See \cite{marshall2002polity}. Specifically, we use the ``polity2'' variable from the Polity IV data.} This measure is computed by subtracting a country's autocracy score from its democracy score, and is scaled from 0 to 20. Previous research has shown that sanctions will be more effective when the target states' domestic institutions are more democratic. Second, we control for the level of internal conflict within  a country using the weighted conflict index from the Cross National Time-series Data Archive.\footnote{\cite{banks2011cross}} The expectation in the extant literature is that countries with higher levels of internal instability would be more likely to comply to sanctions. Finally, we use a logged measure of GDP per capita and the percent change in annual GDP, from the World Bank, to account for the argument that economically successful states are better able to weather the pressures of these agreements.
Below we show our full model specification: 

\begin{align*}
		Compliance_{i,t} =& \\
		&Sanction \; Reciprocity_{j,t-1} + Compliance \; Reciprocity_{j,t-1} + \\
		&No. \; Senders_{j,t} + Distance_{j,t} + Trade_{j,t} + Ally_{j,t} + \\
		&Constraints_{i,t-1} + Ln(GDP \; Capita)_{i,t-1} +\\
		&GDP \; Growth_{i,t-1} + Internal \; Conflict_{i,t} + \epsilon_{i,t}
\end{align*}

where $i$ represents the target of the sanction, $j$ represents the relationship between the set of sender(s) for a particular sanction case and $i$, and $t$ the time period.