\section{Empirics}
\label{empirics}

To test the effects of network pressures on sanction compliance we use the Threat and Imposition of Sanctions (TIES) Database developed by \citet{morgan2009threat}. This database includes over 1,400 sanction case initiations and outcomes from 1945 to 2005. Our focus here is restricted to sanctions that are prompted as the result of an economic issue. The TIES database categorizes the issue(s) involved in the threat or impositions of sanctions, we focus on four:

\begin{itemize}
	\item Release citizens, property, or material
	\item Improve environmental policies
	\item Trade practices
	\item Implement economic reform
\end{itemize}

Restricting our analysis to sanctions stemming from these issues during the period of 1984 to 2005 leaves us with 184 sanction cases. Our unit of analysis is the sanction case-year, providing us with a total of 1,920 observations. Our dependent variable measures whether states are complying. 


We define compliance as:
	\begin{itemize}
		\item Complete/Partial Acquiescence by Target to threat
		\item Negotiated Settlement
		\item Total/Partial Acquiescence by the Target State following sanctions imposition
		\item Negotiated Settlement following sanctions imposition
	\end{itemize}
	
\subsection{Modeling Approach} 

Describe duration model using cox proportional hazards with time varying covariates

\begin{align*}
		Compliance_{i,t} =\; & No. \; Senders_{j,t} + Distance_{j,t} + Trade_{j,t}  + \\
		 &Ally_{j,t} + IGOs_{j,t} + Religion_{j,t} +\\
 		 &Sanc. \; Rec'd_{i,t} + \\
		 &Constraints_{i,t} + GDP \; Capita_{i,t-1} +\\
		 & Internal \; Conflict_{i,t} +\\
		 &Constraints_{i,t}*No. \; Senders_{j} + \epsilon_{i,t}
	\end{align*}
	
\begin{itemize}
	\item $i$ represents the target of the sanction
	\item $j$ represents the relationship between the set of sender(s) for a particular sanction case and $i$
	\item $t$ the time period
\end{itemize}
