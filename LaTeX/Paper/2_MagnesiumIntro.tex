\section*{Introduction}
\label{intro}

Economic sanctions are a frequently used foreign policy tool in the realm of international relations. Typically, one or more states initiate sanctions against another when they perceive the target state as non-cooperative. The trigger for economic sanctions can occur in many contexts: the target state breaks a previous agreement, the target state openly disobeys international law, or the target state engages in behavior that is simply unfavorable to the political preferences of another state. 

Policymakers continually engage in heated debates over the efficacy of sanctions. The motivations for sanction initiation are cross-cutting, spanning a diverse and interdependent mix of policy issues and political actors. While the concept of sanctions--the idea that countries can put pressure via economic ties to other countries in order to influence policy--is relatively straightforward, the study of when and why sanctions work is complex.

Earlier research on sanctions argued that sanctions have little influence on targets.\footnote{\cite{lam1990, dashti1997, morgan1997, drezner1998}} More recent research suggests that the effectiveness of sanctions is dependent on an interaction of several factors, namely: the number of senders sanctioning a target state and the type of issue in dispute;\footnote{\cite{miers2002, morgan2009threat}} the strength of domestic institutions within the target state; and the type of regime governing the target state.\footnote{\cite{mcgillivray2004}} 

The theoretical and empirical literatures on economic sanctions demonstrate that several different, interacting conditions underline sanction outcomes. We argue, however, that scholars have thus far failed to incorporate a key factor into their analysis: reciprocity within the sanction network.\footnote{With the one major exception being \cite{cranmer2014reciprocity}, but the focus of their work has been on the sanctioning process not on compliance, which we expand on in further discussion below.} Drawing on the work in international relations on trade and conflict, we suggest that sanction cases are best conceptualized as a network phenomenon and must be modeled as such. Reciprocity is a not a new concept to the field of international studies, but has its roots in previous theories of cooperation and the evolution of norms between states.\footnote{\cite{richardsonai:1960,choucri:north:1972,goldstein1991reciprocity,rajmaira1990evolving,ward1992reciprocity}} Yet the study of sanctions has not addressed how reciprocity's effect on state behavior might condition the effects of other variables on sanction compliance. 

We analyze this key endogenous structure inherent to network dynamics, reciprocity, and argue that the structure created by reciprocal interactions over time must be accounted for in studies of sanction outcomes. Further, we extend on previous work suggesting duration models as the most appropriate approach for modeling sanctions outcomes by incorporating network measures into the duration framework. In doing so we are able to return to key hypotheses from the literature and assess whether factors such as domestic political institutions and internal stability influence sanction outcomes once network dynamics are adequately incorporated into the model. 

We leverage the network modeling approach to produce an accurate test of when and why sanctions end. In the following section, we review previous work on compliance and introduce the network concept. We then present our central argument and hypotheses; in doing so we articulate the various ways that networks can be conceptualized in this context. Last, we present our findings and review the results.