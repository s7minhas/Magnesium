\section*{Introduction}
\label{intro}

Economic sanctions are a frequently used foreign policy tool in the realm of international relations. Typically, one or more states initiate sanctions against another when they desire to force that state to undertake a policy change. Policy change is expected to occur by depriving the sanction target of trade or other forms of economic exchange with the sanction initiator(s). The trigger for economic sanctions can occur in many contexts: the target state breaks a previous agreement, the target state openly disobeys international law, or the target state engages in behavior that is simply unfavorable to the political preferences of another state. The consequences of sanctions for the populations in target countries can be severe, including increased unemployment, lost foreign investment, and reduced trade flows \citep{hufbauer2003impact,hufbauer1997us}. 

% They represent something between a “diplomatic
% slap on the wrist” and “more extreme measures such as covert action or
% military measures” (Hufbauer, Schott and Elliott, 1990, p. 11).

% Sanctions may seem appealing in principle, since compared to war they may
% provide a lower-cost method of punishing departures from international standards
% of conduct and of resolving disputes between countries. I

Causing economic disruption is explicitly the intent of sanctions, as Woodrow Wilson told the United States Senate sanctions are a ``peaceful, silent deadly remedy'' for coercing concessions from other states \citep{foley23}. Though sanctions have often succeeded in disrupting economic activity in target states \citep{escriba2010dealing}, their ability to force policy change is one of heated debate among both policymakers and scholars. The motivations for sanction initiation are cross-cutting, spanning a diverse and interdependent mix of policy issues and political actors. While the concept of sanctions -- the idea that countries can put pressure via economic ties to other countries in order to influence policy -- is relatively straightforward, the study of when and why sanctions work is complex.

% However, in recent decades, economic
% sanctions have been pursued for a much broader range of international goals:
% forestalling war; hastening the achievement of freedom and democracy; cleaning
% up the environment; strengthening human rights or labor rights; nuclear nonproliferation;
% the freeing of captured citizens; and the reversal of captures of land.
% Sanctions have become a standard and routine policy tool of nations and international
% organizations for addressing any actions of a targeted nation that the
% targeting nation or group of nations disagreed with. In large measure, sanctions are
% meant to influence the behavior of foreign nations, now or in the future, by
% providing current constraints or promising them for the future. They may also serve
% some role in meeting domestic political concerns

Earlier research on sanctions argued that sanctions have little influence on targets.\footnote{\cite{lam1990, dashti1997, morgan1997, drezner1998}} More recent research suggests that the effectiveness of sanctions is dependent on an interaction of several factors, namely: the number of senders sanctioning a target state and the type of issue in dispute;\footnote{\cite{miers2002, morgan2009threat}} the strength of domestic institutions within the target state; and the type of regime governing the target state.\footnote{\cite{mcgillivray2004}} 

The theoretical and empirical literatures on economic sanctions demonstrate that several different, interacting conditions underline sanction outcomes. We argue, however, that scholars have thus far failed to incorporate a key factor into their analysis of sanction outcomes: reciprocity.\footnote{Previous work by \cite{cranmer2014reciprocity} has highlighted the role that network effects such as reciprocity have in the creation of new sanctions, but they also did not address issues of compliance.} Drawing on the work in international relations on trade and conflict, we suggest that sanction cases are best conceptualized as a network phenomenon and must be modeled as such. Reciprocity is a not a new concept to the field of international studies, but has its roots in previous theories of cooperation and the evolution of norms between states.\footnote{\cite{richardsonai:1960,choucri:north:1972,goldstein1991reciprocity,rajmaira1990evolving,ward1992reciprocity}} Yet the study of sanctions has not addressed how reciprocity's effect on state behavior might condition the effects of other variables on sanction compliance. 

We analyze this key endogenous structure inherent to network dynamics, reciprocity, and argue that the structure created by reciprocal interactions over time must be accounted for in studies of sanction outcomes. Further, we extend on previous work suggesting duration models as the most appropriate approach for modeling sanctions outcomes by incorporating network measures into the duration framework. In doing so we are able to return to key hypotheses from the literature and assess whether factors such as domestic political institutions and internal stability influence sanction outcomes once network dynamics are adequately incorporated into the model. 

We leverage the network modeling approach to produce an accurate test of when and why sanctions end. In the following section, we review previous work on compliance and introduce the network concept. We then present our central argument and hypotheses; in doing so we articulate the various ways that networks can be conceptualized in this context. Last, we present our findings and review the results.

% Wang, McClean, Kuberski
% Why do some economic sanction threats lead to concessions, whereas other threats are followed by sanction imposition? Sanctions of- ten lead to substantial economic losses and even signifi- cant humanitarian suffering and human rights violations in the sanctioned country (Escriba`-Folch and Wright 2010; Wood 2008). At the same time, previous research suggests that sanctions are often ineffective because they rarely deliver the desired outcome for the sender (Pape 1997). Yet, sanctions remain a commonly used foreign policy instrument (Hufbauer et al. 2007).


% The question, "Do economic sanctions work?"
% has been perhapsthe most fundamentalinquiryin the
% literaturedebatingtheeffectivenessofsanctionsa,ndthe
% conventionalwisdomappearsto be thatsanctionsare
% ineffectiveand failed policy instrumentsin the vast
% majorityof cases (Galtung 1967; Wallensteen1968;
% HSE; Pape 1997, 1998; Drury 1998; Elliott 1998).
