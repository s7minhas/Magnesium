\section*{Introduction}
\label{intro}

Economic sanctions are a frequently used foreign policy tool in the realm of international relations. Typically, one or more states initiate sanctions against another to force the target state to change policy. Policy change is expected to occur by depriving the sanction target of trade or other forms of economic exchange with the sanction initiator(s). Triggers for economic sanctions can occur in many contexts: the target state breaks a previous agreement, the target state openly disobeys international law, or the target state engages in behavior that is simply unfavorable to the political preferences of another state. The consequences of sanctions for the populations in target countries can be severe, including increased unemployment, foreign investment loss, and reduced trade flows \citep{hufbauer2003impact,hufbauer1997us}. 

Economic disruption is the general intent of economic sanctions, as Woodrow Wilson told the United States Senate sanctions are a ``peaceful, silent deadly remedy'' for coercing concessions from other states \citep{foley23}. But motivations for sanction initiation are cross-cutting, spanning a diverse and interdependent mix of policy issues and political actors. Though sanctions often succeed in disrupting economic activity in target states \citep{escriba2010dealing}, their ability to force policy change is debated among both policymakers and scholars.  While the concept of sanctions -- the idea that countries can put pressure via economic ties to other countries in order to influence policy -- is relatively straightforward, the study of when and why sanctions work is complex.

Early research argued that sanctions have little influence on targets,\footnote{\cite{lam1990, dashti1997, morgan1997, drezner1998}} but recent work suggests that the effectiveness of sanctions is dependent on the interaction of several factors, namely: the number of senders sanctioning a target state and the type of issue in dispute;\footnote{\cite{miers2002, morgan2009threat}} the strength of domestic institutions within the target state;\footnote{\cite{dashti1997,marinov2005}} and the type of regime governing the target state.\footnote{\cite{mcgillivray2004,lektzian2007,allen2008domestic}} 

% SM Note: were you thinking of some studies when you wrote this: "the strength of domestic institutions within the target state". i couldn't find any specific references to strengths of domestic institutions. I guessed that you were thinking of the Dashti et al and Marinov studies. %yeah I was, does this work then?

We argue that scholars have thus far failed to incorporate a key factor into their analysis of sanction outcomes: reciprocity.\footnote{Previous work by \cite{cranmer2014reciprocity} has highlighted the role that network effects such as reciprocity have in the creation of new sanctions, but they also did not address issues of compliance.} Drawing on the work in international relations on trade and conflict, we suggest that sanction cases are best conceptualized as a network phenomenon and must be addressed both theoretically and empirically in these terms. Reciprocity is not a new concept to the field of international studies, but has its roots in previous theories of cooperation and the evolution of norms between states.\footnote{\cite{richardsonai:1960,choucri:north:1972,goldstein1991reciprocity,rajmaira1990evolving,ward1992reciprocity}} Yet the study of sanctions has not addressed how reciprocity drives  states' strategic calculation of sanction compliance. 

We analyze this key endogenous structure inherent to network dynamics and argue that the structure created by reciprocal interactions over time must be accounted for in studies of sanction outcomes. Further, we incorporate our measures of reciprocal interactions into a duration modeling framework, thus enabling us to explicitly account for interdependencies in the resolution of sanction cases. In doing so we are able to then return to test key hypotheses from the literature and assess the extent to which factors such as domestic political institutions and internal stability influence sanction outcomes once network dynamics are incorporated.  

We leverage our network modeling approach to produce an accurate test of when and why sanctions end. In the following section, we review previous work on compliance and introduce the network concept. We then present our central argument and hypothesis, while doing so we articulate the various ways that networks can be conceptualized in this context. Last, we present our findings and review the results.
