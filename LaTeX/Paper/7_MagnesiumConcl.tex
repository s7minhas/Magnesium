\section*{Conclusion}
\label{conclusion}

%Economic sanctions continue to be a frequently used tool among policymakers and play an important role in global politics. Indeed, in response to the current crisis in the Ukraine, the European Union and the United States both launched sanctions against Russia, and Russia quickly responded with sanctions of its own. For now there is little sign that any of the sanctions initiated in response to Russia's transgressions in Ukraine will be resolved quickly. The exchange between Russia, the European Union, and the West highlights the endogenous nature of sanction initiation. Further the likelihood of a quick resolution is tempered by the fact that these are states without a notable history of positive reciprocal interactions. 

% one key theme of this paper: sanctions are driven by the interdependent nature of the international system. 
% SM: it wasn't clear to me how the Russian case highlights the interdependent nature of the system
% See here for info on Russian sanctions: https://euobserver.com/foreign/125205

We have outlined both theoretical and empirical reasons for how the initiation and duration of sanctions between states is dependent upon network attributes and shown that integrating these network attributes into the study of sanction compliance is essential for robust inferences. Our empirical analyses clearly demonstrate the key role of reciprocity in determining the duration of economic sanctions. We find strong support for the influence of reciprocal compliance, suggesting that the most effective sanctions are likely to be those initiated by a higher number of senders with a history of positive reciprocal interactions. Similarly, we highlight a previously little examined aspect of sanction behavior and show that countries whom have sanctioned each other in the past without complying to one another are unlikely to comply with another in the present. 

Incorporating these network attributes into our models of sanction compliance, clearly enables us to arrive at more precise models of when the targets of sanctions comply. Further, our findings strongly complement the existing literature. We continue to find that wealthier countries are less likely to comply quickly, and we find that multilateral sanctions are more effective. Unlike the extant literature, however, we find little support for other prominent hypotheses such as the argument that more democratic institutions comply sooner and that countries are more likely to comply when faced with sanctions by trading partners. 

More importantly, while much of the research on international sanctions has focused on explaining how monadic characteristics of target states or formalized linkages between states drive sanction outcomes, our study provides a theoretical framework of interstate politics as a nuanced networked phenomenon wherein relations between states develop overtime as a function of continued informational and behavioral exchange. Ignoring network attributes in our study of sanction compliance forces us to assume that the likelihood of any sanction case being resolved is independent of any other. This is an untenable assumption; international politics, more broadly, is a system in which actors are interacting with one another simultaneously across a host of issues, and these interactions provide us with a great deal of information on how we can expect them to behave in the future. Utilizing network theory allows us to incorporate these interactions in a principled way, as we have demonstrated here in our study of sanction compliance.

\newpage
\section*{Appendix}
\label{appendix}

\subsection*{Imputation Procedure}
\label{appImp}

The copula based approach developed by \citet{hoff:2007} is estimated through a Markov chain Monte Carlo (MCMC) algorithm. We run the MCMC for 6,000 imputations, saving every sixth imputation, using the \texttt{sbgcop.mcmc} function in the \texttt{sbgcop} package in $\mathcal{R}$. To account for time trends and obtain better performance from this imputation procedure, we create five lags of each variable, except for polity, prior to imputation. Every imputation of the MCMC leads to the creation of one dataset with all missing values imputed. Running this algorithm on our dataset then produces a total of 1,000 imputed datasets. Results across these 1,000 imputed datasets are then averaged, thereby accounting for a portion of the uncertainty in the imputed values. We then used the average of the results from these 1,000 imputed datasets to generate the regression estimates in table \ref{tab:regResults}. 

% latex table generated in R 3.0.2 by xtable 1.7-3 package
% Wed Jun 25 01:22:06 2014
\begin{table}[ht]
\centering
{\normalsize
\begin{tabular}{lccc}
 Variable & Model 1 & Model 2 & Model 3 \\ 
  \hline
\hline
Compliance Reciprocity$_{j,t-1}$ &  &  & $0.293^{\ast\ast}$ \\ 
   &  &  & (0.068) \\ 
  Sanction Reciprocity$_{j,t-1}$ &  &  & $-0.13^{\ast\ast}$ \\ 
   &  &  & (0.033) \\ 
   \hline
Number of Senders$_{j,t}$ &  & $0.358^{\ast\ast}$ & $0.33^{\ast\ast}$ \\ 
   &  & (0.065) & (0.067) \\ 
  Distance$_{j,t}$ &  & $0.438^{\ast\ast}$ & $0.408^{\ast\ast}$ \\ 
   &  & (0.198) & (0.203) \\ 
  Trade$_{j,t}$ &  & $97.145^{\ast\ast}$ & $102.839^{\ast\ast}$ \\ 
   &  & (24.315) & (24.462) \\ 
  Ally$_{j,t}$ &  & 0.025 & 0.056 \\ 
   &  & (0.196) & (0.193) \\ 
   \hline
Polity$_{i,t-1}$ & $-0.013^{\ast}$ & -0.007 & -0.008 \\ 
   & (0.007) & (0.008) & (0.008) \\ 
  Ln(GDP per capita)$_{i,t-1}$ & $-0.126^{\ast\ast}$ & -0.019 & -0.005 \\ 
   & (0.06) & (0.075) & (0.078) \\ 
  GDP Growth$_{i,t-1}$ & 0.024 & $0.042^{\ast\ast}$ & 0.032 \\ 
   & (0.02) & (0.021) & (0.021) \\ 
  Population$_{i,t-1}$ & -0.072 & 0.041 & $0.137^{\ast}$ \\ 
   & (0.054) & (0.064) & (0.07) \\ 
  Internal Conflict$_{i,t-1}$ & -0.004 & 0 & -0.011 \\ 
   & (0.017) & (0.018) & (0.018) \\ 
   \hline
n & 6140 & 6106 & 6106 \\ 
  Events & 160 & 159 & 159 \\ 
  Likelihood ratio test & 17.31 (0) & 59.16 (0) & 77.77 (0) \\ 
   \hline
\hline
\end{tabular}
}
\caption{Duration model on unimputed data with time varying covariates estimated using Cox Proportional Hazards. Standard errors in parentheses. $^{**}$ and $^{*}$ indicate significance at $p< 0.05 $ and $p< 0.10 $, respectively.} 
\label{tab:regResultsNoImp}
\end{table}

\FloatBarrier

Regression results on the original dataset are shown in table \ref{tab:regResultsNoImp}. The results, particularly for our reciprocity variables, are nearly identical. 

\newpage

\subsection*{Summary Statistics}
\label{appSumm}

Below we show summary statistics for the imputed, table~\ref{tab:summNoImp}, and original, table~\ref{tab:summImp}, datasets.

% latex table generated in R 3.1.2 by xtable 1.7-4 package
% Sat May 23 14:51:36 2015
\begin{table}[ht]
\centering
{\normalsize
\begin{tabular}{lcccccc}
 Variable & N & Mean & Median & Std. Dev. & Min. & Max. \\ 
  \hline
\hline
Compliance & 3764 & 0.02 & 0 & 0.14 & 0 & 1 \\ 
  Compliance Reciprocity$_{j,t-1}$ & 3764 & 2.05 & 0.19 & 4.3 & -1.52 & 17.25 \\ 
  Sanction Reciprocity$_{j,t-1}$ & 3764 & 7.16 & 2.8 & 9.95 & -2.62 & 43.87 \\ 
  Number of Senders$_{j,t}$ & 3764 & 1.26 & 1 & 0.96 & 1 & 5 \\ 
  Distance$_{j,t}$ & 3764 & 0.12 & 0 & 0.32 & 0 & 1 \\ 
  Trade$_{j,t}$ & 3764 & 0.01 & 0.01 & 0 & 0 & 0.07 \\ 
  Ally$_{j,t}$ & 3764 & 0.51 & 1 & 0.5 & 0 & 1 \\ 
  Polity$_{i,t-1}$ & 3764 & 16.49 & 20 & 5.96 & 0 & 20 \\ 
  Ln(GDP per capita)$_{i,t-1}$ & 3764 & 8.91 & 9.41 & 1.51 & 4.5 & 10.94 \\ 
  GDP Growth$_{i,t-1}$ & 3764 & 3.56 & 3.41 & 3.69 & -30.51 & 42.41 \\ 
  Population$_{i,t-1}$ & 3764 & 17.95 & 17.89 & 1.37 & 12.65 & 20.99 \\ 
  Internal Conflict$_{i,t-1}$ & 3764 & 2.56 & 1 & 4.56 & 0 & 87 \\ 
   \hline
\hline
\end{tabular}
}
\caption{Summary statistics of parameters included in duration model using imputed data.} 
\label{tab:summImp}
\end{table}

\FloatBarrier

% latex table generated in R 3.1.2 by xtable 1.7-4 package
% Sat May 23 14:51:45 2015
\begin{table}[ht]
\centering
{\normalsize
\begin{tabular}{lcccccc}
 Variable & N & Mean & Median & Std. Dev. & Min. & Max. \\ 
  \hline
\hline
Compliance & 3579 & 0.02 & 0 & 0.13 & 0 & 1 \\ 
  Compliance Reciprocity$_{j,t-1}$ & 3579 & 2.1 & 0.28 & 4.35 & -1.52 & 17.25 \\ 
  Sanction Reciprocity$_{j,t-1}$ & 3579 & 7.37 & 2.94 & 10.11 & -2.62 & 43.87 \\ 
  Number of Senders$_{j,t}$ & 3579 & 1.25 & 1 & 0.92 & 1 & 5 \\ 
  Distance$_{j,t}$ & 3579 & 0.12 & 0 & 0.32 & 0 & 1 \\ 
  Trade$_{j,t}$ & 3579 & 0.01 & 0.01 & 0 & 0 & 0.04 \\ 
  Ally$_{j,t}$ & 3579 & 0.54 & 1 & 0.5 & 0 & 1 \\ 
  Polity$_{i,t-1}$ & 3579 & 16.86 & 20 & 5.63 & 0 & 20 \\ 
  Ln(GDP per capita)$_{i,t-1}$ & 3579 & 9.02 & 9.58 & 1.44 & 4.52 & 10.94 \\ 
  GDP Growth$_{i,t-1}$ & 3579 & 3.53 & 3.32 & 3.58 & -22.93 & 19.69 \\ 
  Population$_{i,t-1}$ & 3579 & 18.03 & 17.91 & 1.34 & 14.64 & 20.99 \\ 
  Internal Conflict$_{i,t-1}$ & 3579 & 2.56 & 1 & 4.54 & 0 & 87 \\ 
   \hline
\hline
\end{tabular}
}
\caption{Summary statistics of parameters included in duration model using original data.} 
\label{tab:summNoImp}
\end{table}

\FloatBarrier

\subsection*{Security Sanctions}
\label{appSecSanc}

Below we show the results for model 3 from Table~\ref{tab:regResults} when limiting our analysis to sanctions that were imposed for security related reasons.

% latex table generated in R 3.2.2 by xtable 1.7-4 package
% Sat Jul 16 11:01:12 2016
\begin{table}[ht]
\centering
{\normalsize
\begin{tabular}{lcc}
 Variable & Model 1 & Model 2 \\ 
  \hline
\hline
Compliance Reciprocity$_{j,t-1}$ & $0.68^{\ast\ast}$ & $0.75^{\ast\ast}$ \\ 
   & (0.34) & (0.25) \\ 
  Sanction Reciprocity$_{j,t-1}$ & $-0.52^{\ast\ast}$ & $-0.48^{\ast\ast}$ \\ 
   & (0.23) & (0.18) \\ 
   \hline
Number of Senders$_{j}$ & 0.1 & $0.11^{\ast}$ \\ 
   & (0.07) & (0.06) \\ 
  Distance$_{j}$ & 0.39 & $0.51^{\ast}$ \\ 
   & (0.34) & (0.28) \\ 
  Trade$_{j,t-1}$ & -2.99 & -2.73 \\ 
   & (5.72) & (4.88) \\ 
  Ally$_{j,t-1}$ & -0.31 & -0.18 \\ 
   & (0.29) & (0.25) \\ 
   \hline
Polity$_{i,t-1}$ & 0.02 & 0.02 \\ 
   & (0.02) & (0.02) \\ 
  Ln(GDP per capita)$_{i,t-1}$ & -0.02 & -0.03 \\ 
   & (0.1) & (0.09) \\ 
  GDP Growth$_{i,t-1}$ & 0 & 0 \\ 
   & (0.02) & (0.02) \\ 
  Population$_{i,t-1}$ & -0.05 & -0.02 \\ 
   & (0.09) & (0.07) \\ 
  Internal Conflict$_{i,t-1}$ & 0.01 & 0 \\ 
   & (0.02) & (0.02) \\ 
   \hline
n & 1535 & 2245 \\ 
  Events & 71 & 101 \\ 
  Likelihood ratio test & 12.17 (0.35) & 17.02 (0.11) \\ 
   \hline
\hline
\end{tabular}
}
\caption{Here we focus on predicting the time until compliance for sanctions not related to economic issues. The first column shows duration model results on unimputed data with time varying covariates estimated using Cox Proportional Hazards, and the second with using imputed data. Standard errors in parentheses. $^{**}$ and $^{*}$ indicate significance at $p< 0.05 $ and $p< 0.10 $, respectively.} 
\label{tab:regResultsNonEconSanctions}
\end{table}

\FloatBarrier

\newpage