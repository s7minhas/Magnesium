\section*{Conclusion}
\label{conclusion}

We have outlined both theoretical and empirical reasons for why sanctioning behavior between states constitutes a network, and thus requires scholars to incorporate network attributes into the study of sanction compliance. We have then demonstrated the key role of reciprocity in determining the duration of economic sanctions in the international network, while also assessing longstanding hypothesis from the literature. In doing so, we are able to construct a more accurate representation of sanction dynamics than has yet been presented in the literature. 

We find strong support for the influence of reciprocal compliance as well as the number of sender states within the sanction network, suggesting that the most effective sanctions are likely to be those composed of a higher number of senders with this shared compliance history. Similarly, we highlight a previously underexamined aspect of sanction behavior and show that countries whom have sanctioned each other in the past without complying to one another are not liekly to comply to another in the present. We also find support that bolsters previous claims from the literature: even when accounting for network effects, trade partners are more likely to successfully utilize sanctions against target states.  




\newpage
\section*{Appendix}
\label{appendix}

\subsection*{Imputation Procedure}

The copula based approach developed by \citet{hoff:2007} is estimated through a Markov chain Monte Carlo (MCMC) algorithm. We run the MCMC for 6,000 imputations, saving every sixth imputation, using the \texttt{sbgcop.mcmc} function in the \texttt{sbgcop} package in $\mathcal{R}$. To account for time trends and obtain better performance from this imputaiton procedure, we create five lags of each variable, except for polity, prior to imputation. Every imputation of the MCMC leads to the creation of one dataset with all missing values imputed. Running this algorithm on our dataset then produces a total of 1,000 imputed datasets. Results across these 1,000 imputed datasets are then averaged, thereby accounting for a portion of the uncertainty in the imputed values. We then used the average of the results from these 1,000 imputed datasets to generate the regression estimates in table \ref{tab:regResults}. 

Regression results on the unimputed dataset are shown in table \ref{tab:regResultsNoImp}. The results, particularly for our reciprocity variables, are nearly identical. 

% latex table generated in R 3.1.2 by xtable 1.7-4 package
% Sun May 17 06:32:03 2015
\begin{table}[ht]
\centering
{\normalsize
\begin{tabular}{lccc}
 Variable & Model 1 & Model 2 & Model 3 \\ 
  \hline
\hline
Compliance Reciprocity$_{j,t-1}$ &  &  & $0.36^{\ast\ast}$ \\ 
   &  &  & (0.114) \\ 
  Sanction Reciprocity$_{j,t-1}$ &  &  & $-0.155^{\ast\ast}$ \\ 
   &  &  & (0.058) \\ 
   \hline
Number of Senders$_{j,t}$ &  & $0.439^{\ast\ast}$ & $0.429^{\ast\ast}$ \\ 
   &  & (0.092) & (0.093) \\ 
  Distance$_{j,t}$ &  & $1.314^{\ast\ast}$ & $1.345^{\ast\ast}$ \\ 
   &  & (0.295) & (0.3) \\ 
  Trade$_{j,t}$ &  & $80^{\ast}$ & $93.478^{\ast\ast}$ \\ 
   &  & (46.616) & (45.695) \\ 
  Ally$_{j,t}$ &  & -0.065 & -0.064 \\ 
   &  & (0.289) & (0.297) \\ 
   \hline
Polity$_{i,t-1}$ & 0.05 & $0.087^{\ast\ast}$ & $0.083^{\ast\ast}$ \\ 
   & (0.03) & (0.032) & (0.033) \\ 
  Ln(GDP per capita)$_{i,t-1}$ & $-0.393^{\ast\ast}$ & $-0.372^{\ast\ast}$ & $-0.337^{\ast\ast}$ \\ 
   & (0.103) & (0.126) & (0.131) \\ 
  GDP Growth$_{i,t-1}$ & 0.041 & 0.06 & 0.044 \\ 
   & (0.036) & (0.038) & (0.037) \\ 
  Population$_{i,t-1}$ & $-0.232^{\ast\ast}$ & -0.126 & -0.036 \\ 
   & (0.091) & (0.111) & (0.12) \\ 
  Internal Conflict$_{i,t-1}$ & -0.003 & 0.002 & 0 \\ 
   & (0.03) & (0.029) & (0.029) \\ 
   \hline
n & 3592 & 3579 & 3579 \\ 
  Events & 64 & 64 & 64 \\ 
  Likelihood ratio test & 21.25 (0) & 56.76 (0) & 66.98 (0) \\ 
   \hline
\hline
\end{tabular}
}
\caption{Duration model on unimputed data with time varying covariates estimated using Cox Proportional Hazards. Standard errors in parentheses. $^{**}$ and $^{*}$ indicate significance at $p< 0.05 $ and $p< 0.10 $, respectively.} 
\label{tab:regResultsNoImp}
\end{table}

\FloatBarrier

\newpage