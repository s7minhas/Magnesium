In this paper we investigate when and why states comply with economic sanctions.  The literature has suggested two ways of modeling this phenomenon. First, previous literature has suggested a duration modeling approach to capture the time it takes for a sanction to ``work". This approach, however, has failed to carefully account for important dynamics relevant to the modeling of sanction outcomes. Namely, present duration approaches fail to incorporate the network pressures instrinstic to international sanction processes. Target states face a network of sanctioners, not just an individual sender state. We present a duration model that incorporates the interdependent nature of the international system. In doing so we are able to test two prominent hypotheses from the literature: (1) does dependence between the target state and its sanctioning network decrease the time until target compliance; and (2) are democratic states more susceptible to these network pressures than others?