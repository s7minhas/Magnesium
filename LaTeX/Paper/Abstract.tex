This article explores when and why states comply with sanctions. Previous literature has suggested a duration modeling approach is needed to adequately capture the time it takes for a sanction to "work." This approach, however, has failed to carefully account for important dynamics relevant to the modeling of sanction outcomes. Namely, present duration approaches fail to incorporate the network effects intrinsic to international sanction processes. At any given time, target states typically face both a network of sanctioners for an individual sanction case, as well as a general network of sanctioners including senders from multiple cases. We present a model that incorporates this interdependent nature of the international system by including network effects within the duration model. In addition, we are able to test whether traditional conditions that the literature claims as critical for predicting sanction compliance, such as domestic institutions, are still influential once network dynamics are adequately modeled. In doing so we are able to test two key hypothesis: (1) does dependence between the target state and its sanctioning network increase the probability of target compliance; and (2) do domestic institutions condition network effects?