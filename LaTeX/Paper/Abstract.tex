In this paper we investigate when and why states comply with different types of sanctions (e.g., economic dispute, territorial dispute). The approach the extant literature has taken in modeling sanction compliance is 


 extant literature on sanction compliance has advanced in two primary ways of modeling this phenomenon. First, previous literature has suggested a duration modeling approach to capture the time it takes for a sanction to ``work". Second,  

effects of domestic institutions and stability within the target state to explain the time until compliance. These approaches, however, fail to account for important dynamics relevant to the modeling of sanction outcomes. Namely, the network pressures instrinstic to international sanction processes. Target states face a network of sanctioners, not just an individual sender state. We present a duration model that incorporates the interdependent nature of the international system. In doing so we are able to test two prominent hypotheses from the literature: (1) how and what typedoes dependence between the target state and its sanctioning network decrease the time until target compliance; and (2) are democratic states more susceptible to these network pressures than others?