\section{Introduction}
\label{intro}

I completely agree that understanding why states try to string together as broad a base as possible when constructing sanctions is key. But we should take that a step further to understanding if there are a set of states which are crucial to determining the power of sanctions in bringing about change. For example, the US and EU put sanctions on N. Korea but what do they care, they have no strong economic ties with them nor do they share any alliances with them, etc. But if China were to join then N. Korea may begin to pay attention because it relies on China for economic support and in many other ways. This is what I was trying to get at with the economic and reputational argument. The idea that sanctions are not just about finding ways to harm a state directly but also about isolating it from the rest of the international system. 

From the economic side, the US will try to get as many of Magnesia's trading partners to join the sanctions because it knows that's the best way to make Magnesia really feel the pain of the sanctions. For example, with Iran the US spent a good deal of effort roping in countries like China and Russia because they were Iran's major export market. Having them participate in sanctions deprives Iran of important sources of income for their economy. 

From the reputational side, I was being completely unclear, lol. But I was using it as a catch-all for all the other non-economic ties that bind states - so forget about phrasing it as reputation. Also I wasn't trying to imply that there would be costs to sanction joiners from the sanctioned state in the future, there very well might be but I think that's a future question. All I am saying is that there are non-economic ties that bind states such as things like shared cultural heritage and maybe even geographic proximity. When states with these close ties join in sanctions against poor, old Magnesia then that would do more damage to regime stability in Magnesia than if the sanctioning states were just a bunch of gun-toting hillbillies a world away. 

I agree that this latter argument is underdeveloped but I think it would be interesting for us to be able to talk about the proximity of states on some space other than simply geographic, or economic. For example, don't we think that sanctions on Syria from the US mean less to it than sanctions from Saudi Arabia and that sanctions from Saudi Arabia mean less to it than sanctions from Iran? I don't think that we can capture that type of variation with purely an economic variable. 

In terms of types of sanctions yeah I think we should be as broad as possible but the key point here is that sanctions, at least from our network perspective, are only effective if they can isolate the target state from those that it relies on or considers as its peers. 

We haven't at all talked about the all important monadic level here yet either...countries like Syria, N Korea, and Iran feel free to plough forward with their plans despite a bevy of sanctions from multiple countries. Additionally, Iran saw no regime change despite years of sanctions from everybody. Last there is a major sample bias problem as well. Maybe target states always back down before its close economic and other peer countries sign onto sanctions, so just the threat of them joining in the sanctions is enough to make the target state back down. 