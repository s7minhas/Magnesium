\section*{Theory}

The standing hypothesis in the literature is that countries facing economic/diplomatic sanctions should have higher levels of internal instability or a higher likelihood of regime change. Yet few empirical studies actually find that this relationship holds in any substantive way. Our hypothesis is that some sanctions hurt more than others, and the amount of hurt is not contingent on just the type of sanction but on the relationship of the target state to the senders of the sanction. Not accounting for the relationship between actors may explain the null findings with regards to the effect of sanctions that so many empirical studies have determined.   \\

Yet, we have no clear hypothesis as to what relational factors may matter more or less, or whether this hypothesis is best tested through some combination of relational factors. Much of the IR literature would point to trading relationships as an important linkage between states that senders of sanctions could leverage against targets. However, our preliminary examination of this hypothesis found little evidence of a relationship (This is not to say that we should no longer consider trade linkages but we should develop other ideas as well). Other relational factors that we may want to consider are:

\begin{itemize}
	\item Distance
	\item Common majority religion
	\item Common dominant ethnic group
	\item Common political regimes
	\item Whether one of the senders is a major power
\end{itemize}

This is a paltry list that I have assembled but one reason why it's hard to think of these linkages is because the literature has done such little work in determining reasons why some states are able to leverage influence over others.

\section*{Scope \& Revised Theory???}

In terms of scope, the type of sanctions that we would obviously want to focus on are those defined as having the goal of destabilizing the regime. However, there are just 22 of these types of sanctions in the dataset so we will need to be more flexible. The only type of sanctions that I would exclude, using the categories defined by the issue variable of the TIES user manual, are:

\begin{itemize}
	\item Improve environmental policies
	\item Implement Economic Reform
	\item Other
\end{itemize}

Given the types of sanctions that we are including many of which may not even implicitly have the goal of destabilizing a regime, I am questioning our choice of internal instablity as a dependent variable. To some extent, what I'm wondering is if we may want to consider using a binary version of the final outcome variable from the TIES dataset instead as the DV. \\

Thus our theory would now be that the success of a particular sanction is contingent on the relationship between the senders and receiver of that sanction. Doesn't this seem like a cleaner test to you than trying to parse out a relationship between political instability and sanctions?

Employing this approach might even allow us to more carefully use relational factors within the model. For example, we could subset the type of sanctions we use in our analysis as only those involving trade, go through the process of constructing the weighting matrix for sanctions by the network of trading relationships, and then test to see whether or not target countries are more likely to acquiesce to sender demands when the senders are its top trading partners. \\

Similarly, we could also the same with other types of sanctions such as contain political influence, solve territorial dispute, improve human rights and use weighting matrices that would be particularly relevant in those issues like common political regime or ethnic groups. \\

This would be a big turn in the project. Alternatively we could also do a duration type analysis here where we look at the time it takes for a resolution to be reached instead of just a binary variable indicating that a resolution has been reached.

\section*{Modeling Strategy}

In terms of the modeling strategy, we already have the idea of incorporating network ties and we already have a method for generating the necessary variable to test the importance of these network ties. 